\chapter{Voice Dialog Flow}

Il seguente capitolo presenta degli esempi di interazioni fra l'utente e il dispositivo \glossario{Alexa}, esse potranno essere soggette a modifiche e ampliamenti nelle fasi successive del  \glossario{progetto}, a seconda delle esigenze e richieste del committente e del proponente.
Le interazioni sono strutturate come segue:
\begin{itemize}
	\item \textbf{U:} rappresenta l'utente
	\item \textbf{A:} rappresenta Alexa,
	\item \textbf{[Data]:} rappresenta un dato il cui contenuto è determinato da risorse non presenti nel dispositivo al momento dell'interazione
\end{itemize}


\section{Intents}
Per \glossario{intent} si intende la richiesta da parte dell'utente di effettuare una particolare operazione al fine di gestire l'esecuzione della \glossario{skill}. 
Questa sezione si occupa di elencare i principali \glossario{intents} a disposizione dell'utente, insieme agli output previsti una volta ricevuta la richiesta.
Si fa riferimento alla documentazione fornita da \glossario{Amazon} per lo sviluppo di una \glossario{skill} e la sua implementazione ottimale.\footnote{\url{https://developer.amazon.com/it/documentation}}


\subsection{Help Intent}
Permette all'utente di ottenere ulteriori informazioni, come il numero di workflow disponibili e il loro contenuto.
\subsubsection{English}
\begin{itemize}
	\intent{help}
	\intent{help me}
	\intent{can you help me}
	\intent{tell me my workflows}
	\intent{i'd like a list of my workflows}	
\end{itemize}
\subsubsection{Italiano}
	\begin{itemize}
	\intent{aiuto}	
	\intent{aiutami}
	\intent{aiuto per favore}
	\intent{dimmi i miei workflow}
	\intent{vorrei una lista dei miei workflow}
	\end{itemize}

\subsection{Cancel Intent}
Permette all'utente di cancellare un'azione in esecuzione (come l'avvio di un workflow).
\subsubsection{English}
\begin{itemize}
	\intent{cancel}
	\intent{never mind}
	\intent{forget it}
\end{itemize}

\subsubsection{Italiano}
\begin{itemize}
	\intent{annulla}
	\intent{non importa}
	\intent{lascia stare}
\end{itemize}

\subsection{Stop Intent}
Permette all'utente di uscire dalla skill.
\subsubsection{English}
\begin{itemize}	
	\intent{stop}
	\intent{off}
	\intent{shut up}
\end{itemize}

\subsubsection{Italiano}

\begin{itemize}
	\intent{ferma}
	\intent{smettila}
	\intent{basta}
	\intent{stop}
	\intent{spegni}
\end{itemize}


\subsection{Next Intent}
Consente di passare a un blocco successivo di un workflow, sia nel caso in cui esso sia ancora in esecuzione oppure abbia terminato le proprie operazioni. 
\subsubsection{English}
\begin{itemize}
	\intent{next}
	\intent{skip}
	\intent{skip forward}	
\end{itemize}

\subsubsection{Italiano}

\begin{itemize}
	
	\intent{prossimo}
	\intent{prossima}
	\intent{successivo}
	
\end{itemize}


\subsection{No Intent}
Consente all'utente di fornire una risposta negativa in seguito a una domanda di Alexa.
\subsubsection{English}
\begin{itemize}
	
	\intent{no}
	\intent{no thanks}
	
\end{itemize}

\subsubsection{Italiano}
\begin{itemize}	
	\intent{No}	
	\intent{No grazie}
	
\end{itemize}



\subsection{Pause Intent}
Permette di fermare temporaneamente l'esecuzione di un blocco all'interno di un workflow.
\subsubsection{English}
\begin{itemize}
	
	\intent {pause}
	\intent {pause that}
	
\end{itemize}

\subsubsection{Italiano}
\begin{itemize}
	
	\intent{metti in pausa}
	\intent{mettila in pausa}
	
\end{itemize}


\subsection{Resume Intent}
Premette di riprendere l'esecuzione di un blocco all'interno di un workflow precedentemente fermato.
\subsubsection{English}
\begin{itemize}
	
	\intent{resume}
	\intent{continue}
	\intent{keep going"}
	
\end{itemize}

\subsubsection{Italiano}
\begin{itemize}
	\intent{riprendi}
	\intent{continua}
	\intent{riprendilo}
	
\end{itemize}



\subsection{Yes Intent}
Consente all'utente di fornire una risposta positiva in seguito a una domanda di Alexa.
\subsubsection{English}
\begin{itemize}
	
	\intent{yes}
	\intent{yes please}
	\intent{sure"}
	
\end{itemize}

\subsubsection{Italiano}
\begin{itemize}
	\intent{sì}	
	\intent{sì per favore}	
	\intent{certo}		
\end{itemize}





\section{Avvio Skill}
\subsection{Italiano}





\subsection{Inglese}
Una conversazione tipica con Alexa per accendere la skill e aprire un workflow potrebbe essere la seguente:\\

U: Alexa, open MegAlexa \\
A: Welcome to MegAlexa, which workflow would you like to open?\\
U: Show my workflow\\
A: Your workflow are [Workflow]x3, do you want to hear more?\\
U: Yes, go on\\
A: Alrigh, there are also [Workflow] and [Workflow], which one should I open?\\
U: Initiate [Workflow] \\

Mentre è possibile che Alexa non dica nulla e che l'utente sappia già quale workflow scegliere, pertanto l'utente entrerà immediatamente nel workflow da lui detto: \\
U: Alexa, open MegAlexa\\
U: Launch [Workflow]\\

\section{Esempi Workflow}
\subsection{Workflow: Mattina}
\subsubsection{Italiano}
\subsubsection{Inglese}


\subsection{Workflow: Social}
\subsubsection{Italiano}
\subsubsection{Inglese}