\chapter{Introduzione}

\section{Scopo del prodotto}
Lo scopo del progetto è quello di sviluppare un applicativo Mobile in grado di creare delle routine personalizzate per gli utenti gestibili tramite\glossario{Alexa}di\glossario{Amazon}. L'obbiettivo è quello di creare\glossario{skill}in grado di avviare\glossario{workflow}creati dagli utenti fornendogli dei \glossario{connettori}.
\section{Glossario}
Al fine di evitare ogni ambiguità di linguaggio e massimizzare la comprensione dei documenti, i termini tecnici, di dominio, gli acronimi e le parole che necessitano di essere chiarite, sono riportate nel \glossariodocumento.\\
Le occorrenze dei vocaboli presenti nel \textit{Glossario} è marcata da una "G" maiuscola in pedice.

\section{Riferimenti}

\subsection{Riferimenti normativi}
\begin{itemize}
	
	\item \textbf{Capitolato$_{G}$ d'appalto C4} \footnote{\url{https://www.math.unipd.it/~tullio/IS-1/2018/Progetto/C4.pdf}};
	\item \textbf{Norme di Progetto:} \textit{\normediprogetto};
	\item \textbf{Verbale di incontro esterno:} \textit{VerbaleEsterno20181205} avvenuto presso la proponente Zero12 in data 2018-12-05;

\end{itemize}

\subsection{Riferimenti informativi}
\begin{itemize}
	
	\item \textbf{Glossario:} \textit{\glossariodocumento};
	\item \textbf{Presentazione del capitolato d'appalto C4} \footnote{\url{https://www.math.unipd.it/~tullio/IS-1/2018/Progetto/C4.pdf}};
	\item \textbf{Slide del corso:} lucidi forniti per il corso di Ingegneria del Software\footnote{\url{https://www.math.unipd.it/~tullio/IS-1/2018/}};
	\item \textbf{Documentazione di Amazon$_{G}$\footnote{\url{https://developer.amazon.com/it/documentation}}}.
\end{itemize}
