\chapter{Introduzione}

\section{Scopo del Documento}
Lo scopo di tale documento è chiarire e analizzare i \glossario{requisiti} identificati dal gruppo \glossario{ZeroSeven} per lo sviluppo del progetto \glossario{MegAlexa}, presentato nel \glossario{capitolato} scelto C4.
I contenuti di questo documento costituiranno poi la base per lo sviluppo architetturale del software, in conformità con le direttive concordate con la proponente zero12 s.r.l. e con i committenti Vardanega Tullio e Cardin Riccardo.

\section{Obbiettivo del prodotto}
Lo scopo del prodotto è la realizzazione di un applicativo mobile(compatibile con sistemi operativi \glossario{Android}) che permetta di creare e gestire \glossario{Workflow} contententi \glossario{skill} integrate ad \glossario{Amazion Alexa}.
%TODO: specificare a grandi linee quali tecnologie verranno coinvolte (AWS, Node, Aurora serverless, DynamoDB)% 

\section{Riferimenti}

\subsection{Riferimenti normativi}
\begin{itemize}
	
	\item \textbf{Capitolato d'appalto C4} \footnote{\url{https://www.math.unipd.it/~tullio/IS-1/2018/Progetto/C4.pdf}};
	\item \textbf{Norme di Progetto v 1.0.0};
	\item \textbf{Verbale di incontro esterno:} \textit{Verbale5122018} avvenuto presso la sede dell'azienda committente zero12 in data 2018-12-5;

\end{itemize}

\subsection{Riferimenti informativi}
\begin{itemize}
	
	\item \textbf{Glossario 1.0.0:} documento creato al fine di chiarire precisamente il significato di termini che potrebbero risultare ambigui, tale identificazione avviene tramite una \textit{G} maiuscola posta a pedice; 
	\item \textbf{Presentazione del capitolato d'appalto C4} \footnote{\url{https://www.math.unipd.it/~tullio/IS-1/2018/Progetto/C4.pdf}};
	\item \textbf{Slide del corso:} lucidi forniti per il corso di Ingenieria del Software. \footnote{\url{https://www.math.unipd.it/~tullio/IS-1/2018/}}
\end{itemize}
