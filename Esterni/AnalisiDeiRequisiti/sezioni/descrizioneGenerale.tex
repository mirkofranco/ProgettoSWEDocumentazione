\chapter{Descrizione generale} \label{DescrizioneGenerale}

\section{Funzioni del prodotto}
Il prodotto è un'applicazione Android e consentirà all'utente di impostare una \glossario{skill}, questa è composta dai vari \glossario{workflow} che l'utente crea. In ogni workflow sono presenti diversi blocchi per creare la routine. Le funzioni principali del prodotto sono:
\begin{itemize}
	\item Permettere il login all'utente;
	\item Permettere la registrazione all'utente;
	\item Creare dei\glossario{workflow}personalizzabili tramite l'aggiunta dei seguenti blocchi:
	\begin{itemize}
		\item Sveglia
		\item Pin
		\item Lista
		\item Twitter
		\item Mail
		\item Blocco di testo
		\item Calendario
		\item Notizie (lettura delle notizie tramite il blocco di Feed RSS)
		\item Sport (lettura delle notizie tramite il blocco di Feed RSS)
		\item Filtro
		\item Meteo
		\item Musica, tramite Amazon Music 
	\end{itemize}
	\item Permettere la modifica e la cancellazione delle\glossario{routine};
	\item Possibilità di eseguire i workflow tramite \glossario{Alexa};
	\item Ottenere un output vocale dal dispositivo\glossario{Alexa}in base alla routine eseguita.
\end{itemize}
In ogni momento l'utente potrà chiedere informazioni riguardo lo stato attuale del workflow o chiedere aiuto nella scelta di questo, permettendo quindi all'utente di non perdersi all'interno del workflow, questo viene agevolato da Amazon tramite le funzionalità offerte dagli \textit{Intents}, descritti nella sezione \ref{help}.

\section{Caratteristiche degli utenti}
Il prodotto è sviluppato per i privati e le aziende che vogliono utilizzare l'assistente vocale\glossario{Alexa}di\glossario{Amazon}.
L'utente, per accedere al sistema, dovrà essere registrato con una propria mail e password e aver effettuato un \textit{account}\glossario{linking}con il proprio account Amazon personale. Dopo il primo accesso all'applicazione, i successivi saranno automatici e non ci sarà bisogno di inserire nuovamente l'account Amazon, a meno che l'utente non voglia entrare con un account diverso, per far ciò è necessario effettuare il log out e fare un nuovo log in.

