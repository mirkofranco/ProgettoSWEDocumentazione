\chapter{Descrizione generale}

\section{Scopo del prodotto}
L'obbiettivo del prodotto è quello di permettere ad un utente di creare uno o più\glossario{workflow}personali leggibili tramite un dispositivo\glossario{Alexa}.
Per poter usufruire delle funzionalità di tale\glossario{prodotto}l'utente dovrà prima effettuare l'autenticazione all'applicativo con apposite username e password.

\section{Funzioni del prodotto}
\begin{itemize}
	\item Permettere il login all'utente;
	\item Permettere la registrazione all'utente;
	\item Creare dei\glossario{workflow}personalizzabili;
	\item Pemettere la modifica e la cancellazione delle\glossario{routine};
	\item Possibilità di eseguire i workflow tramite Alexa;
	\item Ottenere un output vocale dal dispositivo\glossario{Alexa}in base alla routine eseguita.
\end{itemize}

\section{Caratteristiche degli utenti}
Il prodotto è sviluppato per i privati e le aziende che vogliono utilizzare l'assistente vocale\glossario{Alexa}di\glossario{Amazon}.
L'utente, per accedere al sistema, dovrà essere registrato con una propria mail e password e aver effettuato un \textit{account}\glossario{linking}con il proprio account Amazon personale.


\section{Vincoli di progettazione}
\subsection{Requisiti desiderabili}
\begin{itemize}
	\item Realizzare una mobile app per dispositivi\glossario{Android};
	\item Il\glossario{prodotto}deve essere sviluppato principalmente in lingua inglese;
	\item Le\glossario{routine}di un utente devono essere univoche;
	\item L'applicativo deve utilizzare dei\glossario{connettori}generalizzati;
	\item Impedire che vengano eseguiti contemporaneamente più\glossario{workflow}della stessa\glossario{skill};
	\item I workflow creati vanno memorizzati nei servizi web a supporto della skill;
	\item Il prodotto deve saper riconoscere gli utenti attraverso delle chiavi salvate nel database.
\end{itemize}
\subsection{Requisiti opzionali}
\begin{itemize}
	\item Sviluppo del prodotto per dispositivi\glossario{iOS};
	\item Sviluppo del prodotto con interfaccia web.
\end{itemize}