\chapter{Descrizione generale}

\section{Scopo del prodotto}
L'obbiettivo del prodotto è quello di permettere ad un utente di creare uno o più workflow personali leggibili tramite un dispositivo Alexa.
Per poter usufruire delle funzionalità di tale prodotto l'utente dovrà prima effettuare l'autenticazione all'applicativo con apposite username e password.
Il prodotto realizzato sarà multilingua, quindi potrà essere utilizzato da un numero più ampio di utenti.


\section{Funzioni del prodotto}
\begin{itemize}
	\item Permettere il login all'utente;
	\item Creare delle skill per formare dei workflow personali;
	\item Pemettere la modifica e la cancellazione di skill o dei workflow;
	\item Inviare ad Alexa queste routine per essere elaborati;%ATTENZIONE: Non sono convinto che questo sia corretto; alexa può elaborare 1/2 skill alla volta e non ha l'infrastruttura per schedulare più skill mandate in blocco: i workflow vanno elaborati PRIMA e poi inviati ad Alexa tramite AWS%
	\item Mostrare i risultati tramite GUI mobile oppure "a voce" tramite dispositivo Alexa;
\end{itemize}

\section{Caratteristiche degli utenti}
Il prodotto è sviluppato per i privati e le aziende che vogliono utilizzare l'assistente vocale Alexa di Amazon.
L'utente, per accedere al sistema, deve avere un nome utente e una password personale ricevuta tramite registrazione al sito di Amazon(?).

\section{Vincoli di progettazione}
\subsection{Requisiti desiderabili}
\begin{itemize}
	\item Realizzare una mobile app per dispositivi Android;
	\item
	\item
	\item
\end{itemize}
\subsection{Requisiti opzionali}
\begin{itemize}
	\item Sviluppo del prodotto per dispositivi iOS;
	\item Sviluppo del prodotto con interfaccia web;
\end{itemize}