\chapter{Casi d'uso}
In questa sezione sono elencati i casi d'uso del progetto MegAlexa dedotti da un'attenta indagine ed analisi da parte dei membri del gruppo sugli attori principali del sistema, sulle loro caratteristiche e possibilità.
Ogni caso d'uso è identificato da un codice univoco e possiede una struttura interna accuratamente definita nel documento Norme Di Progetto 
v1.0.0.


\section{Interfaccia}
Ogni caso d'uso deve avere la seguente nomenclatura:

\begin{center}
	\textbf{UC$\Bigl\{$XX$\Bigr\}$.$\Bigl\{$YY$\Bigr\}$}
\end{center}
dove:
\begin{itemize}
	\item \textbf{UC:} Use Case
	\item \textbf{{XX}:} numero che identifica i casi d'uso.
	\item \textbf{{YY}:} numero progressivo che identifica i sottocasi, esso può, a sua volta, includere altri sottocasi.
\end{itemize}


\noindent\fbox{%
	\parbox{\textwidth}{%
		\subsubsection{UC0: Nome caso d'uso}
		
		\textbf{Descrizione}\\
		
		\textbf{Precondizione}\\
		
		\textbf{Postcondizione}\\
		
		\textbf{Scenario principale}
		\begin{enumerate}
			\item Lorem
			\item Ipsum
		\end{enumerate}
		\textbf{Estensioni}
		\begin{enumerate}
			\item[1a] Lorem
			\item[1b] Ipsum
		\end{enumerate}
		
		
		
		
	}%
}

\section{Attori dei casi d'uso}
\textbf{Attori primari}
	\begin{itemize}
	\item \textbf{Utente non autenticato}: si riferisce all'utente del sistema che non ha ancora eseguito il login;
	\item \textbf{Utente autenticato}: si riferisce all'utente del sistema che ha effettuato il login ed è stato autenticato.
	\end{itemize}
\textbf{Attori secondari}
	\begin{itemize}
	\item \textbf{Amazon}:/(da completare)
\end{itemize}





\section{Caso d'uso UC 1: Scenario principale dell'utente non autenticato}
\begin{itemize}
	\item \textbf{Attori primari}: Utente non autenticato;
	\item \textbf{Attori secondari}: Amazon;
	\item \textbf{Descrizione:} L'utente ha avviato l'applicazione e questa è pronta all'uso. L'utente, non ancora autenticato, può registrarsi al nostro servizio o effettuare il login, nel caso fosse già registrato;
	\item \textbf{Precondizione:} L'applicazione è avviata e pronta all'uso;
	\item \textbf{Postcondizione:} L'applicazione ha ricevuto tutte le informazioni dell'utente non autenticato sulle operazioni che vuole eseguire.
	\item \textbf{Flusso principale degli eventi}:
		\begin{enumerate}
			\item L'utente ha la possibilità di: Registrazione (UC 1.1);
			\item L'utente ha la possibilità di: Login (UC 1.2).
		\end{enumerate}
	\item \textbf{Estensioni}:
		\begin{enumerate}
			\item Visualizzazione errore sulla registrazione (UC 1.3);
			\item Autenticazione fallita (UC 1.4).
		\end{enumerate}
\end{itemize}
\section{Caso d'uso UC 1.1}

\section{Caso d'uso UC 1.2}

\section{Caso d'uso UC 2: Scenario principale dell'utente autenticato}
\begin{itemize}
	\item \textbf{Attori primari}: Utente autenticato;
	\item \textbf{Attori secondari}: ;
	\item \textbf{Descrizione:}
	\item \textbf{Flusso principale degli eventi}:
	\begin{enumerate}
		\item L'utente può creare un Workflow(UC 2.1);
		\item L'utente può modificare un Workflow(UC 2.2);
	\end{enumerate}
	\item \textbf{Estensioni}:

\end{itemize}

\section{UC 2.1}

\section{UC 2.2}

