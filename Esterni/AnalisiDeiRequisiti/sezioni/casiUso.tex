\chapter{Casi d'uso}

\section{Introduzione}

\section{Interfaccia}
Ogni caso d'uso deve avere la seguente nomenclatura:

\begin{center}
	\textbf{UC$\Bigl\{$XX$\Bigr\}$.$\Bigl\{$YY$\Bigr\}$}
\end{center}
dove:
\begin{itemize}
	\item \textbf{UC:} Use Case
	\item \textbf{{XX}:} numero che identifica i casi d'uso.
	\item \textbf{{YY}:} numero progressivo che identifica i sottocasi, esso può, a sua volta, includere altri sottocasi.
\end{itemize}


\noindent\fbox{%
	\parbox{\textwidth}{%
		\subsubsection{UC0: Nome caso d'uso}
		
		\textbf{Descrizione}\\
		
		\textbf{Precondizione}\\
		
		\textbf{Postcondizione}\\
		
		\textbf{Scenario principale}
		\begin{enumerate}
			\item Lorem
			\item Ipsum
		\end{enumerate}
		\textbf{Estensioni}
		\begin{enumerate}
			\item[1a] Lorem
			\item[1b] Ipsum
		\end{enumerate}
		
		
		
		
	}%
}

\section{Caso d'uso UC1: Operazioni ad alto livello}
\begin{itemize}
	\item \textbf{Attori}: Utente non autenticato;
	\item \textbf{Descrizione:} L'utente ha avviato l'applicazione e questa è pronta all'uso. L'utente, non ancora autenticato, può registrarsi al nostro servizio o effettuare il login, nel caso fosse già registrato;
	\item \textbf{Precondizione:} L'applicazione è avviata e pronta all'uso;
	\item \textbf{Postcondizione:} L'applicazione ha soddisfatto le volontà dell'utente;
	\item \textbf{Flusso principale degli eventi}:
		\begin{enumerate}
			\item L'utente ha la possibilità di: Registrazione (UC 1.1);
			\item L'utente ha la possibilità di: Login (UC 1.2).
		\end{enumerate}
	\item \textbf{Estensioni}:
		\begin{enumerate}
			\item Visualizzazione errore sulla registrazione (UC 1.3);
			\item Autenticazione fallita (UC 1.4)
		\end{enumerate}
\end{itemize}