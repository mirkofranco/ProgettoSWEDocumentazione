\chapter{Requisiti}
Vengono riportati i requisiti individuati, essi sono stati basandosi sui casi d'uso, sul capitolato, sui diversi incontri avuti con l'azienda \textit{s.r.l. Zero12} e per necessità interne. \\
I requisiti verranno inseriti con la seguenti nomenclatura:
\begin{center}
	\textbf{R$\Bigl\{$A$\Bigr\}$$\Bigl\{$B$\Bigr\}$$\Bigl\{$XX$\Bigr\}$.$\Bigl\{$YY$\Bigr\}$}
\end{center}
dove:
\begin{itemize}
	\item \textbf{A:} corrisponde a uno dei seguenti requisiti:
	\begin{itemize}
		\item 1: funzionale
		\item 2: di qualità
		\item 3: di prestazione
		\item 4: di vincolo
	\end{itemize}
	\item \textbf{B:} corrisponde a uno dei seguenti requisiti:
	\begin{itemize}
		\item O: obbligatorio
		\item F: facoltativo
		\item D: desiderabile
	\end{itemize}
	\item \textbf{{XX}:} numero che identifica i requisiti.
	\item \textbf{{YY}:} numero progressivo che identifica i sottocasi, esso può, a sua volta, includere altri sottocasi.
\end{itemize}

\section{Requisiti funzionali}
\begin{tabularx}{\textwidth}{|C|C|C|C|}
	\hline
     \textbf{Codice identificativo} & \textbf{Tipologia} & \textbf{Descrizione} & \textbf{Fonti} \\
    \hline
	\endhead
	R1O01& Funzionale Obbligatorio &L'utente non autenticato deve potersi registrare con un account amazon& UC1.1	\\
	\hline
	R1O02& Funzionale Obbligatorio &L'utente non autenticato deve poter effettuare il login automatico con l'acount amazon  & UC1.3\\
	\hline
	R1O03& Funzionale Obbligatorio &L'utente autenticato deve poter effettuare il logout & UC2.5\\
	\hline
	R1O04& Funzionale Obbligatorio &L'utente non autenticato deve poter effettuare il login manuale con l'account amazon& UC1.2\\
	\hline
	R1O05& Funzionale Obbligatorio&L'utente autenticato deve poter modificare l'account amazon& UC2.4\\
	\hline
	R1O06&Funzionale Obbligatorio &L'utente autenticato deve poter creare un workflow& UC2.1\\
	\hline
	R1O06.01&Funzionale Obbligatorio& L'utente autenticato deve poter inserire il nome del workflow  & UC2.1.1\\
	\hline
	R1O06.02&Funzionale Obbligatorio&L'utente autenticato deve poter aggiungere un blocco e configurarlo  &UC2.1.2 UC2.1.3\\
	\hline
	R1O07&Funzionale Obbligatorio& L'utente autenticato deve poter eliminare un workflow  &UC2.2\\
	\hline
	R1O08&Funzionale Obbligatorio &L'utente autenticato deve poter modificare un workflow&UC2.3\\
	\hline
	R1O09& Funzionale Obbligatorio&L'utente deve poter inserire un pin per proteggere il workflow & Interno \\
	\hline
	R1O10&Funzionale Obbligatorio & L'utente deve poter gestire una lista& Interno\\
	\hline
	R1O11&Funzionale Obbligatorio  &L'utente autenticato deve poter impostare una sveglia attraverso l'echo &Interno\\
	\hline
	R1O12&Funzionale Obbligatorio  &L'utente deve poter mettere in pausa un workflow attraverso l'echo&Interno \\
	\hline
	R1O013&Funzionale Obbligatorio&Deve essere disponibile un sistema di aiuto per gli utenti che dice in qualsiasi istante in quale punto dell'esecuzione si trova l'utente& Interno\\
	\hline
\end{tabularx}


\section{Requisiti di qualità}
\begin{tabularx}{\textwidth}{|C|C|C|}
	\hline
	\textbf{Codice identificativo} & \textbf{Descrizione} & \textbf{Fonti} \\
	\hline
	\endhead
	R2O01 & Deve essere fornita la documentazione dettagliata di tutte le \glossario{API} & Capitolato\\
	\hline
	R2O02 & Devono essere forniti i diagrammi UML relativi agli Use Cases di progetto  & Capitolato\\
	\hline
	R2O03 & Devono essere forniti i \textit{Voice Dialog Flow} & Capitolato\\
	\hline
	R2O04 & Deve essere fornito lo \textit{Schema Design} relativo alla base dati  & Capitolato\\
	\hline
	R2O05 & Deve essere fornito il \textit{Piano di test di unità} & Capitolato\\
	\hline
	R2O06 & Deve essere consegnato il \textit{Bug Reporting} per il tracciamento di
	tutte le problematiche o anomalie del sistema & Capitolato\\
	\hline
	R2O07 & Tutta la documentazione prodotta in italiano dal team deve avere
	un indice di Gulpease compreso tra 60 e 100 & Interno\\
	\hline
	R2O08 & Tutta la documentazione prodotta in inglese dal team deve avere
	un indice per la formula di Flesh compreso tra 80 e 100 & Interno\\
	\hline
	R2O09 &  & Capitolato\\
	\hline
\end{tabularx}

\section{Requisiti di prestazione}

\section{Requisiti di vincolo}


\section{Riepilogo dei requisiti}