\chapter{Requisiti}
Vengono riportati i requisiti individuati, essi sono stati basandosi sui casi d'uso, sul capitolato, sui diversi incontri avuti con l'azienda \textit{s.r.l. Zero12} e per necessità interne. \\
I requisiti verranno inseriti con la seguenti nomenclatura:
\begin{center}
	\textbf{R$\Bigl\{$A$\Bigr\}$$\Bigl\{$B$\Bigr\}$$\Bigl\{$XX$\Bigr\}$.$\Bigl\{$YY$\Bigr\}$}
\end{center}
dove:
\begin{itemize}
	\item \textbf{A:} corrisponde a uno dei seguenti requisiti:
	\begin{itemize}
		\item 1: funzionale
		\item 2: di qualità
		\item 3: di prestazione
		\item 4: di vincolo
	\end{itemize}
	\item \textbf{B:} corrisponde a uno dei seguenti requisiti:
	\begin{itemize}
		\item O: obbligatorio
		\item F: facoltativo
		\item D: desiderabile
	\end{itemize}
	\item \textbf{{XX}:} numero che identifica i requisiti.
	\item \textbf{{YY}:} numero progressivo che identifica i sottocasi, esso può, a sua volta, includere altri sottocasi.
\end{itemize}

\section{Requisiti funzionali}
\begin{tabularx}{\textwidth}{|C|C|C|C|}
	\hline
     \textbf{Codice identificativo} & \textbf{Tipologia} & \textbf{Descrizione} & \textbf{Fonti} \\
    \hline
	\endhead
	R1O01& Funzionale Obbligatorio &L'applicazione deve permettere all'utente non autenticato di registrarsi con un account amazon& UC1.1	\\
	\hline
	R1O02& Funzionale Obbligatorio &L'applicazione deve permettere all'utente non autenticato di effettuare il login automatico con l'account amazon & UC1.3\\
	\hline
	R1O03& Funzionale Obbligatorio &L'applicazione deve permettere all'utente autenticato di  effettuare il logout & UC2.5\\
	\hline
	R1O04& Funzionale Obbligatorio &L'applicazione deve permettere all'utente non autenticato di effettuare il login manuale con l'account amazon& UC1.2\\
	\hline
	R1O05& Funzionale Obbligatorio&L'applicazione deve permettere all'utente autenticato di modificare l'account amazon& UC2.4\\
	\hline
	R1O06& Funzionale Obbligatorio&L'utente tramite Alexa deve poter inserire un pin per proteggere il workflow & Interno \\
	\hline
	R1O07&Funzionale Obbligatorio & L'utente tramite Alexa deve poter gestire una lista& Interno\\
	\hline
	R1O08&Funzionale Obbligatorio &L'utente tramite Alexa deve poter impostare una sveglia &Interno\\
	\hline
	R1O09&Funzionale Obbligatorio  &L'utente tramite Alexa deve poter mettere in pausa un workflow &Interno \\
	\hline
	R1O10&Funzionale Obbligatorio&Alexa deve fornire un sistema di aiuto che dice all'utente in qualsiasi istante in quale punto dell'esecuzione si trova & Interno\\
	\hline
	R1F10&Funzionale Facoltativo  &L'utente tramite Alexa può impostare un timer &Interno \\
	\hline
	R1F11&Funzionale Facoltativo  &L'utente tramite Alexa può in grado di ascoltare la radio &Interno \\
	\hline	
	R1F12&Funzionale Facoltativo  &L'utente tramite Alexa potrebbe controllare l'illuminazione &Interno \\
	\hline
	R1F12.01&Funzionale Facoltativo  &L'utente ha la possibilità di accendere le luci &Interno \\
	\hline
	R1F12.02&Funzionale Facoltativo  &L'utente ha la possibilità di spegnere le luci &Interno \\
	\hline
	R1D01&Funzionale Desiderabile  &L'utente tramite Alexa può pubblicare un tweet all'interno di Twitter &Interno \\
	\hline
	R1D02&Funzionale Desiderabile  &L'utente tramite Alexa può ascoltare la musica tramite il servizio di Spotify &Interno \\
	\hline
	R1D03&Funzionale Desiderabile  &L'utente tramite Alexa può mettere in pausa il workflow &Interno \\
	\hline
	R1D04&Funzionale Desiderabile  &L'utente tramite Alexa può utilizzare un blocco per la pianificazione dell'apertura di un altro blocco in automatico &Interno \\
	\hline
	R1D05&Funzionale Desiderabile  &L'utente tramite Alexa può ascoltare le notizie riguardanti la borsa e le cryptovalute &Interno \\
	\hline
	\caption{Tabella requisiti funzionali}
\end{tabularx}


\section{Requisiti di qualità}
\begin{tabularx}{\textwidth}{|C|C|C|}
	\hline
	\textbf{Codice identificativo} & \textbf{Descrizione} & \textbf{Fonti} \\
	\hline
	\endhead
	R2O01 & Deve essere fornita la documentazione dettagliata di tutte le \glossario{API} & Capitolato\\
	\hline
	R2O02 & Devono essere forniti i diagrammi UML relativi agli Use Cases di progetto  & Capitolato\\
	\hline
	R2O03 & Devono essere forniti i \textit{Voice Dialog Flow} & Capitolato\\
	\hline
	R2O04 & Deve essere fornito lo \textit{Schema Design} relativo alla base dati  & Capitolato\\
	\hline
	R2O05 & Deve essere fornito il \textit{Piano di test di unità} & Capitolato\\
	\hline
	R2O06 & Deve essere consegnato il \textit{Bug Reporting} per il tracciamento di
	tutte le problematiche o anomalie del sistema & Capitolato\\
	\hline
	R2O07 & Tutta la documentazione prodotta in italiano dal team deve avere
	un indice di Gulpease compreso tra 60 e 100 & Interno\\
	\hline
	R2O08 & Tutta la documentazione prodotta in inglese dal team deve avere
	un indice per la formula di Flesh compreso tra 80 e 100 & Interno\\
	\hline
	R2O09 & Tutta la documentazione rispetta le norme delle \textit{Norme di Progetto v1.0.0} e le metriche del \textit{Piano di Qualifica v1.0.0}& Interno\\
	\hline
	R2O10 & Il codice prodotto rispetta le norme delle \textit{Norme di Progetto v1.0.0} e le metriche del \textit{Piano di Qualifica v1.0.0}& Interno\\
	\hline
    \caption{Tabella requisiti di qualità}
\end{tabularx}

\section{Requisiti di prestazione}

\section{Requisiti di vincolo}
\begin{tabularx}{\textwidth}{|C|C|C|C|}
	\hline
	\textbf{Codice Identificativo}& \textbf{Descrizione} & \textbf{Descrizione} & \textbf{Fonti}\\
	\hline	
	\endhead
	R4O01&Vincolo Obbligatorio  &L'utente tramite Alexa deve poter ascoltare i tweet & Verbale Esterno 2018-12-05 \\
	\hline
	R4O02&Vincolo Obbligatorio  &L'utente tramite Alexa deve poter inviare delle mail & Verbale Esterno 2018-12-05 \\
	\hline
	R4O03&Vincolo Obbligatorio  &L'utente tramite Alexa deve poter leggere le proprie mail & Verbale Esterno 2018-12-05 \\
	\hline
	R4O04&Vincolo Obbligatorio  &Alexa deve essere in grado di riprodurre un blocco di testo personalizzato & Verbale Esterno 2018-12-05 \\
	\hline
	R4O05&Vincolo Obbligatorio  &L'utente tramite Alexa deve poter ascoltare il risultato prodotto dal feed RSS & Verbale Esterno 2018-12-05 \\
	\hline
	R4O06 &Vincolo Obbligatorio& L'utente potrà ascoltare news da un sito di notizie tramite Alexa &Verbale Esterno 2018-12-05\\
	\hline
	R4O07 &Vincolo Obbligatorio& L'utente potrà ascoltare notizie sportive tramite Alexa  &Verbale Esterno 2018-12-05\\
	\hline
	R4O08 &Vincolo Obbligatorio&L'utente potrà ascoltare gli eventi presenti nel proprio calendario personale tramite Alexa  &Verbale Esterno 2018-12-05\\
	\hline
	R4O09 &Vincolo Obbligatorio& L'utente potrà inserire eventi in programma nel proprio calendario personale  tramite Alexa& Verbale Esterno 2018-12-05\\
	\hline
	R4O10 &Vincolo Obbligatorio &L'utente potrà applicare un filtro sul numero di risultati ricevuti in output da Alexa &Verbale Esterno 2018-12-05\\
	\hline
	R4O11 &Vincolo Obbligatorio &L'utente potrà ricevere informazioni sul meteo tramite Alexa& Verbale Esterno 2018-12-05\\
	\hline
	R4O12&Vincolo Obbligatorio&L'utente potrà predisporre il suo workflow personalizzato per la riproduzione di musica tramite Amazon Music tramite Alexa& Verbale Esterno 2018-12-5 \\
	\hline
	R4O13&Vincolo Obbligatorio&L'applicazione deve permettere all'utente autenticato la creazione di un workflow& Capitolato UC2.1\\
	\hline
	R4O13.01&Vincolo Obbligatorio& L'applicazione deve permettere all'utente autenticato di inserire il nome del workflow  & UC2.1.1\\
	\hline
	R1O13.02&Vincolo Obbligatorio&L'applicazione deve permettere all'utente autenticato di aggiungere un blocco & Capitolato UC2.1.2\\
	\hline	
	R1O13.03&Vincolo Obbligatorio&L'applicazione deve permettere all'utente autenticato di configurare un blocco&UC2.1.3\\
	\hline
	R4O14&Vincolo Obbligatorio&L'applicazione deve permettere all'utente autenticato di eliminare un workflow&UC2.2\\
	\hline
	R4O15&Vincolo Obbligatorio& L'applicazione deve permettere all'utente autenticato di  modificare un workflow  &UC2.3\\
	\hline
	R4O15.01&Vincolo Obbligatorio& L'applicazione deve permettere all'utente autenticato di cambiare il nome di un workflow  &UC2.3.1\\
	\hline
	R4O15.02&Vincolo Obbligatorio& L'applicazione deve permettere all'utente autenticato di modificare i blocchi contenuti in un workflow  &UC2.3.4\\
	\hline
	\caption{Tabella requisiti di vincolo}
\end{tabularx}

\section{Riepilogo dei requisiti}
\newpage
\section{Tracciamento requisiti-fonti}
\begin{tabularx}{\textwidth}{|C|C|}
	\hline
	\textbf{Requisito} & \textbf{Fonti} \\
	\hline
	\endhead
	R1O01& UC1.1\\
	\hline
	R1O02& UC1.3\\
	\hline
	R1O03& UC2.5\\
	\hline
	R1O04& UC1.2\\
	\hline
	R1O05& UC2.4\\
	\hline
	R1O06& UC2.1\\
	\hline
	R1O06.01& UC2.1.1\\
	\hline
	R1O06.02& UC2.1.2\\
	\hline
	R1O06.03& UC2.1.3\\
	\hline
	R1O07& UC2.2\\
	\hline	
	R1O08& UC2.3\\
	\hline	
	R1O09& Interno\\
	\hline	
	R1O10& Interno\\
	\hline
	R1O11& Interno\\
	\hline	
	R1O12& Interno\\
	\hline
	R1O13& Interno\\
	\hline
	R2O01 &Capitolato\\
	\hline
	R2O02 &Capitolato\\
	\hline
	R2O03 &Capitolato\\
	\hline
	R2O04 &Capitolato\\
	\hline
	R2O05 &Capitolato\\
	\hline
	R2O06 &Capitolato\\
	\hline
	R2O07 &Interno\\
	\hline
	R2O08 &Interno\\
	\hline
	R4O06 &Interno\\
	\hline
	\caption{Tabella requisiti-fonti}
\end{tabularx}