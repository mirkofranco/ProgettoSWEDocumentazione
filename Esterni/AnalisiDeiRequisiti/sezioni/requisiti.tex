\chapter{Requisiti} \label{Requisiti}
Vengono riportati i\glossario{requisiti}individuati basandosi sui casi d'uso, sul capitolato$_{G}$, sui diversi incontri avuti con l'azienda \textit{Zero12 s.r.l.} e per necessità interne. \\


\subsection{Requisiti Funzionali}
\normalsize
\begin{longtable}{|c|>{\centering}m{7cm}|c|}
	\hline 
	\textbf{Id Requisito} & \textbf{Descrizione} & \textbf{Stato}\\
	\hline
	\endhead
	\hypertarget{RFO1}{RFO1} & L'applicazione deve permettere all'utente non autenticato di registrarsi con un account Amazon & \textit{Non Soddisfatto}\\ \hline
	
	\hypertarget{RFO2}{RFO2} & L'applicazione deve permettere all'utente non autenticato di effettuare il login automatico con l'account Amazon & \textit{Non Soddisfatto}\\ \hline
	
	\hypertarget{RFO3}{RFO3} & L'applicazione deve permettere all'utente non autenticato di effettuare il login manuale con l'account Amazon & \textit{Non Soddisfatto}\\ \hline
	
	\hypertarget{RFO4}{RFO4} & L'applicazione deve permettere all'utente autenticato di effettuare il logout & \textit{Non Soddisfatto}\\ \hline
	
	\hypertarget{RFO5}{RFO5} & L'applicazione deve permettere all'utente autenticato di modificare l'account Amazon & \textit{Non Soddisfatto}\\ \hline
	
	\hypertarget{RFO6}{RFO6} & L'utente tramite Alexa deve poter impostare una sveglia & \textit{Non Soddisfatto}\\ \hline
	
	\hypertarget{RFO6.1}{RFO6.1} & L'utente tramite Alexa deve poter creare una nuova sveglia & \textit{Non Soddisfatto}\\ \hline
	
	\hypertarget{RFO6.2}{RFO6.2} & L'utente tramite Alexa deve poter modificare l'orario di una sveglia & \textit{Non Soddisfatto}\\ \hline
	
	\hypertarget{RFO6.3}{RFO6.3} & L'utente tramite Alexa deve poter posporre una sveglia mentre questa sta suonando & \textit{Non Soddisfatto}\\ \hline
	
	\hypertarget{RFO6.4}{RFO6.4} & L'utente tramite Alexa deve poter interrompere una sveglia mentre questa sta suonando & \textit{Non Soddisfatto}\\ \hline
	
	\hypertarget{RFO7}{RFO7} & L'utente tramite Alexa deve poter inserire un Pin per proteggere il workflow & \textit{Non Soddisfatto}\\ \hline
	
	\hypertarget{RFO8}{RFO8} & L'utente tramite Alexa deve poter gestire una lista & \textit{Non Soddisfatto}\\ \hline
	
	\hypertarget{RFO9}{RFO9} & L'utente tramite Alexa deve poter ascoltare i tweet & \textit{Non Soddisfatto}\\ \hline
	
	\hypertarget{RFO10}{RFO10} & L'utente tramite Alexa deve poter inviare delle mail & \textit{Non Soddisfatto}\\ \hline
	
	\hypertarget{RFO11}{RFO11} & L'utente tramite Alexa deve poter leggere le proprie mail & \textit{Non Soddisfatto}\\ \hline
	
	\hypertarget{RFO12}{RFO12} & Alexa deve essere in grado di riprodurre un blocco di testo personalizzato & \textit{Non Soddisfatto}\\ \hline
	
	\hypertarget{RFO13}{RFO13} & L'utente potrà ascoltare gli eventi presenti nel proprio calendario personale tramite Alexa & \textit{Non Soddisfatto}\\ \hline
	
	\hypertarget{RFO14}{RFO14} & L'utente potrà inserire eventi in programma nel proprio calendario personale tramite Alexa & \textit{Non Soddisfatto}\\ \hline
	
	\hypertarget{RFO15}{RFO15} & L'applicazione deve permettere all'utente autenticato la creazione di un workflow & \textit{Non Soddisfatto}\\ \hline
	
	\hypertarget{RFO15.1}{RFO15.1} & L'applicazione deve permettere all'utente autenticato di inserire il nome del workflow & \textit{Non Soddisfatto}\\ \hline
	
	\hypertarget{RFO15.2}{RFO15.2} & L'applicazione deve permettere all'utente autenticato di aggiungere un blocco & \textit{Non Soddisfatto}\\ \hline
	
	\hypertarget{RFO16}{RFO16} & L'applicazione deve permettere all'utente autenticato di eliminare un workflow & \textit{Non Soddisfatto}\\ \hline
	
	\hypertarget{RFO17}{RFO17} & L'applicazione deve permettere all'utente autenticato di modificare un workflow & \textit{Non Soddisfatto}\\ \hline
	
	\hypertarget{RFO17.1}{RFO17.1} & L'applicazione deve permettere all'utente autenticato di cambiare il nome di un workflow & \textit{Non Soddisfatto}\\ \hline
	
	\hypertarget{RFO17.2}{RFO17.2} & L'applicazione deve permettere all'utente autenticato di modificare i blocchi contenuti in un workflow & \textit{Non Soddisfatto}\\ \hline
	
	\hypertarget{RFO18}{RFO18} & L'utente tramite Alexa deve poter ascoltare il risultato prodotto dal feed RSS & \textit{Non Soddisfatto}\\ \hline
	
	\hypertarget{RFO19}{RFO19} & L'utente potrà ascoltare news da un sito di notizie tramite Alexa & \textit{Non Soddisfatto}\\ \hline
	
	\hypertarget{RFO20}{RFO20} & L'utente potrà ascoltare notizie sportive tramite Alexa & \textit{Non Soddisfatto}\\ \hline
	
	\hypertarget{RFO21}{RFO21} & L'utente potrà applicare un filtro sul numero di risultati ricevuti in output da Alexa & \textit{Non Soddisfatto}\\ \hline
	
	\hypertarget{RFO22}{RFO22} & L'utente potrà ricevere informazioni sul meteo tramite Alexa & \textit{Non Soddisfatto}\\ \hline
	
	\hypertarget{RFO23}{RFO23} & L'utente potrà predisporre il suo workflow personalizzato per la riproduzione di musica con Amazon Music tramite Alexa & \textit{Non Soddisfatto}\\ \hline
	
	\hypertarget{RFO24}{RFO24} & L'utente tramite Alexa deve poter mettere in pausa un workflow & \textit{Non Soddisfatto}\\ \hline
	
	\hypertarget{RFO25}{RFO25} & Alexa deve fornire un sistema di aiuto che dice all'utente in qualsiasi istante in quale punto dell'esecuzione si trova & \textit{Non Soddisfatto}\\ \hline
	
	\hypertarget{RFF26}{RFF26} & I dati utente verranno automaticamente sincronizzati con il proprio account Amazon & \textit{Non Soddisfatto}\\ \hline
	
	\hypertarget{RFF27}{RFF27} & L'utente deve poter rispondere ad un tweet dopo che Alexa l'ha letto & \textit{Non Soddisfatto}\\ \hline
	
	\hypertarget{RFF28}{RFF28} & L'utente deve poter rispondere ad una mail dopo che Alexa l'ha letta & \textit{Non Soddisfatto}\\ \hline
	
	\hypertarget{RFF29}{RFF29} & L'utente deve poter inviare messaggi di testo su Telegram tramite Alexa a una persona o ad un gruppo & \textit{Non Soddisfatto}\\ \hline
	
	\hypertarget{RFF30}{RFF30} &  L'utente deve poter leggere tramite Alexa i nuovi messaggi testuali ricevuti da una persona o un gruppo su Telegram & \textit{Non Soddisfatto}\\ \hline
	
	\hypertarget{RFF31}{RFF31} & L'utente deve poter inviare messaggi audio su Telegram tramite Alexa a una persona o un gruppo & \textit{Non Soddisfatto}\\ \hline
	
	\hypertarget{RFF32}{RFF32} & L'utente deve poter modificare un evento nel calendario tramite Alexa & \textit{Non Soddisfatto}\\ \hline
	
	\hypertarget{RFF33}{RFF33} & L'utente potrà predisporre il suo workflow personalizzato per la riproduzione di musica tramite Youtube & \textit{Non Soddisfatto}\\ \hline
	
	\hypertarget{RFF34}{RFF34} & L'utente tramite Alexa può impostare un timer & \textit{Non Soddisfatto}\\ \hline
	
	\hypertarget{RFF35}{RFF35} & L'utente tramite Alexa può in grado di ascoltare la radio & \textit{Non Soddisfatto}\\ \hline
	
	\hypertarget{RFF36}{RFF36} & L'utente tramite Alexa potrebbe controllare l'illuminazione & \textit{Non Soddisfatto}\\ \hline
	
	\hypertarget{RFF36.1}{RFF36.1} & L'utente tramite Alexa può accendere le luci & \textit{Non Soddisfatto}\\ \hline
	
	\hypertarget{RFF36.2}{RFF36.2} & L'utente tramite Alexa può spegnere le luci & \textit{Non Soddisfatto}\\ \hline
	
	\hypertarget{RFD37}{RFD37} & L'utente tramite Alexa può pubblicare un tweet all'interno di Twitter & \textit{Non Soddisfatto}\\ \hline
	
	\hypertarget{RFD38}{RFD38} & L'utente tramite Alexa può ascoltare la musica tramite il servizio di Spotify & \textit{Non Soddisfatto}\\ \hline
	
	\hypertarget{RFD39}{RFD39} & L'utente tramite Alexa può utilizzare un blocco per la pianificazione dell'apertura di un altro blocco in automatico & \textit{Non Soddisfatto}\\ \hline
	
	\hypertarget{RFD40}{RFD40} & L'utente tramite Alexa può ascoltare le notizie riguardanti la borsa e le cryptovalute & \textit{Non Soddisfatto}\\ \hline
	
	\caption[Requisiti Funzionali]{Requisiti Funzionali}
	\label{tabella:req0}
\end{longtable}
\clearpage
\subsection{Requisiti Prestazionali}
\normalsize
\begin{longtable}{|c|>{\centering}m{7cm}|c|}
	\hline 
	\textbf{Id Requisito} & \textbf{Descrizione} & \textbf{Stato}\\
	\hline
	\endhead
	\hypertarget{RPO1}{RPO1} & L'applicazione Android deve essere compatibile dalla versione 4.4 & \textit{Non Soddisfatto}\\ \hline
	
	\caption[Requisiti Prestazionali]{Requisiti Prestazionali}
	\label{tabella:req1}
\end{longtable}
\clearpage
\subsection{Requisiti di Qualità}
\normalsize
\begin{longtable}{|c|>{\centering}m{7cm}|c|}
	\hline 
	\textbf{Id Requisito} & \textbf{Descrizione} & \textbf{Stato}\\
	\hline
	\endhead
	\hypertarget{RQO1}{RQO1} & Deve essere fornita la documentazione dettagliata di tutte le API & \textit{Non Soddisfatto}\\ \hline
	
	\hypertarget{RQO2}{RQO2} & Devono essere forniti i diagrammi UML relativi agli Use Cases di progetto & \textit{Non Soddisfatto}\\ \hline
	
	\hypertarget{RQO3}{RQO3} & Devono essere forniti i Voice Dialog Flow & \textit{Non Soddisfatto}\\ \hline
	
	\hypertarget{RQO4}{RQO4} & Deve essere fornito lo Schema Design relativo alla base dati & \textit{Non Soddisfatto}\\ \hline
	
	\hypertarget{RQO5}{RQO5} & Deve essere fornito il \textit{Piano di test di unità} & \textit{Non Soddisfatto}\\ \hline
	
	\hypertarget{RQO6}{RQO6} & Deve essere consegnato il Bug Reporting per il tracciamento di tutte le problematiche o anomalie del sistema & \textit{Non Soddisfatto}\\ \hline
	
	\hypertarget{RQO7}{RQO7} & Tutta la documentazione prodotta in italiano dal team deve avere un indice di Gulpease compreso tra 60 e 100 & \textit{Non Soddisfatto}\\ \hline
	
	\hypertarget{RQO8}{RQO8} & Tutta la documentazione rispetta le \textit{Norme di Progetto v2.0.0} e le metriche del \textit{Piano di Qualifica v2.0.0} & \textit{Non Soddisfatto}\\ \hline
	
	\hypertarget{RQO9}{RQO9} & Il codice prodotto rispetta le \textit{Norme di Progetto v2.0.0} e le metriche del \textit{Piano di Qualifica v2.0.0} & \textit{Non Soddisfatto}\\ \hline
	
	\caption[Requisiti di Qualità]{Requisiti di Qualità}
	\label{tabella:req2}
\end{longtable}
\clearpage
\subsection{Requisiti di Vincolo}
\normalsize
\begin{longtable}{|c|>{\centering}m{7cm}|c|}
	\hline 
	\textbf{Id Requisito} & \textbf{Descrizione} & \textbf{Stato}\\
	\hline
	\endhead
	\hypertarget{RVO1}{RVO1} & L'applicazione sarà utilizzabile da versioni Android & \textit{Non Soddisfatto}\\ \hline
	
	\hypertarget{RVO2}{RVO2} & L'applicazione deve essere sviluppata utilizzando il linguaggio di programmazione Kotlin & \textit{Non Soddisfatto}\\ \hline
	
	\hypertarget{RVO3}{RVO3} & Per lo sviluppo in back-end si utilizza NodeJS versione 10.15.1 & \textit{Non Soddisfatto}\\ \hline
	
	\hypertarget{RVO4}{RVO4} & Per i test viene utilizzato il Framework Jasmine versione 3.3.0 & \textit{Non Soddisfatto}\\ \hline
	
	\hypertarget{RVO5}{RVO5} & Viene utilizzato il framework Express versione 4.x per NodeJS & \textit{Non Soddisfatto}\\ \hline
	
	\hypertarget{RVO6}{RVO6} & Viene utilizzato DynamoDB come database NoSQL offerto dai servizi Amazon & \textit{Non Soddisfatto}\\ \hline
	
	\hypertarget{RVO7}{RVO7} & Si utilizza AWS Lambda per l'esecuzione del codice nel server & \textit{Non Soddisfatto}\\ \hline
	
	\hypertarget{RVO8}{RVO8} & Viene utilizzato NPM per la gestione della build del codice prodotto & \textit{Non Soddisfatto}\\ \hline
	
	\caption[Requisiti di Vincolo]{Requisiti di Vincolo}
	\label{tabella:req3}
\end{longtable}
\clearpage

\section{Tracciamento requisiti-fonti}
\begin{tabularx}{\textwidth}{|C|C|}
	\hline
	\textbf{Requisito} & \textbf{Fonte} \\
	\hline
	\endhead
	R1O01& UC1.1\\
	\hline
	R1O02& UC1.3\\
	\hline
	R1O03& UC2.5\\
	\hline
	R1O04& UC1.2\\
	\hline
	R1O05& UC2.4\\
	\hline
	R1O06& Interno\\
	\hline
	R1O07& Interno\\
	\hline	
	R1O08& Interno\\
	\hline
	R1O09& Interno\\
	\hline	
	R1O10& Interno\\
	\hline
	R1F01& Interno\\
	\hline
	R1F02& Interno\\
	\hline
	R1F03& Interno\\
	\hline
	R1F04& Interno\\
	\hline
	R1F05& Interno\\
	\hline
	R1F06& Interno\\
	\hline
	R1F07& Interno\\
	\hline
	R1F08& Interno\\
	\hline
	R1F09& Interno\\
	\hline
	R1F10& Interno\\
	\hline
	R1F11& Interno\\
	\hline
	R1F12& Interno\\
	\hline
	R1D01& Interno\\
	\hline
	R1D02& Interno\\
	\hline
	R1D03& Interno\\
	\hline
	R1D04& Interno\\
	\hline
	R2O01&\glossario{Capitolato}\\
	\hline
	R2O02&\glossario{Capitolato}\\
	\hline
	R2O03&\glossario{Capitolato}\\
	\hline
	R2O04&\glossario{Capitolato}\\
	\hline
	R2O05&\glossario{Capitolato}\\
	\hline
	R2O06&\glossario{Capitolato}\\
	\hline
	R2O07&Interno\\
	\hline
	R2O08&Interno\\
	\hline
	R2O09&Interno\\
	\hline
	R2O10&Interno\\
	\hline
	R3O01&Interno\\
	\hline
	R4O01&Verbale Esterno 2018-12-05\\
	\hline
	R4O02&Verbale Esterno 2018-12-05\\
	\hline
	R4O03&Verbale Esterno 2018-12-05\\
	\hline
	R4O04&Verbale Esterno 2018-12-05\\
	\hline
	R4O05&Verbale Esterno 2018-12-05\\
	\hline
	R4O06&Verbale Esterno 2018-12-05\\
	\hline
	R4O07&Verbale Esterno 2018-12-05\\
	\hline
	R4O08&Verbale Esterno 2018-12-05\\
	\hline
	R4O09&Verbale Esterno 2018-12-05\\
	\hline
	R4O10&Verbale Esterno 2018-12-05\\
	\hline
	R4O11&Verbale Esterno 2018-12-05\\
	\hline
	R4O12&Verbale Esterno 2018-12-05\\
	\hline
	R4O13&\glossario{Capitolato}\\& UC2.1\\
	\hline
	R4O13.01&\glossario{Capitolato}\\&UC2.1.1\\
	\hline
	R4O13.02&\glossario{Capitolato}\\& UC2.1.2\\
	\hline
	R4O13.03&\glossario{Capitolato}\\&UC2.1.3\\
	\hline
	R4O14&Verbale Esterno 2018-12-27\\&UC2.2\\
	\hline
	R4O15&Verbale Esterno 2018-12-27\\&UC2.3\\
	\hline
	R4O15.01&Verbale Esterno 2018-12-27\\&UC2.3.1\\
	\hline
	R4O15.02&Verbale Esterno 2018-12-27\\&UC2.3.4\\
	\hline
	\caption{Tabella requisiti-fonti}
\end{tabularx}


\newpage
\section{Tracciamento fonti-requisiti}
\begin{tabularx}{\textwidth}{|C|C|C|}
	\hline
	\textbf{Fonte} &\textbf{Descrizione} & \textbf{Requisito} \\
	\hline
	\endhead
	Capitolato& &R2O01\\&&R2O02\\&&R2O03\\&&R2O04\\&&R2O05\\&&R2O06\\&&R4O13\\&&R4O13.01\\&&R4O13.02\\&&R4O13.03\\
	\hline
	Verbale Esterno 2018-12-05&&R4O01\\&&R4O02\\&&R4O03\\&&R4O04\\&&R4O05\\&&R4O06\\&&R4O07\\&&R4O08\\&&R4O09\\&&R4O10\\&&R4O11\\&&R4O12\\
	\hline
	Verbale Esterno 2018-12-27&&R4O14\\&&R4O15\\&&R4O15.01\\&&R4O15.02\\
	\hline
	UC1.1  &Registrazione &R1O01\\
	\hline
	UC1.3  &Login automatico&R1O02\\
	\hline
	UC2.5   &Logout&R1O03\\
	\hline
	UC1.2 &Login&R1O04\\
	\hline
	UC2.4 &Modifica account Amazon&R1O05\\
	\hline
	UC2.1 &Creazione workflow&R4O13\\
	\hline
	UC2.1.1 &Scelta nome&R4O13.01\\
	\hline
	UC2.1.2 &Scelta blocchi&R4O13.02\\
	\hline
	UC2.1.3 &Configurazione blocchi&R4O13.03\\
	\hline
	UC2.2 &Eliminazione workflow&R4O14\\
	\hline
	UC2.3 &Modifica workflow&R4O15\\
	\hline	
	UC2.3.1 &Modifica nome workflow&R4O15-01\\
	\hline
	UC2.3.4 &Modifica blocco&R4O15-02\\
	\hline
	\caption{Tabella fonti-requisiti}
\end{tabularx}

\subsection{Riepilogo Requisiti}
\normalsize
\begin{longtable}{|c|c|c|c|}
	\hline 
	\textbf{Tipo} & \textbf{Obbligatorio} & \textbf{Desiderabile} & \textbf{Facoltativo}\\
	\hline
	Funzionale & 33 & 4 & 13\\ \hline
	Prestazionale & 1 & 0 & 0\\ \hline
	Di Qualità & 9 & 0 & 0\\ \hline
	Di Vincolo & 8 & 0 & 0\\ \hline
	\caption[Riepilogo Requisiti]{Riepilogo Requisiti}
	\label{tabella:riepilogorequi}
\end{longtable}
\clearpage
