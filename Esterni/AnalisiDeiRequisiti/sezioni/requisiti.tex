\chapter{Requisiti} \label{Requisiti}
Vengono riportati i\glossario{requisiti}individuati basandosi sui casi d'uso, sul \textit{capitolato$_{G}$}, sui diversi incontri avuti con l'azienda \textit{Zero12 s.r.l.} e per necessità interne. \\

Ogni\glossario{requisito}deve avere la seguente nomenclatura:
\begin{center}
	\textbf{R$\Bigl\{$A$\Bigr\}$$\Bigl\{$B$\Bigr\}$$\Bigl\{$XX$\Bigr\}$.$\Bigl\{$YY$\Bigr\}$}
\end{center}
dove:
\begin{itemize}
	\item \textbf{A:} corrisponde a uno dei seguenti requisiti:
	\begin{itemize}
		\item F: funzionale;
		\item Q: di qualità;
		\item P: di prestazione;
		\item V: di vincolo.
	\end{itemize}
	\item \textbf{B:} corrisponde a uno dei seguenti requisiti:
	\begin{itemize}
		\item O: obbligatorio;
		\item F: facoltativo;
		\item D: desiderabile.
	\end{itemize}
	\item \textbf{{XX}:} numero che identifica i\glossario{requisiti};
	\item \textbf{{YY}:} numero progressivo che identifica i sottocasi, esso può, a sua volta, includere altri sottocasi.
\end{itemize}
\pagebreak
\section{Requisiti Funzionali}
\normalsize
\begin{longtable}{|c|>{\centering}m{7cm}|c|}
	\hline 
	\textbf{Id Requisito} & \textbf{Descrizione} & \textbf{Stato}\\
	\hline
	\endhead
	\hypertarget{RFO1}{RFO1} & L'applicazione deve permettere all'utente non autenticato di registrarsi con un account Amazon & \textit{Non Soddisfatto}\\ \hline
	
	\hypertarget{RFO2}{RFO2} & L'applicazione deve permettere all'utente non autenticato di effettuare il login automatico con l'account Amazon & \textit{Non Soddisfatto}\\ \hline
	
	\hypertarget{RFO3}{RFO3} & L'applicazione deve permettere all'utente non autenticato di effettuare il login manuale con l'account Amazon & \textit{Non Soddisfatto}\\ \hline
	
	\hypertarget{RFO4}{RFO4} & L'applicazione deve permettere all'utente autenticato di effettuare il logout & \textit{Non Soddisfatto}\\ \hline
	
	\hypertarget{RFO5}{RFO5} & L'applicazione deve permettere all'utente autenticato di modificare l'account Amazon & \textit{Non Soddisfatto}\\ \hline
	
	\hypertarget{RFO6}{RFO6} & L'utente tramite Alexa deve poter impostare una sveglia & \textit{Non Soddisfatto}\\ \hline
	
	\hypertarget{RFO6.1}{RFO6.1} & L'utente tramite Alexa deve poter creare una nuova sveglia & \textit{Non Soddisfatto}\\ \hline
	
	\hypertarget{RFO6.2}{RFO6.2} & L'utente tramite Alexa deve poter modificare l'orario di una sveglia & \textit{Non Soddisfatto}\\ \hline
	
	\hypertarget{RFO6.3}{RFO6.3} & L'utente tramite Alexa deve poter posporre una sveglia mentre questa sta suonando & \textit{Non Soddisfatto}\\ \hline
	
	\hypertarget{RFO6.4}{RFO6.4} & L'utente tramite Alexa deve poter interrompere una sveglia mentre questa sta suonando & \textit{Non Soddisfatto}\\ \hline
	
	\hypertarget{RFO7}{RFO7} & L'utente tramite Alexa deve poter inserire un Pin per proteggere il workflow & \textit{Non Soddisfatto}\\ \hline
	
	\hypertarget{RFO8}{RFO8} & L'utente tramite Alexa deve poter gestire una lista & \textit{Non Soddisfatto}\\ \hline
	
	\hypertarget{RFO9}{RFO9} & L'utente tramite Alexa deve poter ascoltare i tweet & \textit{Non Soddisfatto}\\ \hline
	
	\hypertarget{RFO10}{RFO10} & L'utente tramite Alexa deve poter inviare delle mail & \textit{Non Soddisfatto}\\ \hline
	
	\hypertarget{RFO11}{RFO11} & L'utente tramite Alexa deve poter leggere le proprie mail & \textit{Non Soddisfatto}\\ \hline
	
	\hypertarget{RFO12}{RFO12} & Alexa deve essere in grado di riprodurre un blocco di testo personalizzato & \textit{Non Soddisfatto}\\ \hline
	
	\hypertarget{RFO13}{RFO13} & L'utente potrà ascoltare gli eventi presenti nel proprio calendario personale tramite Alexa & \textit{Non Soddisfatto}\\ \hline
	
	\hypertarget{RFO14}{RFO14} & L'utente potrà inserire eventi in programma nel proprio calendario personale tramite Alexa & \textit{Non Soddisfatto}\\ \hline
	
	\hypertarget{RFO15}{RFO15} & L'applicazione deve permettere all'utente autenticato la creazione di un workflow & \textit{Non Soddisfatto}\\ \hline
	
	\hypertarget{RFO15.1}{RFO15.1} & L'applicazione deve permettere all'utente autenticato di inserire il nome del workflow & \textit{Non Soddisfatto}\\ \hline
	
	\hypertarget{RFO15.2}{RFO15.2} & L'applicazione deve permettere all'utente autenticato di aggiungere un blocco & \textit{Non Soddisfatto}\\ \hline
	
	\hypertarget{RFO16}{RFO16} & L'applicazione deve permettere all'utente autenticato di eliminare un workflow & \textit{Non Soddisfatto}\\ \hline
	
	\hypertarget{RFO17}{RFO17} & L'applicazione deve permettere all'utente autenticato di modificare un workflow & \textit{Non Soddisfatto}\\ \hline
	
	\hypertarget{RFO17.1}{RFO17.1} & L'applicazione deve permettere all'utente autenticato di cambiare il nome di un workflow & \textit{Non Soddisfatto}\\ \hline
	
	\hypertarget{RFO17.2}{RFO17.2} & L'applicazione deve permettere all'utente autenticato di modificare i blocchi contenuti in un workflow & \textit{Non Soddisfatto}\\ \hline
	
	\hypertarget{RFO18}{RFO18} & L'utente tramite Alexa deve poter ascoltare il risultato prodotto dal feed RSS & \textit{Non Soddisfatto}\\ \hline
	
	\hypertarget{RFO19}{RFO19} & L'utente potrà ascoltare news da un sito di notizie tramite Alexa & \textit{Non Soddisfatto}\\ \hline
	
	\hypertarget{RFO20}{RFO20} & L'utente potrà ascoltare notizie sportive tramite Alexa & \textit{Non Soddisfatto}\\ \hline
	
	\hypertarget{RFO21}{RFO21} & L'utente potrà applicare un filtro sul numero di risultati ricevuti in output da Alexa & \textit{Non Soddisfatto}\\ \hline
	
	\hypertarget{RFO22}{RFO22} & L'utente potrà ricevere informazioni sul meteo tramite Alexa & \textit{Non Soddisfatto}\\ \hline
	
	\hypertarget{RFO23}{RFO23} & L'utente potrà predisporre il suo workflow personalizzato per la riproduzione di musica con Amazon Music tramite Alexa & \textit{Non Soddisfatto}\\ \hline
	
	\hypertarget{RFO24}{RFO24} & L'utente tramite Alexa deve poter mettere in pausa un workflow & \textit{Non Soddisfatto}\\ \hline
	
	\hypertarget{RFO25}{RFO25} & Alexa deve fornire un sistema di aiuto che dice all'utente in qualsiasi istante in quale punto dell'esecuzione si trova & \textit{Non Soddisfatto}\\ \hline
	
	\hypertarget{RFF26}{RFF26} & I dati utente verranno automaticamente sincronizzati con il proprio account Amazon & \textit{Non Soddisfatto}\\ \hline
	
	\hypertarget{RFF27}{RFF27} & L'utente deve poter rispondere ad un tweet dopo che Alexa l'ha letto & \textit{Non Soddisfatto}\\ \hline
	
	\hypertarget{RFF28}{RFF28} & L'utente deve poter rispondere ad una mail dopo che Alexa l'ha letta & \textit{Non Soddisfatto}\\ \hline
	
	\hypertarget{RFF29}{RFF29} & L'utente deve poter inviare messaggi di testo su Telegram tramite Alexa a una persona o ad un gruppo & \textit{Non Soddisfatto}\\ \hline
	
	\hypertarget{RFF30}{RFF30} &  L'utente deve poter leggere tramite Alexa i nuovi messaggi testuali ricevuti da una persona o un gruppo su Telegram & \textit{Non Soddisfatto}\\ \hline
	
	\hypertarget{RFF31}{RFF31} & L'utente deve poter inviare messaggi audio su Telegram tramite Alexa a una persona o un gruppo & \textit{Non Soddisfatto}\\ \hline
	
	\hypertarget{RFF32}{RFF32} & L'utente deve poter modificare un evento nel calendario tramite Alexa & \textit{Non Soddisfatto}\\ \hline
	
	\hypertarget{RFF33}{RFF33} & L'utente potrà predisporre il suo workflow personalizzato per la riproduzione di musica tramite Youtube & \textit{Non Soddisfatto}\\ \hline
	
	\hypertarget{RFF34}{RFF34} & L'utente tramite Alexa può impostare un timer & \textit{Non Soddisfatto}\\ \hline
	
	\hypertarget{RFF35}{RFF35} & L'utente tramite Alexa può ascoltare la radio & \textit{Non Soddisfatto}\\ \hline
	
	\hypertarget{RFF36}{RFF36} & L'utente tramite Alexa può controllare l'illuminazione, se compatibile con il dispositivo & \textit{Non Soddisfatto}\\ \hline
	
	\hypertarget{RFF36.1}{RFF36.1} & L'utente tramite Alexa può accendere le luci & \textit{Non Soddisfatto}\\ \hline
	
	\hypertarget{RFF36.2}{RFF36.2} & L'utente tramite Alexa può spegnere le luci & \textit{Non Soddisfatto}\\ \hline
	
	\hypertarget{RFD37}{RFD37} & L'utente tramite Alexa può pubblicare un tweet all'interno di Twitter & \textit{Non Soddisfatto}\\ \hline
	
	\hypertarget{RFD38}{RFD38} & L'utente tramite Alexa può ascoltare la musica tramite il servizio di Spotify & \textit{Non Soddisfatto}\\ \hline
	
	\hypertarget{RFD39}{RFD39} & L'utente tramite Alexa può utilizzare un blocco per la pianificazione dell'apertura di un altro blocco in automatico & \textit{Non Soddisfatto}\\ \hline
	
	\hypertarget{RFD40}{RFD40} & L'utente tramite Alexa può ascoltare le notizie riguardanti la borsa e le criptovalute & \textit{Non Soddisfatto}\\ \hline
	
	\caption[Requisiti Funzionali]{Requisiti Funzionali}
	\label{tabella:req0}
\end{longtable}
\clearpage
\section{Requisiti Prestazionali}
\normalsize
\begin{longtable}{|c|>{\centering}m{7cm}|c|}
	\hline 
	\textbf{Id Requisito} & \textbf{Descrizione} & \textbf{Stato}\\
	\hline
	\endhead
	\hypertarget{RPO1}{RPO1} & L'applicazione Android deve essere compatibile dalla versione 4.4 & \textit{Non Soddisfatto}\\ \hline
	
	\caption[Requisiti Prestazionali]{Requisiti Prestazionali}
	\label{tabella:req1}
\end{longtable}
\clearpage
\section{Requisiti di Qualità}
\normalsize
\begin{longtable}{|c|>{\centering}m{7cm}|c|}
	\hline 
	\textbf{Id Requisito} & \textbf{Descrizione} & \textbf{Stato}\\
	\hline
	\endhead
	\hypertarget{RQO1}{RQO1} & Deve essere fornita la documentazione dettagliata di tutte le API & \textit{Non Soddisfatto}\\ \hline
	
	\hypertarget{RQO2}{RQO2} & Devono essere forniti i diagrammi UML relativi agli Use Cases di progetto & \textit{Non Soddisfatto}\\ \hline
	
	\hypertarget{RQO3}{RQO3} & Devono essere forniti i Voice Dialog Flow & \textit{Non Soddisfatto}\\ \hline
	
	\hypertarget{RQO4}{RQO4} & Deve essere fornito lo Schema Design relativo alla base dati & \textit{Non Soddisfatto}\\ \hline
	
	\hypertarget{RQO5}{RQO5} & Deve essere fornito il \textit{Piano di test di unità} & \textit{Non Soddisfatto}\\ \hline
	
	\hypertarget{RQO6}{RQO6} & Deve essere consegnato il Bug Reporting per il tracciamento di tutte le problematiche o anomalie del sistema & \textit{Non Soddisfatto}\\ \hline
	
	\hypertarget{RQO7}{RQO7} & Tutta la documentazione prodotta in italiano dal team deve avere un indice di Gulpease compreso tra 60 e 100 & \textit{Non Soddisfatto}\\ \hline
	
	\hypertarget{RQO8}{RQO8} & Tutta la documentazione rispetta le \textit{Norme di Progetto v2.0.0} e le metriche del \textit{Piano di Qualifica v2.0.0} & \textit{Non Soddisfatto}\\ \hline
	
	\hypertarget{RQO9}{RQO9} & Il codice prodotto rispetta le \textit{Norme di Progetto v2.0.0} e le metriche del \textit{Piano di Qualifica v2.0.0} & \textit{Non Soddisfatto}\\ \hline
	
	\caption[Requisiti di Qualità]{Requisiti di Qualità}
	\label{tabella:req2}
\end{longtable}
\clearpage
\section{Requisiti di Vincolo}
\normalsize
\begin{longtable}{|c|>{\centering}m{7cm}|c|}
	\hline 
	\textbf{Id Requisito} & \textbf{Descrizione} & \textbf{Stato}\\
	\hline
	\endhead
	\hypertarget{RVO1}{RVO1} & L'applicazione sarà utilizzabile da dispositivi Android & \textit{Non Soddisfatto}\\ \hline
	
	\hypertarget{RVO2}{RVO2} & L'applicazione deve essere sviluppata utilizzando il linguaggio di programmazione Kotlin & \textit{Non Soddisfatto}\\ \hline
	
	\hypertarget{RVO3}{RVO3} & Per lo sviluppo in back-end si utilizza NodeJS versione 10.15.1 & \textit{Non Soddisfatto}\\ \hline
	
	\hypertarget{RVO4}{RVO4} & Per i test viene utilizzato il framework Mocha insieme alla libreria Chai & \textit{Non Soddisfatto}\\ \hline
	
	\hypertarget{RVO5}{RVO5} & Viene utilizzato il framework Express versione 4.16.4 per NodeJS & \textit{Non Soddisfatto}\\ \hline
	
	\hypertarget{RVO6}{RVO6} & Viene utilizzato DynamoDB come database NoSQL offerto dai servizi Amazon Web Services & \textit{Non Soddisfatto}\\ \hline
	
	\hypertarget{RVO7}{RVO7} & Si utilizza AWS Lambda per l'esecuzione del codice nel server & \textit{Non Soddisfatto}\\ \hline
	
	\hypertarget{RVO8}{RVO8} & Viene utilizzato NPM versione 6.4.1 per la gestione della build del codice prodotto & \textit{Non Soddisfatto}\\ \hline
	
	\caption[Requisiti di Vincolo]{Requisiti di Vincolo}
	\label{tabella:req3}
\end{longtable}
\clearpage

\section{Tracciamento Requisiti-Fonti}
\normalsize
\begin{longtable}{|>{\centering}m{5cm}|m{5cm}<{\centering}|}
	\hline
	\textbf{Id Requisito} & \textbf{Fonti}\\
	\hline
	\endhead
	\hyperlink{RFO1}{RFO1} & \hyperref[UC1.1]{UC1.1}\\ \hline
	
	\hyperlink{RFO2}{RFO2} & \hyperref[UC1.3]{UC1.3}\\ \hline
	
	\hyperlink{RFO3}{RFO3} & \hyperref[UC1.2]{UC1.2}\\ \hline
	
	\hyperlink{RFO4}{RFO4} & \hyperref[UC2.5]{UC2.5}\\ \hline
	
	\hyperlink{RFO5}{RFO5} & \hyperref[UC2.4]{UC2.4}\\ \hline
	
	\hyperlink{RFO6}{RFO6} & \hyperref[UC2.1.2.6]{UC2.1.2.6}\\ \hline
	
	\hyperlink{RFO6.1}{RFO6.1} & \hyperref[UC2.1.2.6]{UC2.1.2.6}\\ \hline
	
	\hyperlink{RFO6.2}{RFO6.2} & \hyperref[UC2.1.2.6]{UC2.1.2.6}\\ \hline
	
	\hyperlink{RFO6.3}{RFO6.3} & \hyperref[UC2.1.2.6]{UC2.1.2.6}\\ \hline
	
	\hyperlink{RFO6.4}{RFO6.4} & \hyperref[UC2.1.2.6]{UC2.1.2.6}\\ \hline
	
	\hyperlink{RFO7}{RFO7} & \hyperref[UC2.1.2.1]{UC2.1.2.1}\\ \hline
	
	\hyperlink{RFO8}{RFO8} & \hyperref[UC2.1.2.5]{UC2.1.2.5}\\ \hline
	
	\hyperlink{RFO9}{RFO9} & \hyperref[UC2.1.2.4]{UC2.1.2.4} \\& Verbale Esterno 2018-12-05\\ \hline
	
	\hyperlink{RFO10}{RFO10} & \hyperref[UC2.1.2.7]{UC2.1.2.7} \\& Verbale Esterno 2018-12-05\\ \hline
	
	\hyperlink{RFO11}{RFO11} & \hyperref[UC2.1.2.7]{UC2.1.2.7} \\& Verbale Esterno 2018-12-05\\ \hline
	
	\hyperlink{RFO12}{RFO12} & \hyperref[UC2.1.2.3]{UC2.1.2.3} \\& Verbale Esterno 2018-12-05\\ \hline
	
	\hyperlink{RFO13}{RFO13} & \hyperref[UC2.1.2.9]{UC2.1.2.9} \\& Verbale Esterno 2018-12-05\\ \hline
	
	\hyperlink{RFO14}{RFO14} & \hyperref[UC2.1.2.9]{UC2.1.2.9} \\& Verbale Esterno 2018-12-05\\ \hline
	
	\hyperlink{RFO15}{RFO15} & \hyperref[UC2.1]{UC2.1} \\& Verbale Esterno 2018-12-05\\ \hline
	\pagebreak
	\hyperlink{RFO15.1}{RFO15.1} & \hyperref[UC2.1.1]{UC2.1.1} \\& Verbale Esterno 2018-12-05\\ \hline
	
	\hyperlink{RFO15.2}{RFO15.2} & \hyperref[UC2.1.2]{UC2.1.2} \\& Verbale Esterno 2018-12-05\\ \hline
	
	\hyperlink{RFO16}{RFO16} & \hyperref[UC2.2]{UC2.2} \\& Verbale Esterno 2018-12-05\\ \hline
	
	\hyperlink{RFO17}{RFO17} & \hyperref[UC2.3]{UC2.3} \\& Verbale Esterno 2018-12-05\\ \hline
	
	\hyperlink{RFO17.1}{RFO17.1} & \hyperref[UC2.3.1]{UC2.3.1} \\& Verbale Esterno 2018-12-05\\ \hline
	
	\hyperlink{RFO17.2}{RFO17.2} & \hyperref[UC2.3.4]{UC2.3.4} \\& Verbale Esterno 2018-12-05\\ \hline
	
	\hyperlink{RFO18}{RFO18} & \hyperref[UC2.1.2.16]{UC2.1.2.16} \\& Verbale Esterno 2018-12-05\\ \hline
	
	\hyperlink{RFO19}{RFO19} & \hyperref[UC2.1.2.16]{UC2.1.2.16} \\& Verbale Esterno 2018-12-05\\ \hline
	
	\hyperlink{RFO20}{RFO20} & \hyperref[UC2.1.2.16]{UC2.1.2.16} \\& Verbale Esterno 2018-12-05\\ \hline
	
	\hyperlink{RFO21}{RFO21} & \hyperref[UC2.1.2.17]{UC2.1.2.17} \\& Verbale Esterno 2018-12-05\\ \hline
	
	\hyperlink{RFO22}{RFO22} & \hyperref[UC2.1.2.18]{UC2.1.2.18} \\& Verbale Esterno 2018-12-05\\ \hline
	
	\hyperlink{RFO23}{RFO23} & \hyperref[UC2.1.2.19]{UC2.1.2.19} \\& Verbale Esterno 2018-12-05\\ \hline
	
	\hyperlink{RFO24}{RFO24} & \hyperref[Interno]{Interno}\\ \hline
	
	\hyperlink{RFO25}{RFO25} & \hyperref[Interno]{Interno}\\ \hline
	
	\hyperlink{RFF26}{RFF26} & \hyperref[Interno]{Interno}\\ \hline
	
	\hyperlink{RFF27}{RFF27} & \hyperref[UC2.1.2.4]{UC2.1.2.4}\\ \hline
	
	\hyperlink{RFF28}{RFF28} & \hyperref[UC2.1.2.7]{UC2.1.2.7}\\ \hline
	
	\hyperlink{RFF29}{RFF29} & \hyperref[UC2.1.2.8]{UC2.1.2.8}\\ \hline
	
	\hyperlink{RFF30}{RFF30} & \hyperref[UC2.1.2.8]{UC2.1.2.8}\\ \hline
	
	\hyperlink{RFF31}{RFF31} & \hyperref[UC2.1.2.8]{UC2.1.2.8}\\ \hline
	
	\hyperlink{RFF32}{RFF32} & \hyperref[UC2.1.2.9]{UC2.1.2.9}\\ \hline
	
	\hyperlink{RFF33}{RFF33} & \hyperref[UC2.1.2.10]{UC2.1.2.10}\\ \hline
	
	\hyperlink{RFF34}{RFF34} & \hyperref[UC2.1.2.11]{UC2.1.2.11}\\ \hline
	
	\hyperlink{RFF35}{RFF35} & \hyperref[UC2.1.2.12]{UC2.1.2.12}\\ \hline
	
	\hyperlink{RFF36}{RFF36} & \hyperref[UC2.1.2.13]{UC2.1.2.13}\\ \hline
	
	\hyperlink{RFF36.1}{RFF36.1} & \hyperref[UC2.1.2.13]{UC2.1.2.13}\\ \hline
	
	\hyperlink{RFF36.2}{RFF36.2} & \hyperref[UC2.1.2.13]{UC2.1.2.13}\\ \hline
	
	\hyperlink{RFD37}{RFD37} & \hyperref[UC2.1.2.4]{UC2.1.2.4}\\ \hline
	
	\hyperlink{RFD38}{RFD38} & \hyperref[UC2.1.2.14]{UC2.1.2.14}\\ \hline
	
	\hyperlink{RFD39}{RFD39} & \hyperref[UC2.1.2.2]{UC2.1.2.2}\\ \hline
	
	\hyperlink{RFD40}{RFD40} & \hyperref[UC2.1.2.15]{UC2.1.2.15}\\ \hline
	
	\hyperlink{RPO1}{RPO1} & \hyperref[Verbale Esterno 2018-12-05]{Verbale Esterno 2018-12-05}\\ \hline
	
	\hyperlink{RQO1}{RQO1} & \hyperref[Capitolato]{Capitolato}\\ \hline
	
	\hyperlink{RQO2}{RQO2} & \hyperref[Capitolato]{Capitolato}\\ \hline
	
	\hyperlink{RQO3}{RQO3} & \hyperref[Capitolato]{Capitolato}\\ \hline
	
	\hyperlink{RQO4}{RQO4} & \hyperref[Capitolato]{Capitolato}\\ \hline
	
	\hyperlink{RQO5}{RQO5} & \hyperref[Capitolato]{Capitolato}\\ \hline
	
	\hyperlink{RQO6}{RQO6} & \hyperref[Capitolato]{Capitolato}\\ \hline
	
	\hyperlink{RQO7}{RQO7} & \hyperref[Interno]{Interno}\\ \hline
	
	\hyperlink{RQO8}{RQO8} & \hyperref[Interno]{Interno}\\ \hline
	
	\hyperlink{RQO9}{RQO9} & \hyperref[Interno]{Interno}\\ \hline
	
	\hyperlink{RVO1}{RVO1} & \hyperref[Verbale Esterno 2018-12-05]{Verbale Esterno 2018-12-05}\\ \hline
	
	\hyperlink{RVO2}{RVO2} & \hyperref[Capitolato]{Capitolato}\\ \hline
	
	\hyperlink{RVO3}{RVO3} & \hyperref[Capitolato]{Capitolato}\\ \hline
	
	\hyperlink{RVO4}{RVO4} & \hyperref[Interno]{Verbale Esterno 2019-02-15}\\ \hline
	
	\hyperlink{RVO5}{RVO5} & \hyperref[Interno]{Interno}\\ \hline
	
	\hyperlink{RVO6}{RVO6} & \hyperref[Capitolato]{Capitolato}\\ \hline
	
	\hyperlink{RVO7}{RVO7} & \hyperref[Capitolato]{Capitolato}\\ \hline
	
	\hyperlink{RVO8}{RVO8} & \hyperref[Interno]{Interno}\\ \hline
	
	\caption[Tracciamento Requisiti-Fonti]{Tracciamento Requisiti-Fonti}
	\label{tabella:requi-fonti}
\end{longtable}
\clearpage

\section{Tracciamento Fonti-Requisiti}
\normalsize
\begin{longtable}{|>{\centering}m{5cm}|m{5cm}<{\centering}|}
	\hline
	\textbf{Fonte} & \textbf{Id Requisiti}\\
	\hline
	\endhead
	\hyperlink{Capitolato}{Capitolato} & \hyperlink{RQO1}{RQO1}\\
	& \hyperlink{RQO2}{RQO2}\\
	& \hyperlink{RQO3}{RQO3}\\
	& \hyperlink{RQO4}{RQO4}\\
	& \hyperlink{RQO5}{RQO5}\\
	& \hyperlink{RQO6}{RQO6}\\
	& \hyperlink{RVO2}{RVO2}\\
	& \hyperlink{RVO3}{RVO3}\\
	& \hyperlink{RVO6}{RVO6}\\
	& \hyperlink{RVO7}{RVO7}\\ \hline
	\hyperlink{Interno}{Interno} & \hyperlink{RFO24}{RFO24}\\
	& \hyperlink{RFO25}{RFO25}\\
	& \hyperlink{RFF26}{RFF26}\\
	& \hyperlink{RQO7}{RQO7}\\
	& \hyperlink{RQO8}{RQO8}\\
	& \hyperlink{RQO9}{RQO9}\\
	& \hyperlink{RVO5}{RVO5}\\
	& \hyperlink{RVO8}{RVO8}\\ \hline
	\hyperlink{UC1.1}{UC1.1} & \hyperlink{RFO1}{RFO1}\\ \hline
	\hyperlink{UC1.2}{UC1.2} & \hyperlink{RFO3}{RFO3}\\ \hline
	\hyperlink{UC1.3}{UC1.3} & \hyperlink{RFO2}{RFO2}\\ \hline
	\hyperlink{UC2.1}{UC2.1} & \hyperlink{RFO15}{RFO15}\\ \hline
	\hyperlink{UC2.1.1}{UC2.1.1} & \hyperlink{RFO15.1}{RFO15.1}\\ \hline
	\hyperlink{UC2.1.2}{UC2.1.2} & \hyperlink{RFO15.2}{RFO15.2}\\ \hline
	\hyperlink{UC2.1.2.1}{UC2.1.2.1} & \hyperlink{RFO7}{RFO7}\\ \hline
	\hyperlink{UC2.1.2.10}{UC2.1.2.10} & \hyperlink{RFF33}{RFF33}\\ \hline
	\hyperlink{UC2.1.2.11}{UC2.1.2.11} & \hyperlink{RFF34}{RFF34}\\ \hline
	\hyperlink{UC2.1.2.12}{UC2.1.2.12} & \hyperlink{RFF35}{RFF35}\\ \hline
	\hyperlink{UC2.1.2.13}{UC2.1.2.13} & \hyperlink{RFF36}{RFF36}\\
	& \hyperlink{RFF36.1}{RFF36.1}\\
	& \hyperlink{RFF36.2}{RFF36.2}\\ \hline
	\hyperlink{UC2.1.2.14}{UC2.1.2.14} & \hyperlink{RFD38}{RFD38}\\ \hline
	\hyperlink{UC2.1.2.15}{UC2.1.2.15} & \hyperlink{RFD40}{RFD40}\\ \hline
	\hyperlink{UC2.1.2.16}{UC2.1.2.16} & \hyperlink{RFO18}{RFO18}\\
	& \hyperlink{RFO19}{RFO19}\\
	& \hyperlink{RFO20}{RFO20}\\ \hline
	\hyperlink{UC2.1.2.17}{UC2.1.2.17} & \hyperlink{RFO21}{RFO21}\\ \hline
	\hyperlink{UC2.1.2.18}{UC2.1.2.18} & \hyperlink{RFO22}{RFO22}\\ \hline
	\hyperlink{UC2.1.2.19}{UC2.1.2.19} & \hyperlink{RFO23}{RFO23}\\ \hline
	\hyperlink{UC2.1.2.2}{UC2.1.2.2} & \hyperlink{RFD39}{RFD39}\\ \hline
	\hyperlink{UC2.1.2.3}{UC2.1.2.3} & \hyperlink{RFO12}{RFO12}\\ \hline
	\hyperlink{UC2.1.2.4}{UC2.1.2.4} & \hyperlink{RFO9}{RFO9}\\
	& \hyperlink{RFF27}{RFF27}\\
	& \hyperlink{RFD37}{RFD37}\\ \hline
	\hyperlink{UC2.1.2.5}{UC2.1.2.5} & \hyperlink{RFO8}{RFO8}\\ \hline
	\hyperlink{UC2.1.2.6}{UC2.1.2.6} & \hyperlink{RFO6}{RFO6}\\
	& \hyperlink{RFO6.1}{RFO6.1}\\
	& \hyperlink{RFO6.2}{RFO6.2}\\
	& \hyperlink{RFO6.3}{RFO6.3}\\
	& \hyperlink{RFO6.4}{RFO6.4}\\ \hline
	\hyperlink{UC2.1.2.7}{UC2.1.2.7} & \hyperlink{RFO10}{RFO10}\\
	& \hyperlink{RFO11}{RFO11}\\
	& \hyperlink{RFF28}{RFF28}\\ \hline
	\hyperlink{UC2.1.2.8}{UC2.1.2.8} & \hyperlink{RFF29}{RFF29}\\
	& \hyperlink{RFF30}{RFF30}\\
	& \hyperlink{RFF31}{RFF31}\\ \hline
	\hyperlink{UC2.1.2.9}{UC2.1.2.9} & \hyperlink{RFO13}{RFO13}\\
	& \hyperlink{RFO14}{RFO14}\\
	& \hyperlink{RFF32}{RFF32}\\ \hline

	\hyperref[UC1.1]{UC1.1} & \hyperlink{RFO1}{RFO1}\\ \hline
	\hyperref[UC1.2]{UC1.2} & \hyperlink{RFO3}{RFO3}\\ \hline
	\hyperref[UC1.3]{UC1.3} & \hyperlink{RFO2}{RFO2}\\ \hline
	\hyperref[UC2]{UC2} & \hyperlink{RFO15}{RFO15}\\ \hline
	\hyperref[UC2.1.1]{UC2.1.1} & \hyperlink{RFO15.1}{RFO15.1}\\ \hline
	\hyperref[UC2.1.2]{UC2.1.2} & \hyperlink{RFO15.2}{RFO15.2}\\ \hline
	\hyperref[UC2.1.2.1]{UC2.1.2.1} & \hyperlink{RFO7}{RFO7}\\ \hline
	\hyperref[UC2.1.2.2]{UC2.1.2.2} & \hyperlink{RFD39}{RFD39}\\ \hline
	\hyperref[UC2.1.2.3]{UC2.1.2.3} & \hyperlink{RFO12}{RFO12}\\ \hline
	\hyperref[UC2.1.2.4]{UC2.1.2.4} & \hyperlink{RFO9}{RFO9}\\
	& \hyperlink{RFF27}{RFF27}\\
	& \hyperlink{RFD37}{RFD37}\\ \hline
	\hyperref[UC2.1.2.5]{UC2.1.2.5} & \hyperlink{RFO8}{RFO8}\\ \hline
	\hyperref[UC2.1.2.6]{UC2.1.2.6} & \hyperlink{RFO6}{RFO6}\\
	& \hyperlink{RFO6.1}{RFO6.1}\\
	& \hyperlink{RFO6.2}{RFO6.2}\\
	& \hyperlink{RFO6.3}{RFO6.3}\\
	& \hyperlink{RFO6.4}{RFO6.4}\\ \hline
	\hyperref[UC2.1.2.7]{UC2.1.2.7} & \hyperlink{RFO10}{RFO10}\\
	& \hyperlink{RFO11}{RFO11}\\
	& \hyperlink{RFF28}{RFF28}\\ \hline
	\hyperref[UC2.1.2.8]{UC2.1.2.8} & \hyperlink{RFF30}{RFF30}\\
	& \hyperlink{RFF31}{RFF31}\\ \hline
	\hyperref[UC2.1.2.9]{UC2.1.2.9} & \hyperlink{RFO13}{RFO13}\\
	& \hyperlink{RFO14}{RFO14}\\
	& \hyperlink{RFF32}{RFF32}\\ \hline
	\hyperref[UC2.1.2.10]{UC2.1.2.10} & \hyperlink{RFF33}{RFF33}\\ \hline
	\hyperref[UC2.1.2.11]{UC2.1.2.11} & \hyperlink{RFF34}{RFF34}\\ \hline
	\hyperref[UC2.1.2.12]{UC2.1.2.12} & \hyperlink{RFF35}{RFF35}\\ \hline
	\hyperref[UC2.1.2.13]{UC2.1.2.13} & \hyperlink{RFF36}{RFF36}\\
	& \hyperlink{RFF36.2}{RFF36.2}\\ \hline
	\hyperref[UC2.1.2.14]{UC2.1.2.14} & \hyperlink{RFD38}{RFD38}\\ \hline
	\hyperref[UC2.1.2.15]{UC2.1.2.15} & \hyperlink{RFD40}{RFD40}\\ \hline
	\hyperref[UC2.1.2.16]{UC2.1.2.16} & \hyperlink{RFO18}{RFO18}\\
	& \hyperlink{RFO19}{RFO19}\\
	& \hyperlink{RFO20}{RFO20}\\ \hline
	\hyperref[UC2.1.2.17]{UC2.1.2.17} & \hyperlink{RFO21}{RFO21}\\ \hline
	\hyperref[UC2.1.2.18]{UC2.1.2.18} & \hyperlink{RFO22}{RFO22}\\ \hline
	\hyperref[UC2.1.2.19]{UC2.1.2.19} & \hyperlink{RFO23}{RFO23}\\ \hline
	\hyperlink{UC2.2}{UC2.2} & \hyperlink{RFO16}{RFO16}\\ \hline
	\hyperlink{UC2.3}{UC2.3} & \hyperlink{RFO17}{RFO17}\\ \hline
	\hyperlink{UC2.3.1}{UC2.3.1} & \hyperlink{RFO17.1}{RFO17.1}\\ \hline
	\hyperlink{UC2.3.4}{UC2.3.4} & \hyperlink{RFO17.2}{RFO17.2}\\ \hline
	\hyperlink{UC2.4}{UC2.4} & \hyperlink{RFO5}{RFO5}\\ \hline
	\hyperlink{UC2.5}{UC2.5} & \hyperlink{RFO4}{RFO4}\\ \hline
	\hyperlink{Verbale Esterno 2018-12-05}{Verbale Esterno 2018-12-05} &
	 \hyperlink{RPO1}{RPO1}\\ & \hyperlink{RVO1}{RVO1}\\ & \hyperlink{RFO9}{RFO9}\\ & \hyperlink{RFO10}{RFO10}\\ & \hyperlink{RFO11}{RFO11}\\ & \hyperlink{RFO12}{RFO12}\\ & \hyperlink{RFO13}{RFO13}\\ & \hyperlink{RFO14}{RFO14}\\ & \hyperlink{RFO15}{RFO15}\\ & \hyperlink{RFO15.1}{RFO15.1}\\ & \hyperlink{RFO15.2}{RFO15.2}\\ & \hyperlink{RFO16}{RFO16}\\ & \hyperlink{RFO17}{RFO17}\\ & \hyperlink{RFO17.1}{RFO17.1}\\ & 
	\hyperlink{RFO17.2}{RFO17.2}\\ & \hyperlink{RFO18}{RFO18}\\ & \hyperlink{RFO19}{RFO19}\\ & \hyperlink{RFO20}{RFO20}\\ & \hyperlink{RFO21}{RFO21}\\ & \hyperlink{RFO22}{RFO22}\\ & \hyperlink{RFO23}{RFO23}\\ \hline
	\hyperlink{Verbale Esterno 2019-02-15}{Verbale Esterno 2019-02-15} &
	\hyperlink{RVO4}{RVO4}\\ \hline
	\hyperref[UC2.2]{UC2.2} & \hyperlink{RFO16}{RFO16}\\ \hline
	\hyperref[UC2.3]{UC2.3} & \hyperlink{RFO17}{RFO17}\\ \hline
	\hyperref[UC2.3.1]{UC2.3.1} & \hyperlink{RFO17.1}{RFO17.1}\\ \hline
	\hyperref[UC2.3.4]{UC2.3.4} & \hyperlink{RFO17.2}{RFO17.2}\\ \hline
	\hyperref[UC2.4]{UC2.4} & \hyperlink{RFO5}{RFO5}\\ \hline
	\hyperref[UC2.5]{UC2.5} & \hyperlink{RFO4}{RFO4}\\ \hline
	\caption[Tracciamento Fonti-Requisiti]{Tracciamento Fonti-Requisiti}
	\label{tabella:fonti-requi}
\end{longtable}
\clearpage

\subsection{Riepilogo Requisiti}
\normalsize
\begin{longtable}{|c|c|c|c|}
	\hline 
	\textbf{Tipo} & \textbf{Obbligatorio} & \textbf{Desiderabile} & \textbf{Facoltativo}\\
	\hline
	Funzionale & 33 & 4 & 13\\ \hline
	Prestazionale & 1 & 0 & 0\\ \hline
	Di Qualità & 9 & 0 & 0\\ \hline
	Di Vincolo & 8 & 0 & 0\\ \hline
	\caption[Riepilogo Requisiti]{Riepilogo Requisiti}
	\label{tabella:riepilogorequi}
\end{longtable}
\clearpage
