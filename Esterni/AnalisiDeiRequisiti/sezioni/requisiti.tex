\chapter{Requisiti}
Vengono riportati i requisiti individuati, essi sono stati basandosi sui casi d'uso, sul capitolato, sui diversi incontri avuti con l'azienda \textit{s.r.l. Zero12} e per necessità interne. \\
I requisiti verranno inseriti con la seguenti nomenclatura:
\begin{center}
	\textbf{R$\Bigl\{$A$\Bigr\}$$\Bigl\{$B$\Bigr\}$$\Bigl\{$XX$\Bigr\}$.$\Bigl\{$YY$\Bigr\}$}
\end{center}
dove:
\begin{itemize}
	\item \textbf{A:} corrisponde a uno dei seguenti requisiti:
	\begin{itemize}
		\item 1: funzionale
		\item 2: di qualità
		\item 3: di prestazione
		\item 4: di vincolo
	\end{itemize}
	\item \textbf{B:} corrisponde a uno dei seguenti requisiti:
	\begin{itemize}
		\item O: obbligatorio
		\item F: facoltativo
		\item D: desiderabile
	\end{itemize}
	\item \textbf{{XX}:} numero che identifica i requisiti.
	\item \textbf{{YY}:} numero progressivo che identifica i sottocasi, esso può, a sua volta, includere altri sottocasi.
\end{itemize}

\section{Requisiti funzionali}

\section{Requisiti di qualità}

\section{Requisiti di vincolo}

\section{Riepilogo dei requisiti}