\newglossaryentry{Account}
{
	name={Account linking},
	description={Collegamento che consente di connettere l’utente con un account utente in
		un sistema diverso}
}

\newglossaryentry{Activity}
{
	name={Activity},
	description={Classe scritta in java che si occupa di creare una finestra in cui è possibile inserire un'interfaccia utente}
}

\newglossaryentry{Adapter}
{
	name={Adapter},
	description={Pattern strutturale basato su classi ed oggetti che risolve il problema dell'interoperabilità tra interfacce differenti}
}

\newglossaryentry{Agile}
{
	name={Agile},
	description={Questo tipo di metodo prevede un approccio poco strutturato e con lo scopo di consegnare al cliente un prodotto funzionante e di qualità in brevi tempi}
}

\newglossaryentry{Alexa}
{
	name={Alexa},
	description={Assistente personale che anima gli Amazon Echo}
}

\newglossaryentry{Amazon}
{
	name={Amazon},
	description={Azienda di commercio elettronico con sede negli USA}
}

\newglossaryentry{Android}
{
	name={Android},
	description={Sistema operativo per dispositivi mobili sviluppato da Google Inc}
}

\newglossaryentry{API}
{
	name={API},
	description={(Application Programming Interface) sono degli strumenti di programmazione che le software house mettono a disposizione degli sviluppatori per aiutarli
		nella creazione di applicazioni}
}

\newglossaryentry{API Gateway}
{
	name={API Gateway},
	description={Servizio di AWS per la creazione, manutenzione e protezione di API}
}

\newglossaryentry{Asana}
{
	name={Asana},
	description={Asana è un applicativo web pensato per aiutare i team di sviluppo ad organizzare e monitorare le attività da svolgere in un progetto}
}

\newglossaryentry{Assertion library}
{
	name={Assertion library},
	description={Libreria di strumenti usati per testare il codice riducendo così la quantità di If statements}
}

\newglossaryentry{Astah}
{
	name={Astah UML},
	description={Software gratuito per la creazione dei diagrammi UML}
}

\newglossaryentry{Aurora}
{
	name={Aurora Serverless},
	description={Configurazione di dimensionamento automatico on demand per Amazon Aurora}
}

\newglossaryentry{AWS}
{
	name={Amazon Web Services},
	description={Insieme di servizi di cloud computing offerti da Amazon. Comprende servizi di calcolo, di rete e di storage di basi di dati}
}
\newglossaryentry{AWSCloudWatch}
{
	name={AWS CloudWatch},
	description={CloudWatch raccoglie dati di monitoraggio e operativi sotto forma di log, parametri ed eventi, fornendo una visualizzazione unificata delle risorse AWS, sulle applicazioni e i servizi eseguiti in AWS e su server locali}
}
\newglossaryentry{AWSLambda}
{
	name={AWS~Lambda},
	description={Servizio serverless offerto da AWS che consente di eseguire codice in risposta ad eventi}
	{\newpage}
}

\newglossaryentry{Backup}
{
	name={Backup},
	description={Atto di ottenere una o più copie dei propri file e salvarli in un supporto esterno al computer}
}

\newglossaryentry{Baseline}
{
	name={Baseline},
	description={Rappresenta un punto di riferimento dal quale calcolare l’avanzamento del lavoro in un progetto}
}

\newglossaryentry{Bug}
{
	name={Bug},
	description={Errore di scrittura nel codice sorgente del software}
	{\newpage}
}

\newglossaryentry{Capitolato}
{
	name={Capitolato},
	description={Documento che descrive in linea generale le caratteristiche tecniche del prodotto richiesto}
}

\newglossaryentry{Chai}
{
	name={Chai },
	description={É una libreria di asserzioni per Node.js che può essere accoppiata con i framework di test di javascript}
}

\newglossaryentry{Conference Call}
{
	name={Conference Call},
	description={Modalità di comunicazione a distanza tramite telefono}
}

\newglossaryentry{Connettore}
{
	name={Connettore},
	description={Descrive un modulo la cui funzione è favorire la comunicazione e collaborazione fra altri moduli}
	{\newpage}
}

\newglossaryentry{Design Pattern}
{
	name={Design Pattern},
	description={Soluzione progettuale generale per la risoluzione di un problema ricorrente}
}

\newglossaryentry{Diagrammi di Gantt}
{
	name={Diagrammi di Gantt},
	description={Diagramma che permette la rappresentazione grafica di un calendario di attività}
}

\newglossaryentry{DynamoDB}
{
	name={DynamoDB},
	description={Database non sequenziale messo a disposizione da AWS che può contenere dati di tipo documento e di tipo chiave-valore}
	{\newpage}
}

\newglossaryentry{Express}
{
	name={Express},
	description={Framework open source per applicazioni web per Node.js}
	{\newpage}
}

\newglossaryentry{Fragment}
{
	name={Fragment},
	description={Componente utilizzato per la creazione di un'interfaccia utente Android che fa parte di Activity}
}

\newglossaryentry{Framework}
{
	name={Framework},
	description={Architettura logica di supporto su cui è possibile realizzare un software, semplificando il lavoro al programmatore}
	{\newpage}
}

\newglossaryentry{ganttProject}
{
	name={GanttProject},
	description={Software open-source per la creazione dei diagrammi di Gantt}
}

\newglossaryentry{GitHub}
{
	name={GitHub},
	description={GitHub è un servizio di hosting per lo sviluppo di progetti software che usa il sistema di controllo di versione Git}
}

\newglossaryentry{Google Hangouts}
{
	name={Google Hangouts},
	description={Software di messaggistica istantanea e VoIP sviluppato da Google}
}

\newglossaryentry{gradle}
{
	name={Gradle},
	description={Sistema open-source usato per l'automazione dello sviluppo}
	{\newpage}
}

\newglossaryentry{harvest}
{
	name={Harvest},
	description={Servizio web, integrabile ad Asana, utilizzato dal gruppo per la contabilità delle ore di lavoro al progetto}
	{\newpage}
}

\newglossaryentry{Indice di Gulpease}
{
	name={Indice di Gulpease},
	description={L’indice di Gulpease è un indice di leggibilità di un testo tarato sulla lingua
		italiana. Rispetto ad altri ha il vantaggio di utilizzare la lunghezza delle parole in lettere anziché in sillabe, semplificandone il calcolo automatico}
}

\newglossaryentry{Inspection}
{
	name={Inspection},
	description={Atto di verificare eseguendo una lettura mirata}
}

\newglossaryentry{Instagantt}
{
	name={Instagantt},
	description={Applicativo web che crea diagrammi di Gantt per le attività di Asana}
}

\newglossaryentry{iOS}
{
	name={iOS},
	description={Sistema operativo per smartphone sviluppato da Apple}
}

\newglossaryentry{Istanbul}
{
	name={Istanbul},
	description={É una libreria per eseguire la copertura del codice in JavaScript. Produrrà informazioni sulla copertura del codice sulla riga di comando e genererà un report HTML completo}
	{\newpage}
}

\newglossaryentry{Jasmine}
{
	name={Jasmine},
	description={Framework open source di testing per Javascript}
}

\newglossaryentry{JavaDoc}
{
	name={JavaDoc},
	description={Applicativo contenuto nel JDK che genera automaticamente la documentazione di codice sorgente scritto in java}
}

\newglossaryentry{Javascript}
{
	name={Javascript},
	description={Javascript è un linguaggio di scripting pensato per la programmazione web lato client}
	{\newpage}
}

\newglossaryentry{Kotlin}
{
	name={Kotlin},
	description={Linguaggio di programmazione per sistemi Android}
	{\newpage}
}

\newglossaryentry{MegAlexa}
{
	name={MegAlexa},
	description={Nome del capitolato scelto}
}

\newglossaryentry{Milestone}
{
	name={Milestone},
	description={Una milestone è una data di calendario associata ad uno specifico insieme di baseline}
}

\newglossaryentry{Mocha}
{
	name={Mocha},
	description={Mocha è un framework di test JavaScript per i programmi Node.js}
}

\newglossaryentry{Mock}
{
	name={Mock},
	description={Oggetti simulati che nei test unitari automatici riproducono il comportamento degli oggetti reali}
	{\newpage}
}

\newglossaryentry{NodeJS}
{
	name={Node.js},
	description={Piattaforma ad eventi per il motore Javascript V8. L’esecuzione di javascript avviene lato server}
}

\newglossaryentry{npm}
{
	name={npm},
	description={Gestore di pacchetti usato dall'ambiente Node.Js. Consiste di un client a riga di comando e di un database di pacchetti}
	{\newpage}
}

\newglossaryentry{pragmaDB}
{
	name={PragmaDB},
	description={Database utilizzato dal team per inserire i dati e generare i documenti}
}

\newglossaryentry{Processo}
{
	name={Processo},
	description={Insieme di attività correlate e coese che rispondono ad un bisogno e che, consumando risorse, trasformano necessità in prodotti}
}

\newglossaryentry{Prodotto}
{
	name={Prodotto},
	description={Qualsiasi bene o servizio in grado di soddisfare un bisogno o un’esigenza}
}

\newglossaryentry{Progetto}
{
	name={Progetto},
	description={Insieme di attività e compiti che devono raggiungere un obiettivo secondo delle specifiche e dei tempi prefissati, utilizzando determinate risorse}
}

\newglossaryentry{Proof Of Concept}
{
	name={Proof Of Concept},
	description={Prototipo che realizza alcune delle funzionalità principali del prodotto finale}
}

\newglossaryentry{Proponente}
{
	name={Proponente},
	description={Colui che presenta una proposta, in questo caso il capitolato riguardante il progetto}
	{\newpage}
}

\newglossaryentry{Requisito}
{
	name={Requisito},
	description={Funzionalità che il prodotto software deve avere}
}

\newglossaryentry{REST API}
{
	name={REST API},
	description={Design pattern basato su HTTP e che utilizza le API}
}

\newglossaryentry{Routine}
{
	name={Routine},
	description={Insieme di attività che si ripetono nel tempo}
	{\newpage}
}

\newglossaryentry{Scrum}
{
	name={Scrum},
	description={Modello di sviluppo Agile per la gestione di un progetto che prevede che l’avanzamento avvenga tramite una serie di sprint. Ad ogni riunione viene pianificato uno sprint}
}

\newglossaryentry{Skill}
{
	name={Skill},
	description={Per skill si intende un applicativo sviluppato su Amazon Alexa in grado di eseguire un compito specifico}
}

\newglossaryentry{Skype}
{
	name={Skype},
	description={Software che permette di effettuare chiamate e videochiamate utilizzando la rete internet}
}

\newglossaryentry{SonarQube}
{
	name={SonarQube},
	description={Strumento che effettua l’analisi statica del codice di vari linguaggi di programmazione}
}

\newglossaryentry{SPICE}
{
	name={SPICE},
	description={Corrisponde allo standard ISO/IEC 15504 e serve per la valutazione di un processo software}
}

\newglossaryentry{Sprint}
{
	name={Sprint},
	description={Indica un’unità di base dello sviluppo in SCRUM ed ha una durata fissata. Ogni sprint è preceduto da una riunione di pianificazione in cui vengono stabiliti gli obbiettivi e stimati i tempi}
}

\newglossaryentry{Stakeholders}
{
	name={Stakeholders},
	description={(portatori di interesse) Persone che agiscono in qualità di fornitori, di committenti o di clienti}
}

\newglossaryentry{SVG}
{
	name={SVG},
	description={(Scalable Vector Graphics) Tipo di formato delle immagini}
}

\newglossaryentry{Swift}
{
	name={Swift},
	description={Linguaggio di programmazione per sistemi iOS e MacOS}
	{\newpage}
}

\newglossaryentry{Telegram}
{
	name={Telegram},
	description={Servizio di messaggistica istantanea basato su cloud che permette lo scambio di file multimediali}
}

\newglossaryentry{test Driven Development}
{
	name={Test Driven Development},
	description={Il Test Driven Development (TDD) è una pratica di sviluppo software molto diffusa nelle metodologie agili.
		Consiste nella scrittura dei test prima dell'implementazione delle funzionalità seguito da un refactoring del codice scritto}
}

\newglossaryentry{TeXstudio}
{
	name={TeXstudio},
	description={Editor di testo per il linguaggio di markup \LaTeX}
}

\newglossaryentry{Ticket}
{
	name={Ticket},
	description={Notifica per segnalare ai membri del gruppo la presenza di un’attività da svolgere per l’avanzamento del progetto}
}

\newglossaryentry{travis CI}
{
	name={Travis CI},
	description={Applicazione web usata per la continous integration e per i test sui progetti software}
}

\newglossaryentry{TypeScript}
{
	name={TypeScript},
	description={Linguaggio di programmazione sviluppato da Microsoft, la cui compilazione genera codice JavaScript standard}
	{\newpage}
}

\newglossaryentry{UML}
{
	name={UML},
	description={(Unified Modeling Language) Linguaggio di modellazione grafica basato sul paradigma orientato agli oggetti}
}

\newglossaryentry{Upper Camel Case}
{
	name={Upper Camel Case},
	description={Notazione che indica parole unite insieme la cui lettera iniziale è maiuscola}
	{\newpage}
}

\newglossaryentry{Verifica}
{
	name={Verifica},
	description={Accertare che l’esecuzione di specifiche attività non abbia prodotto errori. Viene fatta più di una volta nel corso di un progetto}
}

\newglossaryentry{viewModel}
{
	name={ViewModel},
	description={Classe scritta in java progettata per archiviare e gestire i dati relativi all'interfaccia utente}
}

\newglossaryentry{Voice Dialog Flow}
{
	name={Voice Dialog Flow},
	description={Documento di testo che descrive l'interazione fra l'utente e il dispositivo Alexa nel corso dell'esecuzione della skill}
	{\newpage}
}

\newglossaryentry{Walkthrough}
{
	name={Walkthrough},
	description={È un metodo di verifica dei requisiti che consiste in un'analisi dettagliata e approfondita di un documento di testo, senza l'utilizzo di strumenti di supporto (come le liste di controllo)}
}

\newglossaryentry{Workflow}
{
	name={Workflow},
	description={Consiste in una serie di connettori eseguiti da Alexa in successione uno dopo l'altro secondo le esigenze dell'utente}
	{\newpage}
}

\newglossaryentry{xML}
{
	name={XML (eXtensible Markup Language)},
	description={Metalinguaggio che permette di definire dei linguaggi di markup (ad es. HTML)}
}

\newglossaryentry{xML Schema}
{
	name={XML Schema},
	description={Documento che fornisce una descrizione formale di una grammatica per un linguaggio di markup basato su XML}
}