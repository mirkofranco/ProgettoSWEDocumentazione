\newpage
\section{S}\label{l:S}

\textbf{SCRUM}\index{SCRUM}\newline Modello di sviluppo Agile per la gestione di un progetto che prevede che l'avanzamento avvenga tramite una serie di sprint. Ad ogni riunione viene pianificato uno sprint.\\
\newline
\textbf{Skill}\index{Skill}\newline Per skill si intende la capacità di fare bene qualcosa.\\
\newline
\textbf{Skype}\index{Skype}\newline Software che permette di effetturare chiamate e videochiamate utilizzando la rete internet.\\
\newline
\textbf{SonarQube}\index{SonarQube}\newline Strumento che effettua l'analisi statica del codice di vari linguaggi di programmazione.\\
\newline
\textbf{SPICE}\index{SPICE}\newline Corrisponde allo standard ISO/IEC 15504 e serve per la valutazione di un processo software.\\
\newline
\textbf{Sprint}\index{Sprint}\newline Indica un'unità di base dello sviluppo in SCRUM ed ha una durata fissata. Ogni sprint è preceduto da una riunione di pianificazione in cui vengono stabiliti gli obbiettivi e stimati i tempi.\\
\newline
\textbf{Stakeholders}\index{Stakeholders}\newline (portatori di interesse) Persone che agiscono in qualità di fornitori,di committenti o di clienti.\\
\newline
\textbf{SVG}\index{SVG}\newline (Scalable Vector Graphics) Tipo di formato delle immagini.\\
\newline
\textbf{Swift}\index{Swift}\newline Linguaggio di programmazione per sistemi iOS e MacOS.