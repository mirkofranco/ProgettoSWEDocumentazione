\chapter{Qualità di Processo}
\label{processo} 
\section{Scopo}
Al fine di garantire la massima efficacia del prodotto finale è necessario controllare e misurare i processi che concorrono alla sua realizzazione; a tal fine si fa riferimento allo standard ISO/IEC 15504 conosciuto anche come SPICE per una valutazione complessiva sulla maturità dei processi e il loro miglioramento continuo.\\
Viene applicato inoltre il ciclo di Deming (PDCA), al fine di assicurare un miglioramento continuo delle attività di processo. 

\section{Processi}\label{processi}
Vengono elencati gli obiettivi perseguiti dal gruppo per assicurare la qualità di processo, insieme alle metriche adottate per misurare ciascuno di essi.
\subsection{QP001: Pianificazione delle attività di progetto, valutazione e controllo dei processi}
Verificare che il processo di pianificazione di progetto risulti adeguato è un obiettivo di primaria importanza: è opportuno quindi assicurarsi che la pianificazione temporale (comprendente l'assegnazione di\glossario{task}in conformità agli incrementi previsti nel calendario) e i costi sostenuti siano supportati da metriche in grado di verificare che lo stato del progetto sia conforme alle pianificazioni precedentemente prodotte.


\subsubsection{Obiettivi}
\begin{itemize}
	\item \textbf{Pianificazione temporale}
	\begin{itemize}
		\item \textbf{Gestione delle task}: verificare che l'assegnazione e il completamento delle task sia conforme a quanto pianificato e che non si presentino scostamenti;
		\item \textbf{Calendario}: la pianificazione delle task da assegnare deve essere adatta alla quantità di lavoro richiesto per il suo completamento.
	\end{itemize}
	\item \textbf{Budget}: accertarsi che i costi attuali derivanti dal lavoro svolto siano il più possibile conformi a quanto preventivato.
\end{itemize}
\subsubsection{Metriche utilizzate}
\begin{itemize}
	\item \textbf{MP001}: Schedule Variance;
	\item \textbf{MP002}: Cost Variance.
\end{itemize}


\subsection{QP002: Miglioramento continuo delle attività di processo}
Lo standard SPICE definisce dei livelli di maturità da perseguire per assicurare il miglioramento continuo delle attività di processo, obiettivo da perseguire se viene adottato il modello PDCA per lo sviluppo di un progetto software.
\subsubsection{Obiettivi}
\begin{itemize}
	\item \textbf{Maturità dei processi}: un processo diventa maturo quando il suo risultato è prevedibile prima della sua attuazione e risulta ottimizzato quando le risorse da lui impiegate vengono utilizzate con la massima efficienza possibile: vista l'inesperienza del gruppo, ottenere la piena ottimalità di tutti i processi è un obbiettivo molto difficile da raggiungere, tuttavia, attraverso il processo di miglioramento continuo è possibile prevedere una piena maturità degli stessi.\\
	A tal scopo si fa riferimento alle metriche definite dallo standard ISO/IEC 15504 per la verifica della qualità dei processi.
\end{itemize}

\subsubsection{Metriche utilizzate}
\begin{itemize}
	\item \textbf{MP003}: SPICE capability level;
	\item \textbf{MP004}: SPICE process attributes.
\end{itemize}

\subsection{QP003: Analisi e prevenzione dei rischi}
Il processo intende individuare e prevenire i rischi che potrebbero insorgere durante l'attività di progetto:

\subsubsection{Obiettivi}
\begin{itemize}
	\item \textbf{Individuazione dei rischi}: per ogni fase del progetto, verranno analizzati e attualizzati i rischi che potrebbero insorgere, cercando, ove possibile, di automatizzare le procedure che mitigano la loro occorrenza; non vengono definiti range di accettazione e ottimalità per le metriche adottate in questo processo.
\end{itemize}

\subsubsection{Metriche utilizzate}
\begin{itemize}	
	\item \textbf{MP005}: Occorrenza rischi non previsti;
	\item \textbf{MP006}: Indisponibilità dei servizi.
\end{itemize}
\subsection{QP004: Verifica del software}
Il processo si occupa di verificare che il software prodotto sia conforme ai\glossario{requisiti}stabiliti con la proponente e la committente, che sia privo di errori e che il codice scritto risulti chiaro, conciso ed efficiente.
\subsubsection{Obiettivi}
\begin{itemize}
	\item \textbf{Chiarezza del codice}: il codice prodotto deve risultare il più possibile chiaro, devono essere seguite le norme descritte nelle \textit{Norme di Progetto} ed esso deve essere supportato da commenti che chiariscano il funzionamento delle unità di codice a cui fanno riferimento;
	\item \textbf{Prevenzione degli errori}: ogni unità di codice deve risultare il più possibile privo di errori e di\glossario{bug}prima del suo utilizzo;
	\item \textbf{Passaggio dei Test}: ogni unità di codice fa riferimento a un test di unità a lui assegnato, è obbligatorio che tutti i test siano definiti pre-sviluppo e che ogni unità di codice passi il test ad essa riferita.\\
	Per ulteriori informazioni si rimanda alla sezione \ref{test}.
\end{itemize}
\subsubsection{Metriche utilizzate}

\begin{itemize}
	\item \textbf{MP007}: Complessità ciclomatica;
	\item \textbf{MP008}: Numero di parametri per metodo;
	\item \textbf{MP009}: Numero di livelli di annidamento;
	\item \textbf{MP010}: Attributi per classe.
	
	
\end{itemize}

\subsection{QP005: Gestione dei Test}\label{test}
L'obiettivo di tale processo è misurare l'efficacia del piano di test adottato: esso deve fornire risultati quantificabili sulla qualità del codice prodotto e permettere azioni correttive mirate.\\
Si fa riferimento a scopo informativo al libro \textit{Software Testing Fundamentals: Methods and Metrics} scritto da Marnie L. Hutcheson per la definizione delle metriche adottate e per la stesura del piano dei test di unità (vedi sezione \ref{rfinf}).
\subsubsection{Obiettivi}
\begin{itemize}
	\item \textbf{Qualità del piano di test.} 
\end{itemize}
\subsubsection{Metriche utilizzate}
\begin{itemize}
	\item \textbf{MP011}: Tempo medio del team di sviluppo per la risoluzione di errori;
	\item \textbf{MP012}: Efficienza della progettazione dei test.
\end{itemize}


\subsection{QP006: Versionamento e build}
\subsubsection{Obiettivi}
\begin{itemize}
	\item \textbf{Correttezza dei commit}: Ogni commit deve superare i controlli automatici, nel caso in cui un commit presenti degli errori  essi dovranno essere corretti immediatamente;
	\item \textbf{Dimensione dei commit}: Le modifiche apportate tramite un commit alla repository dovranno avere una dimensione opportunamente scelta, commit troppo piccoli potrebbero risultare di scarsa utilità, mentre quelli troppo grandi non permettono di fornire sufficienti informazioni sullo stato del ciclo di vita del software;
	\item  \textbf{Frequenza dei commit}: I commit effettuati dai membri del gruppo devono essere frequenti, poichè essi mantengono aggiornato il software con le modifiche più recenti e favoriscono il dialogo tra i membri del gruppo.
	
\end{itemize}
\subsubsection{Metriche utilizzate}
\begin{itemize}
	\item \textbf{MP013:} Percentuale build superate;
	\item \textbf{MP014:} Media commit giornaliera.
\end{itemize}

\subsection{QP007: Conformità dei requisiti}
Per assicurare la conformità del prodotto finale(e dei requisiti individuati che ne fanno parte), il gruppo utilizzerà le seguenti metriche:

\subsubsection{Metriche utilizzate}
\begin{itemize}
	\item \textbf{MP015:} Percentuale requisiti obbligatori soddisfatti;
	\item \textbf{MP016:} Percentuale requisiti desiderabili soddisfatti.
\end{itemize}

\section{Tabella dei processi}
\label{Tab2.1}
\begin{center}
	\begin{tabularx}{\textwidth}{|c|c|}
		\hline 
		\textbf{ID} & \textbf{Nome} \\ 
		\hline 
		QP001 &  Pianificazione delle attività di progetto,
		valutazione e controllo dei processi \\ 
		\hline 
		QP002 &  Miglioramento continuo delle attività di processo \\ 
		\hline 
		QP003 &  Analisi e prevenzione dei rischi \\ 
		\hline 
		QP004 &  Verifica del software \\ 
		\hline 
		QP005 &  Gestione dei test \\ 
		\hline 
		QP006 &  Conformità dei requisiti \\ 
		\hline 
		QP007 &  Versionamento e build \\ 
		\hline
		\caption{Tabella dei processi}
	\end{tabularx}
\end{center}
\section{Tabella delle metriche}
\label{Tab2.2}
La tabella elenca le metriche per le quali è stato definito un range di accettazione e di ottimalità
\begin{tabularx}{\textwidth}{|c|c|X|X|}
	\hline 
	\textbf{ID} & \textbf{Nome} & \textbf{Range di accettazione}  & \textbf{Range di ottimalità}  \\ 
	\hline
	\multicolumn{4}{|c|}{\textbf{QP001}} \\
	\hline
	MP001 & Schedule Variance & >= -3 giorni & >=0 giorni \\ 
	\hline
	MP002 &Cost Variance& >= - 4 &  >=0 \\ 
	\hline
	\multicolumn{4}{|c|}{\textbf{QP002}} \\
	\hline
	MP003 & SPICE capability level & 3-5 & 4-5 \\ 
	\hline 
	MP004 & SPICE process attributes & L -(>50\% - 85\%)  & F >85\% - 100\% \\ 
	\hline
	\multicolumn{4}{|c|}{\textbf{QP004}} \\
	\hline 
	MP005 & Occorrenza rischi non previsti &   0-4 &0 \\
	\hline 
	MP006 & Indisponibilità dei servizi & 0   & 0 \\
	\hline
	\multicolumn{4}{|c|}{\textbf{QP004}} \\
	\hline
	MP007 & Complessità ciclomatica & 1-15 & 1-10 \\ 
	\hline 
	MP008 & Numero di parametri per metodo & 0-8 & 0-4 \\ 
	\hline 
	MP009& Numero di livelli di annidamento & 1-6 & 1-3 \\ 
	\hline 
	MP010 & Attributi per classe & 0-16 & 3-8 \\ 
	\hline 
	\multicolumn{4}{|c|}{\textbf{QP005}} \\
	\hline
	MP011 &\makecell{Tempo medio del team di sviluppo \\ per la risoluzione di errori} & 0h-4h & 0h-2h \\ 
	\hline 
	MP012 & Efficienza della progettazione dei test  & 0.5h-3.5h  & 1h-2h \\ 
	\hline 
	\multicolumn{4}{|c|}{\textbf{QP006}} \\
	\hline
	MP013 & Percentuale build superate& 60\% & 80\%\\
	\hline
	MP014 & Media commit giornaliera & >=25 & >=35\\
	\hline
	\multicolumn{4}{|c|}{\textbf{QP007}} \\
	\hline
	MP015& \makecell{Percentuale requisiti \\ obbligatori soddisfatti}& 100\% & 100\%  \\
	\hline
	MP016& \makecell{Percentuale requisiti \\ desiderabili soddisfatti}& >=50\% &>=50\% \\
	\hline
	\caption{Tabella delle metriche dei processi}
\end{tabularx}