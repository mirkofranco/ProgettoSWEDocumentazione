\chapter{Qualità di Processo}
\label{processo} 
\section{Scopo}
Al fine di garantire la massima efficacia del prodotto finale è necessario controllare e misurare i processi che concorrono alla sua realizzazione; a tal fine si fa riferimento ai principi stabiliti dallo standard ISO/IEC 12207 per la suddivisione dei processi in moduli disaccoppiati ma comunque coesi fra loro e alle responsabilità derivanti da ciascuno di essi.
Viene applicato inoltre il ciclo di Deming (PDCA), al fine di assicurare un miglioramento continuo delle attività di processo. 

\section{Processi}

\subsection{Pianificazione}
Per misurare l'efficacia della pianificazione vengono adottate le seguenti metriche 

\subsubsection{BCWP: Budgeted Cost Of Work Performed}\label{bcwp}
Utilizzato per il calcolo di Cost Variance e Schedule Variance, rappresenta (in giorni) il valore delle attività svolte.
\begin{itemize}
	\item Range accettazione: [>= 0];
	\item Range ottimale: [>= 0].
\end{itemize}
\subsubsection{BCWS: Budgeted Cost of Work Scheduled}\label{bcws}
Rappresenta il costo in giorni preventivato per il processo in esame (è detto anche Planned Value):
\begin{itemize}
\item Range accettazione: [>=0].
\item Range ottimale: [>= 0].
\end{itemize}

\subsubsection{ACWP: Actual Cost of Work Performed}\label{acwp}
Rappresenta il costo (in \euro) effettivamente sostenuto al momento del calcolo:
\begin{itemize}
	\item Range accettazione: [>=0].
	\item Range ottimale: [>= 0].
\end{itemize}

\subsubsection{SV: Schedule Variance}
Metrica che indica se si è in anticipo o in ritardo rispetto alla schedulazione delle attività di progetto.
Essa è il risultato della seguente formula:\\
\begin{center}
	
	$SV = $\hyperref[bcwp]{BCWP} $-$\hyperref[bcws]{BCWS}
	
\end{center}

\begin{itemize}
	\item Range accettazione: $-BCWS *10\%$;
	\item Range ottimale: [>= 0].
\end{itemize}



\subsection{Miglioramento}
%todo
%STANDARD SPICE??%
\subsection{Costo}
Per verificare che i costi siano conformi a quanto preventivato nel \textit{Piano di Progetto}, ciascun processo viene misurato tramite la sua \textbf{Cost Variance(CV)}, un valore positivo indica il rispetto dei costi preventivati, essa viene calcolata nel seguente modo:\\ 

\begin{center}
	\begin{math}
	CV = BCWP - ACWP
	\end{math}
\end{center}
\begin{itemize}
	\item[] \textbf{ACWP} (Actual Cost of Work Performed) rappresenta il costo(in giorni) effettivamente sostenuto al momento del calcolo. 
\end{itemize}
Dove: BCWP e ACWP sono descritte rispettivamente nelle sezioni \ref{bcwp} e \ref{acwp}.

\begin{itemize}
	\item Range accettazione: Viene accettato uno scostamento massimo del 10\% rispetto ai costi preventivati;
	\item Range ottimale: [>= 0].
\end{itemize}


\subsection{Analisi dei rischi}
%todo
%spice +sv
\section{Metriche}
tabella-metriche