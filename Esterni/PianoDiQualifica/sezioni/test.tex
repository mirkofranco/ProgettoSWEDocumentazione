\chapter{Specifica test}
\label{test}
\section{Test di Sistema}
\normalsize
\begin{longtable}{|c|>{}m{8cm}|c|}
\hline 
\textbf{Id Test} & \textbf{Descrizione} & \textbf{Stato}\\
\hline
\endhead
\hypertarget{TSFO2}{TSFO2} & Il gruppo esegue l'applicazione e controlla che il login funziona correttamente. & \textit{Non Implementato}\\ \hline
\hypertarget{TSFO15}{TSFO15} & Eseguire l'applicazione, creare un nuovo workflow, aggiunge dei blocchi a piacere e salvare il workflow. Bisogna verificare che il workflow sia presente nel database. & \textit{Non Implementato}\\ \hline
\hypertarget{TSFO16}{TSFO16} & Avviare l'applicazione ed eliminare un workflow. & \textit{Non Implementato}\\ \hline
\hypertarget{TSFO17.1}{TSFO17.1} & Avviare l'applicazione e modificare il nome di un workflow, poi verificare che sia stato effettivamente modificato nel database. & \textit{Non Implementato}\\ \hline
\hypertarget{TSFO17.2}{TSFO17.2} & Il test consiste nell' eseguire l'applicazione e modificare un workflow aggiungendo un blocco, modificandone un altro ed eliminandone un terzo. In seguito verificare che il database sia stato effettivamente modificato. & \textit{Non Implementato}\\ \hline
\caption[Test di Sistema]{Test di Sistema}
\label{tabella:test1}
\end{longtable}
\clearpage
\subsection{Tracciamento Test di Sistema-Requisiti}
Ogni test di sistema viene eseguito per validare un requisito, definito nell'\textit{analisi dei requisiti 2.0.0}
\normalsize
\begin{longtable}{|>{\centering}m{5cm}|m{5cm}<{\centering}|}
\hline
\textbf{Test} & \textbf{Requisito}\\
\hline
\endhead
\hyperlink{TSFO2}{TSFO2} & RFO2\\ \hline
\hyperlink{TSFO15}{TSFO15} & RFO15\\ \hline
\hyperlink{TSFO16}{TSFO16} & RFO16\\ \hline
\hyperlink{TSFO17.1}{TSFO17.1} & RFO17.1\\ \hline
\hyperlink{TSFO17.2}{TSFO17.2} & RFO17.2\\ \hline
\caption[Tracciamento Test di Sistema-Requisiti]{Tracciamento Test di Sistema-Requisiti}
\label{tabella:ts-requi}
\end{longtable}

\section{Test di Integrazione}
\normalsize
\begin{longtable}{|c|>{}m{8cm}|c|}
\hline 
\textbf{Id Test} & \textbf{Descrizione} & \textbf{Stato}\\
\hline
\endhead
\hypertarget{TI1}{TI1} & Viene testata l'applicazione Android ad ogni sua modifica nel repository Github, attraverso Travis-CI. L'applicazione deve compilare e non produrre warning. & \textit{Non Eseguito}\\ \hline
\hypertarget{TI2}{TI2} & La skill Android è soggetta a integrazione continua, attraverso Travis-CI. & \textit{Non Eseguito}\\ \hline
\hypertarget{TI3}{TI3} & La skill viene pubblicata in automatico (ad ogni modifica nel repository) in Aws Lambda e viene eseguito un test per controllare che funzioni. Questo test non assicura il funzionamento della skill, quindi è necessario un controllo umano (più controlli sono richiesti in caso di un numero alto di commit nel repository). & \textit{Non Eseguito}\\ \hline
\hypertarget{TI4}{TI4} & Ad ogni commit nel repository, Travis-CI controlla che il collegamento tra la Skill e AWS API-Gateway funzioni. Questo viene fatto attraverso una semplice chiamata post a una funzione MOCK (per ridurre il traffico a DynamoDB e alle lambda). & \textit{Non Implementato}\\ \hline
\hypertarget{TI5}{TI5} & Controllare che il collegamento tra API-gateway, Lambda e DynamoDB funzioni, eseguendo dei test automatici appositi. Questa operazione deve essere eseguita poco frequentemente, in quanto potrebbe aumentare i costi dei servizi AWS. & \textit{Non Implementato}\\ \hline
\caption[Test di Integrazione]{Test di Integrazione}
\label{tabella:test2}
\end{longtable}

\subsection{Tracciamento Test di Integrazione-Componenti}
Nella seguente tabella viene riportato il tracciamento tra test di Integrazione e Package, definiti in \textit{norme di progetto 2.0.0}.
\normalsize
\begin{longtable}{|>{\centering}m{3cm}|m{9cm}<{\centering}|}
\hline
\textbf{Test} & \textbf{Componente}\\
\hline
\endhead
\hyperlink{TI1}{TI1} & \texttt{megalexa}\\ \hline
\hyperlink{TI2}{TI2} & {\texttt{MegAlexaSkill}}\\ \hline
\hyperlink{TI3}{TI3} & {\texttt{MegAlexaSkill::lambda}}\\ \hline
\hyperlink{TI4}{TI4} & {\texttt{MegAlexaSkill::lambda::connection}}\\ \hline
\hyperlink{TI5}{TI5} & {\texttt{megalexa::adapters}}\\ \hline
\caption[Tracciamento Test di Integrazione-Componenti]{Tracciamento Test di Integrazione-Componenti}
\label{tabella:ts-requi}
\end{longtable}

\section{Test di Unità}
Questa sezione verrà compilata durante il periodo di progettazione di dettaglio e codifica, per applicare il test driven developlment (come richiesto dalla proponente).