\chapter{Resoconto delle attività di verifica}
\label{resoconto}
\section{Analisi}
Nel periodo antecedente la Revisione dei Requisiti sono stati verificati i documenti ed i processi applicando quanto descritto nelle \textit{Norme di Progetto v1.0.0}.\\
L'analisi statica è stata effettuata secondo i criteri e le modalità indicate nella sezione \ref{AnalisiStatica}.\\ 
Per gli errori riscontrati effettuando \textit{walkthrough$_{G}$}, si è provveduto a correggere le anomalie riscontrate e sono stati riportati nella lista di controllo nelle \textit{Norme di progetto v1.0.0} per permettere di effettuare inspection successivamente.\\
L'\textit{inspection$_{G}$} viene effettuata utilizzando la lista di controllo precedentemente stilata. \\
Si sono poi calcolate le metriche descritte nella sezione \ref{IndicediGulpease}.\\
L'avanzamento dei processi è stato poi valutato secondo le metriche descritte nella \hyperref[ProcessoMetriche]{sezione 3.1} 
\subsection{Verifica dei processi}
Per il\glossario{processo}di stesura dei documenti, il calcolo delle metriche di Budget Variance e di Schedule Variance è stato effettuato sul valore complessivo delle ore impiegate dal totale dei componenti del gruppo.\\
Per le successive fasi del \textit{progetto$_{G}$}, il gruppo si propone di automatizzare il processo di calcolo delle ore impiegate, con il dettaglio puntuale dei singoli processi.
Lo Schedule Variance totale è di -1 ore e il Budget Variance totale equivale a -25\euro.
\begin{comment}
\begin{tabularx}{\textwidth}{|C|C|C|}
	\hline
	\textbf{Macro-Attività}& \textbf{SV}&\textbf{BV}\\
	\hline
	\textit{Norme di Progetto}    & \euro & \euro\\
	\textit{Piano di Progetto}    & \euro & \euro\\
	\textit{Studio di Fattibilità} & \euro & \euro\\
	\textit{Analisi dei Requisiti}& \euro & \euro\\
	\textit{Piano di Qualifica}   & \euro & \euro\\
	\textit{Glossario}            & \euro & \euro\\
	\hline
	\caption{Esito verifica processi}
\end{tabularx}
\end{comment}
\\
\subsection{Verifica dei documenti}
\begin{tabularx}{\textwidth}{|C|c|C|}
	\hline
	\textbf{Documento}& \textbf{Indice di Gulpease}&\textbf{Esito}\\
	\hline
	\endhead
	\textit{Norme di Progetto}    & 76 & Superato \\
	\textit{Piano di Progetto}    & 64 & Superato \\
	\textit{Studio di Fattibilità} & 61 & Superato\\
	\textit{Analisi dei Requisiti}& 80 & Superato \\
	\textit{Piano di Qualifica}   & 67 & Superato \\
	\textit{Glossario}            & 68 & Superato \\
	\hline
	\caption{Esito della verifica documenti}
\end{tabularx}

\section{Revisione Analisi}
\label{revisione}
Durante il breve periodo di Revisione Analisi, il gruppo si è preparato allo sviluppo del POC e ha apportato delle correzione ai documenti, migliorando i propri processi. 
\subsection{Verifica dei processi}
I miglioramenti principali (tutti descritti nelle \textit{norme di progetto v2.0.0}) sono stati:
\begin{itemize}
	\item automatizzato il calcolo delle ore di lavoro integrando \glossario{Harvest} ad \glossario{Asana};
	\item automatizzato il calcolo dell'indice di gulpease, tramite script;
	\item se dei documenti contenenti degli errori grammaticali raggiungono la repository, un bot avvisa per email chi ha commesso l'errore e invia una notifica al gruppo.
\end{itemize}

\subsubsection{MP004 Schedule variance}
Schedule variance
\newpage
\subsubsection{MP024 Budget variance}
\begin{figure} [h]
    \centering
	\includegraphics[scale=0.5]{./images/bvra.png}
	\caption{\textit{MP024 - Revisione Analisi}}\label{}
\end{figure}

\section{Progettazione della base tecnologica}
\label{progettazione}
\subsubsection{MP004 Schedule variance}
Schedule variance

\subsubsection{MP024 Budget variance}
\begin{figure} [h]
    \centering
	\includegraphics[scale=0.5]{./images/bvp.png}
	\caption{\textit{MP024 - Progettazione della base tecnologica}}\label{}
\end{figure}

\subsubsection{MP005: SPICE capability level}
Di seguito vengono riportati i livelli di maturità raggiunti dai processi eseguiti durante lo sviluppo del \glossario{poc};
\begin{figure} [h]
    \centering
	\includegraphics[scale=0.5]{./images/15504.PNG}
    \caption{\textit{MP005 - ISO/IEC 15504 }}\label{}
\end{figure}

\subsubsection{MP020 Percentuale build superate}
Viene fatta distinzione tra Android e Skill, in quanto vengono contenute in repository diversi.
Le build non superate sono 24 su 134 per la Skill e 29 su 232 per Android.
\begin{figure} [h]
    \centering
	\includegraphics[scale=0.5]{./images/StatobuildTravis-ciandroid.png}
    \caption{\textit{MP020 - Android - Progettazione della base tecnologica}}\label{}
\end{figure}
\begin{figure} [h]
    \centering
	\includegraphics[scale=0.5]{./images/StatobuildTravis-ciSkill.png}
    \caption{\textit{MP020 - Skill - Progettazione della base tecnologica}}\label{}
\end{figure}

\subsection{MP007:  Occorrenza rischi non previsti}
\textbf{rischi non previsti: 1}\\
Un aggiornamento automatico di \glossario{Android Studio} completamente rimosso una libreria utilizzata dall'applicazione mobile, quindi il gruppo ha perso tempo per implementane una alternativa. Questo è successo perché la libreria in questione era deprecata. Per evitare problemi simili, l'utilizzo di librerie deprecate è stato vietato, come descritto nelle \textit{norme di progetto}.

\subsection{MP008: Indisponibilità dei servizi}
\textbf{indisponibilità dei servizi: 0}\\
Durante il periodo di progettazione della base tecnologica, il gruppo non ha riscontrato problemi riguardanti il downtime di servizi esterni.

\subsubsection{MP021: Media commit per settimana}
Come si può notare dai grafici, il numero di commit è stato abbastanza costante, con un aumento del carico di lavoro durante la fine di febbraio.
\begin{figure} [h]
    \centering
	\includegraphics[scale=0.7]{./images/dailycommits_kotlin.PNG}
    \caption{\textit{MP021 - Android- Progettazione della base tecnologica}}\label{}
\end{figure}
\begin{figure} [h]
    \centering
	\includegraphics[scale=0.7]{./images/daycommits_js.PNG}
    \caption{\textit{MP021 - Skill - Progettazione della base tecnologica}}\label{}
\end{figure}

\newpage
\section{Progettazione di dettaglio e codifica}
Questa sezione verrà compilata alla fine del periodo di Progettazione della base tecnologica.
\section{Verifica e collaudo}
Questa sezione verrà compilata alla fine del periodo di Verifica e collaudo.