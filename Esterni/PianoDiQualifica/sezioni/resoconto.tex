\chapter{Resoconto delle attività di verifica}
\section{Analisi}
Nel periodo antecedente la Revisione dei Requisiti sono stati verificati i documenti ed i processi applicando quanto descritto nelle \textit{Norme di Progetto v1.0.0}.\\
L'analisi statica è stata effettuata secondo i criteri e le modalità indicate nella sezione \ref{AnalisiStatica}.\\ 
Per gli errori riscontrati effettuando \textit{walkthrough$_{G}$}, si è provveduto a correggere le anomalie riscontrate e sono stati riportati nella lista di controllo nelle \textit{Norme di progetto v1.0.0} per permettere di effettuare inspection successivamente.\\
L'\textit{inspection$_{G}$} viene effettuata utilizzando la lista di controllo precedentemente stilata. \\
Si sono poi calcolate le metriche descritte nella sezione \ref{IndicediGulpease}.\\
L'avanzamento dei processi è stato poi valutato secondo le metriche descritte nella \hyperref[ProcessoMetriche]{sezione 3.1} 
\subsection{Verifica dei processi}
Per il\glossario{processo}di stesura dei documenti, il calcolo delle metriche di Budget Variance e di Schedule Variance è stato effettuato sul valore complessivo delle ore impiegate dal totale dei componenti del gruppo.\\
Per le successive fasi del \textit{progetto$_{G}$}, il gruppo si propone di automatizzare il processo di calcolo delle ore impiegate, con il dettaglio puntuale dei singoli processi.
Lo Schedule Variance totale è di -1 ore e il Budget Variance totale equivale a -25\euro.
\begin{comment}
\begin{tabularx}{\textwidth}{|C|C|C|}
	\hline
	\textbf{Macro-Attività}& \textbf{SV}&\textbf{BV}\\
	\hline
	\textit{Norme di Progetto}    & \euro & \euro\\
	\textit{Piano di Progetto}    & \euro & \euro\\
	\textit{Studio di Fattibilità} & \euro & \euro\\
	\textit{Analisi dei Requisiti}& \euro & \euro\\
	\textit{Piano di Qualifica}   & \euro & \euro\\
	\textit{Glossario}            & \euro & \euro\\
	\hline
	\caption{Esito verifica processi}
\end{tabularx}
\end{comment}
\\
\subsection{Verifica dei documenti}
\begin{tabularx}{\textwidth}{|C|c|C|}
	\hline
	\textbf{Documento}& \textbf{Indice di Gulpease}&\textbf{Esito}\\
	\hline
	\endhead
	\textit{Norme di Progetto}    & 76 & Superato \\
	\textit{Piano di Progetto}    & 64 & Superato \\
	\textit{Studio di Fattibilità} & 61 & Superato\\
	\textit{Analisi dei Requisiti}& 80 & Superato \\
	\textit{Piano di Qualifica}   & 67 & Superato \\
	\textit{Glossario}            & 68 & Superato \\
	\hline
	\caption{Esito della verifica documenti}
\end{tabularx}



