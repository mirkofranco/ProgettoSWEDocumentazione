\chapter{Qualità di Processo}
\section{Obbiettivi}
Affinché la qualità di prodotto sia garantita è necessario perseguire al qualità dei processi che concorrono alla sua realizzazione. Per raggiungere questo obbiettivo viene adottato lo standard ISO/IEC 15504 denominato SPICE che fornisce strumenti per valutare la bontà dei processi che vengono realizzati.
Per applicare correttamente questo standard è necessario utilizzare il ciclo di Deming (ciclo PCDA) che fornisce una metodologia per il controllo dei processi che consente di migliorarne in modo continuativo la qualità. 
\section{Procedure di controllo di qualità dei processi}
La qualità dei processi viene garantita dall'applicazione del ciclo PDCA. Grazie a tale principio è possibile ottenere un miglioramento continuo della qualità di ogni processo e come diretta conseguenza si ottiene il miglioramento dei prodotti ottenuti.
Per avere controllo dei processi, e di conseguenza migliorarne la qualità, bisogna: 
\begin{itemize}
		\item Definire in modo dettagliato i processi;
		\item Assegnare in modo chiaro le risorse ai processi;
		\item Avere controllo sui processi.
\end{itemize} 
La realizzazione di questi punti è descritta in modo dettagliato nel \textit{Piano di Progetto v1.0.0}.
\section{Metriche}
Come metrica per i processi si è deciso di utilizzare indici che ci permettano di valutare costi e tempi. 
\subsection{Schedule Variance (SV)}
Indica se si è in linea, in anticipo o in ritardo rispetto alla pianificazione temporale indicata delle attività. 
Se SV > 0 significa che il gruppo sta procedendo con maggiore velocità rispetto a quanto pianificato, viceversa se negativo. 
Essendo stati inseriti slack durante la pianificazione delle attività dei processi, il valore di tale indice è inizialmente positivo.
\begin{itemize}
	\item Range accettazione: >= -(costo preventivo fase x 10\%)
	\item Range ottimale: [>= 0] 
\end{itemize}
\subsection{Budget Variance (BV)}
Indica se alla data corrente corrente si è speso di più o meno rispetto a quanto pianificato.
Se BV >= 0 significa che si sta usando più budget di quanto preventivato, negativo viceversa.
\begin{itemize}
	\item  Range accettazione: >= -( costo preventivo fase x 10\%)
	\item Range ottimale: >= 0
\end{itemize}
