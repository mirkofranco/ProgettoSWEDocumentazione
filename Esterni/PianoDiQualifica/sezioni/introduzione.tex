\chapter{Introduzione}
\section{Scopo del documento}
Il \texttt{Piano di Qualifica} ha lo scopo di definire gli obbiettivi di qualità che il gruppo perseguita per il proprio prodotto. Per ottenere tali obbiettivi è necessario un processo di verifica continua di ogni attività. Questo consente di rilevare e correggere le anomalie riscontrate tempestivamente. 
\section{Scopo del prodotto}
Lo scopo del progetto è quello di creare un applicativo Web e Mobile in grado di creare delle routine personalizzate per gli utenti gestibili tramite Alexa di Amazon. L'obbiettivo è quello di creare skill in grado di avviare workflow creati dagli utenti fornendogli dei connettori.
\section{Glossario}
Al fine di evitare ogni ambiguità di linguaggio e massimizzare la comprensione dei documenti, i termini tecnici, di dominio, gli acronimi e le parole che necessitano di essere chiarite, sono riportate nel \texttt{Glossario v1.0.0}.\\
Ogni occorrenza di vocaboli presenti nel \texttt{Glossario} è marcata da una "G" maiuscola in pedice.
\section{Riferimenti}
\subsection{Normativi}
\begin{itemize}
	\item  \textbf{Norme di Progetto}: \texttt{Norme di Progetto v1.0.0}
	\item \textbf{Capitolato d'appalto C4}: MegAlexa: arricchitore di skill di Amazin Alexa
\end{itemize}
\subsection{Informativi}
\begin{itemize}
	\item \textbf{Piano di Progetto}: \texttt{Piano di Progetto v1.0.0}
\end{itemize}
