\chapter{Introduzione}
\label{introduzione}
\section{Scopo del documento}
Il \textit{Piano di Qualifica} ha lo scopo di definire gli obbiettivi di qualità che il gruppo perseguita per il proprio prodotto. Per ottenere tali obbiettivi è necessario un processo di verifica continua di ogni attività. Questo consente di rilevare e correggere le anomalie riscontrate tempestivamente.\\
Questo documento descrive nel dettaglio la qualità dei processi più vicini nel tempo e ad alto livello quelli più lontani, per poi essere aggiornato con nuovi contenuti ogni volta che il gruppo lo ritiene necessario.
\section{Scopo del prodotto}
Lo scopo del progetto è sviluppare un applicativo mobile in grado di creare delle routine personalizzate per gli utenti gestibili tramite\glossario{Alexa}di \textit{Amazon$_{G}$}. L'obbiettivo è creare\glossario{skill}in grado di avviare\glossario{workflow}creati dagli utenti fornendogli dei \textit{connettori$_{G}$}.
\section{Glossario}
Al fine di evitare ogni ambiguità di linguaggio e massimizzare la comprensione dei documenti, i termini tecnici, di dominio, gli acronimi e le parole che necessitano di essere chiarite, sono riportate nel \glossariodocumento.\\
Ogni occorrenza di vocaboli presenti nel \textit{Glossario} è marcata da una "G" maiuscola in pedice.
\section{Riferimenti}
\subsection{Normativi}
\label{RiferimentiNormativi}
\begin{itemize}
	\item  \textbf{Norme di Progetto}: \normediprogetto;
	\item \textbf{Capitolato$_{G}$ C4}: \textit{MegAlexa$_{G}$}: arricchitore di skill di Amazon Alexa\footnote{\url{https://www.math.unipd.it/~tullio/IS-1/2018/Progetto/C4.pdf}}.
\end{itemize}
\subsection{Informativi}
\label{RiferimentiInformativi}
\begin{itemize}\label{rfinf}
	\item \textbf{Piano di Progetto}: \pianodiprogetto;
	\item \textbf{Complessità ciclomatica}\footnote{\url{https://it.wikipedia.org/wiki/Complessit\%C3\%A0_ciclomatica}};
	\item \textbf{Software Testing Fundamentals: Methods and Metrics} di Marnie L. Hutcheson, Wiley Publishing, Inc. capitolo 2,4 e 5.
	\item \textbf{Ciclo di Deming};
	\footnote{\url{https://it.wikipedia.org/wiki/Ciclo_di_Deming}}.
\end{itemize}
