\chapter{Qualità di Prodotto}
\label{prodotto}
\section{scopo}

\section{Qualità documento}
\label{documento}
Il team si impegna a produrre dei documenti di alta qualità, rispettando le seguenti caratteristiche.
\subsection{Ortografia}
Un documento, per essere privo di errori grammaticali e ortografici, viene controllato su diversi ambienti: durante la redazione tramite il controllo automatico integrato nell'ambiente di lavoro; nel repository condiviso tramite il correttore automatico eseguito da Travis-ci (con notifica in caso di errori); durante la verifica da parte di un \glossario{Verificatore}.
\begin{itemize}
    \item \textbf{Metrica:} Il numero di errori trovati durante la verifica;
    \item \textbf{Soglia di accettabilità:} si accettano al massimo due errori dopo la verifica da parte dei \glossario{Verificatori};
    \item \textbf{Soglia di ottimalità:} la soglia di ottimalità verrà raggiunta quando il documento sarà completamente privo di errori.
\end{itemize}
\subsection{Comprensibilità e leggibilità}
Per misurare la leggibilità di un documento il gruppo ha scelto di utilizzare l'indice di \glossario{Gulpease}. Questo viene calcolato automaticamente ogni volta che il documento viene modificato nel repository condiviso.
\begin{itemize}
    \item \textbf{Metrica:} indice di \glossario{Gulpease};
    \item \textbf{Soglia di accettabilità:} viene accettato un indice di Gulpease non inferiore a 40;
    \item \textbf{Soglia di ottimalità:} la soglia di ottimalità verrà raggiunta con l'indice di Gulpease maggiore di 50.
\end{itemize}
\subsection{Correttezza dei contenuti}
La correttezza del documento è data anche dalla coerenza dei contenuti. Gli \glossario{Analisti} devono redigere dei buoni documenti, i verificatori devono controllarli e correggerli.
\begin{itemize}
    \item \textbf{Metrica:} numero di errori inerenti alla correttezza dei documenti;
    \item \textbf{Soglia di accettabilità:} vengono accettati al massimo 3 errori di questo tipo dopo la verifica da parte dei \glossario{Verificatori};
    \item \textbf{Soglia di ottimalità:} la soglia di ottimalità viene raggiunta quando il documento è privo di errori dopo la verifica.
\end{itemize}
\subsection{Adesione alla norme interne}
I contenuti di un documento devono rispettare le Norme di Progetto. I \glossario{Verificatori} hanno il compito di avvisare gli \glossario{Analisti} in caso di mancato rispetto delle norme.
\begin{itemize}
    \item \textbf{Metrica:} numero di errori inerenti alle Norme di Progetto;
    \item \textbf{Soglia di accettabilità:} vengono accettati non più di 3 errori di questo tipo dopo la verifica da parte dei \glossario{Verificatori};
    \item \textbf{Soglia di ottimalità:} la soglia di ottimalità viene raggiunta se il documento è privo di errori dopo la verifica.
\end{itemize}

\section{Qualità del software}
\label{software}
(-> principi iso9126. guardare sezioni vecchie) 

\subsection{Funzionalità}

\subsection{Affidabilità}

\subsection{Efficienza}

\subsection{Usabilità}

\subsection{Portabilità}

\section{Metriche}