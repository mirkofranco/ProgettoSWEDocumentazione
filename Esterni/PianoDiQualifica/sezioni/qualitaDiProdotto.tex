\chapter{Qualità di Prodotto}
\label{prodotto}
\section{Scopo}
Per riuscire a garantire una buona qualità di prodotto, sono state individuale nello
standard ISO/IEC 25010 le principale caratteristiche da possedere nel ciclo di vita del prodotto definendone le sotto-caratteristiche che le identificano e le metriche per la valutazione.

\section{Qualità documento}
\label{documento}
Il team si impegna a produrre dei documenti di alta qualità, rispettando le seguenti caratteristiche.
\subsection{Ortografia}
\subsubsection{obbiettivi}
Un documento, per essere privo di errori grammaticali e ortografici, viene controllato su diversi ambienti: durante la redazione, tramite il controllo automatico integrato nell'ambiente di lavoro; nel repository condiviso, tramite il correttore automatico eseguito da Travis-ci (con notifica in caso di errori); durante la verifica, da parte di un\glossario{Verificatore}.\\
Le metriche utilizzate per la valutazione, definite nelle Norme di progetti in ??, sono le seguenti:
\begin{itemize}
    \item MPR001 numero di errori ortografici.
\end{itemize}
\subsection{Comprensibilità e leggibilità}
\subsubsection{obbiettivi}
Per misurare la leggibilità di un documento il gruppo ha scelto di utilizzare l'indice di\glossario{Gulpease}. Questo viene calcolato automaticamente ogni volta che il documento viene modificato nel repository condiviso.\\
Le metriche utilizzate per la valutazione, definite nelle Norme di progetti in ??, sono le seguenti:
\begin{itemize}
    \item MPR002 indice di Gulpease$_{G}$.
\end{itemize}
\subsection{Correttezza dei contenuti}
\subsubsection{obbiettivi}
La correttezza del documento è data anche dalla coerenza dei contenuti. Gli \glossario{Analisti} devono redigere dei buoni documenti, i verificatori devono controllarli e correggerli.\\
Le metriche utilizzate per la valutazione, definite nelle Norme di progetti in ??, sono le seguenti:
\begin{itemize}
    \item MPR003 Numero di errori inerenti alla correttezza dei documenti.
\end{itemize}
\subsection{Adesione alla norme interne}
\subsubsection{obbiettivi}
I contenuti di un documento devono rispettare le Norme di Progetto. I \glossario{Verificatori} hanno il compito di avvisare gli \glossario{Analisti} in caso di mancato rispetto delle norme.\\
Le metriche utilizzate per la valutazione, definite nelle Norme di progetti in ??, sono le seguenti:
\begin{itemize}
    \item MPR004 numero di errori inerenti alle Norme di Progetto.
\end{itemize}

\section{Qualità del software}

\subsection{Funzionalità}
Rappresenta la capacità del prodotto software di provvedere le funzionalità necessarie a soddisfare i requisiti individuati nel documento Analisi dei Requisiti v2.0.0. 
\subsubsection{Obbiettivi }Il prodotto dovrà possedere le seguenti caratteristiche:
\begin{itemize}
	\item \textbf{Efficacia funzionale:} Indice che determina il grado di copertura dei requisiti;
	\item \textbf{Correttezza:} Indice che determina la correttezza dei risultati forniti dal software.
\end{itemize}

Le metriche utilizzate per la valutazione, definite nelle Norme di progetti in ??, sono le seguenti:
\begin{itemize}
	\item MPR005 Completezza dell'implementazione funzionale;
	\item MPR006 Correttezza rispetto alle attese.
\end{itemize}

\subsection{Affidabilità}
Rappresenta la capacità del prodotto software di svolgere correttamente le sue funzionalità mantenendo delle buone prestazioni al verificarsi di situazioni anomale.

\subsubsection{Obbiettivi } Il prodotto dovrà possedere le seguenti caratteristiche :
\begin{itemize}
	\item \textbf{Tolleranza agli errori:} Il\glossario{prodotto}software continua a lavorare correttamente in presenza di errori dovuti a uno scorretto uso dell'applicativo;
	\item \textbf{Recuperabilità:} Nel caso in cui si presenta un'anomalia, l'applicativo è in grado di recuperare i dati e ripristinare lo stato interrotto.
\end{itemize}

Le metriche utilizzate per la valutazione, definite nelle Norme di progetto in ??, sono le seguenti:
\begin{itemize}
	\item MPR007 Totalità di failure.
\end{itemize}


\subsection{Efficienza}
Rappresenta la capacità di un prodotto software di realizzare le funzioni richieste nel minor tempo possibile e con l'uso del minimo numero di risorse necessarie. 
\subsubsection{Obbiettivi } Il prodotto dovrà possedere le seguenti caratteristiche :
\begin{itemize}
	\item \textbf{Comportamento rispetto al tempo:} per svolgere le  funzioni richieste il prodotto software deve fornire adeguati tempi di risposta ed elaborazione;
	\item \textbf{Utilizzo delle risorse:} il software nello svolgimento delle funzionalità deve utilizzare un appropriato numero e tipo di risorse.
\end{itemize}
Le metriche utilizzate per la valutazione, definite nelle Norme di progetto in ??, sono le seguenti:
\begin{itemize}
	\item MPR008 Tempo di risposta.
\end{itemize}

\subsection{Usabilità}
L'usabilità rappresenta il grado di facilità e soddisfazione con cui si compie l'interazione tra l'uomo e il \textit{prodotto$_{G}$}, ovvero l'efficacia, l'efficienza e la soddisfazione con le quali gli utenti raggiungono determinati obbiettivi in determinati contesti.
\subsubsection{Obbiettivi } Il prodotto dovrà possedere le seguenti caratteristiche :
\begin{itemize}
	\item \textbf{Apprendibilità:}l'utente deve essere in grado di riconoscere le funzionalità offerte dal software e comprendere se sono utili per il proprio scopo; ---->(prima)Indica il rapporto tra tempo necessario per l'apprendimento e quantità di informazioni correttamente apprese.
	La curva d'apprendimento sarà differente in base al dispositivo considerato in quanto l'applicazione mobile avrà funzionalità più complesse rispetto all'echo che dovrà essere utilizzato da un qualsiasi utente; Dunque verranno suddivisi i requisiti in base al dispositivo:
	\begin{itemize}
		\item \textbf{Curva d'apprendimento App};
		\item \textbf{Curva d'apprendimento Alexa$_{G}$}.
		\item \textbf{Protezione dall'errore:} Rappresenta il grado con cui il\glossario{prodotto} protegge l'utente dal commettere errori;
		\item \textbf{Estetica dell'interfaccia utente:} Rappresenta la capacità dell'interfaccia di evitare smarrimento e confusione da parte dell'utente.
		L'interfaccia dell'echo ha un'usabilità differente in quanto si parla di interfaccia vocale, dunque soggetta a diversi tempi di reattività e apprendimento rispetto ad un'interfaccia mobile;
		\item \textbf{Accessibilità:} Si intende la possibilità di fornire i servizi anche a coloro che sono affetti da disabilità temporanee e non, che quindi utilizzano tecnologie ausiliarie.
		Nel caso dell'echo alcune disabilità sono per ora vincolanti in quanto presuppongono necessariamente l'utilizzo della voce.
	\end{itemize}
	Le metriche utilizzate per la valutazione, definite nelle Norme di progetto in ??, sono le seguenti:
	\begin{itemize}
		\item MPR009 Comprensibilità delle funzioni offerte;
		\item MPR010 Facilità di apprendimento;
	\end{itemize}
\end{itemize}

\subsection{Manutenibilità}
Rappresenta la capacità del\glossario{prodotto}di essere modificato tramite correzioni, miglioramenti e adattamenti.
\subsubsection{Obbiettivi } Il prodotto dovrà possedere le seguenti caratteristiche :
\begin{itemize}
	\item \textbf{Analizzabilità:} Il software deve poter essere analizzato per poter trovare gli errori;
	\item \textbf{Modificabilità:} Il prodotto deve permettere la modifica delle sue parti;
	\item \textbf{Modularità:} Il prodotto è diviso in parti che svolgono compiti ben precisi;
	\item \textbf{Riusabilità:} Le parti del software possono essere riusate in altre applicazioni;
	\item \textbf{Testabilità:} Il software deve essere testabile per consentire la validazione e l'approvazione di modifiche.
\end{itemize}	Le metriche utilizzate per la valutazione, definite nelle Norme di progetto in ??, sono le seguenti:
\begin{itemize}
	\item MPR011 Capacità di analisi failure;
	\item MPR012 Impatto delle modifiche
\end{itemize}

\section{Tabella delle metriche}
\begin{center}
	\begin{tabularx}{\textwidth}{|c|c|c|c|}
		\hline 
		\textbf{ID} & \textbf{Nome} & \textbf{Range di accettazione}  & \textbf{Range di ottimalità}  \\ 
		\hline
	\end{tabularx}
\end{center}