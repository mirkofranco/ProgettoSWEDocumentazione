\chapter{Visione generale della strategia di verifica}
\section{Obbiettivi}
La seguente sezione intende descrivere sia gli obiettivi di qualità relativi al prodotto
richiesto dal committente, che quelli relativi ai processi necessari al suo completamento.
\subsection{Qualità di processo}
Affinché la qualità di prodotto sia garantita è necessario perseguire al qualità dei processi che concorrono alla sua realizzazione. Per raggiungere questo obbiettivo viene adottato lo standard ISO/IEC 15504 denominato SPICE che fornisce strumenti per valutare la bontà dei processi che vengono realizzati.
Per applicare correttamente questo standard è necessario utilizzare il ciclo di Deming (ciclo PCDA) che fornisce una metodologia per il controllo dei processi che consente di migliorarne in modo continuativo la qualità. 
\subsection{Qualità di prodotto}
Al fine di aumentare la possibilità di successo del prodotto in termini di conformità ai requisiti, idoneità all'uso, soddisfazione del committente è necessario fissare alcuni obbiettivi qualitativi e verificare che questi vengano rispettati.
Lo standard ISO/IEC 9126:2001 fornisce gli obbiettivi e le metriche necessarie a misurare il raggiungimento di tali obbiettivi.
\section{Organizzazione}
Ognuna delle fasi del \glossario{progetto} descritte nel \textit{Piano di Progetto v1.0.0} necessità di diverse attività di verifica e approvazione:\\\newline
 \textbf{Analisi:} in questa fase si fa riferimento ai metodi di verifica descritti nella \hyperref[Metodi]{sezione 2.12} sui documenti prodotti dal gruppo.\\ \newline
Concluso il processo di verifica ha inizio il processo di approvazione.Durante tale processo sarà compito del \textit{Responsabile di Progetto} accertarsi che il livello macroscopico dei prodotti sia conforme a quanto pianificato e progettato precedentemente.
\section{Pianificazione strategica e temporale}
Avendo come obbiettivo principale il rispetto della pianificazione descritta nel \texttt{Piano di Progetto v1.0.0}, è necessario che l'attività di verifica di ogni attività sia sistematica e organizzata nel migliore dei modi. 
Rispettando ciò si riesce a individuare e quindi correggere gli eventuali errori rilevati il prima possibile, evitandone una pericolosa proliferazione. 
Le metodologie per l'individuazione e la correzione degli errori sono descritte nelle \texttt{Norme di Progetto v.1.0.0}.
\section{Reponsabilità}
Per garantire che il processo di verifica sia efficace ed efficiente vengono attribuite delle responsabilità a dei specifici ruoli di progetto.
I ruoli che possiedono la responsabilità del processo di verifica sono il \textit{Responsabile} e i \textit{Verificatori}, fermo restando che ogni membro del gruppo è responsabile di quanto prodotto.
\section{Risorse} Per la realizzazione del prodotto software sono necessarie sia risorse tecnologiche che umane.
\begin{itemize}
	\item \textbf{Risorse Umane:} descritte con maggiore dettaglio nel \textit{Piano di Progetto v.1.0.0}
	\subitem	\textit{Responsabile di Progetto;}
	\subitem	\textit{Amministratore di Progetto;}
	\subitem	\textit{Analista}
	\subitem	\textit{Progettista;}
	\subitem	\textit{Programmatore;}
	\subitem	\textit{Verificatore.}
	
	\item \textbf{Risorse Software:} descritte con maggiore dettaglio nelle \textit{Norme di Progetto}; comprendono tutti gli strumenti utilizzati dal gruppo per
	\subitem Gestione della \glossario{repository};
	\subitem Sviluppo del codice;
	\subitem Creazione dei diagrammi in \glossario{UML};
	\subitem Assegnazione di compiti per lo svolgimento delle \glossario{attività};
	\subitem Gestione dei test e dell'analisi del codice;
	\subitem Automatizzazione del processo di verifica;

	\item \textbf{Risorse Hardware}
	Ogni membro del gruppo ha a disposizione un computer personale contenente tutti gli strumenti descritti nel \textit{Piano di Qualifica} e nelle 
	\textit{Norme di Progetto}.
	
\section{Tecniche di analisi}
\subsection{Analisi statica} \label{AnalisiStatica}
L’analisi statica, applicata alla documentazione e al
codice, permette di effettuare la verifica di quanto prodotto individuando errori ed
anomalie; essa può essere svolta in due modi distinti ma complementari.

\begin{itemize}
\item \textbf{Walkthrough:} questa metodologia prevede una revisione ad ampio
spettro parallelizzabile, così da stilare una prima lista degli errori più
comuni. Risulta fondamentale nelle prime fasi di sviluppo del progetto
dal momento che è il più semplice da imparare, di contro però è molto
dispendioso in termini di efficienza;

\item \textbf{Inspection:} risulta essere molto più veloce perché attraverso una lista di controllo degli
errori e delle misurazioni effettuate permette un'analisi più efficace delle
criticità, omettendo le parti che non presentano problematiche.
\end{itemize}
	
\subsection{Analisi dinamica}	L'analisi dinamica al contrario dell'analisi statica è applicabile unicamente
al software e non ai documenti. Il processo prevede la realizzazione e l'esecuzione
di test sul codice. Ogni test per essere valido deve essere ripetibile,
cioè attraverso il medesimo input di partenza si deve poter risalire a un dato
output. A tal fine è necessario che per ogni test vengano segnati i seguenti
riferimenti:

\begin{itemize}
	\item \textbf{Ambiente}: sistemi software e hardware utilizzati nel corso dei
	test;
	\item \textbf{Stato iniziale}: lo stato di partenza del prodotto prima del test;
	\item \textbf{Input};
	\item \textbf{Output};
	\item \textbf{Avvisi}: un eventuale insieme di istruzioni riguardanti l'esecuzione del
test e i suoi risultati.

\end{itemize}

Sono inoltre identificabili 5 tipi diversi di test:
\begin{itemize}
	\item \textbf{Test di unità:} Per unità si intende una piccola quantità di software che è utile verificare singolarmente (spesso sviluppata da un singolo \textit{Programmatore}).
	Testare la corretta esecuzione di queste porzioni di codice permette di evitare possibili errori di implementazione in fase di sviluppo; 
	\item \textbf{Test di integrazione:} Servono a verificare che la combinazione di 2 o più unità
	software funzioni come previsto. \\
	Questo tipo di test serve ad individuare errori residui nella realizzazione dei singoli
	moduli.
	L'esecuzione di tali test prevede l'aggiunta di componenti non ancora sviluppate in modo fittizio, allo scopo di non influenzare in alcuna maniera l'esito delle analisi;
	\item \textbf{Test di sistema:} Comprendono tutti i test di validazione del prodotto software una volta integrati tutti i componenti: l'esecuzione 
	di tali test avviene solamente quando si ritiene che il prodotto abbia raggiunto la sua versione definitiva.\\
	Tali test si occupano inoltre di verificare che la copertura dei requisiti software stabiliti in fase di Analisi
	di Dettaglio sia totale.
	\item \textbf{Test di regressione:} Da effettuare nel caso in cui certe parti del prodotto software vengano modificate nuovamente una volta complete; ciò implica l'esecuzione multipla di test di unità e di integrazione per i moduli interessati.
	\item \textbf{Test di accettazione:}È il test di collaudo del prodotto software che viene eseguito in presenza del proponente.
	Se tale collaudo viene superato positivamente si può procedere al rilascio ufficiale del
	prodotto sviluppato.
\end{itemize}
	
\end{itemize}	
\chapter{Misure e metriche}\label{Metriche}
Il processo di verifica, per essere utile ed informativo, deve essere quantificabile. Le misure effettuate attraverso il processo di verifica devono quindi basarsi su metriche decise a priori.
Nel caso vi fossero metriche approssimate e/o incerte, queste verranno corrette in modo incrementale.
Essendo la natura delle metriche molto variabile, vi sono due tipologie di range:
\begin{itemize}
	\item \textbf{Accettazione}: valori richiesti affinché  il prodotto sia accettato;
	\item \textbf{Ottimale}: valore entro cui dovrebbe collocarsi la misurazione. Tale range non è vincolante, ma fortemente consigliato.
\end{itemize}
\section{Metriche per ila qualità di processo}
Come metrica per i processi si è deciso di utilizzare indici che ci permettano di valutare costi e tempi. 
\subsection{Schedule Variance (SV)}
Indica se si è in linea, in anticipo o in ritardo rispetto alla pianificazione temporale indicata delle attività. 
Se SV > 0 significa che il gruppo sta procedendo con maggiore velocità rispetto a quanto pianificato, viceversa se negativo. 
Essendo stati inseriti slack durante la pianificazione delle attività dei processi, il valore di tale indice è inizialmente positivo.
\begin{itemize}
	\item Range accettazione: >= -(costo preventivo fase x 10\%)
	\item Range ottimale: [>= 0] 
\end{itemize}
\subsection{Budget Variance (BV)}
Indica se alla data corrente corrente si è speso di più o meno rispetto a quanto pianificato.
Se BV >= 0 significa che si sta usando più budget di quanto preventivato, negativo viceversa.
\begin{itemize}
	\item  Range accettazione: >= -( costo preventivo fase x 10\%)
	\item Range ottimale: >= 0
\end{itemize}
\section{Metriche per la qualità di prodotto}
\subsection{Indice di Gulpease} \label{IndicediGulpease}
 L'indice di Gulpease è un indice di leggibilità di un testo tarato sulla lingua italiana. Vengono tenute in considerazione due variabili linguistiche:  la lunghezza della parola e la lunghezza della frase rispetto al numero di lettere.
\[ IG = 89+ \frac{300*nf - 10*nl}{np} \]
Dove:
\begin{itemize}
	\item nf è il numero delle frasi;
	\item nl è il numero delle lettere;
	\item np è il numero delle parole.
\end{itemize}
Il risultato è un numero compreso nel range di valori $0 \le{IG} \le{100}$, dove il valore "100" indica la leggibilità più alta e "0" la leggibilità più bassa.
\begin{itemize}
	\item Range accettazione: [40 - 100]
	\item Range ottimale: [50 - 100]
\end{itemize}
\subsection{Complessità ciclomatica}
Metrica software che indica la complessità di un programma tenendo in considerazione \glossario{moduli}, \glossario{funzioni}, \glossario{metodi} e \glossario{classi}.
Nello specifico, essa è calcolata tramite il grafo di controllo di flusso del programma, dove i nodi sono gruppi indivisibili di istruzioni e gli archi connettono due nodi se il secondo gruppo di istruzioni può essere eseguito immediatamente dopo il primo , e il suo valore è determinato dal numero di cammini linearmente indipendenti all'interno del codice sorgente. 
\'E quindi opportuno definire un valore di complessità ciclomatica preciso: valori alti sono indice di scarsa manutenibilità del codice e valori bassi potrebbero indicare scarsa efficienza dei metodi.
Esso fornisce, inoltre, un indice del carico di lavoro richiesto per la fase di testing (un valore alto richiede più test per una copertura completa).
Il range di ottimalità stabilito varia da 0 a 10, come suggerito dall'ideatore della materica Thomas J. McCabe.  

\begin{itemize}
	\item \textbf{Range accettazione:} [1 - 15];
	\item \textbf{Range ottimale:} [1 - 10].
\end{itemize}

\subsection{Numero di parametri per metodo}
Definire un range relativo al numero di parametri permette di individuare possibili errori nella progettazione (nel caso in cui un metodo abbia un numero di parametri eccessivo).
 
\begin{itemize}
	\item \textbf{Range accettazione:} [0 - 8];
	\item \textbf{Range ottimale:} [0 - 4].
\end{itemize}  


\subsection{Numero di livelli di annidamento}
Metrica per indicare il numero di chiamate annidate di procedure controllate all'interno dei metodi.\newline
Un valore elevato è indice di un basso livello di astrazione del codice e una complessità eccessivamente elevata. 

\begin{itemize}
	\item \textbf{Range accettazione:} [1 - 6];
	\item \textbf{Range ottimale:} [1 - 3].
\end{itemize}  


\subsection{Attributi per classe}
Un numero elevato di attributi in una classe potrebbe essere indice di un errore di progettazione.
Viene quindi definita una metrica che identifichi range accettabili e ottimali per questo parametro.
Nel caso in cui una classe abbia un numero eccessivo di parametri, valutare la possibilità di suddividere la stessa in più classi, suddividendo quindi le funzioni ad essa assegnate.

\begin{itemize}
	\item \textbf{Range accettazione:} [0 - 16];
	\item \textbf{Range ottimale:} [3 - 8].
\end{itemize}  

\subsection{Linee di codice per commento}
Metrica identificata dal rapporto tra linee di codice e linee di commento: risulta utile per garantire una maggiore manutenibilità del codice.

\begin{itemize}
	\item \textbf{Range accettazione:} [>=0.25];
	\item \textbf{Range ottimale:} [>=0.30]\footnote{ricavato dal rapporto 22/78 \url{https://www.ohloh.net/p/firefox/factoids\#FactoidCommentsLow}}.
\end{itemize}  

\subsection{Flusso di informazioni}
Metrica proposta da S. Henry e D. Kafura che misura il flusso di informazioni così definito:

\begin{itemize}
	\item \textbf{fan-in:} numero di moduli che passano informazioni dentro al modulo in esame;
	\item \textbf{fan-out:}numero di moduli a cui il modulo in esame passa informazioni.
\end{itemize} 
il valore viene calcolato tramite questa funzione:
\begin{center}
(lunghezzafunzione)$^2\times fan-in\times fan-out$
\end{center}


\subsection{Accoppiamento}
La definizione dei range di ottimalità e accettazione delle metriche presenti in questa sezione viene rimandata alla fase di progettazione nel dettaglio 
\subsubsection{Accoppiamento Afferente}
Indica la dipendenza di classi esterne a un package rispetto alle classi interne contenute nel package stesso.
Tale valore, sebbene non indicante necessariamente errori di progettazione, può evidenziare criticità riguardanti la robustezza e l'utilità del package a
cui fa riferimento.
Un valore basso può evidenziare scarsa utilità del package, mentre un valore alto necessita di ulteriori verifiche di robustezza, in quanto rappresentante un punto critico nel software.
\subsubsection{Accoppiamento Efferente}
Indica la dipendenza di classi interne a un package rispetto alle classi esterne ad esso.
Un valore basso indica un forte indipendenza del package rispetto al resto del sistema.

\subsection{Copertura del codice}
Indica la percentuale di istruzioni che sono eseguite durante i test.
Maggiore è la percentuale di istruzioni coperte dai test eseguiti, maggiore sarà la probabilità che le componenti testate abbiano una ridotta quantità di errori.
Il valore di tale indice può essere abbassato da metodi molto semplici che non richiedono
testing. Esempi di questi metodi sono: get e set.
Parametri utilizzati:

\begin{itemize}
	
	\item \textbf{Range accettazione:} [42\%-100\%];
	\item \textbf{Range ottimale:} [65\%-100\%].
	
\end{itemize}

\subsection{Riassunto metriche}
\begin{center}
\begin{tabularx}{\textwidth}{|X|X|X|X|}

	\hline
	\textbf{Codice} & \textbf{Nome} & 	\textbf{Range di accettazione} & \textbf{Range di ottimalità}\\
	\endhead
	\hline
			MP001 & Schedule Variance& >= -(c.p.f \footnote{costo preventivo fase} x 10\%)& >=0  \\
	\hline
	MP002 & Budget Variance & >= -(c.p.f x 10\%)& >=0 \\
	\hline
	MP003 & Indice di Gulpease & 40-100 & 50-100\\
	\hline
	MPR001 & Complessità ciclomatica & 1-15 & 1-10 \\
	\hline
	MPR002 & Numero di parametri per metodo & 0-8 & 0-4 \\
	\hline
	MPR003 & Numero di livelli di annidamento & 1-6 & 1-3 \\
	\hline
	MPR004 & Attributi per classe & 0-16 & 3-8 \\
	\hline
	MPR005 & Linee di codice per commento & >=0.25 & >=0.30 \\
	\hline
	MPR006 & Flusso di informazioni & - & - \\
	\hline
	MPR007 & Accoppiamento Afferente & N.D & N.D. \\
	\hline
	MPR008 & Accoppiamento Efferente& N.D & N.D. \\
	\hline
	MPR009 & Copertura del codice & 42\%-100\% & 65\%-100\% \\
	\hline
	\caption{Tabella delle metriche}
\end{tabularx}
\end{center}

\section{Metodi}\label{Metodi}
Questa sezione ha lo scopo di definire precisamente le procedure di analisi statica e dinamica del codice.
\subsection{Analisi dei processi}
Processo che si identifica in queste fasi:
\begin{enumerate}
	\item \textbf{Analisi delle metriche:} Alla conclusione di ogni fase del progetto, per ogni macro-attività, definita nel
	\textit{Piano di Progetto} , si calcolano gli indici definiti nella \hyperref[Metriche]{Sezione 2.8}. Al fine
	di avere un indice complessivo di fase dovrà essere inoltre calcolato il valore medio
	di tali indici.
	\item \textbf{Analisi del grafico PDCA:} Con l'utilizzo dei dati forniti in output dal grafico PDCA, si possono trarre considerazioni sullo svolgimento generale dei processi.
	I dati forniti sono visuali e quindi non numericamente identificabili, tuttavia risultano utili per evidenziare errori di pianificazione o criticità presenti nello svolgimento dei processi.
\end{enumerate}
\subsection{Analisi dei documenti}
Viene definito un protocollo per la verifica dei documenti.
\begin{enumerate}
	\item \textbf{Controllo e sintattico del periodo:} Utilizzando \glossario{TexStudio} è possibile evidenziare gran parte degli errori grammaticali presenti in un documento, tuttavia è comunque necessario un \glossario{walkthrough} da parte di un \glossario{verificatore} per l'individuazione di errori aggiuntivi sfuggiti al controllo automatico;
	\item \textbf{Rispetto delle norme di progetto:} Nelle \textit{Norme di Progetto} sono definite alcune direttive tipografiche e relative alla struttura dei documenti.
	I \glossario{verificatori} dovranno effettuare una \glossario{inspection} del documento per individuare errori di questo tipo.
	\item \textbf{Lista di controllo} il \textit{Verificatore}, dovrà utilizzare la lista di controllo per i documenti, e verificare che gli errori più frequenti non siano presenti;
	\item \textbf{Verifica delle proprietà di glossario:}Seguendo le \textit{Norme di Progetto}, è possibile utilizzare uno script automatico per l'individuazione di tutte le occorrenze di una parola del glossario in un documento.\\
	Per ogni occorrenza viene applicato lo stile  predefinito dal comando \textbackslash glossario.
	Il \glossario{Verificatore} dovrà quindi effettuare un ulteriore controllo allo scopo di constatare che l’output finale sia conforme a quanto precisato nelle \textit{Norme di Progetto};
	\item \textbf{Calcolo dell'indice di Gulpease:} Utilizzando gli strumenti automatici definiti nelle \textit{Norme di Progetto},su ogni documento redatto il \\ \glossario{Verificatore} deve calcolare l’indice di leggibilità.
	Si renderà necessario un ulteriore \glossario{walkthrough} nel caso in cui l’indice risultasse troppo basso, allo scopo di snellire e accorciare frasi troppo lunghe e complesse.
	\item \textbf{Miglioramento Processo} Per avere un miglioramento del processo di verifica, quando i Verificatori eseguono
	walkthrough di un documento, dovranno riportare gli errori più frequentemente
	trovati. Grazie a tale pratica sarà possibile eseguire inspection su tali errori nelle
	verifiche future.
	
\end{enumerate}


\chapter{Gestione amministrativa della revisione}
\section{Comunicazione e risoluzione di anomalie}

Nel caso in cui un \textit{Verificatore} riscontri un'anomalia intesa come:

\begin{itemize}
	\item Violazione delle metriche disposte nel \hyperref[Metriche]{capitolo 3}
	\item Incongruenza del prodotto con quanto specificato nell'\textit{Analisi dei Requisiti v 1.0.0}
	\item Violazione delle norme tipografiche descritte nell \textit{Norme di Progetto}
\end{itemize}

egli potrà aprire una \glossario{task} su \glossario{Asana} per la sua risoluzione, seguendole procedure descritte nelle \textit{Norme di Progetto v 1.0.0}.


