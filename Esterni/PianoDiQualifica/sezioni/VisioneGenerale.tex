\chapter{Visione generale della strategia di verifica}
\section{Obbiettivi}
\subsection{Qualità di processo}
Affinché la qualità di prodotto sia garantita è necessario perseguire al qualità dei processi che concorrono alla sua realizzazione. Per raggiungere questo obbiettivo viene adottato lo standard ISO/IEC 15504 denominato SPICE che fornisce strumenti per valutare la bontà dei processi che vengono realizzati.
Per applicare correttamente questo standard è necessario utilizzare il ciclo di Deming (ciclo PCDA) che fornisce una metodologia per il controllo dei processi che consente di migliorarne in modo continuativo la qualità. 
\subsection{Qualità di prodotto}
Al fine di aumentare la possibilità di successo del prodotto in termini di conformità ai requisiti, idoneità all'uso, soddisfazione del committente è necessario fissare alcuni obbiettivi qualitativi e verificare che questi vengano rispettati.
Lo standard ISO/IEC 9126:2001 fornisce gli obbiettivi e le metriche necessarie a misurare il raggiungimento di tali obbiettivi.
\section{Procedure di controllo di qualità dei processi}
La qualità dei processi viene garantita dall'applicazione del ciclo PDCA. Grazie a tale principio è possibile ottenere un miglioramento continuo della qualità di ogni processo e come diretta conseguenza si ottiene il miglioramento dei prodotti ottenuti.
Per avere controllo dei processi, e di conseguenza migliorarne la qualità, bisogna: 
\begin{itemize}
	\item Definire in modo dettagliato i processi;
	\item Assegnare in modo chiaro le risorse ai processi;
	\item Avere controllo sui processi.
\end{itemize} 
La realizzazione di questi punti è descritta in modo dettagliato nel \textit{Piano di Progetto v1.0.0}.
\section{Procedure di controllo di qualità di prodotto}
Il controllo di qualità del prodotto è garantito: 
\begin{itemize}
	\item \textbf{Quality Assurance}:insieme di attività realizzate per garantire il raggiungimento degli obbiettivi di qualità;
	\item \textbf{Verifica}: processo che determina se l'output di un'attività è consistente, coretto e completo. La verifica viene effettuata lungo tutta la durata del progetto. Questo permette di arrivare pronti al collaudo finale;
	\item \textbf{Validazione}: conferma in modo oggettivo e formale che il prodotto corrisponda ai requisiti.
\end{itemize}
\section{Organizzazione}
\section{Pianificazione strategica e temporale}
Avendo come obbiettivo principale il rispetto della pianificazione descritta nel \texttt{Piano di Progetto v1.0.0}, è necessario che l'attività di verifica di ogni attività sia sistematica e organizzata nel migliore dei modi. 
Rispettando ciò si riesce a individuare e quindi correggere gli eventuali errori rilevati il prima possibile, evitandone una pericolosa proliferazione. 
Le metodologie per l'individuazione e la correzione degli errori sono descritte nelle \texttt{Norme di Progetto v.1.0.0}.
\section{Reponsabilità}
Per garantire che il processo di verifica sia efficace ed efficiente vengono attribuite delle responsabilità a dei specifici ruoli di progetto.
I ruoli che possiedono la responsabilità del processo di verifica sono il \textit{Responsabile} e i \textit{Verificatori}, fermo restando che ogni membro del gruppo è responsabile di quanto prodotto.
\section{Risorse}
\section{Tecniche di Analisi}