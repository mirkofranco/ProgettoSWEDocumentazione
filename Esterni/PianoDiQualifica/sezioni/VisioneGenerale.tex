\chapter{Visione generale della strategia di verifica}
\section{Obbiettivi}
\subsection{Qualità di processo}
Affinché la qualità di prodotto sia garantita è necessario perseguire al qualità dei processi che concorrono alla sua realizzazione. Per raggiungere questo obbiettivo viene adottato lo standard ISO/IEC 15504 denominato SPICE che fornisce strumenti per valutare la bontà dei processi che vengono realizzati.
Per applicare correttamente questo standard è necessario utilizzare il ciclo di Deming (ciclo PCDA) che fornisce una metodologia per il controllo dei processi che consente di migliorarne in modo continuativo la qualità. 
\subsection{Qualità di prodotto}
Al fine di aumentare la possibilità di successo del prodotto in termini di conformità ai requisiti, idoneità all'uso, soddisfazione del committente è necessario fissare alcuni obbiettivi qualitativi e verificare che questi vengano rispettati.
Lo standard ISO/IEC 9126:2001 fornisce gli obbiettivi e le metriche necessarie a misurare il raggiungimento di tali obbiettivi.
\section{Procedure di controllo di qualità dei processi}
La qualità dei processi viene garantita dall'applicazione del ciclo PDCA. Grazie a tale principio è possibile ottenere un miglioramento continuo della qualità di ogni processo e come diretta conseguenza si ottiene il miglioramento dei prodotti ottenuti.
Per avere controllo dei processi, e di conseguenza migliorarne la qualità, bisogna: 
\begin{itemize}
	\item Definire in modo dettagliato i processi;
	\item Assegnare in modo chiaro le risorse ai processi;
	\item Avere controllo sui processi.
\end{itemize} 
La realizzazione di questi punti è descritta in modo dettagliato nel \textit{Piano di Progetto v1.0.0}.
\section{Procedure di controllo di qualità di prodotto}
Il controllo di qualità del prodotto è garantito: 
\begin{itemize}
	\item \textbf{Quality Assurance}:insieme di attività realizzate per garantire il raggiungimento degli obbiettivi di qualità;
	\item \textbf{Verifica}: processo che determina se l'output di un'attività è consistente, coretto e completo. La verifica viene effettuata lungo tutta la durata del progetto. Questo permette di arrivare pronti al collaudo finale;
	\item \textbf{Validazione}: conferma in modo oggettivo e formale che il prodotto corrisponda ai requisiti.
\end{itemize}
\section{Organizzazione}
\section{Pianificazione strategica e temporale}
Avendo come obbiettivo principale il rispetto della pianificazione descritta nel \texttt{Piano di Progetto v1.0.0}, è necessario che l'attività di verifica di ogni attività sia sistematica e organizzata nel migliore dei modi. 
Rispettando ciò si riesce a individuare e quindi correggere gli eventuali errori rilevati il prima possibile, evitandone una pericolosa proliferazione. 
Le metodologie per l'individuazione e la correzione degli errori sono descritte nelle \texttt{Norme di Progetto v.1.0.0}.
\section{Reponsabilità}
Per garantire che il processo di verifica sia efficace ed efficiente vengono attribuite delle responsabilità a dei specifici ruoli di progetto.
I ruoli che possiedono la responsabilità del processo di verifica sono il \textit{Responsabile} e i \textit{Verificatori}, fermo restando che ogni membro del gruppo è responsabile di quanto prodotto.
\section{Risorse}
\section{Msiure e metriche}
Il processo di verifica, per essere utile ed informativo, deve essere quantificabile. Le misure effettuate attraverso il processo di verifica devono quindi basarsi su metriche decise a priori.
Nel caso vi fossero metriche approssimate e/o incerte, queste verranno corrette in modo incrementale.
Essendo la natura delle metriche molto variabile, vi sono due tipologie di range:
\begin{itemize}
	\item \textbf{Accettazione}: valori richiesti affinchè il prodotto sia accettato;
	\item \textbf{Ottimale}: valore entro cui dovrebbe collocarsi la misurazione. Tale range non è vincolante, ma fortemente consigliato.
\end{itemize}
\section{Metriche per ila qualità di processo}
Come metrica per i processi si è deciso di utilizzare indici che ci permettano di valutare costi e tempi. 
\subsection{Schedule Variance (SV)}
Indica se si è in linea, in anticipo o in ritardo rispetto alla pianificazione temporale indicata delle attività. 
Se SV > 0 significa che il gruppo sta procedendo con maggiore velocità rispetto a quanto pianificato, viceversa se negativo. 
Essendo stati inseriti slack durante la pianificazione delle attività dei processi, il valore di tale indice è inizialmente positivo.
\begin{itemize}
	\item Range accettazione: >= -(costo preventivo fase x 10\%)
	\item Range ottimale: [>= 0] 
\end{itemize}
\subsection{Budget Variance (BV)}
Indica se alla data corrente corrente si è speso di più o meno rispetto a quanto pianificato.
Se BV >= 0 significa che si sta usando più budget di quanto preventivato, negativo viceversa.
\begin{itemize}
	\item  Range accettazione: >= -( costo preventivo fase x 10\%)
	\item Range ottimale: >= 0
\end{itemize}
\section{Metriche per la qualità di prodotto}
\subsection{Indice di Gulpease}
 L'indice di Gulpease è un indice di leggibilità di un testo tarato sulla lingua italiana. Vengono tenute in considerazione due variabili linguistiche:  la lunghezza della parola e la lunghezza della frase rispetto al numero di lettere.
\[ IG = 89+ \frac{300*nf - 10*nl}{np} \]
Dove:
\begin{itemize}
	\item nf è il numero delle frasi;
	\item nl è il numero delle lettere;
	\item np è il numero delle parole.
\end{itemize}
Il risultato è un numero compreso nel range di valori $0 \le{IG} \le{100}$, dove il valore "100" indica la leggibilità più alta e "0" la leggibilità più bassa.
\begin{itemize}
	\item Range accettazione: [40 - 100]
	\item Range ottimale: [50 - 100]
\end{itemize}
\subsection{Complessità ciclomatica}
Metrica software che indica la complessità di un programma tenedo in considerazione \glossario{moduli}, \glossario{funzioni}, \glossario{metodi} e \glossario{classi}.
Nello specifico, essa è calcolata tramite il grafo di controllo di flusso del programma, dove i nodi sono gruppi indivisibili di istruzioni e gli archi connettono due nodi se il secondo gruppo di istruzioni può essere eseguito immediatamente dopo il primo , e il suo valore è determinato dal numero di cammini linearmente indipendenti all'interno del codice sorgente. 
\'E quindi opportuno definire un valore di complessità ciclomatica preciso: valori alti sono indice di scarsa manutenibilità del codice e valori bassi potrebbero indicare scarsa efficienza dei metodi.
Esso fornisce, inoltre, un indice del carico di lavoro richiesto per la fase di testing (un valore alto richiede più test per una copertura completa).
Il range di ottimalità stabilito varia da 0 a 10, come suggerito dall'ideatore della materica Thomas J. McCabe.  


