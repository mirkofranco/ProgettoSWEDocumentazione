\chapter{Qualità di Prodotto}
\section{Obbiettivi}


\section{Qualità dei documenti}
I documenti prodotti dal gruppo ZeroSeven devono essere leggibili e corretti. Vengono utilizzate le seguenti metriche, definite nelle \textit{Norme di Progetto}:
\begin{itemize}
    \item \textbf{MD001 Indice di Gulpease};
    \item \textbf{MD002 Ortografia};
    \item \textbf{MD003 Formula di Flesch:} per i Dialog Flow descritti nell' \textit{Analisi dei Requisiti}.
\end{itemize}

\section{Qualità del software}
Segue una descrizione dettagliata delle metriche stabilite dal gruppo per perseguire obbiettivi di qualità software. \newline
Esse costituiscono una dichiarazione d'intenti, e potrà subire modifiche nelle revisioni successive.
\subsection{Complessità ciclomatica}
Metrica software che indica la complessità di un programma tenedo in considerazione \glossario{moduli}, \glossario{funzioni}, \glossario{metodi} e \glossario{classi}.
Nello specifico, essa è calcolata tramite il grafo di controllo di flusso del programma, dove i nodi sono gruppi indivisibili di istruzioni e gli archi connettono due nodi se il secondo gruppo di istruzioni può essere eseguito immediatamente dopo il primo , e il suo valore è determinato dal numero di cammini linearmente indipendenti all'interno del codice sorgente. 
\'E quindi opportuno definire un valore di complessità ciclomatica preciso: valori alti sono indice di scarsa manutenibilità del codice e valori bassi potrebbero indicare scarsa efficienza dei metodi.
Esso fornisce, inoltre, un indice del carico di lavoro richiesto per la fase di testing (un valore alto richiede più test per una copertura completa).
Il range di ottimalità stabilito varia da 0 a 10, come suggerito dall'ideatore della materica Thomas J. McCabe.  





\section{Tabella delle metriche}
\begin{center}
	\begin{tabularx}{\textwidth}{|c|X|c|c|}
		\hline
		\textbf{Codice} & \textbf{Nome} & \textbf{Range di accettazione} & \textbf{Range di ottimalità} \\
		\hline
		MD001 & Indice di Gulpease & 50-100 &60-100\\
		MD002 &Ortografia & 100-100 & 100-100\\
		MD003 &Formula di Flesch& 75-100 &80-100 \\
		\hline
		MS001 & Complessità ciclomatica & 0-16 & 0-10 \\
		
		
	\end{tabularx}
\end{center}