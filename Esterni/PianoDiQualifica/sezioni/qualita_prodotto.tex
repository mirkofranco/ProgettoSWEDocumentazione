\chapter{Qualità di Prodotto}
\section{Scopo}
Per garantire e misurare la qualità dei prodotti sviluppati, il gruppo ha scelto di seguire una parte delle linee guida fornite dallo standard ISO/IEC 25010.
\section{Qualità dei documenti}
I documenti prodotti dal gruppo ZeroSeven devono essere leggibili e corretti. Vengono utilizzate le seguenti metriche, definite nelle \textit{Norme di Progetto}:
\begin{itemize}
    \item \textbf{MD001 Indice di Gulpease};
    \item \textbf{MD002 Ortografia};
    \item \textbf{MD003 Formula di Flesch:} per i Dialog Flow descritti nell' \textit{Analisi dei Requisiti}.
\end{itemize}
\section{Qualità del software}
Segue un elenco dettagliato delle metriche stabilithe dal gruppo per il controllo della qualità del software.

\subsection{Funzionalità}
\begin{itemize}
	\item \textbf{MS001 Efficacia funzionale}
	\item \textbf{MS002 Correttezza}
\end{itemize}
\subsection{Usabilità}
todo
\subsection{Affidabilità}
todo
\subsection{Sicurezza}
\begin{itemize}
	\item \textbf{MS003 Confidenzialità:} percentuale per la quale l'applicazione assicura che i dati siano accessibili solo da utenti autorizzati. E' importante notare che l'\textit{Alexa} di Amazon è uno speaker, pertanto l'output prodotto è accessibile a chiunque nella stanza.
	\item \textbf{MS004 Integrità:} percentuale per la quale l'applicazione previene accessi o modifiche non autorizzati.
	\item \textbf{MS005 Autenticità:} percentuale per la quale l'identità dichiarata di un utente può essere provata.
\end{itemize}
\subsection{Manutenibilità}
Con manutenibilità intendiamo la capacità del prodotto di essere modificato tramite correzioni, miglioramenti e adattamenti.
Nello specifico il software deve avere le seguenti caratteristiche:
\begin{itemize}
	\item \textbf{MM001 Analizzabilità:} Il software deve poter essere analizzato per poter trovare gli errori.
	\item \textbf{MM002 Modificabilità:} Il prodotto deve permettere la modifica delle sue parti.
	\item \textbf{MM003 Modularità:} Il prodotto è diviso in parti che svolgono compiti ben precisi.
	\item \textbf{MM004 Riusabilità:} Le parti del software possono essere riusate in altre applicazioni.
	\item \textbf{MM005 Testabilità:} Il software deve essere testabile per consentire la validazione e l'approvazione di modifiche.
\end{itemize}
\section{Tabella delle metriche}
Nella seguente tabella vengono riportati gli indici con i relativi range di accettazione.\\
\begin{table}[h]
    \begin{center}
      \begin{tabular}{|c|c|c|}
        \hline
        \textbf{ID} & \textbf{Indice}       & \textbf{Range di accettazione}\\
        \hline
        MD001       & Indice di Gulpease    & 60-100\\
        MD002       & Ortografia            & 100-100\\
        MD003       & Formula di Flesch     & 80-100\\\hline
        MS001       & Efficacia funzionale  & 100-100\\
        MS002       & Correttezza  			& 80-100\\
        MS003       & Confidenzialità 	 	& 60-100\\
        MS004       & Integrità  			& 80-100\\
        MS005       & Autenticità		    & 80-100\\\hline
        MM001       & Analizzabilità	    & 80-100\\
        MM002       & Modificabilità		& 80-100\\
        MM003       & Modularità		    & 80-100\\
        MM004       & Riusabilità			& 70-100\\
		MM005       & Testabilità			& 90-100\\
        \hline
      \end{tabular}
      \caption{Tabella metriche qualità di prodotto}
    \end{center}
\end{table}