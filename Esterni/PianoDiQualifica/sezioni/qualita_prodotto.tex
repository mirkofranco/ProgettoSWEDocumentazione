\chapter{Qualità di Prodotto}
\section{Scopo}
Per garantire e misurare la qualità dei prodotti sviluppati, il gruppo ha scelto di seguire una parte delle linee guida fornite dallo standard ISO/IEC 25010.
\section{Qualità dei documenti}
I documenti prodotti dal gruppo ZeroSeven devono essere leggibili e corretti. Vengono utilizzate le seguenti metriche, definite nelle \textit{Norme di Progetto}:
\begin{itemize}
    \item \textbf{MD001 Indice di Gulpease};
    \item \textbf{MD002 Ortografia};
    \item \textbf{MD003 Formula di Flesch:} per i Dialog Flow descritti nell' \textit{Analisi dei Requisiti}.
\end{itemize}
\section{Qualità del software}
Segue un elenco dettagliato delle metriche stabilithe dal gruppo per il controllo della qualità del software.

\subsection{Funzionalità}
\begin{itemize}
	\item \textbf{MS001 Efficacia funzionale}
	\item \textbf{MS002 Correttezza}
\end{itemize}
\subsection{Usabilità}
todo
\subsection{Affidabilità}
Caratteristica che rappresenta la capacità del prodotto software di svolgere correttamente il suo compito, mantenendo delle buone prestazioni al verificarsi di situazioni anomale. 
Il prodotto software dovrà presentare le seguenti caratteristiche:
 \begin{itemize}
	\item \textbf{MS000 Tolleranza agli errori:} Il prodotto software continua a lavorare nel workflow corrente in presenza di errori dovuti a uno scorretto uso dell'applicativo.
	\item \textbf{MS000 Recuperabilità:} Nel caso in cui si presentano errori, dovuti a uno scoretto comando da parte dell'utente,l'echo riesce a recuperare l'esecuzione  del workflow nel punto dove è stato interrotto.
\end{itemize}
\subsection{Sicurezza}
\begin{itemize}
	\item \textbf{MS003 Confidenzialità:} percentuale per la quale l'applicazione assicura che i dati siano accessibili solo da utenti autorizzati. E' importante notare che l'\textit{Alexa} di Amazon è uno speaker, pertanto l'output prodotto è accessibile a chiunque nella stanza.
	\item \textbf{MS004 Integrità:} percentuale per la quale l'applicazione previene accessi o modifiche non autorizzati.
	\item \textbf{MS005 Autenticità:} percentuale per la quale l'identità dichiarata di un utente può essere provata.
\end{itemize}
\subsection{Manutenibilità}
Con manutenibilità intendiamo la capacità del prodotto di essere modificato tramite correzioni, miglioramenti e adattamenti.
Nello specifico il software deve avere le seguenti caratteristiche:
\begin{itemize}
	\item \textbf{MM001 Stabilità:} Il software non deve manifestare comportamenti strani dopo aver effettuato delle modifiche.
	\item \textbf{MM002 Analizzabilità:} Il software deve poter essere analizzato per poter trovare gli errori.
	\item \textbf{MM003 Testabilità:} Il software deve essere testabile per consentire la validazione e l'approvazione di modifiche.
	\item \textbf{MM004 Modificabilità:} Il prodotto deve permettere la modifica delle sue parti.
\end{itemize}
\section{Tabella delle metriche}
Nella seguente tabella vengono riportati gli indici con i relativi range di accettazione.\\
\begin{table}[h]
    \begin{center}
      \begin{tabular}{|c|c|c|}
        \hline
        \textbf{ID} & \textbf{Indice}       & \textbf{Range di accettazione}\\
        \hline
        MD001       & Indice di Gulpease    & 60-100\\
        MD002       & Ortografia            & 100-100\\
        MD003       & Formula di Flesch     & 80-100\\\hline
        MS001       & Efficacia funzionale  & 100-100\\
        MS002       & Correttezza  			& 80-100\\
        MS000       & Tolleranza errori    & 60-100\\
        MS000       & Recuperabilità			& 70-100\\
        MS003       & Confidenzialità 	 	& 60-100\\
        MS004       & Integrità  			& 80-100\\
        MS005       & Autenticità		    & 80-100\\\hline
        MM001       & Stabilità 		    & 80-100\\
        MM002       & Analizzabilità		& 80-100\\
        MM003       & Testabilità		    & 80-100\\
        MM004       & Modificabilità		& 80-100\\
        \hline
      \end{tabular}
      \caption{Tabella metriche qualità di prodotto}
    \end{center}
\end{table}