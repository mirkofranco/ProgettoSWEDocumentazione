\chapter{Qualità di prodotto}
\section{Scopo}
Per garantire e misurare la qualità dei prodotti sviluppati, il gruppo ha scelto di seguire una parte delle linee guida fornite dallo standard ISO/IEC 25010.
\section{Qualità dei documenti}
I documenti prodotti dal gruppo ZeroSeven devono essere leggibili e corretti. Vengono utilizzate le seguenti metriche, definite nelle \textit{Norme di Progetto}:
\begin{itemize}
    \item \textbf{MD001 Indice di Gulpease};
    \item \textbf{MD002 Ortografia};
    \item \textbf{MD003 Formula di Flesch:} per i Dialog Flow descritti nell' \textit{Analisi dei Requisiti}.
\end{itemize}
\section{Qualità del software}
Segue un elenco dettagliato delle metriche stabilithe dal gruppo per il controllo della qualità del software.

\subsection{Funzionalità}
\begin{itemize}
	\item \textbf{MS001 Efficacia funzionale}
	\item \textbf{MS002 Correttezza}
\end{itemize}
\subsection{Usabilità}
todo
\subsection{Affidabilità}
todo
\subsection{Sicurezza}
todo
\subsection{Manutenibilità}
todo
\section{Tabella delle metriche}
Nella seguente tabella vengono riportati gli indici con i relativi range di accettazione.\\
\begin{table}[h]
    \begin{center}
      \begin{tabular}{|c|c|c|}
        \hline
        \textbf{ID} & \textbf{Indice}       & \textbf{Range di accettazione}\\
        \hline
        MD001       & Indice di Gulpease    & 60-100\\
        MD002       & Ortografia            & 100-100\\
        MD003       & Formula di Flesch     & 80-100\\\hline
        MS001       & Efficacia funzionale  & 100-100\\
        MS002       & Correttezza  			& 80-100\\
            
        \hline
      \end{tabular}
      \caption{Tabella metriche qualità di prodotto}
    \end{center}
\end{table}