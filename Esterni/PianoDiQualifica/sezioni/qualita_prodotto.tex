\chapter{Qualità di Prodotto}
\section{Scopo}
Per garantire e misurare la qualità dei prodotti sviluppati, il gruppo ha scelto di seguire una parte delle linee guida fornite dallo standard ISO/IEC 25010.
\section{Qualità dei documenti}
I documenti prodotti dal gruppo ZeroSeven devono essere leggibili e corretti. Vengono utilizzate le seguenti metriche, definite nelle \textit{Norme di Progetto}:
\begin{itemize}
    \item \textbf{MD001 Indice di Gulpease};
    \item \textbf{MD002 Ortografia};
    \item \textbf{MD003 Formula di Flesch:} per i Dialog Flow descritti nell' \textit{Analisi dei Requisiti}.
\end{itemize}
\section{Qualità del software}
Segue un elenco dettagliato delle metriche stabilithe dal gruppo per il controllo della qualità del software.

\subsection{Funzionalità}
\subsection{Fuzionalità}
\begin{itemize}
	\item \textbf{MS001 - Efficacia funzionale:} Indice che determina il grado di copertura di bisogni e obiettivi richiesti dall'utente. Il range stabilito varia da 0 a 100;
	\item \textbf{MS002 - Correttezza:} Indice che determina la precisione e la correttezza degli output forniti dal software. Il range stabilito varia da 0 a 100
	\end{itemize}

\subsection{Usabilità}
l'usabilità definisce il grado di facilità e soddisfazione con cui si compie l'interazione tra l'uomo e il prodotto, ovvero l'efficacia, l'efficienza e la soddisfazione con le quali determinati utenti raggiungono determinati obiettivi in determinati contesti.
\begin{itemize}
	\item \textbf{Curva d'apprendimento:} Indica il rapporto tra tempo necessario per l'apprendimento e quantità di informazioni correttamente apprese.
	La curva d'apprendimento sarà differente in base al dispositivo considerato in quanto l'applicazione mobile avrà funzionalità più complesse rispetto all'echo che dovrà essere utilizzato da un qualsiasi utente;
	Dunque verranno suddivisi i requisiti in base al dispositivo:
	\begin{itemize}
		\item \textbf{MS003 Curva d'apprendimento App};
		\item \textbf{MS004 Curva d'apprendimento Alexa}.
	\end{itemize}
	\item \textbf{MS005 Protezione dall'errore:} Rappresenta il grado in cui il prodotto protegge l'utente dal commettere errori.
	\item \textbf{MS006 Estetica dell'interfaccia utente:} Rappresenta la capacità dell'interfaccia di evitare smarrimento.
	L'interfaccia dell'echo ha un'usabilità differente  in quanto si parla di interfaccia vocale, dunque soggetta a diversi tempi di reattività e apprendimento rispetto ad un'interfaccia mobile.
	\item \textbf{MS007 Accessibilità:}  Si intende la possibilità di fornire i servizi anche a coloro che sono affetti da disabilità temporanee e non, che quindi utilizzano tecnologie ausiliarie.
	Nel caso dell'echo alcune disabilità sono per ora vincolanti in quanto presuppongono necessariamente l'utilizzo della voce.
\end{itemize}
\subsection{Affidabilità}
Caratteristica che rappresenta la capacità del prodotto software di svolgere correttamente il suo compito, mantenendo delle buone prestazioni al verificarsi di situazioni anomale. 
Il prodotto software dovrà presentare le seguenti caratteristiche:
 \begin{itemize}
	\item \textbf{MS008 Tolleranza agli errori:} Il prodotto software continua a lavorare nel workflow corrente in presenza di errori dovuti a uno scorretto uso dell'applicativo.
	\item \textbf{MS009 Recuperabilità:} Nel caso in cui si presentano errori, dovuti a uno scoretto comando da parte dell'utente,l'echo riesce a recuperare l'esecuzione  del workflow nel punto dove è stato interrotto.
\end{itemize}
\subsection{Sicurezza}
\begin{itemize}
	\item \textbf{MS010 Confidenzialità:} percentuale per la quale l'applicazione assicura che i dati siano accessibili solo da utenti autorizzati. E' importante notare che l'\textit{Alexa} di Amazon è uno speaker, pertanto l'output prodotto è accessibile a chiunque nella stanza.
	\item \textbf{MS011 Integrità:} percentuale per la quale l'applicazione previene accessi o modifiche non autorizzati.
	\item \textbf{MS012 Autenticità:} percentuale per la quale l'identità dichiarata di un utente può essere provata.
\end{itemize}
\subsection{Manutenibilità}
Con manutenibilità intendiamo la capacità del prodotto di essere modificato tramite correzioni, miglioramenti e adattamenti.
Nello specifico il software deve avere le seguenti caratteristiche:
\begin{itemize}
	\item \textbf{MM001 Analizzabilità:} Il software deve poter essere analizzato per poter trovare gli errori.
	\item \textbf{MM002 Modificabilità:} Il prodotto deve permettere la modifica delle sue parti.
	\item \textbf{MM003 Modularità:} Il prodotto è diviso in parti che svolgono compiti ben precisi.
	\item \textbf{MM004 Riusabilità:} Le parti del software possono essere riusate in altre applicazioni.
	\item \textbf{MM005 Testabilità:} Il software deve essere testabile per consentire la validazione e l'approvazione di modifiche.
\end{itemize}
\section{Tabella delle metriche}
Nella seguente tabella vengono riportati gli indici con i relativi range di accettazione.\\
\begin{table}[h]
    \begin{center}
      \begin{tabular}{|c|c|c|}
        \hline
        \textbf{ID} & \textbf{Indice}       & \textbf{Range di accettazione}\\
        \hline
        MD001       & Indice di Gulpease    & 60-100\\
        MD002       & Ortografia            & 100-100\\
        MD003       & Formula di Flesch     & 80-100\\\hline
        MS001       & Efficacia funzionale  & 100-100\\
        MS002       & Correttezza  			& 80-100\\
        MS003       & Curva d'apprendimento App & 70-100\\
        MS004       & Curva d'apprendimento Alexa   & 90-100\\
        MS005       & Protezione dall'errore  & 80-100\\
        MS006       & Estetica dell'interfaccia utente  & 80-100\\
        MS007       & Accessibilità			  & 60-100\\
        MS008       & Tolleranza errori    & 60-100\\
        MS009       & Recuperabilità			& 70-100\\
        MS010       & Confidenzialità 	 	& 60-100\\
        MS011       & Integrità  			& 80-100\\
        MS012       & Autenticità		    & 80-100\\\hline
        MM001       & Analizzabilità	    & 80-100\\
        MM002       & Modificabilità		& 80-100\\
        MM003       & Modularità		    & 80-100\\
        MM004       & Riusabilità			& 70-100\\
		MM005       & Testabilità			& 90-100\\
        \hline
      \end{tabular}
      \caption{Tabella metriche qualità di prodotto}
    \end{center}
\end{table}