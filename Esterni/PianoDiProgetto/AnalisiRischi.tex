\chapter{Analisi dei rischi}
Per aumentare la probabilità di una buona riuscita del progetto viene effettuata un'approfondita analisi dei rischi. Per poter gestire i rischi viene applicata la seguente procedura:
\begin{itemize}
		\item \textbf{Identificazione:} identificare i potenziali rischi che possono minare l'avanzamento del progetto e capirne la cause. I rischi possono afferire alle seguenti categorie:
		\begin{itemize}
			\item \textbf{Progetto}: relativi alle risorse, alla pianificazione e agli strumenti;
			\item \textbf{Prodotto}: relativi alla conformità del prodotto a quanto atteso dal committente;
			\item \textbf{Mercato}: relativi a costi.
		\end{itemize}
		\item \textbf{Analisi}: ne viene valutata la probabilità di occorrenza, la pericolosità e le possibili conseguenze;
		\item \textbf{Pianificazione}: vengono attuate strategie che permettono di evitare i rischi o mitigarne gli effetti qualora si presentassero;
		\item \textbf{Controllo}: viene posta attenzione continua tramite la rilevazione di specifici indicatori raffinando le strategie di pianificazione qualora ve ne fosse il bisogno.
\end{itemize}
\section{Elenco dei rischi}
Ogni rischio viene classificato secondo la seguente convenzione:\\\\
\centering \textbf{R[Tipo][Identificativo]}\\
\begin{itemize}
	\item La lettera "R" è l'abbreviazione di Rischio
	\item Il secondo valore indica il tipo di rischio. Può essere:
		\begin{itemize}
			\item \textbf{P}: indica i rischi legati al progetto, quindi a risorse, pianificazione e strumenti;
			\item \textbf{PR}: indica i rischi legati ai requisiti;
			\item \textbf{M}: indica i rischi legati al mercato.
		\end{itemize}
	\item L'identificato è semplicemente un numero progressivo.
\end{itemize}
\begin{center}
\begin{longtable}{|X|X|X|X|}
		\caption{Elenco dei rischi}\\
		\textbf{Codice Nome} & \textbf{Descrizione} & 	\textbf{Rilevamento} & \textbf{Grado di rischio}\\
		\hline
<<<<<<< HEAD
=======
		\textbf{Codice Nome} & \textbf{Descrizione} & 	\textbf{Rilevamento} & \textbf{Grado di rischio}\\
		\hline
>>>>>>> parent of f72e83e... fix tabella
		Scarsa esprienza RP001 & Nessun membro del gruppo ha esperienza in un progetto di tali dimensioni e complessità. &
		Ogni membro comunicherà al Responsabili eventuali difficoltà incontrate. & Occorrenza: Media Pericolosità: Alta \\
		\hline
		\multicolumn{4}{X}{\textbf{Mitigazione}: Il Responsabile pianificherà le attività cercando di valorizzare capacità e competenze delle singole persone}\\
<<<<<<< HEAD
	
%	\end{tabularx}
\end{longtable}
=======
>>>>>>> parent of f72e83e... fix tabella


