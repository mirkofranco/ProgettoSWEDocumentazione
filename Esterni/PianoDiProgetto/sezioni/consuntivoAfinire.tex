\begin{flushleft}
\chapter{Consuntivo a finire}
Il \textbf{Consuntivo a finire} contiene il prospetto economico che riporta le spese effettivamente sostenute. Vengono riportate le ore impiegate per svolgere i compiti preventivati, sia per
ruolo che per persona. Il bilancio è ottenuto facendo la differenza di ore tra il preventivo e il consuntivo.\\
Il bilancio può essere:
\begin{itemize}
    \item \textbf{Positivo:} se il preventivo ha superato il consuntivo;
    \item \textbf{Negativo:} se il consuntivo ha superato il preventivo;
    \item \textbf{In pari:} se consuntivo e preventivo coincidono.
\end{itemize}
\section{bilancio Analisi}
La tabella riporta la differenza delle ore tra preventivo e consuntivo, divise per ruolo. Il segno negativo indica che le ore effettive superano le ore preventivate.    
\begin{table}[h]
    \begin{tabular}{|l|l|l|}
    \hline
    Ruolo          & Ore & Costo \\ \hline
    Responsabile   & -2  & -60   \\ \hline
    Amministratore & 2   & 40    \\ \hline
    Analista       & 1   & 25    \\ \hline
    Progettista    & 0   & 0     \\ \hline
    Verificatore   & -2  & -30   \\ \hline
    Programmatore  & 0   & 0     \\ \hline
    Totale         & 0   & -25   \\ \hline
    \end{tabular}
\end{table}
In questa tabella  sono riportate le differenze tra le ore di lavoro previste per ogni membro del gruppo con quelle realmente impiegate.

\begin{table}[h]
    \begin{tabular}{|l|l|l|l|l|l|l|l|}
    \hline
    \multirow{2}{*}{Nominativo} & \multicolumn{6}{l|}{Ore per ruolo} & \multirow{2}{*}{Ore totali} \\ \cline{2-7}
                                & Re   & Am  & An  & Pt  & Ve  & Pr  &                             \\ \hline
    Gian Marco Bratzu           &      &     &     &     &     &     & 0                           \\ \hline
    Ludovico Brocca             &      &     &     &     & -1  &     & -1                          \\ \hline
    Bianca Ciuche               & -2   &     & 1   &     &     &     & -1                          \\ \hline
    Andrea Deidda               &      & 2   &     &     &     &     & 2                           \\ \hline
    Matteo Depascale            &      &     &     &     & -1  &     & -2                          \\ \hline
    Mirko Franco                &      &     &     &     &     &     & 0                           \\ \hline
    Stefano Zanatta             &      &     &     &     &     &     & 0                           \\ \hline
    \end{tabular}
    \end{table}
\subsection{Conclusioni}
La stesura del \textit{Piano di Progetto} ha richiesto più lavoro del previsto da parte dei \textbf{Responsabili} e dai \textbf{Verificatori}. Il carico di lavoro degli \textbf{Amministratori} e degli \textbf{Analisti} è stato inferiore di quanto preventivato.\\ Il bilancio totale è negativo, con un deficit di 25\euro. Questo costo non è a carico del committente, in quanto è stato generato dalle attività di Analisi.
\end{flushleft}