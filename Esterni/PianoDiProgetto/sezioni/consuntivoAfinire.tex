\begin{flushleft}
    \chapter{Consuntivo di periodo e preventivo a finire}
    Il \textit{Consuntivo di periodo} contiene il prospetto economico che riporta le spese effettivamente sostenute. Vengono riportate le ore impiegate per svolgere i compiti preventivati, sia per
    ruolo che per persona. Il bilancio è ottenuto facendo la differenza di ore tra il preventivo e il consuntivo.\\
    Il bilancio può essere:
    \begin{itemize}
        \item \textbf{Positivo:} se il preventivo ha superato il consuntivo;
        \item \textbf{Negativo:} se il consuntivo ha superato il preventivo;
        \item \textbf{In pari:} se consuntivo e preventivo coincidono.
    \end{itemize}

   \newpage
    \section{Bilancio Analisi}
    La tabella riporta la differenza delle ore tra preventivo e consuntivo, divise per ruolo. Il segno negativo indica che le ore effettive superano le ore preventivate.  
      
	\begin{table}[!h]
		\begin{center}
			\rowcolors{1}{}{lightgray}
			\begin{tabular}{ccc}
				\rowcolor{coolblack}
				\hline
				\textcolor{white}{Ruolo} & \textcolor{white}{Ore} & \textcolor{white}{Costo in \euro}\\
				\hline
				Responsabile   & 22 (-2)  &  660,00 (-60,00) 	\\ 
				Amministratore & 39 (+2)  &  780,00 (+40,00) 	\\ 
				Analista       & 58 (+1)  &  1450,00 (+25,00)   	\\ 
				Progettista    & -  	 &  - 					\\ 
				Verificatore   & 49(-2)  &  735,00 (-30,00) 	\\ 
				Programmatore  & -       &  -    		 		\\ \hline
				\textbf{Totale}& \textbf{168} & \textbf{3625,00}	\\ \hline 
				\textbf{Totale preventivato}& \textbf{167} & \textbf{3600,00}\\ \hline 
				\textbf{Differenza}& \textbf{-1} & \textbf{-25,00 }	\\ \hline  
			\end{tabular}
			\caption{Differenza delle ore tra preventivo e consuntivo divise per ruolo} 
		\end{center}
	\end{table}
\newpage
    In questa tabella  sono riportate le differenze tra le ore di lavoro previste per ogni membro del gruppo con quelle realmente impiegate.
       \begin{table}[!h]
 	\begin{center}
 		\rowcolors{1}{}{lightgray}
 		\begin{tabularx}{\textwidth}{|c|cccccc|c|}
 			
 			\hline
 			\multirow{2}{*}{Nominativo} & \multicolumn{6}{c|}{Ore per ruolo} & \multirow{2}{*}{Ore totali} \\ \cline{2-7}
 			& Re & Am & An & Pt & Ve & Pr &      \\ \hline
 			\endhead
 			Mirko Franco       &   &    &  &    &  &  & 0    \\ \hline
 			Bianca Ciuche      & -2 &  & +1 &    &  &   & -1       \\ \hline
 			Stefano Zanatta    &   &  &  &  &   &   &    0   \\ \hline
 			Andrea Deidda      &   & +2 &   &   &  &   &  +2  		\\ \hline
 			Ludovico Brocca    &   &  &  &  & -1 &   & -1       \\ \hline
 			Matteo Depascale   &   &  &   &   &  -1 &  & -1  		\\ \hline
 			Gian Marco Bratzu  &   &  &   &   &  &   & 0        \\ \hline
 			
 		\end{tabularx}
 		\caption{Differenza tra le ore di lavoro previste per ogni membro del gruppo con le ore realmente impiegate }
 		\end{center}
	 \end{table}
    \subsection{Conclusioni}
    La stesura del \textit{Piano di Progetto} ha richiesto più lavoro del previsto da parte dei \textit{Responsabili} e dei \textit{Verificatori}. Il carico di lavoro degli \textit{Amministratori} e degli \textit{Analisti} è stato inferiore di quanto preventivato.\\ Il bilancio totale è negativo, con un deficit di 25\euro. Questo costo non è a carico del committente, in quanto è stato generato dalle attività di Analisi.
    
    \newpage
    
\section{Bilancio Revisione analisi}\label{BilRevAn}
La tabella riporta la differenza delle ore tra preventivo e consuntivo, divise per ruolo. Il segno negativo indica che le ore effettive superano le ore preventivate.  
  
\begin{table}[!h]
	\begin{center}
		\rowcolors{1}{}{lightgray}
		\begin{tabular}{ccc}
			\rowcolor{coolblack}
			\hline
			\textcolor{white}{Ruolo} & \textcolor{white}{Ore} & \textcolor{white}{Costo in \euro}\\
			\hline
			Responsabile   & 5 (+2)  &  150,00 (+60,00) 	\\ 
			Amministratore & 5 (-3)  &  100,00 (-60,00) 	\\ 
			Analista       & 9 (-5)  &  225,00 (-125,00)   	\\ 
			Progettista    & -  	 &  - 					\\ 
			Verificatore   & 16(+5)  &  240,00 (+75,00) 	\\ 
			Programmatore  & -       &  -    		 		\\ \hline
			\textbf{Totale}& \textbf{37} & \textbf{765,00}	\\ \hline 
			\textbf{Totale preventivato}& \textbf{35} & \textbf{715,00}\\ \hline 
			\textbf{Differenza}& \textbf{-2} & \textbf{-50,00 }	\\ \hline  
		\end{tabular}
		\caption{Differenza delle ore tra preventivo e consuntivo divise per ruolo} 
	\end{center}
\end{table}
  \clearpage
In questa tabella  sono riportate le differenze tra le ore di lavoro previste per ogni membro del gruppo con quelle realmente impiegate.\\

    \begin{table}[!h]
	\begin{center}
		\rowcolors{1}{}{lightgray}
		\begin{tabularx}{\textwidth}{|c|cccccc|c|}
			
			\hline
			\multirow{2}{*}{Nominativo} & \multicolumn{6}{c|}{Ore per ruolo} & \multirow{2}{*}{Ore totali} \\ \cline{2-7}
			& Re & Am & An & Pt & Ve & Pr &      \\ \hline
			\endhead
			Mirko Franco       &  & -2  & &  & +1 &   & -1   \\ \hline
			Bianca Ciuche      & +2 &    & -2 &    &  &   & 0   \\ \hline
			Stefano Zanatta    &   & -1 & -2 &  & +1 &  & -2    \\ \hline
			Andrea Deidda      &   &  &   &   & +1  &   & +1	\\ \hline
			Ludovico Brocca    &   &  & -1 &   & +1 &   & 0     \\ \hline
			Matteo Depascale   &   &  &   &   &   &   & 0  		\\ \hline
			Gian Marco Bratzu  &   &  &   &   &  &   & 0        \\ \hline
			
		\end{tabularx}
		\caption{Differenza tra le ore di lavoro previste per ogni membro del gruppo con le ore realmente impiegate }
	\end{center}
\end{table}

  \subsection{Conclusioni}
  Le correzioni del \textit{Piano di Qualifica} e del documento \textit{Norme di Progetto} ha richiesto più lavoro del previsto agli {Analisti}. Il carico di lavoro degli \textit{Verificatori} è stato inferiore di quanto preventivato.\\ Il bilancio totale è negativo, con un deficit di 50\euro.
  Un'attenta analisi sulle tecnologie ha portato alla decisione di redistribuire le ore preventivate per la progettazione della base tecnologica, in quanto il carico di lavoro risultava squilibrato a favore dei Progettisti e stabiliva poche ore di programmazione.

\newpage	
	\section{Bilancio Progettazione della base tecnologica} 
	\label{BilProgBT}
  La tabella riporta la differenza delle ore tra preventivo e consuntivo, divise per ruolo. Il segno negativo indica che le ore effettive superano le ore preventivate.
  
	\begin{table}[!h]
	\begin{center}
		\rowcolors{1}{}{lightgray}
		\begin{tabular}{ccc}
			\rowcolor{coolblack}
			\hline
			\textcolor{white}{Ruolo} & \textcolor{white}{Ore} & \textcolor{white}{Costo in \euro}\\
			\hline
			Responsabile   & 7   		&  210,00  			 	\\ 
			Amministratore & 5   		&  100,00 			 	\\ 
			Analista       & 15 (+4)	  	&  375,00 (+100)   	\\ 
			Progettista    & 101 (+5)	&  2222,00 (+110,00) 	\\ 
			Verificatore   & 34 (+5)  	&  510,00 (+75,00) 		\\ 
			Programmatore  & 15 (-14)   &  225,00 (-210,00) 	\\ \hline
			\textbf{Totale}& \textbf{177} & \textbf{3567,00}	\\ \hline 
			\textbf{Totale preventivato}& \textbf{177} & \textbf{3642,00}\\ \hline 
			\textbf{Differenza}& \textbf{0} & \textbf{+75,00 }	\\ \hline  
		\end{tabular}
		
		\caption{Differenza delle ore tra preventivo e consuntivo divise per ruolo} 
	\end{center}
\end{table}
  \clearpage
  
  In questa tabella  sono riportate le differenze tra le ore di lavoro previste per ogni membro del gruppo con quelle realmente impiegate.\\
  
    \begin{table}[!h]
  	\begin{center}
  		\rowcolors{1}{}{lightgray}
    \begin{tabularx}{\textwidth}{|c|cccccc|c|}
  			
  	\hline
  	\multirow{2}{*}{Nominativo} & \multicolumn{6}{c|}{Ore per ruolo} & \multirow{2}{*}{Ore totali} \\ \cline{2-7}
  					  & Re & Am & An & Pt & Ve & Pr &      \\ \hline
  	\endhead
  	Mirko Franco       &   &    &    &    &  &   & 0        \\ \hline
  	Bianca Ciuche      &   &    &    &    &  &   & 0        \\ \hline
  	Stefano Zanatta    &   &    &  & +5 &   & -5  & 0        \\ \hline
  	Andrea Deidda      &   &  &   &   &   &   & 0  		\\ \hline
  	Ludovico Brocca    &   &  &   &   &  &   & 0        \\ \hline
  	Matteo Depascale   &   &  &   &   & -5  & +5  & 0  		\\ \hline
  	Gian Marco Bratzu  &   &  &   &   &  &   & 0        \\ \hline
    
   	\end{tabularx}
  	\caption{Differenza tra le ore di lavoro previste per ogni membro del gruppo con le ore realmente impiegate }
\end{center}
\end{table}
  
  \subsection{Conclusioni}
  Durante il periodo di Progettazione della base tecnologica, il numero di ore preventivate si è dimostrato corretto. Tuttavia è stato necessario ridistribuire le ore a ruoli diversi, poiché per il periodo non è stato necessario utilizzare tutte le ore preventivate per la progettazione della base tecnologica, ma si sono rese necessarie più ore per la programmazione di essa. 
  Il risultato complessivo del periodo è di 0 ore lavorative in meno rispetto al previsto e di un risparmio di 75,00 \euro.
  
  \newpage

  \section{Preventivo a finire}
  La seguente tabella presenta l'attuale preventivo a finire. Nel caso in cui il valore del consuntivo di un determinato periodo non è ancora presente, per il conteggio totale verrà utilizzato il valore del preventivo.
  

  
	\begin{table}[!h]
  
	\begin{center}
  
		\rowcolors{1}{}{lightgray}
  
		\begin{tabular}{ccc}
  
			\rowcolor{coolblack}
  
			\hline
  
			\textcolor{white}{Periodo} & \textcolor{white}{Preventivo in \euro} & \textcolor{white}{Consuntivo in \euro}\\
  
			\hline
			Revisione analisi   & 715,00	&  765,00  			 	\\ 
			Progettazione della base tecnologica &  3642,00 &  3567,00 \\ 
			Progettazione di dettaglio e codifica    & 6404,00  &  -	\\ 
			Validazione e collaudo    & 3018,00 & - 	\\ \hline
			\textbf{Totale rendicontato}& \textbf{13.779,00} & \textbf{13.754,00}	\\ \hline   
  
		\end{tabular}
  
		\caption{Preventivo a finire} 
  
	\end{center}
  
\end{table}
\newpage
\section{Bilancio Progettazione di dettaglio e codifica}\label{BilProgDC}
La tabella riporta la differenza delle ore tra preventivo e consuntivo suddivise per ruolo. Il segno negativo indica una spesa superiore rispetto a quanto preventivato.\\
		\begin{table}[!h]
		\begin{center}
			\rowcolors{1}{}{lightgray}
			\begin{tabular}{ccc}
				\rowcolor{coolblack}
				\hline
				\textcolor{white}{Ruolo} & \textcolor{white}{Ore} & \textcolor{white}{Costo in \euro}\\
				\hline
				Responsabile   & 7(+1)	&  	210 	\\ 
				Amministratore & 15(-3)	&  	300 	\\ 
				Analista       & 11 &   275\\ 
				Progettista    & 115(+2)&   2530	\\ 
				Verificatore   & 69	(+2)&  	1035	\\ 
				Programmatore  & 140(-6)&   2100	\\ \hline
				\textbf{Totale}& \textbf{357} & \textbf{6450}	\\ \hline 
				\textbf{Totale preventivato}& \textbf{353} & \textbf{6404}\\ \hline 
				\textbf{Differenza}& \textbf{-4} & \textbf{-46}	\\ \hline  
			\end{tabular}
			
			\caption{Differenza delle ore tra preventivo e consuntivo divise per ruolo} 
		\end{center}
	\end{table}

	%DESCRIZIONE%
	
	
	\begin{table}[!h]
		\begin{center}
			\rowcolors{1}{}{lightgray}
			\begin{tabularx}{\textwidth}{|c|cccccc|c|}
				
				\hline
				\multirow{2}{*}{Nominativo} & \multicolumn{6}{c|}{Ore per ruolo} & \multirow{2}{*}{Ore totali} \\ \cline{2-7}
				& Re & Am & An & Pt & Ve & Pr &      \\ \hline
				\endhead
				Mirko Franco       &   &  &   & -2  & +6 & -2  & +2        \\ \hline
				Bianca Ciuche      &   & -2  &   &   &  & +3  & +1        \\ \hline
				Stefano Zanatta    &   &  &   &	 +2 &  &  -3 & -1        \\ \hline
				Andrea Deidda      &   &  &   &   & -2 & +2  & 0  		\\ \hline
				Ludovico Brocca    &  +1 &  &   &   &  & -2  & -1        \\ \hline
				Matteo Depascale   &   & -1&  & +2&  &	-6 & -5  		\\ \hline
				Gian Marco Bratzu  & &  &   &   & -2 & +2  & 0        \\ \hline
				
			\end{tabularx}
			\caption{Differenza tra le ore di lavoro previste per ogni membro del gruppo con le ore realmente impiegate }
		\end{center}
	\end{table}

	\subsection{Conclusioni}
	I guadagni riscontrati nei consuntivi dell'ultimo periodo sono stati allocati per la codifica del prodotto, sono sate inoltre redistribuite alcune ore, con una diminuzione di quelle per la progettazione (dovuta a una maggiore semplicità della \glossario{Skill} non considerata in precedenza) e un aumento di quelle da \textit{Amministratore} per la conversione e l'adattamento degli strumenti utilizzati a \glossario{TypeScript}, in sostituzione a JavaScript.\\
	Sono state inoltre aggiunte 4 ore (da \textit{Programmatore}) allo scopo di anticipare i futuri impegni inderogabili dei membri del gruppo, che comporteranno una ripianificazione relativa al prossimo periodo, lasciando il totale del preventivo invariato.\\
	
\section{Preventivo a finire}
	La seguente tabella presenta l’attuale preventivo a finire. Nel caso in cui il
	valore del consuntivo di un determinato periodo non è ancora presente, per
	il conteggio totale verrà utilizzato il valore del preventivo.
	
		\begin{table}[!h]
		\begin{center}
			
			\rowcolors{1}{}{lightgray}
			
			\begin{tabular}{ccc}
				
				\rowcolor{coolblack}
				\hline
				\textcolor{white}{Periodo} & \textcolor{white}{Preventivo in \euro} & \textcolor{white}{Consuntivo in \euro}\\
				\hline
				Revisione analisi& 715,00 & 765,00 \\ 	
				Progettazione della base tecnologica& 3642,00 & 3567,00\\ 
				Progettazione di dettaglio e codifica& 6404,00   & 6450,00\\ 
				Validazione e collaudo & 3018,00 & - \\ \hline
				\textbf{Totale rendicontato}& \textbf{13779,00} & \textbf{13740,00}	\\ \hline   
				
			\end{tabular}
			\caption{Preventivo a finire} 
		\end{center}		
	\end{table}

	\section{Bilancio Validazione e Collaudo}\label{BilVC}
La tabella riporta la differenza delle ore tra preventivo e consuntivo suddivise per ruolo. Il segno negativo indica una spesa superiore rispetto a quanto preventivato.\\
		\begin{table}[!h]
		\begin{center}
			\rowcolors{1}{}{lightgray}
			\begin{tabular}{ccc}
				\rowcolor{coolblack}
				\hline
				\textcolor{white}{Ruolo} & \textcolor{white}{Ore} & \textcolor{white}{Costo in \euro}\\
				\hline
				Responsabile   & 11.50(+0.5)	&  	345 	\\ 
				Amministratore & 16(5)	&  	320 	\\ 
				Analista       & 8.83(3.17) &   220.75\\ 
				Progettista    & 16(-2)&   2530	\\ 
				Verificatore   & 70.50	(-1.50)&  	1057,50	\\ 
				Programmatore  & 28.50(-3.50)&   427.50	\\ \hline
				\textbf{Totale}& \textbf{161.33} & \textbf{2942.75}	\\ \hline 
				\textbf{Totale preventivato}& \textbf{163} & \textbf{3018}\\ \hline 
				\textbf{Differenza}& \textbf{+1.67} & \textbf{+75.25}	\\ \hline  
			\end{tabular}
			
			\caption{Differenza delle ore tra preventivo e consuntivo divise per ruolo} 
		\end{center}
	\end{table}

	%DESCRIZIONE%
	La tabella riporta la differenza delle ore tra preventivo e consuntivo suddivise per membro del gruppo. In questo periodo, nel complessivo, tutti hanno svolto le loro ore, anche se con qualche scostamento. 
	
	\begin{table}[!h]
		\begin{center}
			\rowcolors{1}{}{lightgray}
			\begin{tabularx}{\textwidth}{|c|cccccc|c|}
				
				\hline
				\multirow{2}{*}{Nominativo} & \multicolumn{6}{c|}{Ore per ruolo} & \multirow{2}{*}{Ore totali} \\ \cline{2-7}
				& Re & Am & An & Pt & Ve & Pr &      \\ \hline
				\endhead
				Mirko Franco       &   &  &   & 2.17   & 2 &  & 4.17        \\ \hline
				Bianca Ciuche      &   &   &  &    &  &  & 0        \\ \hline
				Stefano Zanatta    &   & 5 & 4    &	 -2   & -1.5&  -6.5 & -1        \\ \hline
				Andrea Deidda      &   &  &   &   &  &   & 0  		\\ \hline
				Ludovico Brocca    &   &  &   &    &  -2&  & -2        \\ \hline
				Matteo Depascale   &  0.5 & &  &  &  &    & 0.5  		\\ \hline
				Gian Marco Bratzu  & &  &   &   &  &   & 0        \\ \hline
				
			\end{tabularx}
			\caption{Differenza tra le ore di lavoro previste per ogni membro del gruppo con le ore realmente impiegate }
		\end{center}
	\end{table}

	\subsection{Conclusioni}
	Il gruppo ha trovato alcune difficoltà nell' implementazione del collegamento tra le API di Google e l'applicazione Android. Per questo motivo, sono servite più ore da programmatore.\\Le ore preventivate da Analista erano troppe rispetto alle effettive attività da svolgere.\\
	Nello scorso periodo, alcuni membri del gruppo hanno svolto più ore del dovuto, con la prospettiva dello stage, svolgendo meno ore in questo.\\
	In totale, il gruppo ha risparmiato 75.25{\euro} e ha svolto 1.67 ore in meno.

\section{Bilancio complessivo}
	La seguente tabella presenta il bilancio complessivo di tutti i periodi.\\
	Nel complessivo, il gruppo ha risparmiato \textbf{54.25\euro} (non compreso il periodo di Analisi, perché non a carico del committente).\\Le ore effettive superano di 11.33 quelle preventivate (compreso il periodo di Analisi); nei periodi di \textit{Progettazione della base tecnologica} e \textit{Validazione e Collaudo}, il ruolo di programmatore (meno costoso) è servito più di quanto preventivato, a discapito degli Analisti e Progettisti (più costosi).\\
		\begin{table}[!h]
		\begin{center}
			
			\rowcolors{1}{}{lightgray}
			
			\begin{tabular}{ccc}
				
				\rowcolor{coolblack}
				\hline
				\textcolor{white}{Periodo} & \textcolor{white}{Preventivo in \euro} & \textcolor{white}{Consuntivo in \euro}\\
				\hline
				Revisione analisi& 715,00 & 765,00 \\ 	
				Progettazione della base tecnologica& 3642,00 & 3567,00\\ 
				Progettazione di dettaglio e codifica& 6404,00   & 6450,00\\ 
				Validazione e collaudo & 3018,00 & 2942.75\\ \hline
				\textbf{Totale rendicontato}& \textbf{13779,00} & \textbf{13694,00}	\\ \hline   
				
			\end{tabular}
			\caption{Preventivo a finire} 
		\end{center}		
	\end{table}

	La tabella riporta la differenza delle ore tra il preventivo e consuntivo di tutti i periodi, suddivise per membro del gruppo. Un valore positivo indica che sono state svolte meno ore rispetto a quanto preventivato.
	\begin{table}[!h]
		\begin{center}
			\rowcolors{1}{}{lightgray}
			\begin{tabularx}{\textwidth}{|c|cccccc|c|}
				
				\hline
				\multirow{2}{*}{Nominativo} & \multicolumn{6}{c|}{Ore per ruolo} & \multirow{2}{*}{Ore totali} \\ \cline{2-7}
				& Re & Am & An & Pt & Ve & Pr &      \\ \hline
				\endhead
				Mirko Franco       &   & -2 &   &  +0.17  & 8 & -2 & +4.17    \\ \hline
				Bianca Ciuche      &   &  -2 & -1  &    &  &  & -3        \\ \hline
				Stefano Zanatta    &   & +1 & +2   &  & +4.5 & -11.5   &  -4      \\ \hline
				Andrea Deidda      &   & +2 &   &   & -1 & 2   & +3  		\\ \hline
				Ludovico Brocca    &   &  &   &    &  0 & -6  & -6        \\ \hline
				Matteo Depascale   &  0.5 & -1 &  & +2 & -6  & -1  & -5.5 		\\ \hline
				Gian Marco Bratzu  & &  &   &   & -2 & +2  & 0        \\ \hline
				
			\end{tabularx}
			\caption{Differenza tra le ore di lavoro previste per ogni membro del gruppo con le ore realmente impiegate }
		\end{center}
	\end{table}
\end{flushleft}