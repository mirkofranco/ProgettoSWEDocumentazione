\chapter{Consuntivo a finire}
Il \textbf{Consuntivo a finire} contiene il prospetto economico che riporta le spese effettivamente sostenute. Vengono riportate le ore impiegate per svolgere i compiti preventivati, sia per
ruolo che per persona. Il bilancio è ottenuto facendo la differenza di ore tra il preventivo e il consuntivo.\\
Il bilancio può essere:
\begin{itemize}
    \item \textbf{Positivo:} se il preventivo ha superato il consuntivo;
    \item \textbf{Negativo:} se il consuntivo ha superato il preventivo;
    \item \textbf{In pari:} se consuntivo e preventivo coincidono.
\end{itemize}
\section{bilancio Analisi}
La tabella riporta la differenza delle ore tra preventivo e consuntivo, divise per ruolo.  
\begin{table}
    
\end{table}
In questa tabella  sono riportate le differenze tra le ore di lavoro previste per ogni membro del gruppo con quelle realmente impiegate.

\subsection{Conclusioni}