\begin{flushleft}
    \chapter{Consuntivo di periodo}
    Il \textit{Consuntivo di periodo} contiene il prospetto economico che riporta le spese effettivamente sostenute. Vengono riportate le ore impiegate per svolgere i compiti preventivati, sia per
    ruolo che per persona. Il bilancio è ottenuto facendo la differenza di ore tra il preventivo e il consuntivo.\\
    Il bilancio può essere:
    \begin{itemize}
        \item \textbf{Positivo:} se il preventivo ha superato il consuntivo;
        \item \textbf{Negativo:} se il consuntivo ha superato il preventivo;
        \item \textbf{In pari:} se consuntivo e preventivo coincidono.
    \end{itemize}

   \newpage
    \section{Bilancio Analisi}
    La tabella riporta la differenza delle ore tra preventivo e consuntivo, divise per ruolo. Il segno negativo indica che le ore effettive superano le ore preventivate.  
      

        \begin{tabularx}{\textwidth}{|l|l|l|l|l|}
        \hline
        Ruolo          & Ore previste& Differenza ore & Costo\euro  &Differenza costo\euro \\ \hline
        Responsabile   &22& -2 &660 & -60   \\ \hline
        Amministratore &39& 2  &780 & 40    \\ \hline
        Analista       &58& 1 &1.450  & 25    \\ \hline
        Progettista    & 0 &0 &0 & 0     \\ \hline
        Verificatore   & 49&-2 &735 & -30   \\ \hline
        Programmatore  & 0& 0&0   & 0     \\ \hline
        Totale         &168 & -1&3.625   & -25   \\ \hline
        \caption{Differenza delle ore tra preventivo e consultivo divise per ruolo}    
    \end{tabularx}

    In questa tabella  sono riportate le differenze tra le ore di lavoro previste per ogni membro del gruppo con quelle realmente impiegate.
  
        \begin{tabularx}{\textwidth}{|l|l|l|l|l|l|l|l|}
        \hline
        \multirow{2}{*}{Nominativo} & \multicolumn{6}{l|}{Ore per ruolo} & \multirow{2}{*}{Ore totali} \\ \cline{2-7}
                                    & Re   & Am  & An  & Pt  & Ve  & Pr  &                             \\ \hline
                                    \endhead
        Gian Marco Bratzu           &      &     &     &     &     &     & 0                           \\ \hline
        Ludovico Brocca             &      &     &     &     & -1  &     & -1                          \\ \hline
        Bianca Ciuche               & -2   &     & 1   &     &     &     & -1                          \\ \hline
        Andrea Deidda               &      & 2   &     &     &     &     & 2                           \\ \hline
        Matteo Depascale            &      &     &     &     & -1  &     & -1                          \\ \hline
        Mirko Franco                &      &     &     &     &     &     & 0                           \\ \hline
        Stefano Zanatta             &      &     &     &     &     &     & 0                           \\ \hline
     \caption{Differenza tra le ore di lavoro previste per ogni membro del gruppo con le ore realmente impiegate }    
    \end{tabularx}
      
    \subsection{Conclusioni}
    La stesura del \textit{Piano di Progetto} ha richiesto più lavoro del previsto da parte dei \textit{Responsabili} e dei \textit{Verificatori}. Il carico di lavoro degli \textit{Amministratori} e degli \textit{Analisti} è stato inferiore di quanto preventivato.\\ Il bilancio totale è negativo, con un deficit di 25\euro. Questo costo non è a carico del committente, in quanto è stato generato dalle attività di Analisi.
    
    \newpage
    \section{Bilancio Revisione Analisi}
     La tabella riporta la differenza delle ore tra preventivo e consuntivo, divise per ruolo. Il segno negativo indica che le ore effettive superano le ore preventivate.  

     \begin{tabularx}{\textwidth}{|l|l|l|l|l|}
	\hline
	Ruolo              & Ore Previste& Differenza ore & Costo \euro & Differenza Costo \euro \\ \hline
	Responsabile            & 5       & -2  & 150 &  -60 \\ \hline
	Amministratore         & 5       & +3  & 100  &+60 \\ \hline
	Analista                   & 9         & +6   & 225 & +150  \\ \hline
	Progettista              &-           &-  & -  & - \\ \hline
	Verificatore             & 16        &-5  & 240 & -75 \\ \hline
	Programmatore         & -        &-   & -  & - \\ \hline
	Totale                         & 35     &+2   & 715 & +75\\ \hline
	\caption{Differenza delle ore tra preventivo e consultivo divise per ruolo}    
    \end{tabularx}


In questa tabella  sono riportate le differenze tra le ore di lavoro previste per ogni membro del gruppo con quelle realmente impiegate.\\

        \begin{tabularx}{\textwidth}{|l|l|l|l|l|l|l|l|}
	\hline
	\multirow{2}{*}{Nominativo} & \multicolumn{6}{l|}{Ore per ruolo} & \multirow{2}{*}{Ore totali} \\ \cline{2-7}
	& Re   & Am  & An  & Pt  & Ve  & Pr  &                             \\ \hline
	\endhead
	Mirko Franco                &    +2  &     &     &     &   -1  &     & +1                        \\ \hline
	Bianca Ciuche               & -2   &     & +2   &     &     &     & 0                         \\ \hline
	Stefano Zanatta             &      &    +1 & +3   &     & -1   &    & +3                           \\ \hline
	Andrea Deidda               &      &   &     &     &  -2   &     & -2                           \\ \hline
	Ludovico Brocca             &      &     &   +1  &     & -1  &     & 0                        \\ \hline
	Matteo Depascale            &      &     &     &     &   &     & 0                          \\ \hline
	Gian Marco Bratzu           &      &     &     &     &     &     & 0                           \\ \hline
	\caption{Differenza tra le ore di lavoro previste per ogni membro del gruppo con le ore realmente impiegate }    
      \end{tabularx}
  
  \subsection{Conclusioni}
  Le correzioni del \textit{Piano di Qualifica} e del documento \textit{Norme} ha richiesto più lavoro del previsto agli {Analisti}. Il carico di lavoro degli \textit{Verificatori} è stato inferiore di quanto preventivato.\\ Il bilancio totale è negativo, con un deficit di 75\euro.  
   \end{flushleft}