\chapter{Analisi dei rischi}\label{AnRischi}
Per aumentare la probabilità di una buona riuscita del\glossario{progetto}viene effettuata un'approfondita analisi dei rischi. Per poter gestire i rischi viene applicata la seguente procedura:
\begin{itemize}
	\item \textbf{Identificazione:} identificare i potenziali rischi che possono minare l'avanzamento del\glossario{progetto}e capirne la cause. I rischi possono afferire alle seguenti categorie:
	\begin{itemize}
		\item \textbf{Progetto}: relativi alle risorse, alla pianificazione e agli strumenti;
		\item \textbf{Prodotto$_{G}$}: relativi alla conformità del prodotto, a quanto atteso dal committente;
		\item \textbf{Mercato}: relativi a costi.
	\end{itemize}
	\item \textbf{Analisi}: ne viene valutata la probabilità di occorrenza, la pericolosità e le possibili conseguenze;
	\item \textbf{Pianificazione}: vengono attuate strategie che permettono di evitare i rischi o mitigarne gli effetti qualora si presentassero;
	\item \textbf{Controllo}: viene posta attenzione continua tramite la rilevazione di specifici indicatori raffinando le strategie di pianificazione qualora ve ne fosse il bisogno.
\end{itemize}
\section{Elenco dei rischi}
Ogni rischio viene classificato secondo la seguente convenzione:\\\\
\centering \textbf{R[Tipo][Identificativo]}\\
\begin{itemize}
	\item La lettera "R" è l'abbreviazione di Rischio;
	\item Il secondo valore indica il tipo di rischio. Può essere:
	\begin{itemize}
		\item \textbf{P}: indica i rischi legati al \textit{progetto$_{G}$}, quindi a risorse, pianificazione e strumenti;
		\item \textbf{PR}: indica i rischi legati ai \textit{requisiti$_{G}$};
		\item \textbf{M}: indica i rischi legati al mercato.
	\end{itemize}
	\item L'identificato è semplicemente un numero progressivo.
\end{itemize}

\begin{tabularx}{\textwidth}{|X|X|X|X|}
	\hline
	\textbf{Nome} \newline \textbf{Codice} & \textbf{Descrizione} & 	\textbf{Rilevamento} & \textbf{Grado di rischio}\\
	\hline
	\endhead
	Scarsa esperienza \newline RP001 & Nessun membro del gruppo ha esperienza con un progetto di tali dimensioni e complessità. &
	Ogni membro comunicherà al \textit{Responsabile} eventuali difficoltà incontrate. & Occorrenza: Media. \newline Pericolosità: Alta. \\
	\hline
	\multicolumn{4}{|>{\hsize=\dimexpr4\hsize+4\tabcolsep+\arrayrulewidth\relax}X|}{\textbf{Mitigazione}: Il \textit{Responsabile} pianificherà le attività cercando di valorizzare capacità e competenze delle singole persone.}\\
	\hline
   
   
   	Contrasti tra i membri del gruppo \newline RP002 & I gruppi sono stati formati in modo casuale. Ogni componente del gruppo ha idee, carattere e metodologie di lavoro differenti. Questo può portare alla nascita di contrasti creando un clima non collaborativo all'interno del gruppo.  &
 	Ogni membro del gruppo comunicherà al \textit{Responsabile} eventuali difficoltà incontrate. & Occorrenza: Bassa. \newline Pericolosità: Alta. \\
	\hline
   \multicolumn{4}{|>{\hsize=\dimexpr4\hsize+4\tabcolsep+\arrayrulewidth\relax}X|}{\textbf{Mitigazione}: Nel caso di forti contrasti, il \textit{Responsabile} cercherà di mediare il dialogo tra i componenti. Inoltre la pianificazione terrà conto di ciò cercando di minimizzare la possibilità che insorgano situazioni di conflitto.}\\
	\hline
	
	Disponibilità dei membri \newline RP003 &Ciascun componente del gruppo ha impegni personali, imprevisti e necessità differenti. Nel gruppo vi è inoltre uno studente lavoratore. Alcuni membri hanno esami degli anni passati come arretrati. Risulta quindi inevitabile andare incontro a problemi di organizzazione causati da collisioni con impegni personali.  &
	Ogni membro comunicherà tempestivamente al \textit{Responsabile} gli impegni che possono collidere con le attività di \textit{progetto$_{G}$}. & Occorrenza: Media. \newline Pericolosità: Media. \\
	\hline
	\multicolumn{4}{|>{\hsize=\dimexpr4\hsize+4\tabcolsep+\arrayrulewidth\relax}X|}{\textbf{Mitigazione}: A seguito della comunicazione al \textit{Responsabile} di un impegno, esso provvederà a eseguire una nuova pianificazione al periodo problematico. Il carico di lavora verrà ridistribuito tra i componenti con maggiore disponibilità.}\\
	\hline

	
	Tecnologie da utilizzare \newline RP004 & Le tecnologie da usare sono nuove ai componenti del gruppo e/o la documentazione fornita è molto limitata. I componenti del gruppo devono apprendere autonomamente tali tecnologie. Questo può portare a ritardi nello svolgimento del \textit{progetto$_{G}$}. &
	Ogni membro comunicherà al \textit{Responsabile} l'insorgere di grosse difficoltà con le tecnologie da utilizzare. Il \textit{Responsabile} monitorerà la preparazione dei componenti rispetto ai compiti che dovranno svolgere. & Occorrenza: Media. \newline Pericolosità: Alta. \\
	\hline
	\multicolumn{4}{|>{\hsize=\dimexpr4\hsize+4\tabcolsep+\arrayrulewidth\relax}X|}{\textbf{Mitigazione}: I membri con maggiore esperienza e conoscenza di una determinata tecnologia dovranno aiutare il membro in difficoltà. Il carico di lavoro verrà ridistribuito di conseguenza.}\\
	\hline
	
	Strumenti utilizzati \newline RP005 & Il gruppo si affida a tecnologie software di terze parti. Questo può portare a disfunzioni o perdite di dati. &
	Risulta complicato rilevare il problema poiché dipende da terze parti. & Occorrenza: Bassa. \newline Pericolosità: Media. \\
	\hline
	\multicolumn{4}{|>{\hsize=\dimexpr4\hsize+4\tabcolsep+\arrayrulewidth\relax}X|}{\textbf{Mitigazione}: Si cerca di usare strumenti il più affidabili possibile.}\\
	\hline
	 
	 Problemi hardware \newline RP005 & Ogni membro utilizza macchine personali per lavorare al progetto. Guasti hardware potrebbero causare perdite di dati e/o tempo. &
	 Ogni membro comunicherà tempestivamente al gruppo eventuali anomalie riscontrate nelle proprie macchine. & Occorrenza: Bassa. \newline Pericolosità: Media. \\
	 \hline
	 \multicolumn{4}{|>{\hsize=\dimexpr4\hsize+4\tabcolsep+\arrayrulewidth\relax}X|}{\textbf{Mitigazione}: Ogni membro del gruppo dovrà effettuare giornalmente un\glossario{backup}di tutti i dati di \textit{progetto$_{G}$}, compresi file di configurazione e appunti personali riguardanti il \textit{progetto$_{G}$}.}\\
	 \hline
 	
 	 Problemi di connessione \newline RP006 & Visto che non tutti i componenti possono raggiungere sempre facilmente un luogo comune, alcune riunioni potranno essere effettuate tramite\glossario{Hangouts}. Il malfunzionamento della connessione internet di uno dei membri potrebbe impedire la sua partecipazione diretta. &
 	Sarà compito del membro comunicare tempestivamente al gruppo eventuali malfunzionamenti. & Occorrenza: Bassa. \newline Pericolosità: Bassa.\\
 	\hline
 	\multicolumn{4}{|>{\hsize=\dimexpr4\hsize+4\tabcolsep+\arrayrulewidth\relax}X|}{\textbf{Mitigazione}: Qualora un membro non riuscisse a partecipare direttamente alla \textit{conference}\glossario{call}potrà interagire con il resto del gruppo tramite \textit{Telegram$_{G}$}. Sarà suo compito aggiornarsi leggendo il verbale successivamente redatto.}\\
 	\hline
 	
 		
 	Comprensione dei\glossario{requisiti} \newline RPR007 & Esiste la possibilità che il gruppo\glossario{ZeroSeven}interpreti male i requisiti, portando così alla realizzazione di un prodotto non conforme alle richieste della\glossario{proponente}Zero12. & Il gruppo lavorerà a stretto contatto con la proponente. In caso di dubbio sarà premura del gruppo ZeroSeven contattare la proponente per chiedere delucidazioni. & Occorrenza: Media. \newline Pericolosità: Alta. \\
 	\hline
 	\multicolumn{4}{|>{\hsize=\dimexpr4\hsize+4\tabcolsep+\arrayrulewidth\relax}X|}{\textbf{Mitigazione}: I requisiti mal interpretati verranno coretti il prima possibile in modo da realizzare il\glossario{prodotto}richiesto.}\\
 	\hline
	
	Modifica dei requisiti \newline RPR008 & C'è la possibilità che la proponente voglia modificare i requisiti. Questo porterebbe a una modifica dell'\textit{Analisi dei Requisiti} con conseguente ripianificazione del lavoro da svolgere. & Il gruppo lavorerà a stretto contatto con la proponente per minimizzare il rischio di modifiche importanti dei requisti. & Occorrenza: Alta. \newline Pericolosità: Alta. \\
	\hline
	\multicolumn{4}{|>{\hsize=\dimexpr4\hsize+4\tabcolsep+\arrayrulewidth\relax}X|}{\textbf{Mitigazione}: In caso di modifica importante dei requisiti, il gruppo\glossario{ZeroSeven}negozierà tali cambiamenti con la proponente in modo da trovare un compromesso.}\\
	\hline
	
	Costi delle attività \newline RM009 & La pianificazione prevede un costo e un tempo per ogni attività. Essendo il gruppo inesperto è possibile che costi e tempi vengano calcolati in modo errato. & Il \textit{Responsabile} verificherà periodicamente lo stato delle attività in modo da minimizzare il rischio di sforamento di tempi e costi. & Occorrenza: Alta \newline Pericolosità: Alta \\
	\hline
	\multicolumn{4}{|>{\hsize=\dimexpr4\hsize+4\tabcolsep+\arrayrulewidth\relax}X|}{\textbf{Mitigazione}: In caso di forte ritardo o grosso sforamento dei costi il \textit{Responsabile} provvederà a ridistribuire il carico mantenendo come obiettivo primario quello di non sforare le \textit{milestone$_{G}$}.}\\
	\hline	
	\caption{Elenco dei rischi}\\
    \end{tabularx}

