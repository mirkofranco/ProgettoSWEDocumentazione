\chapter{Attualizzazione dei rischi}
\label{Attualizzazione dei rischi}
 \begin{tabularx}{\textwidth}{|X|X|X|}
 	\hline
 	\textbf{Codice} & \textbf{Periodo} & \textbf{Mitigazione} \\
 	\hline
 	\endhead
 	RP003 & Analisi & Si sono verificati problemi causati dalla sovrapposizione delle attività e necessità di progetto con impegni lavorativi e accademici dei membri del gruppo. A causa di ciò alcune volte il lavoro è stato rallentato o non gestito al meglio.\\
 	\hline
 	\multicolumn{3}{|>{\hsize=\dimexpr3\hsize+3\tabcolsep+\arrayrulewidth}X|}{
 	\textbf{Miglioramenti}: Ridistribuzione del carico di lavoro. 
 	}\\
 	\hline
 	RP004 & Analisi & Si sono verificati problemi con il software git a causa della non conoscenza di tale sistema da tutti i componenti del gruppo. Questo a volte ha causato problemi con il coretto utilizzo del sistema di versionamento, con conseguente necessità di tempo per correggere eventuali errori.\\
 	\hline
 	\multicolumn{3}{|>{\hsize=\dimexpr3\hsize+3\tabcolsep+\arrayrulewidth\relax}X|}{
 	\textbf{Miglioramenti}: Ogni persona in difficoltà si è formata in modo autonomo ed è stata sostenuta dai componenti del gruppo con più esperienza. In questo modo si è cercato di evitare il ripetersi di errori dovuti alla non conoscenza di questo software. 
 	}\\
 \hline
     RP001& Progettazione della Base Tecnologica &Si sono verificati problemi riguardo la scarsa esperienza del gruppo legati all'implementazione dell' \textit{API Gateway$_{G}$}. 
     Questo problema si è manifestato negli ultimi giorni e ha causato un aumento del carico di lavoro.\\
     \hline
     \multicolumn{3}{|>{\hsize=\dimexpr3\hsize+3\tabcolsep+\arrayrulewidth\relax}X|}{
     \textbf{Miglioramenti}: Ogni membro dedica più tempo allora studio delle tecnologie da usare.}\\
 \hline
      RP004& Progettazione della Base Tecnologica &Si sono verificati problemi riguardo 
      l'utilizzo dell' \textit{API Gateway$_{G}$}. Questo problema si è manifestato a causa di una scarsa comunicazione con la proponente.\\
 \hline
 \multicolumn{3}{|>{\hsize=\dimexpr3\hsize+3\tabcolsep+\arrayrulewidth\relax}X|}{
 	\textbf{Miglioramenti}: Si provvederà ad una comunicazione più frequente con la proponente.}\\
     	\hline
     	    RP005&Progettazione della Base Tecnologica  &Il gruppo ha utilizzato una libreria che prima della Technology Baseline è stata rimossa dalla versione nuova di \textit{Android Studio$_{G}$}.\\
     	\hline
     	\multicolumn{3}{|>{\hsize=\dimexpr3\hsize+3\tabcolsep+\arrayrulewidth\relax}X|}{
     		\textbf{Miglioramenti}:Prima di utilizzare una libreria verrà controllata se è deprecata o meno.}\\
     	     	\hline
     	RM009&Progettazione della Base Tecnologica & Il gruppo ha preventivato troppe ore da progettista e troppe poche ore da programmatore, come descritto nel consuntivo del \textit{Piano di Progetto v2.0.0}.\\
     	\hline
     	\multicolumn{3}{|>{\hsize=\dimexpr3\hsize+3\tabcolsep+\arrayrulewidth\relax}X|}{
     	\textbf{Miglioramenti}:Maggiore attenzione nella suddivisione delle ore.}\\
     	\hline
     	     	
     	RP004 & Progettazione di dettaglio e codifica & Durante lo svolgimento del progetto alcuni componenti del gruppo inizieranno uno stage formativo ai fini della laurea; questo problema potrebbe portare a inefficienza nello svolgimento delle attività relative al progetto. \\
     	\hline
     		\multicolumn{3}{|>{\hsize=\dimexpr3\hsize+3\tabcolsep+\arrayrulewidth\relax}X|}{
     		\textbf{Miglioramenti}: Il \textit{Responsabile} ha pianificato le attività relative al periodo di validazione e collaudo assecondando le esigenze dei membri impegnati in attività di stage formativo.
     	} \\
     \hline
     RP005 & Progettazione di dettaglio e codifica & L'utilizzo di JavaScript potrebbe aver portato a difficoltà, causate principalmente da mancanze relative al linguaggio stesso (mancanza di tipi di ritorno, interfacce) che ha prodotto codice spesso poco manutenibile.\\
     \hline
     \multicolumn{3}{|>{\hsize=\dimexpr3\hsize+3\tabcolsep+\arrayrulewidth\relax}X|}{
     	\textbf{Miglioramenti}: \'E stata presa la decisione di utilizzare \glossario{TypeScript}, che sopperisce ai problemi sopracitati incorporando il paradigma OOP in JavaScript. La conversione è stata rapida ed efficiente (circa 2 ore di tempo). Durante il deploy, il codice TypeScript viene compilato in JavaScript e raggiunge AWS Lambda grazie a uno script automatico.
     } \\
 	\hline
 	\caption{Attualizzazione dell'analisi dei rischi}
\end{tabularx}