\chapter{Pianificazione}
Lo sviluppo del progetto è diviso in cinque periodi:
\begin{itemize}
    \item \textbf{Analisi} AN;
    \item \textbf{Analisi Dettaglio} AD;
    \item \textbf{Progettazione Architetturale} PA;
    \item \textbf{Progettazione di Dettaglio e Codifica} (PDC);
    \item \textbf{Validazione} (VV);
\end{itemize}
In ogni periodo sono presenti delle attività da svolgere, alle quali sono associate una o più risorse. Ogni attività è sottoposta a verifica, per semplificare il processo di Validazione.\\
Le attività sono suddivise in più sotto-attività.\\
Nel Gantt$_{G}$ vengono riportate:
\begin{itemize}
    \item \textbf{attività:} contengono più sotto-attività. Nel Gantt$_{G}$ sono rappresentate con una linea grigia.
    \item \textbf{milestone$_{G}$:} rappresenta la data ultima prevista per il completanento di un insieme prestabilito di attività. Ha durata di 0 (zero) giorni e coincide con la data della successiva revisione o l'approvazione complessiva di tali attivita. È rappresentata nel $_Gantt$ con un rombo giallo;
    \item \textbf{sotto-attività:} attività atomiche che possono essere svolte da una persona. Nel Gantt$_{G}$ sono rappresentate con una linea blu.  
\end{itemize}
\section{Analisi}
\textbf{Periodo:} da 2018-11-25 a 2019-01-21\\L'Analisi inizia in concomitanza con la pubblicazione dei capitolati d’appalto e termina alla Revisione dei Requisiti (RR).\\
Le attività della fase di Analisi sono:
\begin{itemize}
    \item \textbf{Studio di Fattibilità:} vengono valutati tutti i capitolati d'appalto, la valutazione è basata sull'interesse personale di ogni membro del gruppo, sulla complessità prevista, sui rischi che possono emergere. Viene data anche data una descrizione generale del capitolato e una analisi preliminare.\\Viene svolta come prima attività, in quanto bloccante per l'Analisi dei requisiti;
    \item \textbf{Norme di Progetto:} le norme di progetto contengono le regole che il gruppo dovrà seguire durante l'attuazione di tutte le attività. I verificatori certificheranno il rispetto delle norme;
    \item \textbf{Analisi dei Requisiti:} l'analisi preliminare nello Studio di Fattibilità viene approfondita.\\Tale attività continuerà sino alla data di consegna;
    \item \textbf{Piano di Progetto:} Il Responsabile del gruppo redige il piano di progetto, rispettando i vincoli posti dal proponente.\\Questo documento ha l'obiettivo di regolare le attività svolte dal gruppo;
    \item \textbf{Piano di qualifica:} Il piano di qualifica contiene le strategie utili al raggiungimento degli obiettivi del proponente, del gruppo e inerenti alla qualità dei processi di sviluppo.\\L'obiettivo è quello di rendere la qualità quantificabile e misurabile;
    \item \textbf{Glossario:} scritto in modo incrementale. Contiene la spiegazione dei termini più tecnici utilizzati nei documenti. Questa attività è svolta da tutti i membri del gruppo, in parallelo con la redazione di tutti i documenti.
    \item \textbf{Lettera di presentazione:} documento presentato al committente che permette al gruppo di partecipare alla gara d’appalto per il capitolato.
\end{itemize}
I ruoli maggiormente coinvolti sono: Responsabile, Amministratore e Analista.
\section{Analisi Dettaglio}
\textbf{Periodo:} da 2019-01-21 a 2019-02-15\\
L'\textbf{Revisione Analisi} inizia dopo la \textbf{Revisione dei Requisiti} e termina con l’inizio della \textbf{Progettazione Architetturale}. Questo periodo viene utilizzato dal team per consolidare i requisiti richiesti dal sistema e per migliorare i documenti già redatti.\\I ruoli maggiormente coinvolti sono: \textbf{Responsabile}, \textbf{Amministratore} e \textbf{Analista}.

\section{Progettazione Architetturale}
\textbf{Periodo:} da 2019-02-15 a 2019-03-15\\
La Progettazione Architetturale inizia al termine dell’\textbf{Analisi Dettaglio} e termina con la consegna del prodotto alla \textbf{Revisione di Progetto}.\\
Le attività della fase di \textbf{Progettazione Architetturale} sono:
\begin{itemize}
    \item \textbf{Specifica Tecnica}: il Progettista del gruppo espone le scelte progettuali, ad alto livello, che il prodotto dovrà avere. Vengono descritti i design pattern utilizzati nella creazione del prodotto, l’architettura generale del software, i principali flussi di controllo e il tracciamento dei requisiti;
    \item \textbf{Incremento e Verifica:} tutti i documenti vengono aggiornati in base ai risultati della \textbf{Revisione dei Requisiti}.
\end{itemize}
I ruoli maggiormente coinvolti sono: Responsabile, Amministratore, Progettista, Verificatore e Analista.

\section{Progettazione di Dettaglio e Codifica}
\textbf{Periodo:} da 2019-03-15 a 2019-04-16\\
Questo periodo inizia dopo la \textbf{Revisione di Progetto} e termina con la consegna del prodotto alla \textbf{Revisione di Qualità}.\\Le attività della fase di Progettazione di Dettaglio e Codifica sono:
\begin{itemize}
    \item \textbf{Definizione di Prodotto:} Viene definita approfonditamente la struttura e le relazioni dei componenti del prodotto, basandosi sul documento di \textit{Specifica Tecnica};
    \item \textbf{Codifica:} i programmatori sviluppano quel che è riportato nella \textit{Definizione di Prodotto};
    \item \textbf{Incremento e Verifica:} tutti i documenti vengono aggiornati in base al risultato della Revisione di Progettazione.
\end{itemize}
I ruoli maggiormente coinvolti sono: \textit{Responsabile}, \textit{Amministratore}, \textit{progettista}, \textit{Verificatore} e \textit{Programmatore}.

\section{Validazione e Collaudo}
\textbf{Periodo:} da 2019-04-16 2019-05-17\\
Questo periodo inizia al termine della \textbf{Progettazione di Dettaglio e Codifica} e si conclude con la \textbf{Revisione di Accettazione}.\\Con la \textbf{Validazione} il team svolge le ultime attività di Verifica e prepara il sistema per la Validazione e il rilascio.\\
Le attività sono:
\begin{itemize}
    \item \textbf{Validazione e Collaudo}: il sistema viene validato e collaudato, per dimostrare la sua conformità con le specifiche accordate con il proponente;
    \item \textbf{Incremento e Verifica}: tutti i documenti verranno aggiornati in base al risultato della Revisione di Qualifica.
\end{itemize}
