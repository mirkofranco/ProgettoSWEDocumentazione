\chapter{Pianificazione}
Lo sviluppo del progetto è diviso in cinque periodi:
\begin{itemize}
    \item \textbf{Analisi} (AN);
    \item \textbf{Revisione Analisi} (RE);
    \item \textbf{Progettazione della base tecnologica} (PT);
    \item \textbf{Progettazione di Dettaglio e Codifica} (PDC);
    \item \textbf{Validazione} (VV);
\end{itemize}
In ogni periodo sono presenti delle attività da svolgere, alle quali sono associate una o più risorse. Ogni attività è sottoposta a verifica, per semplificare il processo di Validazione.\\
Le attività sono suddivise in più sotto-attività.\\
Nel \glossario{Gantt} vengono riportate:
\begin{itemize}
    \item \textbf{Attività:} contengono più sotto-attività. Nel Gantt$_{G}$ sono rappresentate con una linea grigia;
    \item \textbf{Milestone$_{G}$:} rappresenta la data ultima prevista per il completamento di un insieme prestabilito di attività. Ha durata di 0 (zero) giorni e coincide con la data della successiva revisione o l'approvazione complessiva di tali attività. È rappresentata nel \glossario{Gantt} con un rombo giallo;
    \item \textbf{Sotto-attività:} attività atomiche che possono essere svolte da una persona. Nel \glossario{Gantt} sono rappresentate con una linea blu.  
\end{itemize}
\section{Analisi}
\textbf{Periodo:} da 2018-11-25 a 2019-01-14\\L'Analisi inizia in concomitanza con la pubblicazione dei capitolati d’appalto e termina alla Revisione dei Requisiti (RR).\\
Le attività della fase di Analisi sono:
\begin{itemize}
    \item \textbf{Studio di Fattibilità:} vengono valutati tutti i capitolati d'appalto, la valutazione è basata sull'interesse personale di ogni membro del gruppo, sulla complessità prevista e sui rischi che possono emergere. Viene data anche una descrizione generale del capitolato e una analisi preliminare.\\Viene svolta come prima attività, in quanto bloccante per l'\textit{Analisi dei requisiti};
    \item \textbf{Norme di Progetto:} le \textit{Norme di Progetto} contengono le regole che il gruppo dovrà seguire durante l'attuazione di tutte le attività. I verificatori certificheranno il rispetto delle norme;
    \item \textbf{Analisi dei Requisiti:} l'analisi preliminare nello Studio di Fattibilità viene approfondita.\\Tale attività continuerà sino alla data di consegna;
    \item \textbf{Piano di Progetto:} Il Responsabile del gruppo redige il \textit{Piano di Progetto}, rispettando i vincoli posti dal proponente.\\Questo documento ha l'obiettivo di regolare le attività svolte dal gruppo;
    \item \textbf{Piano di qualifica:} Il \textit{Piano di Qualifica} contiene le strategie utili al raggiungimento degli obiettivi del proponente, del gruppo e inerenti alla qualità dei processi di sviluppo.\\L'obiettivo è quello di rendere la qualità quantificabile e misurabile;
    \item \textbf{Glossario:} scritto in modo incrementale, contiene la spiegazione dei termini più tecnici utilizzati nei documenti. Questa attività è svolta da tutti i membri del gruppo, in parallelo con la redazione di tutti i documenti.
    \item \textbf{Lettera di presentazione:} documento presentato al committente che permette al gruppo di partecipare alla gara d’appalto per il capitolato.
\end{itemize}
I ruoli maggiormente coinvolti sono: Responsabile, Amministratore e Analista.
\section{Analisi Dettaglio}
\textbf{Periodo:} da 2019-01-14 a 2019-02-07\\
La Revisione Analisi inizia dopo la Revisione dei Requisiti e termina con l’inizio della Progettazione della base tecnologica. Questo periodo viene utilizzato dal team per consolidare i requisiti richiesti dal sistema e per migliorare i documenti già redatti.\\I ruoli maggiormente coinvolti sono: Responsabile, Amministratore e Analista.

\section{Progettazione della base tecnologica}
\textbf{Periodo:} da 2019-02-07 a 2019-03-08\\
La Progettazione della base tecnologica inizia al termine dell’Analisi Dettaglio e termina con la consegna del prodotto alla Revisione di Progetto.\\
Le attività della fase di Progettazione della base tecnologica sono:
\begin{itemize}
    \item \textbf{Technology Baseline}: l'amministratore espone al gruppo, dopo una discussione in modalità \glossario{Agile}, le scelte progettuali che il prodotto dovrà avere.
    \item \textbf{Incremento e Verifica:} tutti i documenti vengono aggiornati in base ai risultati della Revisione dei Requisiti.
    \item \textbf{Glossario:} questa attività consiste nel miglioramento del \textit{Glossario}.
\end{itemize}
I ruoli maggiormente coinvolti sono: Responsabile, Amministratore, Progettista, Verificatore e Analista.

\section{Progettazione di Dettaglio e Codifica}
\textbf{Periodo:} da 2019-03-08 a 2019-04-12\\
Questo periodo inizia dopo la Revisione di Progetto e termina con la consegna del prodotto alla Revisione di Qualità.\\Le attività della fase di Progettazione di Dettaglio e Codifica sono:
\begin{itemize}
    \item \textbf{Definizione di Prodotto:} Viene definita approfonditamente la struttura e le relazioni dei componenti del prodotto, basandosi sul documento di \textit{Specifica Tecnica};
    \item \textbf{Codifica:} i programmatori sviluppano quel che è riportato nella \textit{Definizione di Prodotto};
    \item \textbf{Incremento e Verifica:} tutti i documenti vengono aggiornati in base al risultato della Revisione di Progettazione.
    \item \textbf{Glossario:} questa attività consiste nel miglioramento del \textit{Glossario}.
\end{itemize}
I ruoli maggiormente coinvolti sono: Responsabile, Amministratore, progettista, Verificatore e Programmatore.

\section{Validazione e Collaudo}
\textbf{Periodo:} da 2019-04-12 2019-05-17\\
Questo periodo inizia al termine della Progettazione di Dettaglio e Codifica e si conclude con la Revisione di Accettazione.\\Con la Validazione il team svolge le ultime attività di Verifica e prepara il sistema per la Validazione e il rilascio.\\
Le attività sono:
\begin{itemize}
    \item \textbf{Validazione e Collaudo}: il sistema viene validato e collaudato, per dimostrare la sua conformità con le specifiche accordate con il proponente;
    \item \textbf{Incremento e Verifica}: tutti i documenti verranno aggiornati in base al risultato della Revisione di Qualifica.
    \item \textbf{Glossario:} questa attività consiste nel miglioramento del \textit{Glossario}.
\end{itemize}
