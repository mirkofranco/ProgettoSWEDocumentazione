\chapter{Introduzione}
\section{Scopo del documento}
Il presente documento ha l'intento di descrivere e specificare la pianificazione a cui il gruppo ZeroSeven si atterrà per portare avanti il progetto.
Gli scopi del presente documento sono:
\begin{itemize}
	\item Specificare le risorse disponbili e la loro assegnazione alle attività;
	\item Analizzare i possibili fattori di rischio;
	\item Consuntivare l'utilizzo delle risorse durante lo svolgersi del progetto;
	\item Aggiustare la pianificazione delle risorse durante lo svolgersi del progetto.
\end{itemize}
\section{Scopo del prodotto}
Lo scopo del progetto è quello di creare un applicativo Web e Mobile in grado di creare delle routine personalizzate per gli utenti gestibili tramite Alexa di Amazon. L'obbiettivo è quello di creare skill in grado di avviare workflow creati dagli utenti fornendogli dei connettori.
\section{Glossario}
Al fine di evitare ogni ambiguità di linguaggio e massimizzare la comprensione dei documenti, i termini tecnici, di dominio, gli acronimi e le parole che necessitano di essere chiarite, sono riportate nel \texttt{Glossario v1.0.0}.\\
Ogni occorrenza di vocaboli presenti nel \texttt{Glossario} è marcata da una "G" maiuscola in pedice.
\section{Riferimenti}
\subsection{Normativi}
\begin{itemize}
	\item \textbf{Regole del progetto didattico}
	\item  \textbf{Regolamento organigramma}
	\item  \textbf{Norme di Progetto}: \texttt{Norme di Progetto v1.0.0}
\end{itemize}
\subsection{Informativi}
\begin{itemize}
	\item \textbf{Capitolato d'appalto C4}: MegAlexa: arricchitore di skill di Amazon Alexa
\end{itemize}
\section{Ciclo di vita}
La scelta del ciclo di vita risulta cruciale per organizzare e controllare le attività di progetto. Visti i vincoli e la complessità del progetto viene scelto il modello incrementale. Questo ci permette di effettuare rilasci multipli e frequenti e di classificare i requisiti in ordine di importanza, permettendoci di concentrarci sulle funzionalità con importanza più elevata. Per organizzare ogni incremento prendiamo spunto dal concetto di \texttt{sprint} del Framework \texttt{SCRUM}, essendo anche il progetto didattico in parte orientato verso le metodologie agili. Questo ci permette anche di gestire meglio la complessità di questo progetto.
\section{Scadenze}
Di seguito vengono riportate le scadenze che il gruppo ZeroSeven ha deciso di rispettare e sulle quali si baserà la pianificazione dell'intero progetto:
\begin{itemize}
	\item \textit{Revisione dei Requisiti}: 2018-01-23
	\item \textit{Revisione di Progettazione}: 2018-03-15
	\item \textit{Revisione di Qualifica}: 2018-04-19
	\item \textit{Revisione di Accettazione}: 2018-05-17
\end{itemize}