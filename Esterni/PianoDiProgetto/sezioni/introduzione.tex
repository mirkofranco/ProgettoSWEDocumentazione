\chapter{Introduzione}\label{Introduzione}
\section{Scopo del documento}
Il presente documento ha l'intento di descrivere e specificare la pianificazione a cui il gruppo \textit{ZeroSeven$_{G}$} si atterrà per portare avanti il \textit{progetto$_{G}$}.
Gli scopi del presente documento sono:
\begin{itemize}
	\item Specificare le risorse disponibili e la loro assegnazione alle attività;
	\item Analizzare i possibili fattori di rischio;
	\item Consuntivare l'utilizzo delle risorse durante lo svolgersi del progetto;
	\item Aggiustare la pianificazione delle risorse durante lo svolgersi del progetto.
\end{itemize}
\section{Scopo del prodotto}
Lo scopo del progetto è quello di sviluppare un applicativo Mobile in grado di creare delle routine personalizzate per gli utenti gestibili tramite\glossario{Alexa}di \textit{Amazon$_{G}$}. L'obiettivo è quello di creare\glossario{skill}in grado di avviare\glossario{workflow}creati dagli utenti fornendogli dei\glossario{connettori}.
\section{Glossario}
Al fine di evitare ogni ambiguità di linguaggio e massimizzare la comprensione dei documenti, i termini tecnici, di dominio, gli acronimi e le parole che necessitano di essere chiarite, sono riportate nel \textit{Glossario v2.0.0}.\\
Ogni occorrenza di vocaboli presenti nel \textit{Glossario} è marcata da una "G" maiuscola in pedice.
\section{Riferimenti}
\subsection{Normativi}
\label{normativi}
\begin{itemize}
	\item \textbf{Regole del progetto didattico}\footnote{\url{https://www.math.unipd.it/~tullio/IS-1/2018/Dispense/P01.pdf}};
	\item  \textbf{Regolamento organigramma}\footnote{\url{https://www.math.unipd.it/~tullio/IS-1/2018/Progetto/RO.html}};
	\item  \textbf{Norme di Progetto}: \textit{Norme di Progetto v2.0.0}.
\end{itemize}
\subsection{Informativi}
\begin{itemize}
	\item \textbf{Capitolato$_{G}$  C4}: \textit{MegAlexa$_{G}$}: arricchitore di \textit{skill$_{G}$} di Amazon \textit{Alexa$_{G}$}.
\end{itemize}
\section{Modello  di sviluppo}
\label{Modello di sviluppo}
%La scelta del ciclo di sviluppo risulta cruciale per organizzare e controllare le attività. Visti i vincoli e la complessità del progetto %viene scelto il modello incrementale. Questo ci permette di effettuare rilasci multipli e frequenti e di classificare i requisiti in %ordine di importanza, consentendo di concentrarsi sulle funzionalità con importanza più elevata. Per organizzare ogni %incremento prendiamo spunto dal concetto di\glossario{sprint}del\glossario{Framework}\glossario{Scrum}, essendo anche il %progetto didattico in parte orientato verso le metodologie agili. Questo ci permette di gestire meglio la complessità %del\glossario{progetto}.
La scelta del modello di sviluppo risulta cruciale per organizzare e controllare le attività dell'intero progetto. Visti i vincoli e la complessità del progetto viene scelto di seguire il modello incrementale. Questo ci permette di classificare i requisiti in base alla loro priorità, privilegiando quelli a priorità maggiore per l'effettiva implementazione. Inoltre le funzionalità che implementano i requisiti con maggiore priorità saranno anche quelle soggette a maggior verifica e rilasciando in modo continuo è possibile avere un prototipo funzionante del prodotto anticipatamente. \\
Ogni incremento è della durata di una settimana ed è diviso in fasi:
\begin{itemize}
	\item il primo giorno si assegnano i task ai componenti del gruppo;
	\item nei giorni 2-5 vengono completati i task assegnati;
	\item nei giorni 6-7 viene verificato il lavoro svolto.
\end{itemize}
Il numero di incrementi corrisponde al numero di settimane descritti nel capitolo \ref{Pianificazione}.
\section{Scadenze}
Di seguito vengono riportate le scadenze che il gruppo \textit{ZeroSeven$_{G}$} ha deciso di rispettare e sulle quali si baserà la pianificazione dell'intero progetto:
\begin{itemize}
	\item \textit{Revisione dei Requisiti}: 2019-01-21;
	\item \textit{Revisione di Progettazione}: 2019-03-15;
	\item \textit{Revisione di Qualifica}: 2019-04-19;
	\item \textit{Revisione di Accettazione}: 2019-05-17.
\end{itemize}