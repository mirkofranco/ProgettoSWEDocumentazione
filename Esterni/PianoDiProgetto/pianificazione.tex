\chapter{Pianificazione}
Lo sviluppo del progetto è diviso in cinque periodi:
\begin{itemize}
    \item \textbf{Analisi} AN;
    \item \textbf{Analisi Dettaglio} AD;
    \item \textbf{Progettazione Architetturale} PA;
    \item \textbf{Progettazione di Dettaglio e Codifica} (PDC);
    \item \textbf{Verifica e Validazione} (VV);
\end{itemize}
In ogni periodo sono presenti delle attività da svolgere, alle quali sono associate una o più risorse. La Verifica viene svolta per ogni attività, per semplificare il processo di Validazione.\\
Le attività sono suddivise in più sotto-attività.\\
Nel Gantt$_{G}$ vengono riportate:
\begin{itemize}
    \item attività: le attività possono essere di due tipi:
    \begin{itemize}
        \item Attività critiche: attività dipendenti da altre attività. Un ritardo su queste attività potrebbe rallentare il lavoro del gruppo e causare un ritardo sul raggiungimento della milestone$_{G}$ ;
        \item Attività non critiche: attività che possono essere svolte in parallelo alle attività critiche. Un ritardo su queste attività non provoca ritardi sulle altre;
    \end{itemize}
    \item milestone$_{G}$: rappresenta la data ultima prevista per il completanento di un insieme prestabilito di attività. Ha durata di 0 (zero) giorni e coincide con la data della successiva revisione o l'approvazione complessiva di tali attivita. È rappresentata nel $_Gantt$ con un rombo nero
    \item sotto-attività: le sotto attività sono necessarie per completare l'attività che compongono. Nel Gantt$_{G}$ sono rappresentate con una linea nera. 
\end{itemize}
\section{Analisi}
L'Analisi inizia in concomitanza con la pubblicazione dei capitolati d’appalto e termina alla Revisione dei Requisiti (RR).
Le attività della fase di Analisi sono:
\begin{itemize}
    \item \textbf{Studio di Fattibilità:} vengono valutati tutti i capitolati d'appalto, la valutazione è basata sull'interesse personale di ogni membro del gruppo, sulla complessità prevista, sui rischi che possono emergere. Viene data anche data una descrizione generale del capitolato e una analisi preliminare.\\Viene svolta come prima attività, in quanto bloccante per l'Analisi dei requisiti;
    \item \textbf{Norme di Progetto:} le norme di progetto contengono le regole che il gruppo dovrà seguire durante l'attuazione di tutte le attività. I verificatori certificheranno il rispetto delle norme;
    \item \textbf{Analisi dei Requisiti:} l'analisi preliminare nello Studio di Fattibilità viene approfondita.\\Tale attività continuerà sino alla data di consegna;
    \item \textbf{Piano di Progetto:} Il Responsabile del gruppo redige il piano di progetto, rispettando i vincoli posti dal proponente.\\Questo documento ha l'obiettivo di regolare le attività svolte dal gruppo;
    \item \textbf{Piano di qualifica:} Il piano di qualifica contiene le strategie utili al raggiungimento degli obiettivi del proponente, del gruppo e inerenti alla qualità dei processi di sviluppo.\\L'obiettivo è quello di rendere la qualità quantificabile e misurabile;
    \item \textbf{Glossario:} scritto in modo incrementale. Contiene la spiegazione dei termini più tecnici utilizzati nei documenti. Questa attività è svolta da tutti i membri del gruppo, in parallelo con la redazione di tutti i documenti.
    \item \textbf{Lettera di presentazione:} documento presentato al committente che permette al gruppo di partecipare alla gara d’appalto per il capitolato.
\end{itemize}
I ruoli maggiormente coinvolti sono: Responsabile, Amministratore e Analista.
\section{Analisi Dettaglio}
\section{Progettazione Architetturale}
\section{Progettazione di Dettaglio e Codifica}
\section{Verifica e Validazione}
