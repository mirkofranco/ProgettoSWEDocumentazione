\chapter{Pianificazione}
Lo sviluppo del progetto è diviso in cinque periodi:
\begin{itemize}
    \item \textbf{Analisi} AN;
    \item \textbf{Analisi Dettaglio} AD;
    \item \textbf{Progettazione Architetturale} PA;
    \item \textbf{Progettazione di Dettaglio e Codifica} (PDC);
    \item \textbf{Verifica e Validazione} (VV);
\end{itemize}
In ogni periodo sono presenti delle attività da svolgere, alle quali sono associate una o più risorse. La Verifica viene svolta per ogni attività, per semplificare il processo di Validazione.\\
Le attività più complesse sono suddivise in più sotto-attività.\\Più attività vengono raggruppate in una baseline$_{G}$. Le baseline$_{G}$ permettono al team di misurare il carico di lavoro, individuando dei punti di progresso. Una volta definita, la baseline$_{G}$ può essere modificata solo se è strettamente necessario. Quando tutte le attività che lo caratterizzano sono state soddisfate, il Responsabile può approvare la baseline.\\
Nel Gantt$_{G}$ vengono riportate:
\begin{itemize}
    \item attività: le attività possono essere di due tipi:
    \begin{itemize}
        \item Attività critiche: attività dipendenti da altre attività. Un ritardo su queste attività potrebbe rallentare il lavoro del gruppo e causare un ritardo sul raggiungimento della milestone$_{G}$ ;
        \item Attività non critiche: attività che possono essere svolte in parallelo alle attività critiche. Un ritardo su queste attività non provoca ritardi sulle altre;
    \end{itemize}
    \item milestone$_{G}$: rappresenta la data di raggiungimento di una specifica baseline$_{G}$. Ha durata di 0 (zero) giorni e coincide con la data della successiva revisione o l'approvazione della baseline$_{G}$;
    \item sotto-attività: sono rappresentate in nero. 
\end{itemize}