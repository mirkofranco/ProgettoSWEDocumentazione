\chapter{Estensione delle funzionalità}\label{estensione}
In questo capitolo vengono descritti i passaggi per l'estensione del prodotto \textit{MegAlexa}.

\section{Estensione app}
L'estensione della app avviene mediante l'aggiunta di nuovi blocchi.\\
Segue una lista di attività da attuare che descrive i passi da eseguire per un'estensione corretta:

\begin{itemize}
	\item Definire un blocco che derivi dall'interfaccia \texttt{Block} e, se necessario, l'interfaccia \texttt{Filtrable} posta nel package \textbf{blocks};
	\item Se il blocco richiede una validazione tramite chiamata esterna, definire un\glossario{connettore}che estenda \texttt{Connector} nel package \textbf{connectors};
	\item Se la chiamata esterna prevede l'utilizzo di una\glossario{API}particolare, definire tali interazioni incapsulandole in una nuova classe da posizionare nel package \textbf{network};
	\item Definire le conversioni da\glossario{JSON}e a JSON dichiarando una nuova classe che derivi da \texttt{BlockService} nel package \textbf{service}, per la comunicazione con \textit{API Gateway$_{G}$};
	\item Ampliare la lista presente in \texttt{CreateBlockActivity} (package \textbf{activities}) in modo che presenti un elemento cliccabile in più;
	\item Definire un widget a scelta (\textit{activity} o \textit{fragment}) che rappresenti l'inserimento dei dati da parte dell'utente;
	\item Inserire il listener dell'elemento aggiunto alla lista nel widget appena creato;
	\item Nel caso in cui sia stato creato un fragment ampliare la funzione \textit{onFragmentClick(sender)} in modo che registri e notifichi alle\glossario{activity}interessate le informazioni per l'aggiunta del blocco, in alternativa, se è stata creata una nuova activity, gestire l'interazione con essa mediante le funzioni \textit{startActivityForResult(intent,context)} e \textit{onActivityResult(requestCode,resultCode,data)};
	\item Nella \texttt{CreateWorkflowActivity} e \texttt{ViewBlockActivity}, nelle funzioni \textit{onActivityResult(requestCode,resultCode,data)} 	prelevare i dati passati dalle altre activity e chiamare il viewModel di conseguenza;
	\item Aggiornare il viewModel in modo che supporti l'aggiunta del blocco appena creato.
\end{itemize}



\section{Estensione skill (lambda)}
La Skill può essere estesa in questi modi:
\begin{itemize}
    \item Aggiunta di un nuovo blocco \S\ref{newBlock};
    \item Aggiunta di nuove frasi per Alexa \S\ref{newFrasi};
    \item Aggiunta di un nuovo metodo d'accesso al database \S\ref{newAccesso};
    \item Aggiunta di un nuovo sistema di autenticazione \S\ref{newAuth}.
\end{itemize}
\subsection{Nuovo blocco}\label{newBlock}
Un nuovo blocco può essere \textbf{semplice}, \textbf{filtrabile}, \textbf{filtrabile + elicit}, \textbf{elicit}.
Per creare un nuovo blocco, bisogna seguire i seguenti passaggi:
\begin{itemize}
	\item Creare una classe che rappresenta il blocco ed implementa \textit{Block} (oppure implementa \textit{Filterable} se il blocco è \textbf{filtrabile} o \textbf{filtrabile + elicit}), implementando i metodi rispettando le loro firme;
	\item Nella classe \textit{workflow$_{G}$}, all'attributo \textit{createBlockCommands}, aggiungere una coppia chiave - funzione. La chiave è una stringa contenente il nome del blocco, la funzione è \textit{Promise<Block> => Promise.resolve(new BlockClassName(config))}, dove BlockClassName è il nome della nuova classe;
	\item \textbf{Solo se il blocco è elicit oppure filterabile + elicit:} Creare la classe che rappresenta il connettore del blocco, implementando Connector. Questa classe deve occuparsi di tutte le connessioni con servizi esterni alla \textit{skill$_{G}$}.
\end{itemize}
\subsection{Nuove frasi per Alexa}\label{newFrasi}
Il file phrases-EN.json e phrases-IT.json contengono tutte le frasi personalizzate che Alexa può dire. A ogni parola chiave (uguale per tutte le lingue) sono associate delle frasi (tradotte nella lingua relativa al file).\\
Per inserire una nuova frase, è necessario inserirla nel relativo oggetto associato alla parola chiave. E' importante inserire la frase tradotta per tutti i file phrases-LINGUA.json.\\
Per inserire una nuova parola chiave:
\begin{itemize}
	\item Inserire un oggetto contenente come chiave la \textbf{parola chiave} e come contenuto una lista di stringhe, contenenti le frasi;
	\item Ripetere l'operazione precedente per ogni lingua;
	\item Inserire un nuovo metodo nella classe \textit{PhrasesGenerator.ts}, che ritorna la frase che Alexa deve dire;
	\item L'attributo statico \textbf{jsonPhrases} contiene il file \textit{phrases-LINGUA.json} (la selezione della lingua viene fatta in automatico);
	\item Chiamare questo metodo in modo statico dove è necessario.
\end{itemize}
Per inserire una nuova lingua:
\begin{itemize}
	\item Inserire un nuovo file \textit{phrases-LINGUA.json}, contenente tutte le parole chiavi presenti negli altri file di configurazione e le relative frasi tradotte;
	\item Nel metodo setLanguage aggiungere un case con la nuova lingua.
\end{itemize} 
\subsection{Nuovo metodo d'accesso al database} \label{newAccesso}
Tutti gli accessi al database, per quanto riguarda il download dei \textit{workflow$_{G}$}, avvengono attraverso la classe \textbf{WorkflowService}, nel metodo \textit{workflowFromDatabase}.\\
\'{E} possibile aggiungere una nuova funzione in \textbf{WorkflowService}, contenente un nuovo metodo d'accesso a un diverso database. Per esempio, può essere fatto attraverso una chiamata HTTP a un servizio REST, oppure usando un \textit{web socket$_{G}$}. \'{E} necessario che il valore ritornato sia una lista di blocchi.\\
Lo schema del blocco si può trovare nella cartella \textit{JSONconfigurations}, nel file \textit{JSONconfiguration.ts}, con la chiave (\textbf{BlockJSON}).

\subsection{Nuovo sistema di autenticazione}\label{newAuth}
L'autenticazione, nella versione 1.0.0 della \textit{skill$_{G}$}, avviene solo attraverso Amazon.\\
Il file contenente la richiesta dell'utente (dal Alexa) contiene anche un \textbf{accessToken}. Questo viene utilizzato dalla classe \textbf{User}, dal metodo \textit{credentialsByAccessToken}, per estrarre l'\textbf{AmazonID} dell'utente (lo stesso salvato nel database).\\
Questo sistema è basato su oauth, anche se tutta la parte a basso livello è gestita da Amazon.\\
Per aggiungere un altro sistema di autenticazione basato su oauth:
\begin{itemize}
	\item Creare una nuova configurazione Oauth e ottenere i dati d'accesso (URI, clientID, secret...);
	\item Aprire il pannello di controllo di Alexa\footnote{\url{https://developer.amazon.com/alexa/console/ask}};
	\item Accedere alla skill MegAlexa;
	\item Aprire la sezione "account linking";
	\item Inserire i dati richiesti, generati da \textit{OAuth$_{G}$};
	\item Nella skill MegAlexa, modificare la funzione \textit{credentialsByAccessToken}, nella classe \textbf{User}, così che sia compatibile con il nuovo servizio (ciò che cambia sarà l'ID e la chiamata GET per ottenerlo).
\end{itemize}