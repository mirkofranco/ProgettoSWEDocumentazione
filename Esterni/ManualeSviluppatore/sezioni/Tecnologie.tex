\chapter{Tecnologie utilizzate}
\label{Tecnologie}
\section{Amazon Web Service}
Amazon Web Service è una piattaforma di cloud computing sicura che offre servizi di calcolo, memorizzazione, distribuzione di contenuti e altre funzionalità per aiutare il businesses ad essere scalabile e crescere con facilità. AWS fornisce infatti prodotti e servizi per costruire applicazioni, anche sofisticate, in modo flessibile, scalabile, economico e  con un'ottima resistenza ai guasti. 
\subsection{AWS DynamoDB}
Amazon DynamoDB è un database che supporta i modelli di dati di tipo documento e di tipo chiave-valore che offre prestazioni di pochi millisecondi a qualsiasi livello. Si tratta di un database multi master, multi regione e completamente gestito che offre sicurezza integrata, backup, ripristino e cache in memoria per applicazioni Internet. DynamoDB può gestire oltre 10 trilioni di richieste al giorno e supporta picchi di oltre 20 milioni di richieste al secondo. 
\subsection{AWS Lambda}
AWS Lambda consente di eseguire codice senza dover effettuare il provisioning né gestire il server. Le tariffe sono calcolate in base ai tempi di elaborazione.\\ 
Con Lambda, è possibile eseguire codice per qualunque tipo di applicazione o di servizio back-end, senza alcuna amministrazione. Una volta caricato il codice Lambda si prende carico delle azioni necessarie per eseguirlo e ricalibrarne le risorse con la massima disponibilità. \'E possibile configurare il codice in modo che venga attivato automaticamente da altri servizi AWS oppure che venga richiamato direttamente da qualsiasi app Web o mobile.
\subsection{AWS API Gateway}
AWS API Gateway è un servizio completamente gestito che semplifica la creazione, la pubblicazione, la manutenzione e la protezione delle API su larga scala. Con semplicità è possibile creare e configurare API REST che fungano da "porta di ingresso" per le applicazioni, per consentire l'accesso ai dati, alla logica di business o alle funzionalità dai propri servizi back-end. API Gateway gestisce tutte le attività di accettazione ed elaborazione relative a centinaia di migliaia di chiamate ad API simultanee, inclusi gestione del traffico, controllo di accessi e autorizzazioni, monitoraggio e gestione delle versione delle API. Gateway non prevede alcuna tariffa minima né investimenti iniziali. Vengono addebitati solo i costi di chiamate API ricevute e i volumi di dati trasferiti in uscita e con il modello tariffario a scaglioni di API Gateway potrai ridurre i costi al variare dell'utilizzo delle API.
\subsection{AWS CloudWatch}
AWS CloudWatch è un servizio di monitoraggio e gestione creato per gli sviluppatori, operatori di sistema, ingegneri responsabili del sito e manager IT. CloudWatch fornisce dati e analisi concrete per monitorare le applicazioni, capire e rispondere ai cambiamenti di prestazioni a livello di sistema, ottimizzare l'utilizzo delle risorse e ottenere una visualizzazione unificata dello stato di integrità operativa. 