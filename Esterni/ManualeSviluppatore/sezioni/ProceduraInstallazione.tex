\chapter{Procedura di installazione}

\section{Installazione Skill}
\label{installazioneSkill}
Per ognuno dei sequenti comandi è richiesta l'installazione del package manager \textbf{npm}.
Installazione della skill e delle sue dipendenze:
\begin{itemize}
    \item clonare la repository attraverso il comando \textit{git clone\\https://github.com/sgt390/MegAlexaSkill/};
    \item eseguire il comando \textit{npm install} per installare automaticamente le dipendenze.
\end{itemize}
Pubblicare la skill in AWS Lambda:
\begin{itemize}
    \item installare il programma 7z e inserire il suo eseguibile tra le variabili di sistema;
    \item installare e configurare aws-cli\footnote{\url{https://aws.amazon.com/it/cli/}};
    \item da terminal, eseguire il comando \textit{npm run publish-lambda}.
\end{itemize}
Eseguire i test di unità:
\begin{itemize}
    \item da terminal, eseguire il comando \textit{npm run unitTest}.
\end{itemize}
Eseguire i test di integrazione:
\begin{itemize}
    \item da terminal, eseguire il comando \textit{npm integrationTest}.
\end{itemize}


\section{Installazione App}
\label{installazioneApp}

Per lo sviluppo dell'applicazione è necessario eseguire i seguenti passi:
\begin{itemize}
	\item clonare la repository attraverso il comando \textit{git clone\\https://github.com/sgt390/ProgettoSweCodice/};
	\item fare il login con le proprie credenziali di Amazon developer\footnote{\url{ https://developer.amazon.com}};
	\item cliccare su Settings e successivamente su Security Profiles;
	\item dopo aver creato un profilo di sicurezza aggiungere una chiave API nella sezione Impostazioni Android/Kindle
	\item creare la cartella assets nella repository \texttt{ProgettoSweCodice\textbackslash MegAlexa\textbackslash app\textbackslash  src\textbackslash main\textbackslash}
	\item creare il file \texttt{api\_key.txt} nella cartella \texttt{ProgettoSweCodice\textbackslash MegAlexa\textbackslash app\textbackslash  src\textbackslash main\textbackslash assets} della repository e copiare al suo interno la chiave precedentemente creata.
	
\end{itemize}
