\documentclass[a4paper,12pt]{article}

\usepackage{ZeroSeven}

\titlepage{}

\author{Bianca Ciuche}
\date{2018-12-27}
\intestazioni{Gruppo ZeroSeven}
\pagestyle{myfront}
\begin{document}
\begin{titlepage}
	\centering
	{\huge\bfseries MegAlexa\par}
	Arricchitore di skill di Amazon Alexa
	\line(1,0){350} \\
	{\scshape\LARGE Verbale 2018-12-27 \par}
	\vspace{1cm}
	{\scshape Gruppo ZeroSeven \par}
	\logo
	%devono essere compilati questi campi ogni volta
	\begin{tabular}{c|c}
		{\hfill \textbf{Versione}} 			& 1.0.0				\\
		{\hfill\textbf{Data Redazione}} 	& 2018-12-29		\\ 
		{\hfill\textbf{Redazione}} 			&  		Bianca Ciuche	\\ 
		{\hfill\textbf{Verifica}} 				&  	Stefano Zanatta			\\ 
		{\hfill\textbf{Approvazione}} 		&  		Mirko Franco	\\ 
		{\hfill\textbf{Uso}} 					& 	Esterno	\\ 
		{\hfill\textbf{Distribuzione}} 			& 			Prof. Tullio Vardanega \\ & Prof. Riccardo Cardin \\ & zero12 - Innovation Company \\ & Gruppo ZeroSeven		\\ 
		{\hfill\textbf{Email di contatto}} & zerosevenswe@gmail.com \\
	\end{tabular}
\end{titlepage}
	
		\label{LastFrontPage}
	
	
	\newpage
	\cleardoublepage
	\begin{table}[tbph]
		\centering
		\begin{tabularx}{\textwidth}{|c|c|X|X|c|}
			\hline
			\textbf{Versione} & \textbf{Data} & \textbf{Descrizione} & \textbf{Autore} & \textbf{Ruolo} \\
			\hline
			0.0.1 & 2018-12-28 & Creazione scheletro documento
			& Bianca Ciuche & Analista\\
			\hline
			0.0.2 & 2018-12-29 & Stesura verbale & Bianca Ciuche &Analista \\
			\hline
			0.1.0 & 2018-12-30 & Verifica & Stefano Zanatta &Verificatore \\
			\hline
			1.0.0 & 2018-12-30 & Approvazione per Revisione dei Requisti & Mirko Franco & Responsabile \\
			\hline
		\end{tabularx}
		\caption{Diario delle modifiche}
	\end{table}
	
	\cleardoublepage
	\pagestyle{mymain}
	
	\tableofcontents
	\cleardoublepage

	\section{Informazioni sulla riunione}
	\subsection{Motivo della riunione}
	La riunione è stata effetuata su  \textit{Hangouts} ed è stata convocata attraverso l'invio di una mail all'azienda  \textit{Zero12}. Sono stati esposti e discussi i casi d'uso e i requisiti individuati, dati chiarimenti inerenti il progetto e la sua implementazione.\\
	Di seguito vengono riportate le informazioni dell'incontro:
	\begin{itemize}
		\item \textbf{Luogo e data}: Padova, Giovedì 27 Dicembre 2018
		\item \textbf{Ora di inizio}: 17:00
		\item \textbf{Ora di fine}: 17:40
		\item \textbf{Partecipanti}:  
		\begin{itemize}
			\item Gian Marco Bratzu
			\item Mirko Franco
			\item Ciuche Bianca
			\item Ludovico Brocca
			\item Matteo Depascale
			\item Stefano Zanatta
			\item Sig. Stefano Dindo
		\end{itemize}
	\end{itemize}

	\section{Argomenti discussi}
	La riunione si è focalizzata sulle caratteristiche che la  \textit{skill} dovrà avere: è stata data particolare attenzione sull'implementazione dei \textit{connettori} e dei \textit{workflow}. Il proponente specifica la richiesta di \textit{connettori} generalizzati, in modo che l'utente possa creare dei \textit{workflow} modulari.
	\section{Domande e risposte principali}
	\subsection{Possono essere in esecuzione più \textit{workflow} contemporaneamente?}
	Non possono essere eseguiti contemporaneamente più \textit{workflow} della stessa \textit{skill.}
	\subsection{Dove vengono memorizzati i \textit{workflow}?}
	I \textit{workflow} vengono memorizzati sui servizi web messi a disposizione a supporto della \textit{skill}. La \textit{skill} ha solo servizi esterni, non ci sono servizi memorizzati dentro al \textit{device}. 
	\subsection{Prima di chiamare un \textit{workflow} specifico, bisogna invocare la \textit{skill} che lo contiene ?}
	Si, bisogna prima invocare la \textit{skill} a cui il \textit{workflow} appartiene.
	\subsection{Come vanno risolti i problemi riguardo il riconoscimento di più utenti?}
	Il riconoscimento degli utenti avviene attraverso delle \textit{chiavi} che vengono salvate nel \textit{database}. Una \textit{chiave} corrisponde all'univoca coppia \{nome\_utente, password\} che ha a disposizione i \textit{workflow} che l'utente stesso  ha creato.
	\subsection{I \textit{workflow} di una \textit{skill} possono utilizzare \textit{connettori}  di altre \textit{skill}?}
	Si ma non è una giusta implementazione poiché per fare ciò devono essere installate altre \textit{skill}.
    \subsection{Dal punto di vista della qualità ci sono delle metriche specifiche da rispettare?}
    Una metrica da soddisfare è permettere all'utente di interagire con \textit{l'echo di Alexa} nel linguaggio più naturale possibile, non in maniera robotica, a parte questo non ci sono particolari metriche da rispettare.
    \subsection{È possibile aggiungere dei \textit{blocchi} in futuro?}
    Si, mettendosi d'accordo è possibile aggiungere dei \textit{blocchi}.	
		\label{LastPage}
\end{document}