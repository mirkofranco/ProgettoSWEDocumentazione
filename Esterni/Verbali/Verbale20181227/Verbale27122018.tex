\documentclass[a4paper,12pt]{article}

\usepackage{ZeroSeven}

\titlepage{}

\author{Bianca Ciuche}
\date{2018-12-27}
\intestazioni{\includegraphics[scale=0.3]{images/logo_intestazione}}
\pagestyle{myfront}
\begin{document}
\begin{titlepage}
	\centering
	{\huge\bfseries MegAlexa\par}
	Arricchitore di skill di Amazon Alexa
	\line(1,0){350} \\
	{\scshape\LARGE Verbale Esterno 2018-12-27 \par}
	\vspace{1cm}
	{\scshape Gruppo ZeroSeven \par}
	\logo
	%devono essere compilati questi campi ogni volta
	\begin{tabular}{c|c}
		{\hfill \textbf{Versione}} 			& 1.0.0				\\
		{\hfill\textbf{Data Redazione}} 	& 2018-12-29		\\ 
		{\hfill\textbf{Redazione}} 			&  		Bianca Andreea Ciuche	\\ 
		{\hfill\textbf{Verifica}} 				&  	Stefano Zanatta			\\ 
		{\hfill\textbf{Approvazione}} 		&  		Mirko Franco	\\ 
		{\hfill\textbf{Uso}} 					& 	Esterno	\\ 
		{\hfill\textbf{Distribuzione}} 			& 			Prof. Tullio Vardanega \\ & Prof. Riccardo Cardin \\ & Gruppo ZeroSeven	\\ & Zero12 s.r.l. \\
		{\hfill\textbf{Email di contatto}} & zerosevenswe@gmail.com \\
	\end{tabular}
\end{titlepage}
	
		\label{LastFrontPage}
	
	
	\newpage
	\cleardoublepage
	\begin{table}[tbph]
		\centering
		\begin{tabularx}{\textwidth}{|c|c|X|X|c|}
			\hline
			\textbf{Versione} & \textbf{Data} & \textbf{Descrizione} & \textbf{Autore} & \textbf{Ruolo} \\
			\hline
			1.0.0 & 2018-12-30 & Approvazione per Revisione dei Requisti & Mirko Franco & Responsabile \\
			\hline
			0.1.0 & 2018-12-30 & Verifica & Stefano Zanatta &Verificatore \\
			\hline
			0.0.2 & 2018-12-29 & Stesura verbale & Bianca Andreea Ciuche &Analista \\
			\hline
			0.0.1 & 2018-12-28 & Creazione template documento
			& Ludovico Brocca & Amministratore\\
			\hline
		\end{tabularx}
		\caption{Diario delle modifiche}
	\end{table}
	
	\cleardoublepage
	\pagestyle{mymain}
	
	\tableofcontents
	\cleardoublepage

	\section{Informazioni sulla riunione}
	\subsection{Motivo della riunione}
	La riunione è stata effetuata su\glossario{Hangouts}ed è stata convocata attraverso l'invio di una mail all'azienda Zero12. Sono stati esposti e discussi i casi d'uso e i\glossario{requisiti}individuati, dati chiarimenti inerenti il\glossario{progetto}e la sua implementazione.\\
	Di seguito vengono riportate le informazioni dell'incontro:
	\begin{itemize}
		\item \textbf{Luogo e data}: Padova, Giovedì 27 Dicembre 2018;
		\item \textbf{Ora di inizio}: 17:00;
		\item \textbf{Ora di fine}: 17:40;
		\item \textbf{Partecipanti}:  
		\begin{itemize}
			\item Gian Marco Bratzu;
			\item Mirko Franco;
			\item Ciuche Bianca;
			\item Ludovico Brocca;
			\item Matteo Depascale;
			\item Stefano Zanatta;
			\item Sig. Stefano Dindo.
		\end{itemize}
	\end{itemize}

	\section{Argomenti discussi}
	La riunione si è focalizzata sulle caratteristiche che la\glossario{skill}dovrà avere: è stata data particolare attenzione sull'implementazione dei\glossario{connettori}e dei\glossario{workflow}. La proponente specifica la richiesta di connettori generalizzati, in modo che l'utente possa creare dei workflow modulari.
	\section{Domande e risposte principali}
	\subsection{Possono essere in esecuzione più workflow contemporaneamente?}
	Non possono essere eseguiti contemporaneamente più workflow della stessa skill.
	\subsection{Dove vengono memorizzati i workflow?}
	I\glossario{workflow}vengono memorizzati sui servizi web messi a disposizione a supporto della\glossario{skill}. La skill ha solo servizi esterni, non ci sono servizi memorizzati dentro al device. 
	\subsection{Prima di chiamare un workflow specifico, bisogna invocare la skill che lo contiene ?}
	Si, bisogna prima invocare la skill a cui il workflow appartiene.
	\subsection{Come vanno risolti i problemi riguardo il riconoscimento di più utenti?}
	Il riconoscimento degli utenti avviene attraverso delle chiavi che vengono salvate nel database. Una chiave corrisponde all'univoca coppia \{nome\_utente, password\} che ha a disposizione i\glossario{workflow}che l'utente stesso  ha creato.
	\subsection{I workflow di una skill possono utilizzare connettori  di altre skill?}
	Si ma non è una giusta implementazione poiché per fare ciò devono essere installate altre\glossario{skill}.
    \subsection{Dal punto di vista della qualità ci sono delle metriche specifiche da rispettare?}
    Una metrica da soddisfare è permettere all'utente di interagire con l'echo di\glossario{Alexa}nel linguaggio più naturale possibile, non in maniera robotica, a parte questo non ci sono particolari metriche da rispettare.
    \subsection{È possibile aggiungere dei blocchi in futuro?}
    Si, mettendosi d'accordo è possibile aggiungere dei blocchi.
		\label{LastPage}
\end{document}