\documentclass[a4paper,12pt]{article}

\usepackage{ZeroSeven}

\titlepage{}
\author{Matteo Depascale}
\date{2018-12-08}
\intestazioni{\includegraphics[scale=0.3]{images/logo_intestazione}}
\pagestyle{myfront}
\begin{document}
\begin{titlepage}
	\centering
	{\huge\bfseries MegAlexa\par}
	Arricchitore di skill di Amazon Alexa
	\line(1,0){350} \\
	{\scshape\LARGE Verbale Esterno del 2018-12-05 \par}
	\vspace{1cm}
	{\scshape Gruppo ZeroSeven \par}
	\logo
	%devono essere compilati questi campi ogni volta
	\begin{tabular}{c|c}
		{\hfill \textbf{Versione}} 			& 1.0.0				\\
		{\hfill\textbf{Data Redazione}} 	& 2018-12-08		\\
		{\hfill\textbf{Redazione}} 			&  		Matteo Depascale			\\
		{\hfill\textbf{Verifica}} 				&  		Bianca Andreea Ciuche		\\
		{\hfill\textbf{Approvazione}} 		&  		Mirko Franco	\\
		{\hfill\textbf{Uso}} 					& 	Esterno	\\
		{\hfill\textbf{Distribuzione}} 			& 			Prof. Tullio Vardanega \\ & Prof. Riccardo Cardin \\ & Gruppo ZeroSeven \\ & Zero12 s.r.l. \\
		{\hfill\textbf{Email di contatto}} & zerosevenswe@gmail.com \\
	\end{tabular}
\end{titlepage}



	\label{LastFrontPage}


	\newpage
	\cleardoublepage
		\begin{table}[tbph]
		\centering
		\begin{tabularx}{\textwidth}{|c|c|X|X|c|}
			\hline
			\textbf{Versione} & \textbf{Data} & \textbf{Descrizione} & \textbf{Autore} & \textbf{Ruolo} \\
			\hline
			1.0.0 & 2018-12-13 & Approvazione per Revisione dei Requisiti & Mirko Franco & Responsabile \\
			\hline
			0.0.3 & 2018-12-11 & Verifica con esito positivo & Bianca Andreea Ciuche & Verificatore \\
			\hline
			0.0.2 & 2018-12-08 & Stesura verbale & Matteo Depascale & Verificatore \\
			\hline
			0.0.1 & 2018-12-08 & Creazione template documento
			& Ludovico Brocca & Amministratore\\
			\hline
		\end{tabularx}
		\caption{Diario delle modifiche}
	\end{table}

	\cleardoublepage
	\pagestyle{mymain}

	\tableofcontents
	\cleardoublepage

	\section{Informazioni sulla riunione}
	\subsection{Motivo della riunione}
	La riunione è stata convocata attraverso l'invio di una mail all'azienda Zero12 per avere un quadro più completo del\glossario{capitolato}e per ottenere delle delucidazioni su argomenti poco chiari. \\
	Di seguito vengono riportate le informazioni dell'incontro:
	\begin{itemize}
		\item \textbf{Luogo e data}: sede dell'azienda Zero12, Mercoledì 5 Dicembre 2018;
		\item \textbf{Ora di inizio}: 15:00;
		\item \textbf{Ora di fine}: 16:00;
		\item \textbf{Partecipanti}:
		\begin{itemize}
		\item Gian Marco Bratzu;
		\item Mirko Franco;
		\item Andrea Deidda;
		\item Matteo Depascale;
		\item Sig. Stefano Dindo.
		\end{itemize}
	\end{itemize}

	\section{Domande e risposte}
	\subsection{Introduzione al progetto}
	Per i primi minuti viene data un'introduzione al core del\glossario{progetto}: un utente, registrato con successo alla piattaforma, avrà la possibilità di creare dei\glossario{workflow}nei quali si ha  la possibilità di scegliere un titolo e  dei blocchi prefatti, per esempio:
	\begin{itemize}
		\item blocco 1: contenente del testo;
		\item blocco 2: contenente il feeder RSS reader dell'URL proposto;
		\item blocco 3: contenente un filtro.
	\end{itemize}
	A questo punto, dopo aver salvato, ci si aspetta che ci sia una\glossario{skill}che chiede quale\glossario{workflow}vuole essere eseguito. Verrà quindi letto il messaggio scritto nel blocco 1 e successivamente il contenuto del blocco 2 dopo aver filtrato in base alle scelte del blocco 3.

	\subsection{Un utente può avere più workflow?}
	Sì, non è possibile avere più\glossario{workflow}con lo stesso nome.

	\subsection{Dovranno essere sviluppati i connettori?}
	Sì, verranno decisi assieme quali connettori sviluppare.

	\subsection{E' possibile sviluppare il progetto solamente in mobile?}
	Sì, anche se sviluppare il\glossario{progetto}tramite interfaccia web sarebbe  più semplice poiché basterebbe renderla responsive per averla a disposizione anche nel mobile.

	\subsection{E' possibile fare anche una versione sperimentale per iOS?}
	Certo, inoltre l'idea del progetto è stata presa da un'applicazione per\glossario{iOS}chiamata Shortcuts. \\
	Volendo fare sia\glossario{Android}che iOS è importante che le due applicazioni non siano completamente sbilanciate, in quanto avere un'applicazione iOS con poche funzionalità è abbastanza inutile e pertanto è preferibile non farla.

	\subsection{Possibili moduli da implementare?}
	Ci saranno i seguenti moduli di base:
	\begin{itemize}
		\item modulo testo, utile per l'iterazione con l'utente;
		\item modulo feeder RSS;
		\item modulo social (Twitter piuttosto che Facebook);
		\item modulo filtro;
		\item modulo musica;
		\item modulo per l'invio di messaggi o mail.
	\end{itemize}

	\subsection{In che modo sarà possibile fare i test?}
	Sarà possibile svolgere i test utilizzando l'account developer di\glossario{Alexa}. E' disponibile una sezione test dove poter interagire con l'assistente vocale, l'unico svantaggio è che non supporta il ritorno di un file audio. Inoltre sarà possibile svolgere i test con\glossario{Amazon}Echo prendendo un appuntamento con l'azienda.

	\subsection{Che cosa si intende per Voice Dialog Flow?}
	Il \textit{Voice Dialog}\glossario{Flow}è un documento (di testo) di story telling che può essere visto come una chat messaggio tra l'utente che interagisce e Alexa che risponde. \\
	Questo argomento verrà spiegato più in dettaglio durante la lezione del 14 Dicembre. Dovrà essere incluso nel documento \textit{Analisi dei Requisiti}.

	\subsection{Tipo di basi di dati da utilizzare?}
	Si consiglia di usare come NoSQL:\glossario{DynamoDB}piuttosto che MongoDB, entrambi utilizzano JSON. Essendo entrambi i database documentali, si dovrà spiegare la struttura del documento che andremo a fare a supporto del progetto. Oppure si potrà valutare l'opzione di Amazon \textit{Aurora}\glossario{Serverless}utilizzando quindi un database relazionale.

	\subsection{I documenti devono essere redatti in inglese?}
	Non è necessario, però i commenti del codice andranno fatti in inglese.

	\subsection{Lingue da implementare nel progetto?}
	Come lingua principale ci sarà inglese, mentre come secondaria Italiano.

	\subsection{Cosa si intende per piano di test di unità?}
	Il piano di test di unità è principalmente usato per la parte server, ovvero bisognerà fare dei test per verificare che l'\textit{API$_{G}$} creata risponda alle richieste. \\
	Inoltre nel blog di\glossario{Amazon}sono presenti degli articoli che suggeriscono come implementare i test in modo  automatico sulle \textit{skill$_{G}$}, facendo attenzione poiché è un ambito abbastanza inesplorato. \\
	Dato che bisognerà utilizzare un approccio basato sui test, conviene predisporre un piano di test su un foglio excel formato nella maniera seguente:
	\begin{itemize}
		\item lista dei test;
		\item descrizione;
		\item risultato atteso;
		\item chi ha fatto il test;
		\item risultato reale.
	\end{itemize}

	\subsection{Principali tool da utilizzare?}
	L'azienda Zero12 utilizza CodeCommit, CodePipeline, CodeDeploy per quanto riguarda \textit{Amazon Web}\glossario{Services}. \\
	Sarà inoltre possibile utilizzare servizi che al gruppo\glossario{ZeroSeven}sono più conosciuti in quanto non sono oggetto di valutazione, in caso contrario l'azienda Zero12 fornirà servizi di formazioni in merito.

	\subsection{Quali strumenti verranno messi a disposizione?}
	Verrà messa a disposizione una repository dove caricare il codice sorgente, oppure potrà essere concordato l'utilizzo di\glossario{GitHub}. \\
	Ci sarà la possibilità di concordare delle sessioni di formazioni su argomenti non trattati a lezione, come ad esempio le architetture serverless. \\
	Per quanto riguarda l'account AWS, l'azienda è disponibile a condividere il proprio account se Amazon metterà a disposizione dei crediti.
	\label{LastPage}

\end{document}
