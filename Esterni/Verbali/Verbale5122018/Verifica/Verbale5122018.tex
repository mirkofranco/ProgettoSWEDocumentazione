\documentclass[a4paper,12pt]{article}

\usepackage{ZeroSeven}

\titlepage{}

\author{Matteo Depascale}
\date{8-12-2018}
\intestazioni{Gruppo ZeroSeven}
\pagestyle{myfront}
\begin{document}
\begin{titlepage}
	\centering
	{\huge\bfseries MegAlexa\par}
	Arricchitore di skill di Amazon Alexa
	\line(1,0){350} \\
	{\scshape\LARGE Verbale 5-12-2018 \par}
	\vspace{1cm}
	{\scshape Gruppo ZeroSeven \par}
	\logo
	%devono essere compilati questi campi ogni volta
	\begin{tabular}{c|c}
		{\hfill \textbf{Versione}} 			& 0.0.2				\\
		{\hfill\textbf{Data Redazione}} 	& 8-12-2018		\\ 
		{\hfill\textbf{Redazione}} 			&  		Matteo Depascale			\\ 
		{\hfill\textbf{Verifica}} 				&  				\\ 
		{\hfill\textbf{Approvazione}} 		&  			\\ 
		{\hfill\textbf{Uso}} 					& 	Esterno	\\ 
		{\hfill\textbf{Distribuzione}} 			& 			Prof. Tullio Vardanega \\ & Prof. Riccardo Cardin \\ & zero12 - Innovation Company \\ & Gruppo ZeroSeven		\\ 
		{\hfill\textbf{Email di contatto}} & zerosevenswe@gmail.com \\
	\end{tabular}
\end{titlepage}
	

	
	\label{LastFrontPage}
	

	\newpage
	\cleardoublepage
		\begin{table}[tbph]
		\centering
		\begin{tabularx}{\textwidth}{|c|c|X|X|c|}
			\hline
			\textbf{Versione} & \textbf{Data} & \textbf{Descrizione} & \textbf{Autore} & \textbf{Ruolo} \\
			\hline
			0.0.1 & 2018-12-8 & Creazione scheletro documento
			& Matteo Depascale & Verificatore\\
			\hline
			0.0.2 & 2018-12-8 & Stesura verbale & Matteo Depascale & Verificatore \\
			\hline
		\end{tabularx}
		\caption{Diario delle modifiche}
	\end{table}
	
	\cleardoublepage
	\pagestyle{mymain}
	
	\tableofcontents
	\cleardoublepage
	
	\section{Informazioni sulla riunione}
	\subsection{Motivo della riunione}
	La riunione è stata convocata attraverso l'invio di una mail all'azienda zero12 per avere un quadro più completo del capitolato e per ottenere delle delucidazioni su argomenti poco chiari. \\
	Di seguito vengono riportate le informazioni dell'incontro:
	\begin{itemize}
		\item \textbf{Luogo e data}: sede dell'azienda \textbf{zero12}, Mercoledì 5 Dicembre 2018
		\item \textbf{Ora di inizio}: 15:00
		\item \textbf{Ora di fine}: 16:00
		\item \textbf{Partecipanti}:  
		\begin{itemize}
		\item Gian Marco Bratzu
		\item Mirko Franco
		\item Andrea Deidda
		\item Matteo Depascale
		\end{itemize}
	\end{itemize}

	\section{Domande e risposte}
	\subsection{Introduzione al progetto}
	Per i primi minuti viene data un'introduzione al \textit{core} del progetto:  un utente, registrato con successo alla piattaforma, avrà la possibilità di creare dei \textit{workflow}  nei quali si ha  la possibilità di scegliere un titolo e  dei blocchi prefatti, per esempio:
	\begin{itemize}
		\item blocco 1: contenente del testo
		\item blocco 2: contenente il \textit{feeder RSS reader} dell'\textit{URL} proposto
		\item blocco 3: contenente un filtro
	\end{itemize}
	A questo punto, dopo aver salvato, ci si aspetta che ci sia una \textit{skill} che chiede quale \textit{workflow} vuole essere eseguito. Verrà quindi letto il messaggio scritto nel blocco 1 e successivamente il contenuto del blocco 2 dopo aver filtrato in base alle scelte del blocco 3.
	
	\subsection{Un utente può avere più \textit{workflow}?}
	Sì, non è possibile avere più \textit{workflow} con lo stesso nome.
	
	\subsection{Dovranno essere sviluppati i connettori?}
	Sì, verranno decisi assieme quali connettori sviluppare.
	
	\subsection{E' possibile sviluppare il progetto solamente in mobile?}
	Sì, anche se sviluppare il progetto  tramite interfaccia web sarebbe  più semplice poichè basterebbe renderla responsive per averla a disposizione anche nel mobile.
	
	\subsection{E' possibile fare anche una versione sperimentale per iOS?}
	Certo, inoltre l'idea del progetto è stata presa da un'applicazione per iOS chiamata \textit{Shortcuts}. \\
	Volendo fare sia \textit{Android} che \textit{iOS} è importante che le due applicazioni non siano completamente sbilanciate, in quanto avere un'applicazione \textit{iOS} con poche funzionalità è abbastanza inutile e pertanto è preferibile non farla.
	
	\subsection{Possibili moduli da implementare?}
	Ci saranno i seguenti moduli di base :
	\begin{itemize}
		\item modulo testo, utile per l'iterazione con l'utente 
		\item modulo \textit{feeder RSS} 
		\item modulo social (\textit{Twitter} piuttosto che \textit{Facebook})
		\item modulo filtro
		\item modulo musica
		\item modulo per l'invio di messaggi o mail
	\end{itemize}

	\subsection{In che modo sarà possibile fare i test?}
	Sarà possibile svolgere i test utilizzando l'account \textit{developer} di \textit{Alexa}. E' disponibile una sezione test dove poter interagire con l'assistente vocale, l'unico svantaggio è che non supporta il ritorno di un file audio. Inoltre sarà possibile svolgere i test con \textit{Amazon Echo} prendendo un appuntamento con l'azienda.

	\subsection{Che cosa si intende per \textit{Voice Dialog Flow}?}
	E' un documento (di testo) di \textit{story telling} che può essere visto come una chat messaggio tra l'utente che interagisce e \textit{Alexa} che risponde. \\
	Questo argomento verrà spiegato più in dettaglio durante la lezione del 14 Dicembre.Dovrà essere incluso nel documento dell'\textbf{Analisi dei Requisiti}.
	
	\subsection{Schema di basi di dati da utilizzare?}
	Si consiglia di usare come \textit{NoSQL}: \textbf{dynamodb} piuttosto che \textbf{MongoDB}, entrambi utilizzano \textit{Jason}. Essendo entrambi i database documentali, si dovrà spiegare la struttura del documento che andremo a fare a supporto del progetto. Oppure si potrà valutare l'opzione di \textbf{Amazon Aurora Serverless} utilizzando quindi un \textit{MySQL}.
	
	\subsection{I documenti devono essere redatti in Inglese?}
	Non è necessario, però i commenti del codice andranno fatti in inglese.
	
	\subsection{Lingue da implementare nel progetto?}
	Come lingua principale ci sarà Inglese, mentre come secondaria Italiano.
	
	\subsection{Cosa si intende per piano di test di unità?}
	Il piano di test di unità è principalmente usato per la parte server, ovvero bisognerà fare dei test per verificare che l'\textit{API} creata risponda alle richieste. \\
	Inoltre nel blog di \textit{Amazon} sono presenti degli articoli che suggeriscono come implementare i test in modo  automatico sulle \textit{skills}, facendo attenzione poichè è un ambito abbastanza inesplorato. \\
	Dato che bisognerà utilizzare un approccio basato sui test, conviene predisporre un piano di test su un foglio \textit{excel} formato nella maniera seguente:
	\begin{itemize}
		\item lista dei test
		\item descrizione
		\item risultato atteso
		\item chi ha fatto il test
		\item risultato reale
	\end{itemize}
	
	\subsection{Principali \textit{tool} da utilizzare?}
	L'azienda zero12 utilizza \textit{CodeCommit}, \textit{CodePIpeline}, \textit{CodeDeploy} per quanto riguarda \textit{Amazon Web Services}. \\
	Sarà inoltre possibile utilizzare servizi che al gruppo \textit{ZeroSeven} sono più conosciuti in quanto questi servizi non sono oggetto di valutazione, in caso contrario l'azienda \textbf{zero12} fornirà servizi di formazioni in merito.
	
	\subsection{Quali strumenti verranno messi a disposizione?}
	Verrà messa a disposizione una \textit{repository} dove caricare il codice sorgente, oppure potrà essere concordato l'utilizzo di \textit{GitHub}. \\
	Ci sarà la possibilità di concordare delle sessioni di formazioni su argomenti non trattati a lezione, come ad esempio \textit{le architetture serverless}. \\
	Per quanto riguarda l'account \textit{AWS}, l'azienda è disponibile a condividere il proprio account se \textit{Amazon} metterà a disposizione dei crediti.	
	\label{LastPage}

\end{document}
