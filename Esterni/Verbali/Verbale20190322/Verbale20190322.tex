\documentclass[a4paper,12pt]{article}
\usepackage{ZeroSeven}

\titlepage{}
\author{Andrea Deidda}
\date{2019-03-22}
\intestazioni{\includegraphics[scale=0.3]{images/logo_intestazione}}
\pagestyle{myfront}
\begin{document}
	\begin{titlepage}
		\centering
		{\huge\bfseries MegAlexa\par}
		Arricchitore di skill di Amazon Alexa
		\line(1,0){350} \\
		{\scshape\LARGE Verbale Esterno del 2019-03-22 \par}
		\vspace{1cm}
		{\scshape Gruppo ZeroSeven \par}
		\logo
		%devono essere compilati questi campi ogni volta
		\begin{tabular}{c|c}
			{\hfill \textbf{Versione}} 			& 1.0.0\\
			{\hfill\textbf{Data Redazione}} 	& 2019-03-22\\ 
			{\hfill\textbf{Redazione}} 			&  Andrea Deidda\\ 
			{\hfill\textbf{Verifica}} 			&   Matteo Depascale \\ 
			{\hfill\textbf{Approvazione}} 		&  	Ludovico Brocca \\ 
			{\hfill\textbf{Uso}} 					& 	Esterno	\\ 
			{\hfill\textbf{Distribuzione}} 			& 			Prof. Tullio Vardanega \\ & Prof. Riccardo Cardin \\ & Gruppo ZeroSeven \\ & Zero12 s.r.l.	\\ 
			{\hfill\textbf{Email di contatto}} & zerosevenswe@gmail.com \\
		\end{tabular}
	\end{titlepage}

	\label{LastFrontPage}
	
	\newpage
	\cleardoublepage
	\begin{table}[tbph]
		\centering
		\begin{tabularx}{\textwidth}{|c|c|X|X|c|}
			\hline
			\textbf{Versione} & \textbf{Data} & \textbf{Descrizione} & \textbf{Autore} & \textbf{Ruolo} \\
			\hline
			1.0.0 & 2019-03-25 & Approvazione documento per il rilascio & Ludovico Brocca  & Responsabile \\
			\hline
			0.1.0 & 2019-03-24 & Verifica documento & Matteo Depascale  & Verificatore \\
			\hline
			0.0.2 & 2019-03-22 & Stesura verbale & Andrea Deidda  & Analista \\
			\hline
			0.0.1 & 2018-12-08 & Creazione template documento & Ludovico Brocca & Amministratore\\
			\hline
		\end{tabularx}
		\caption{Registro delle modifiche}
	\end{table}
	\cleardoublepage
	\pagestyle{mymain}
	
	\tableofcontents
	\cleardoublepage
	\section{Informazioni sulla riunione}
	\subsection{Motivo della riunione}\`{E} stata richiesta questa riunione per aggiornare la\glossario{proponente}sul lavoro che stiamo svolgendo e per chiarire alcuni dubbi sorti durante implementazione del\glossario{prodotto}software.
	
	\begin{itemize}
		\item \textbf{Luogo e data}: \textit{Google Hangouts$_{G}$}, Venerdì 22 Marzo 2019;
		\item \textbf{Ora di inizio}: 9:30;
		\item \textbf{Ora di fine}: 10:30;
		\item \textbf{Partecipanti}:  
		\begin{itemize}
			\item Gian Marco Bratzu;
			\item Bianca Ciuche;
			\item Ludovico Brocca;
			\item Andrea Deidda;
			\item Matteo Depascale;
			\item Stefano Zanatta;
			\item Sig. Stefano Dindo.
		\end{itemize}
	\end{itemize}
	
	\section{Resoconto}
	\subsection{Argomenti discussi}
	Nel corso della riunione effettuata mediante Google Hangouts sono state discusse varie problematiche emerse durante la progettazione e l'implementazione dei blocchi dell'app.
	\begin{itemize}
		\item {Si è discusso di una possibile soluzione al problema di come attivare\glossario{Alexa}una volta che l'utente ha creato un blocco sveglia all'interno del suo \textit{workflow$_{G}$}. Dopo un'analisi approfondita del problema si è ritenuto che il blocco Sveglia possa essere trascurato dall'implementazione;}
		\item {Il blocco Pianificazione presenta gli stessi problemi del blocco sveglia, e per questo è stato rimosso dai requisiti obbligatori};
		\item {La progettazione del blocco Invio Email si è dimostrato un compito più complesso di quanto preventivato in periodo di Analisi. Per questo motivo il gruppo, con il consenso della proponente, i requisiti riguardanti l'invio email sono stati scartati};
		\item {Si è discusso con la proponente se i manuali utente e sviluppatore dovessero essere scritti in italiano, in inglese o in entrambe le lingue. La decisione viene lasciata al gruppo;}
		\item {Data l'inesperienza del gruppo, si è chiesto alla\glossario{proponente}se fosse disponibile a fare incontri di formazione riguardanti i test di unità eseguiti con\glossario{Mocha}e \textit{Chai$_{G}$}. La proponente ha espresso la sua disponibilità nel fare questi incontri in futuro, presso l'azienda o tramite Conference call;}
		\item {Si è discusso infine di un problema riscontrato durante l'implementazione del blocco Lista: se l'utente chiede, tramite comando vocale, ad Alexa di rimuovere o aggiungere un elemento alla lista, potrebbero verificarsi problemi di comprensione tra il dispositivo e l'utente e così apportare modifiche sbagliate alla lista. Il team propone di permettere solo tramite app, all'utente, l'aggiunta e la rimozione degli elementi nella lista. La proponente consiglia di far chiedere ad Alexa, all'utente, la conferma delle modifiche che si vogliono apportare alla lista.}
	\end{itemize}
	\subsection{Tracciamento decisioni}
	\begin{table}[tbph]
		\centering
		\begin{tabularx}{\textwidth}{|C|C|}
			\hline
			\textbf{Codice } & \textbf{Decisione} \\
			\hline
			01-VER\_2019-03-22& Rimozione del blocco Sveglia.\\
			\hline
			02-VER\_2019-03-22& Il gruppo può decidere liberamente se scrivere i manuali in italiano e/o inglese.\\
			\hline
			03-VER\_2019-03-22& In futuro, se necessario, il gruppo chiederà degli incontri di formazione alla proponente.\\
			\hline
			04-VER\_2019-03-22& Rimozione del blocco Pianificazione\\
			\hline
			05-VER\_2019-03-22& Rimozione del blocco Invio Email\\
			\hline
		\end{tabularx}
		\caption{Tracciamento decisioni}
	\end{table}
	\label{LastPage}
\end{document}