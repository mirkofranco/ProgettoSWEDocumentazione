\chapter{Introduzione}
\section{Obiettivo del documento}
L’obiettivo delloStudio di fattibilit\'a \'e quello di fornire le motivazioni che hanno portato alla scelta delcapitolato C4 “MegAlexa” piuttosto che uno degli altri capitolati proposti.
\section{Glossario} 
Ogni termine che necessita di un chiarimento o di una spiegazione verrà segnato con una $_{G}$.
\section{Riferimenti}
\subsection{Normativi}
\begin{itemize}
	\item Norme di Progetto: %todo
\end{itemize}
\subsection{Informativi}
\begin{itemize}
\item \textbf{Capitolato 1: Butterfly} - monitor per processi CI/CD;
\item \textbf{Capitolato 2: Colletta} - piattaforma raccolta dati di analisi di testo
\item \textbf{Capitolato 3: G\&B} - monitoraggio intelligente di processi DevOps
\item \textbf{Capitolato 4: MegAlexa} - arricchitore di skill di Amazon Alexa
\item \textbf{Capitolato 5: P2PCS} - piattaforma di peer-to-peer car sharing
\item \textbf{Capitolato 6: Soldino} - piattaforma Ethereum per pagamenti IVA

\end{itemize}
