\chapter{Processi di Supporto}
\section{Gestione della configurazione}
\subsection{Versionamento}
Per un progetto di tali dimensioni risulta necessario mantenere uno storico completo di tutte le modifiche apportate a qualunque file o documento non generato automaticamente da IDE o simili. Inoltre è necessario uno strumento che permetta il lavoro collaborativo. Per questo scopo si è scelto l'utilizzo del sistema di versionamento\glossario{Git}, attraverso il servizio\glossario{GitHub}.
\subsubsection{Comandi basilari}
Si riportano di seguito i comandi fondamentali del software Git per fornire un punto di appoggio a tutti i membri del gruppo. Tutti i comandi, tranne quello per la creazione della copia locale, sono da eseguire all'interno della cartella contenente il repository.
\subsubsection{Creazione della copia locale}
La creazione della copia locale viene effettuata attraverso il comando:\\
\texttt{git clone https://github.com/nomeaccount/nomerepostory}.
\subsubsection{Comandi principali}
\begin{itemize}
	\item \texttt{git status}: mostra lo stato del repository locale con i file modificati, i file in area di staging, i file tracciati e non tracciati;
	\item \texttt{git checkout}: permette di cambiare branch attivo nel repository locale;
	\item \texttt{git add nomeFile}: permette di aggiungere un file all'area di staging;
	\item \texttt{git commit -m "Descrizione commit"}: fa il commit dei file presenti in area di staging nel repository locale;
	\item \texttt{git push}: aggiorna sovrascrivendo il repository remoto con quello locale;
	\item \texttt{git pull}: aggiorna sovrascrivendo il repository locale con quello remoto.
\end{itemize}
\subsubsection{Git Flow}
Per facilitare la collaborazione si sceglie usare il git flow \footnote{\url{https://it.atlassian.com/git/tutorials/comparing-workflows/gitflow-workflow}}\glossario{workflow}. L'Amministratore si assicurerà che tutte le macchine siano configurate correttamente per il suo utilizzo.
\subsubsection{Gestione dei rilasci}
L'amministratore si occuperà di creare i rami di feature necessari per una proficua collaborazione tra i membri nel repository. Ogni documento (o parte di esso) e ogni funzionalità software verrà implementata in un ramo di feature. Quando la funzionalità diventa stabile e corretta verrà rilasciata in un ramo di integrazione chiamato "develop". Prima di ogni revisione verrà effettuato un rilascio nel ramo master a partire dall'ultimo commit. Tale rilascio costituirà\glossario{baseline}e sarà punto di partenza verso la prossima\glossario{milestone}. Tale attività verrà effettuata nei tempi descritti nel \texttt{Piano di Progetto v1.0.0}. 


\section{Documentazione}
Il\glossario{processo}di Documentazione stabilisce una serie di norme e convenzioni per la stesura dei documenti così da renderli formali, validi e coerenti.
Un documento viene definito formale quando viene approvato dal \textit{Responsabile}.
Ogni documento formale verrà classificato come:
\begin{itemize}
	\item \textbf{Interno:} documenti ad uso esclusivo del gruppo\glossario{ZeroSeven}, quali \textit{Norme di progetto} e \textit{Studio di fattibilità};
	\item \textbf{Esterno:} documenti consultabili anche da attori esterni al gruppo ZeroSeven.
	
\end{itemize}

\subsection{Nomenclatura}
Tutti i documenti formali, esclusi i verbali, seguiranno questo sistema di nomenclatura:
\begin{itemize}
	\item \textbf{Nome del documento:} Il nome del documento deve essere scritto in \textit{Upper Camel}\glossario{Case}.
	NormeDiProgetto per il documento corrente;
	\item \textbf{Versione:} La documentazione prodotta deve essere corredata del numero di versione secondo la seguente codifica:
	
	\textbf{v.X.Y.Z}
	
	dove:
	\begin{itemize}
		\item \textbf{X:} indica il numero crescente di uscite formali. Sarà compito del Responsabile azzerare gli indici Y e Z ad ogni rilascio;
		\item \textbf{Y:} indica lo stato del documento secondo la seguente numerazione:
		\begin{enumerate}
			\setcounter{enumi}{-1}
			\item per la documentazione in fase di sviluppo;
			\item per la documentazione in fase di\glossario{verifica};
			\item per la documentazione in fase di approvazione;
			\item per la documentazione approvata e formale.
		\end{enumerate}
		\item \textbf{Z:} indica il numero crescente di modifiche apportate al documento. Ogni modifica deve essere riscontrabile con il diario delle modifiche. Deve essere azzerato quando il responsabile approva il documento. 	
	\end{itemize}
\end{itemize}

I documenti verranno citati secondo il formato NomeDocumento v.X.Y.Z mentre i file saranno rinominati \texttt{NomeDocumento\_v.X.Y.Z.pdf} \\
I verbali sia interni che esterni seguiranno la nomenclatura VerbaleUsoData.

\subsubsection{Ciclo di vita della documentazione}
Ogni documento formale deve passare gli stadi di ``Sviluppo”, ``Verifica” e ``Approvato”.
\begin{itemize}
	\item \textbf{Sviluppo:} inizia con la creazione del documento e termina con la conclusione	della stesura di tutte le sue parti. In questa fase i Redattori aggiungono le parti assegnate tramite\glossario{ticket};
	
	\item  \textbf{Verifica:} il documento entra nella fase di verifica dopo l’assegnazione di un ticket a un \textit{Verificatore} da
	parte del Responsabile. I Verificatori effettueranno le procedure di controllo
	dello stesso.
	Al termine del controllo in caso positivo il documento entra automaticamente
	in fase di ``Approvazione”, altrimenti i loro riscontri vengono consegnati
	al \textit{Responsabile}, che provvederà ad assegnare nuovamente il
	documento ad un Redattore attraverso una nuova fase di Sviluppo;
	
	\item  \textbf{Approvazione:} l’approvazione di un documento coincide con il superamento
	positivo della verifica.
	Sarà onere del Responsabile decidere, dopo un attenta lettura, se approvare il documento per il rilascio esterno o se è necessario modificare il documento;

	
	\item  \textbf{Approvato:} il documento è pronto per il rilascio esterno.
\end{itemize}


\subsection{Struttura dei documenti}
Al fine di uniformare la struttura grafica e di permettere ai membri del gruppo di concentrarsi solo sulla stesura del contenuto è stato creato un template \LaTeX.
Ogni documento sarà composto da:
\begin{itemize}
	\item Frontespizio;
	\item Diario delle modifiche;
	\item Indice;
	\item Elenco delle immagini e tabelle (se presenti);
	\item Contenuto delle pagine interne.
\end{itemize}
\subsubsection{Frontespizio}
La prima pagina di tutti i documenti dovrà essere così composta:
\begin{itemize}
	\item Titolo del\glossario{progetto};
	\item Titolo del documento;
	\item Logo e nome del gruppo;
	\item Descrizione in forma tabellare contenente informazioni importanti quali:
	\begin{itemize}
		\item Versione del documento;
		\item Data di Redazione;
		\item Redattore;
		\item \textit{Verifica$_{G}$};
		\item Approvazione;
		\item Uso;
		\item Distribuzione;
		\item Email di contatto.
	\end{itemize}
\end{itemize}

\subsubsection{Diario delle modifiche}
Successivo al frontespizio deve essere sempre presente un registro riassuntivo delle modifiche del documento in forma tabellare contenente:
\begin{itemize}
\item Versione dopo la modifica;
\item Data della pubblicazione della modifica;
\item Breve descrizione della modifica effettuata;
\item Autore della modifica;
\item Ruolo.
\end{itemize}
\subsubsection{Indice}
Tutti i documenti, eccezion fatta per i verbali, devono contenere un indice il quale permette una visione macroscopica del contenuto del documento,
permettendo una lettura ipertestuale e non necessariamente sequenziale.
 \subsubsection{Elenco delle immagini e tabelle} 
Dopo il diario delle modifiche di ogni documento dovrà essere presente un \textit{Elenco delle figure} ed un \textit{Elenco delle tabelle}.

\subsubsection{Struttura delle pagine interne}
Le pagine interne dei documenti rispettano i canoni previsti dal template \LaTeX e oltre al contenuto interno sono composte da: \\

\begin{itemize}
	\item \textbf{intestazione:}
	\begin{itemize}
		\item logo del gruppo posto a sinistra;
		\item nome del capitolo corrente a destra. 
	\end{itemize}
	\item \textbf {Piè di pagina:}
	\begin{itemize}
	\item titolo del documento completo di versione, posto a sinistra;
	\item numero di pagina posto a destra.
	\end{itemize}

\end{itemize} 
\subsection{Norme tipografiche}
Questa sezione racchiude le convenzioni riguardanti tipografia, ortografia e uno stile uniforme e disciplinato per tutti i documenti.
\subsubsection{Punteggiatura}
Ogni segno della punteggiatura va sempre unito all'ultima lettera della parola che lo precede e separato con uno spazio dalla lettera iniziale della parola che lo segue.
Le lettere maiuscole vanno poste solo dopo il punto, il punto di domanda, il punto esclamativo e all’inizio di ogni elemento di un elenco puntato, oltre che dove previsto dalla lingua italiana. È inoltre utilizzata l’iniziale maiuscola nel nome del team, del\glossario{progetto}, dei documenti, dei ruoli di progetto e delle fasi di lavoro.
\subsubsection{Formati}
\begin{itemize}
\item  Date: le date presenti nei documenti devono seguire la notazione definita dallo standard ISO 8601:2004 YYYY-MM-DD

dove:
\begin{itemize}
	\item  YYYY: rappresenta l’anno scritto utilizzando quattro cifre;
	\item MM: rappresenta il mese scritto utilizzando due cifre;
	\item DD: rappresenta il giorno scritto utilizzando due cifre.
\end{itemize}
  \item  Monospace: sarà utilizzato il carattere monospace per formattare il testo contenente parti di codice, comandi, nomi di classi;
  \item  Percorsi: per gli indirizzi email e web  deve essere utilizzato il comando
  \LaTeX \url, mentre per gli indirizzi relativi va usato il formato monospace;
  \item Maiuscolo: l'utilizzo del carattere maiuscolo per l'intera parola è riservato esclusivamente agli acronimi.
\end{itemize}
\subsubsection{Stile del testo}
\begin{itemize}
	\item \textbf{Corsivo:}
	il corsivo deve essere utilizzato esclusivamente nei seguenti casi:
	\begin{itemize}
		\item Ruoli: Dovrà essere utilizzato il corsivo quando si parla di ruoli di progetto (es. \textit{Analista});
		\item Documenti: Il nome di un documento andrà scritto in corsivo (es. \textit{Norme di progetto}).
	\end{itemize}
	\item  \textbf{Grassetto:} il grassetto può essere utilizzato esclusivamente negli elenchi puntati per dare risalto al concetto sviluppato;
	\item  \textbf{Sottolineato:} Non è previsto l'uso del testo sottolineato.
	\end{itemize}
\subsubsection{Norme redazionali} 

\begin{itemize}
	\item \textbf{Elenchi:} Al termine di ogni punto di un elenco verrà utilizzato il carattere (;) eccetto per l'ultimo punto per il quale verrà utilizzato il carattere (.);
	\item \textbf{Riferimenti informativi:} Ogni riferimento a\glossario{prodotti}, guide, software,
	libri esterno al progetto dovrà essere indicato tramite un’annotazione a piè di pagina;
	\item \textbf{\LaTeX:} Ogni riferimento a \LaTeX verrà scritto utilizzando il comando \texttt{\textbackslash LaTeX};
	\item \textbf{Comandi \LaTeX:} sono stati realizzati dei comandi personalizzati al fine di evitare errori di battitura e unificare tutti i documenti facilitando il lavoro del gruppo.
	\begin{itemize}
		\item \textbf{ \textbackslash intestazioni}: inserisce un'intestazione personalizzata con il nome del documento e il logo\glossario{ZeroSeven};
		\item \textbf{ \textbackslash mailzeroseven}: inserisce l'indirizzo email del gruppo per un eventuale contatto;
		\item \textbf{ \textbackslash progetto}: inserisce il nome del\glossario{progetto};
		\item \textbf{ \textbackslash logo}: inserisce il logo del gruppo ZeroSeven;
		\item \textbf{ \textbackslash glossario}: indica un termine da inserire nel glossario(marcato da una G maiuscola a pedice).	
	\end{itemize}
	
	\item \textbf{Sigle:} Nonostante sarà preferito l'utilizzo delle parole per intero potranno essere utilizzate le seguenti sigle:
	\begin{itemize}
	\item ADR = Analisi dei Requisiti;
	\item NDP = Norme di Progetto;
	\item PDP = Piano di Progetto;
	\item PDQ = Piano di Qualifica;
	\item SDF = Studio di Fattibilità;
	\item RR = Revisione dei Requisiti;
	\item RQ = Revisione di Qualifica;
	\item RP = Revisione di Progetto;
	\item RA = Revisione di Accettazione.
	\end{itemize}
\end{itemize}
\subsubsection{Componenti grafiche}
	\begin{itemize}
	\item \textbf{Tabelle:} 
	Ogni tabella presente all'interno dei documenti deve essere accompagnata da una didascalia,	in cui deve comparire un numero identificativo incrementale per la tracciatura della stessa all'interno del documento;
	\item \textbf{Immagini:}
	Le immagini presenti all'interno dei documenti devono essere nel formato Scalable Vector Graphics (\textit{SVG$_{G}$}). In questo modo si garantisce una maggior qualità dell'immagine in caso di ridimensionamento. Per consentire l’inclusione delle immagini nei documenti,
	le immagini dovranno essere convertite nel formato PDF. Qualora non sia possibile
	salvare le immagine in formato vettoriale è preferito il formato Portable Network
	Graphics (PNG).
	\end{itemize}
\subsection{Strumenti a supporto della documentazione}
Per la scrittura della documentazione in modo coerente e uniforme si deve usare il linguaggio di markup \LaTeX.
\subsubsection{TeXstudio}
E' utilizzato per la scrittura e il controllo ortografico dei documenti.
\subsubsection{Script automatici}
Viene implementato uno script per permettere il calcolo automatico dell'\textit{indice di}\glossario{Gulpease}.

\section{Garanzia di qualità}
\subsection{Classificazione metriche e obbiettivi}
Questa\glossario{processo}definisce norme e struttura delle metriche, obbiettivi di qualità, metodologie e strumenti per perseguire qualità di processo e di prodotto. 
\subsubsection{Classificazioni degli obbiettivi} 
Gli obbiettivi di qualità inclusi nel \textit{Piano di Qualifica v1.0.0} devono rispettare la seguente notazione: \\
\begin{center}
Q[Tipo][ID] : [Nome].
\end{center}
\begin{itemize}
	\item \textbf{Tipo}: stabilisce se l'obbiettivo si riferisce a\glossario{prodotti}o processi e può assumere i seguenti valori:
	\begin{itemize}
		\item \textbf{P}: per indicare i processi;
		\item \textbf{PR}: per indicare i prodotti.
	\end{itemize}
	\item \textbf{ID}: identifica univocamente l'obbiettivo attraverso un numero progressivo;
	\item \textbf{Descrizione}: breve descrizione dell'obbiettivo di qualità.
\end{itemize}
\subsubsection{Classificazione delle metriche}
Risulta importante fissare delle metriche per permettere di monitorare costantemente la qualità di prodotto e processo.Tali metriche dovranno rispettare la seguente notazione:
M[Tipo][ID] : [Nome].\\
\begin{itemize}
	\item \textbf{Tipo}: stabilisce se la metrica si riferisce a prodotti o processi e può assumere i seguenti valori:
	\begin{itemize}
		\item \textbf{P}: per indicare i processi;
		\item \textbf{PR}: per indicare i prodotti.
	\end{itemize}
	\item \textbf{ID}: identifica univocamente la metrica attraverso un numero progressivo;
	\item \textbf{Descrizione}: breve descrizione della metrica.
\end{itemize}

\section{Verifica}
La\glossario{verifica}è un processo atto a evidenziare ed eliminare la possibile presenza di errori.
Di seguito verranno descritti gli strumenti e le pratiche utilizzate per la verifica del codice e dei documenti durante la loro realizzazione.
Durante questa prima fase di progetto la verifica si è focalizzata principalmente su documenti.
\subsection{Verifica di processi}
Ciascun\glossario{processo}verrà costantemente monitorato in tutta la sua esecuzione e documentato alla fine di ogni periodo nell'appendice "Resoconto della verifica" nel \texttt{Piano di Qualifica v1.0.0}, secondo le metodologie ISO/IEC 15504.

\subsection{Metriche per la qualità di processo}
\subsubsection{MP001: Budgeted Cost Of Work Performed}\label{bcwp}
Utilizzato per il calcolo di Cost Variance e Schedule Variance, rappresenta (in giorni) il valore delle attività svolte.

\subsubsection{MP002: Budgeted Cost of Work Scheduled}\label{bcws}
Rappresenta il costo in giorni preventivato per il processo in esame (è detto anche Planned Value):


\subsubsection{MP003: Actual Cost of Work Performed}\label{acwp}
Rappresenta il costo (in \euro) effettivamente sostenuto al momento del calcolo:

\subsubsection{MP004: Schedule Variance}
Metrica che indica se si è in anticipo o in ritardo rispetto alla schedulazione delle attività di progetto.
Essa è il risultato della seguente formula:\\
\begin{center}
	
	$SV = ${MP001} $-${MP002}
	
\end{center}

\subsubsection{Costo}
Per verificare che i costi siano conformi a quanto preventivato nel \textit{Piano di Progetto}, ciascun processo viene misurato tramite la sua \textbf{Cost Variance(CV)}, un valore positivo indica il rispetto dei costi preventivati, essa viene calcolata nel seguente modo:\\ 
\subsubsection{MP005: Cost Variance}
\begin{center}
	\begin{math}
	CV = MP001 - MP003
	\end{math}
\end{center}
\begin{itemize}
	\item[] \textbf{ACWP} (Actual Cost of Work Performed) rappresenta il costo(in giorni) effettivamente sostenuto al momento del calcolo. 
\end{itemize}
Dove: BCWP e ACWP sono descritte rispettivamente nelle sezioni \ref{bcwp} e \ref{acwp}.

\subsubsection{MP006: SPICE capability level}
Per ogni processo, lo standard 15504 definisce 6 livelli di maturità(da 0 a 5) determinati da un processo di  \textit{Process Assessment} che ha lo scopo di determinare l'effettiva qualità dei processi in uso, un processo raggiunge la sua massima efficacia quando raggiunge il livello 5 (raggiunge quindi l'ottimalità).


\subsubsection{MP007: SPICE process attributes}
Per ogni attributo di processo, lo SPICE definisce 4 livelli (N-P-L-F) di ottimalità riferiti all'attributo stesso, il valore viene dedotto in base ai risultati delle metriche adottate per ciascun processo, un processo raggiunge il livello massimo quando tutte le metriche a lui riferite presentano risultati circoscritti al range di ottimalità definito:

\subsubsection{MP008: Occorrenza rischi non previsti}
Contatore che viene incrementato all'occorrenza di un rischio non elencato nell'analisi dei rischi del \textit{Piano di Progetto}, un alto valore indica l'eccessiva occorrenza dello stesso e la necessità di un'analisi al fine di mitigare il suo impatto nell'attività di progetto.\\
Viene resettato ad ogni inizio di una nuova fase del progetto.

\subsubsection{MP009: Indisponibilità dei servizi}
Contatore che viene incrementato ogni qualvolta uno strumento esterno risulta inutilizzabile a causa di errori non gestibili dai membri del gruppo.\\
Viene resettato ad ogni inizio di una nuova fase del progetto.

\subsubsection{MP010: Complessità ciclomatica}
Metrica software che indica la complessità di un programma tenendo in considerazione moduli, funzioni, metodi e classi.
Nello specifico, essa è calcolata tramite il grafo di controllo di flusso del programma, dove i nodi sono gruppi indivisibili di istruzioni e gli archi connettono due nodi se il secondo gruppo di istruzioni può essere eseguito immediatamente dopo il primo, e il suo valore è determinato dal numero di cammini linearmente indipendenti all'interno del codice sorgente. 
\'E quindi opportuno definire un valore di complessità ciclomatica preciso: valori alti sono indice di scarsa manutenibilità del codice mentre valori bassi potrebbero indicare scarsa efficienza dei metodi.
Esso fornisce, inoltre, un indice del carico di lavoro richiesto per la fase di testing (un valore alto richiede più test per una copertura completa).
Il range di ottimalità stabilito varia da 0 a 10, come suggerito dall'ideatore della metrica Thomas J. McCabe.  

\subsubsection{MP011: Numero di parametri per metodo}
Definire un range relativo al numero di parametri permette di individuare possibili errori nella progettazione (nel caso in cui un metodo abbia un numero di parametri eccessivo).

\subsubsection{MP012: Numero di livelli di annidamento}
Metrica per indicare il numero di chiamate annidate di procedure controllate all'interno dei metodi.\newline
Un valore elevato è indice di un basso livello di astrazione del codice e una complessità eccessivamente elevata. 

\subsubsection{MP013: Attributi per classe}
Un numero elevato di attributi in una classe potrebbe essere indice di un errore di progettazione.
Viene quindi definita una metrica che identifichi range accettabili e ottimali per questo parametro.
Nel caso in cui una classe abbia un numero eccessivo di parametri, valutare la possibilità di suddividere la stessa in più classi, suddividendo quindi le funzioni ad essa assegnate.

\subsubsection{MP014: Linee di codice per commento}
Metrica identificata dal rapporto tra linee di codice e linee di commento: risulta utile per garantire una maggiore manutenibilità del codice.

\subsection{MP015: Flusso di informazioni}
Metrica proposta da S. Henry e D. Kafura che misura il flusso di informazioni così definito:

\begin{itemize}
	\item \textbf{fan-in:} numero di moduli che passano informazioni dentro al modulo in esame;
	\item \textbf{fan-out:} numero di moduli a cui il modulo in esame passa informazioni.
\end{itemize} 
il valore viene calcolato tramite questa funzione:
\begin{center}
	(lunghezzafunzione)$^2\times fan-in\times fan-out$
\end{center}

\subsubsection{Accoppiamento}
La definizione dei range di ottimalità e accettazione delle metriche presenti in questa sezione viene rimandata alla fase di Revisione di Analisi.
\paragraph{MP016: Accoppiamento Afferente}
Indica la dipendenza di classi esterne a un package rispetto alle classi interne contenute nel package stesso.
Tale valore, sebbene non indicante necessariamente errori di progettazione, può evidenziare criticità riguardanti la robustezza e l'utilità del package a cui fa riferimento.\\
Un valore basso può evidenziare scarsa utilità del package, mentre un valore alto necessita di ulteriori verifiche di robustezza, in quanto rappresentante un punto critico nel software.
\paragraph{MP017: Accoppiamento Efferente}
Indica la dipendenza di classi interne a un package rispetto alle classi esterne ad esso.\\
Un valore basso indica un forte indipendenza del package rispetto al resto del sistema.

\subsubsection{MP018: Copertura del codice}
Indica la percentuale di istruzioni che sono eseguite durante i test.
Maggiore è la percentuale di istruzioni coperte dai test eseguiti, maggiore sarà la probabilità che le componenti testate abbiano una ridotta quantità di errori.\\
Il valore di tale indice può essere abbassato da metodi molto semplici che non richiedono testing. Esempi di questi metodi sono: get e set.\\
Parametri utilizzati:

\subsubsection{MP019: efficacia dei test}
Misura la capacità di un test di trovare errori all'interno del codice, essa è calcolata nel seguente modo
\begin{center}
\vspace{1em}
$\frac{\mbox{n bug trovati con il test}}{\mbox{totale dei bug trovati}}$\\

\end{center}
\vspace{1em}
Dove il totale dei bug trovati è risultato dei bug già conosciuti + i nuovi bug scoperti durante le prove di utilizzo del software

\subsubsection{MP020:Test Effort Performance}
Attesta l'effettiva efficacia di un test misurando il numero di bug risolti, l'obiettivo è quello di minimizzare il numero di bug presenti nel software al momento del rilascio.
La metrica risulta dal seguente calcolo:
\begin{center}
\vspace{1em}
	$\frac{\mbox{n bug risolti}}{\mbox{numero di bug trovati}}$

\end{center}

\subsection{Verifica dei prodotti}
La verifica della qualità di prodotto si divide in attività e procedure differenti a seconda del\glossario{prodotto}che si sta verificando. Per i prodotto software si procede con analisi statica e dinamica, mentre per i documenti si applica soltanto analisi statica.
\subsubsection{Verifica dei documenti}
\paragraph{Analisi statica}
Al fine di verificare i documenti verranno utilizzati\glossario{inspection}e\glossario{walkthrought}:
\begin{itemize}
	\item \textbf{Walkthrough:} questa metodologia prevede una lettura profonda e attenta del documento. Gli errori riscontrati verranno corretti e aggiunti all'appendice "Lista di controllo" delle \texttt{Norme di Progetto v1.0.0} per permettere di utilizzare inspection le volte successive;
	\item \textbf{Inspection:} al contrario della metodologia precedente, questa risulta essere molto più veloce perché attraverso una lista di controllo degli errori permette un'analisi più efficace delle criticità, omettendo le parti che non presentano problematiche. 
\end{itemize}
\paragraph{Procedura di controllo dei documenti}
\begin{enumerate}
	\item Viene assegnato un\glossario{ticket}a un \textit{Verificatore};
	\item Vengono controllati gli errori comuni attraverso la lista di controllo;
	\item Viene controllato il rispetto delle \textit{Norme di Progetto};
	\item Viene effettuata una lettura profonda e corrette eventuali nuove anomalie;
	\item Calcolo dell'\textit{indice di}\glossario{Gulpease};
	\item Vengono aggiunte eventuali nuove anomalie alla lista di controllo;
	\item Viene chiuso il ticket.
\end{enumerate}
\subsubsection{Verifica del prodotti software}
\paragraph{Analisi dinamica}
\paragraph{Test di Unità}
\paragraph{Test di Integrazione}
\paragraph{Test di Sistema}
\paragraph{Test di Validazione}
\subsubsection{Procedure di verifica del software}
\paragraph{Strumenti}
\section{Validazione}
Questa sezione verrà sviluppata in un momento successivo alla Revisione dei Requisiti.


%\subsubsection{Analisi dinamica}
%L'analisi dinamica al contrario dell'analisi statica è applicabile unicamente al software e non ai documenti. Il\glossario{processo}prevede %la realizzazione e l'esecuzione di test sul codice. Ogni test per essere valido deve essere ripetibile, cioè attraverso il medesimo input di %partenza si deve poter risalire a un dato output. A tal fine è necessario che per ogni test vengano segnati i seguenti riferimenti: 
%\begin{itemize}
%\item \textbf{Ambiente:} sono i sistemi software e hardware utilizzati nel corso dei test;
%\item \textbf{Stato iniziale:} lo stato di partenza del prodotto prima del test;
%\item \textbf{Input};
%\item \textbf{Output};
%\item \textbf{Avvisi}: un eventuale insieme di istruzioni riguardanti l'esecuzione del test e i suoi risultati.
%\end{itemize}
%\subsubsection{Verifica diagrammi UML}
%Sarà compito dei verificatori accertarsi che tutti i diagrammi\glossario{UML} prodotti rispettino lo standard UML e che siano corretti %semanticamente.
%\subsubsection{Strumenti}
%\paragraph{Verifica ortografica}
%Per la verifica ortografica verrà utilizzato lo strumento integrato in\glossario{TexStudio}, mentre l'analisi del periodo sarà effettuata dai %Verificatori.
%\paragraph{Indice di Gulpease}
%Per il calcolo dell'\textit{indice di}\glossario{Gulpease} il gruppo utilizzerà strumenti automatici disponibili online quali Farfalla Project \footnote{\url{https://farfalla-project.org/readability_static}} e Corrige!it \footnote{\url{http://www.corrige.it}}.

%\subsubsection{Metriche per il controllo della qualità}
%Segue una descrizione dettagliata delle metriche stabilite dal gruppo per il controllo della qualità del\glossario{prodotto}, fare %riferimento al \textit{Piano di Qualifica v 1.0.0} per una valutazione precisa degli indici adottati.\\
%Le linee guida adottate riprendono in parte lo standard di riferimento ISO/IEC 25010.

%\paragraph{Qualità dei documenti}
%\begin{itemize}
%	\item \textbf{MD001 - Indice Gulpease:} L'\textit{indice di}\glossario{Gulpease}è un indice di leggibilità di un testo tarato sulla lingua %italiana. Vengono tenute in considerazione due variabili linguistiche:  la lunghezza della parola e la lunghezza della frase rispetto al %numero di lettere.
%	\[ IG = 89+ \frac{300*nf - 10*nl}{np} \]
%	Dove:
%	\begin{itemize}
%		\item nf è il numero delle frasi;
%		\item nl è il numero delle lettere;
%		\item np è il numero delle parole.
%	\end{itemize}
%	Il risultato è un numero compreso nel range di valori $0 \le{IG} \le{100}$, dove il valore "100" indica la leggibilità più alta e "0" la %leggibilità più bassa.
%	
%	\item \textbf{MD002 - Ortografia:} Indice che rappresenta la percentuale di errori ortografici e lessicali corretti nei vari documenti;
%	
%	\item \textbf{MD003 - Formula di Flesch:} La \textit{formula di}\glossario{Flesch} serve a misurare la leggibilità di un testo in inglese.
%	Il modo di calcolare questo indice è il seguente:
%	
%	\[ F = 206,835 - (0,846*S) - (1,015*P) \]
%	Dove:
%	\begin{itemize}
%		\item S è il numero delle sillabe, calcolato su un campione di 100 parole;
%		\item P è il numero medio di parole per frase.
%	\end{itemize}
%Più questo indice è alto e più il testo risulta semplice da leggere. La leggibilità è alta se F è superiore a 60, media se fra 50 e 60, bassa %sotto 50.
%\end{itemize}
%
%\subsection{Validazione}
%l processo di Validazione verrà svolto nella fase finale del\glossario{progetto}e ha l'obbiettivo di verificare che il\glossario{prodotto}sia %conforme ai\glossario{requisiti}.
