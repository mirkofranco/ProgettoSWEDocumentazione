\chapter{Processi organizzativi}\label{Po}
\section{Gestione di processo}
\subsection{Comunicazioni interne}
Per le comunicazioni interne è stato adottato un gruppo \textit{Telegram$_{G}$}, un'applicazione di messaggistica istantanea dove risulta possibile anche aggiungere Bot per automatizzare compiti ripetitivi, inviare file anche di grosse dimensioni, ricercare messaggi anche marcandoli con un hashtag.
\subsection{Comunicazione esterne}
Le comunicazioni esterne avvengono unicamente per mezzo scritto. 
Queste devono avvenire esclusivamente attraverso l'indirizzo mail del gruppo: \mailzeroseven.
Ogni mail ricevuta a questo indirizzo viene inoltrata automaticamente alle caselle mail di ogni componente del gruppo. L'accesso all'account di gruppo è possibile solo da parte del \textit{Responsabile}.
\subsection{Gestione delle riunioni}
Le riunioni possono essere interne o esterne. 
All'inizio di ogni riunione il Responsabile nomina a turno un segretario che si occuperà di prendere nota di tutto ciò di cui viene discusso oltre ad avere l'onere di far rispettare l'ordine del giorno.
\subsubsection{Verbali di riunione}
Al termine di ogni riunione il segretario ha il compito di redigere il relativo verbale secondo il seguente schema:
\begin{itemize}
	\item \textbf{Verbale [Tipo] [Data]}: nel frontespizio, con il tipo che indica se il verbale è interno o esterno e per data si intende la data nella quale si è svolta la riunione;
	\item \textbf{Informazioni sulla riunione:}
	\begin{itemize}
			\item \textbf{Motivo della riunione};
			\item \textbf{Luogo e data};
			\item \textbf{Ora di inizio e fine};
			\item \textbf{Partecipanti}.
	\end{itemize}
	\item \textbf{Ordine del giorno};
	\item \textbf{Resoconto}.
\end{itemize}
\subsection{Riunioni interne}
La partecipazione alle riunioni interne è permessa solamente ai membri del gruppo \textit{ZeroSeven$_{G}$}. Queste si possono svolgere:
\begin{itemize}
	\item \textbf{In presenza}: incontri di persona;
	\item \textbf{Hangouts}: video chiamate di gruppo attraverso la piattaforma \textit{Google Hangouts$_{G}$}.
\end{itemize}
\subsection{Riunioni esterne}
Le riunioni esterne vedono coinvolti oltre ai membri del gruppo ZeroSeven anche uno o più soggetti esterni. Queste si possono svolgere  in:
\begin{itemize}
	\item \textbf{Sede della Proponente$_{G}$};
	\item \textbf{Torre Archimede};
	\item \textbf{Google Hangouts}.
\end{itemize}
\subsection{Identificazione delle decisioni}
Le decisioni prese durante gli incontri devono essere identificate con un codice univoco per permettere di riferirsi ad esse se necessario, perciò nei verbali sarà presente il tracciamento delle decisioni nel quale verrà associato ad ogni decisione un identificativo che segue questa codifica:
\begin{center}
	\texttt{[NumeroDecisione]-[VER-DATA]}
\end{center}
dove:
\begin{itemize}
	\item \textbf{NumeroDecisione:} è il numero della decisione presa in uno specifico incontro;
	\item \textbf{VER-DATA:} è il nome del verbale dell'incontro in cui è stata presa tale decisione.
\end{itemize}



\subsection{Ruoli di progetto}
I componenti del gruppo di progetto sono tenuti a ripartire tra loro in modo equo e omogeneo i differenti ruoli. A tal fine i ruoli verranno assegnati a rotazione dal \textit{Responsabile}, il quale si assicurerà di non creare conflitti di interesse.
\subsubsection{Responsabile}E' il responsabile ultimo dei risultati del progetto. Sue responsabilità sono l'approvazione dell'emissione di documenti, l'elaborazione e l'emanazione di piani e scadenze. Coordina il gruppo e le attività. E' suo compito gestire le relazioni con il controllo di qualità interno al \textit{progetto$_{G}$}. Infine approva l'offerta e i suoi allegati.
\subsubsection{Amministratore} L'amministratore è responsabile dell'efficienza e dell'operatività dell'ambiente di sviluppo. Si occupa della redazione e dell'attuazione dei piani e delle procedure di Gestione per la qualità. Redige le norme di\glossario{progetto}per conto del responsabile e collabora alla stesura del piano di progetto. Gestisce l'archivio della documentazione di progetto, controlla versioni e configurazioni del \textit{prodotto$_{G}$}.
\subsubsection{Analista} Si occupa dell'attività di analisi, della redazione dello \textit{Studio di Fattibilità} e dell'\textit{Analisi dei requisiti$_{G}$}.
\subsubsection{Progettista} Suo compito è l'attività di progettazione, inoltre redige la specifica tecnica, la \textit{Definizione di\glossario{prodotto}} e la parte programmatica del \textit{Piano di qualifica}. 
\subsubsection{Programmatore} Si occupa delle attività di codifica che hanno lo scopo di realizzare il\glossario{prodotto}e le componenti di ausilio necessarie all'esecuzione delle prove di verifica e di validazione.
\subsubsection{Verificatore} Si occupa dell'attività di \textit{verifica$_{G}$}. Redige la parte retrospettiva del piano di qualifica, che illustra l'esito e la completezza delle verifiche e delle prove effettuate secondo il piano. Si occupa  inoltre dell'attività di controllo, in particolare verifica se i documenti e i prodotti dei processi rispettano le attese e sono conformi alle \textit{Norme di Progetto}. 

\subsection{Ticketing}
Per la gestione di attività e compiti viene utilizzato\glossario{Asana}\footnote{Guida e Documentazione di Asana - \url{https://asana.com/guide}}, che permette di gestire facilmente progetti di grandi dimensioni.
L'assegnazione di compiti tramite questo strumento è affidata al \textit{Responsabile}, seguendo le scadenze precedentemente stabilite nel \textit{Piano di Progetto}.

\section{Calcolo delle ore} \label{calcoloOre}
Viene integrato ad Asana \textit{Harvest$_{G}$}, che permette il time tracking delle ore di lavoro suddivise per ruolo, producendo dei grafici utili ai membri del gruppo per avere un quadro generale dell'andamento del lavoro.
\subsubsection{Avvio del timer}
Sebbene sia possibile inserire le ore manualmente, è consigliato avviare il timer al momento dell'inizio del lavoro, allo scopo di ottenere misurazioni più precise.\\
Per avviare il timer è sufficiente cliccare l'icona in alto a destra una volta aperta la task assegnata su Asana.

\section{Ambiente di lavoro}
Gli strumenti indicati di seguito possono subire variazione durante lo svolgersi dei lavori.
\subsection{Coordinamento}
\subsubsection{Versionamento}
Per il versionamento si sceglie di utilizzare il software Git, attraverso la piattaforma \textit{GitHub$_{G}$}. Questa scelta deriva dal fatto che questo sistema era già conosciuto quasi da tutti i membri del gruppo. Inoltre essendo Git un sistema di versionamento distribuito si riduce il rischio di perdita di dati. 
\subsubsection{Pianificazione e Ticketing}
Per quanto riguarda la gestione delle attività si è scelto di utilizzare \textit{Asana$_{G}$}, in quanto il servizio premium risulta essere gratuito per gli studenti. Asana viene usato in concomitanza con \textit{Instagantt$_{G}$}, in quanto quest'ultimo permette una migliore gestione dei diagrammi.
\subsection{Documentazione}
\subsubsection{\LaTeX}
Per quanto riguarda la stesura della documentazione, si è deciso di utilizzare \LaTeX. Nonostante non sia subito di facile compressione, è stato scelto in quanto garantisce una migliore qualità tipografica dei documenti rispetto a un normale word processor.
\subsubsection{Editor}
 L'editor consigliato per scrivere in \LaTeX è\glossario{TeXstudio}in quanto si presenta come un software molto completo, supporta la codifica UTF-8, implementa il completamento automatico dei comandi e supporta il dizionario italiano. Questo semplifica il lavoro di stesura dei documenti.
 \subsubsection{Script}
Al fine di automatizzare la verifica dei documenti è stato creato il seguente script:
\begin{itemize}
	\item \textit{gulpease.py}: uno script per calcolare l'\textit{indice di Gulpease$_{G}$}.
\end{itemize} 
\subsubsection{Diagrammi UML}
Per modellare con\glossario{UML}il gruppo ZeroSeven ha scelto di utilizzare Astah Professional, disponibile in versione gratuita grazie alla licenza accademica. Astah Professional supporta la sintassi UML 2.x e permette di modellare i diagrammi dei casi d'uso, i diagrammi delle classi, degli oggetti, delle attività e di sequenza. 
\subsubsection{Creazione diagrammi di Gantt}
Lo strumento scelto per la realizzazione dei diagrammi di\glossario{Gantt}è \textit{GanttProject$_{G}$}, in quanto è gratuito, open source e multi piattaforma.
\subsection{Ambiente di sviluppo}
\subsubsection{Sistemi operativi}
I membri del gruppo\glossario{ZeroSeven}potranno lavorare indifferentemente su Windows, Linux o MacOS dal momento che tutti gli strumenti scelti ai fini del progetto sono disponibili per tutti e tre i sistemi operativi citati. 
\subsection{Ambiente di verifica}
\subsubsection{Documenti}
Per quanto riguarda la verifica dei documenti,\glossario{TeXstudio}non ha di default installato il dizionario per il controllo ortografico per la lingua italiana. E' stato quindi preparato un pacchetto da scompattare all'interno della cartella "dictionaries" dentro la directory di installazione di TeXstudio in modo da poter selezionare come lingua preferita l'italiano nella sezione "Controllo Linguistico" nelle impostazioni di TeXstudio.  In caso di difficoltà sarà compito dell'\textit{Amministratore} aiutare i membri del gruppo a effettuare correttamente questa configurazione.
\section{Formazione del personale}
Per quanto riguarda la formazione, i componenti del gruppo ZeroSeven devono provvedere autonomamente con lo studio delle tecnologie e degli strumenti necessari nel corso del progetto, prendendo come riferimento la seguente documentazione:
\begin{itemize}
	\item Per l'utilizzo di \LaTeX: \url{https://www.latex-project.org};
	\item Per l'utilizzo del software Git: \url{https://git-scm.com/docs};
	\item Per l'utilizzo del Gitflow Workflow: \url{https://it.atlassian.com/git/tutorials/comparing-workflows/gitflow-workflow}.
\end{itemize}
I membri del gruppo dovranno essere formati in modo da essere pronti per le attività pianificate nel \textit{Piano di Progetto v2.0.0}. Qualora alcuni membri presentino lacune, essi dovranno informare preventivamente il \textit{Responsabile} che provvederà a ridistribuire il carico di lavoro. 