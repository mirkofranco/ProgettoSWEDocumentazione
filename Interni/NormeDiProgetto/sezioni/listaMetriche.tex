\chapter{Metriche}

\subsection{Metriche per la qualità di processo}
Verranno utilizzate le seguenti metriche per valutare l'efficienza e l'efficacia 
processi.
\subsection{Metriche per la qualità di prodotto}
Verranno utilizzate le seguenti metriche per valutare l'efficienza e l'efficacia dei
prodotti.
\subsubsection{MPR001 Indice di Gulpease}
L’Indice Gulpease è un indice di leggibilità di un testo tarato sulla lingua italiana, per il suo calcolo vengono considerate due variabili linguistiche: la lunghezza delle parole e la lunghezza delle frasi rispetto al numero delle lettere.
\begin{center}{$G=89+\frac{300\times N_F-10\times N_L}{N_P}$}\end{center}

$N_F$ indica il numero delle frasi, $N_L$ indica il numero di lettere e $N_P$ indica il numero di parole nel testo.
\subsubsection{MPR002 Formula di Flesch}

\subsubsection{MPR003 Errori ortografici}

\subsubsection{MPR004 Completezza dell'implementazione funzionale}
Misura la quantità in percentuale di requisiti funzionali soddisfatti dalla corrente implementazione. Viene utilizzata la seguente formula: 
\begin{center}{$CO=\frac{N_{RS}}{N_{RT}}\times 100$}\end{center}
$N_{RS}$  indica il numero di requisiti soddistatti, $N_{RT}$  indica il numero totale di requisiti.

\subsubsection{MPR005 Correttezza  rispetto alle attese}
Misura la percentuale di risultati affini rispetto alle attese. Viene utilizzata la seguente formula:
\begin{center}{$CRA=(1-\frac{N_{TD}}{N_{TE}})\times 100$}\end{center}
${N_{TD}}$ indica il numero di test che producono risultati discordi rispetto alle attese, ${N_{TE}}$ indica il numero totale di test-case eseguiti.

\subsubsection{MPR006 Totalità di failure}
Misura la percentuale di test conclusi con una failure. Viene utilizzata la seguente formula:
\begin{center}{$TF=\frac{N_{FR}}{N_{TE}}\times 100$}\end{center}
${N_{FR}}$ indica il numero di failure rilevati durante l'attività di testing,
${N_{TE}}$ indica il numero totale di test-case eseguiti.

\subsubsection{MPR007 Tempo di risposta}
Misura la differenza media di tempo trascorsa dall’esecuzione di una funzionalità e la restituzione
dell’eventuale risultato. Viene utilizzata la seguente formula:
\begin{center}{$TR=\frac{\sum\limits_{i=1}^n {T_i }}{n}$}\end{center}
${T_i}$ indica il tempo (in secondi) trascorso dalla richiesta di una funzionalità ed il completamento di questa con un eventuale restituzione del risultato;\\

\subsubsection{MPR008 Comprensibilità  delle funzionalità offerte}
Misura la quantità in percentuale di operazioni comprese dall’utente che non richiedono la
consultazione del manuale. Viene utilizzata la seguente formula:
\begin{center}{$C=\frac{N_{FC}}{N_{FO}}\times 100$}\end{center}
${N_{FC}}$ indica il numero di funzionalità comprese in modo immediato dall'utente, ${N_{FO}}$ indica il numero di funzionalità totali offerte dal sistema;
	
\subsubsection{MPR09 Facilità apprendimento}
Misura il tempo medio che occorre ad un utente per imparare ad usare in maniera corretta
una certa funzionalità. Si misura tramite un indicatore numerico che indica i minuti
impiegati da un utente per apprendere il funzionamento di una certa funzionalità;

\subsubsection{MPR010 Capacità di analisi failure}
Misura la quantità in percentuale di failures incontrate di cui sono state tracciate le cause. Viene
utilizzata la seguente formula:
\begin{center}{$CAF=\frac{N_{FI}}{N_{FR}}\times 100 $}\end{center}
${N_{FI}}$ indica il numero di failure delle quali sono state individuate le cause, ${N_{FR}}$ indica il numero di failures rilevate.

\subsubsection{MPR011 Impatto delle modifiche}
Misura la quantità in percentuale di modifiche introdotte per risolvere failures che hanno introdotto nuove failures nel prodotto. Viene utilizzata la seguente formula:
\begin{center}{$IM=\frac{N_{FRE}}{N_{FR}}\times 100 $}\end{center}
${N_{FRF}}$ indica il numero di failure risolte introducentro nuove failure, ${N_{FR}}$ indica il numero di failures risolte.


