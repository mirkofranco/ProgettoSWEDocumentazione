\chapter{Lista di controllo}
\label{ListaControllo}
Durante l'applicazione del walkthrought ai documenti, sono state riportate le tipologie di errori più frequenti. La lista di controllo risultante è la seguente:
\begin{itemize}
	\item \textbf{Norme stilistiche}:
	\begin{itemize}
		\item \textbf{ordine righe nella tabella delle modifiche:} errore riscontrato nei primi documenti, la tabella deve iniziare con la modifica più recente e terminare con quella meno recente;
		\item \textbf{nomi dei documenti in grassetto:} devono essere in corsivo;
		\item evitare espressioni come "il fine di... \textbf{è quello di}": la parte in grassetto è del tutto ridondante;
		\item \textbf{evitare l'uso dell'accento nelle maiuscolo}: è necessario utilizzare l'apostrofo.
	\end{itemize}
	\item \textbf{Italiano}:
	\begin{itemize}
		\item \textbf{evitare il futuro:} quando possibile, il presente aiuta a rendere più chiara la frase;
	\end{itemize}
	\item \textbf{\LaTeX}:
	\begin{itemize}
		\item \textbf{Gestione cartelle in modo ricorsivo:} errore grave. Un file non può essere ripetuto più volte in cartelle diverse.
	\end{itemize}
\end{itemize}