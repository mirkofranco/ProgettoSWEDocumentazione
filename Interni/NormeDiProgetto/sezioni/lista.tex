\chapter{Lista di controllo}
\label{ListaControllo}
Durante l'applicazione del walkthrought ai documenti, sono state riportate le tipologie di errori più frequenti. La lista di controllo risultante è la seguente:
\begin{itemize}
	\item \textbf{Norme stilistiche}:
	\begin{itemize}
		\item \textbf{Ordine righe nel registro delle modifiche:} errore riscontrato nei primi documenti, il registro deve iniziare con la modifica più recente e terminare con quella meno recente;
		\item \textbf{Nomi dei documenti in grassetto:} devono essere in corsivo;
		\item \textbf{evitare espressioni inutili:} \textbf{il fine di}, \textbf{è quello di}", la parte in grassetto è del tutto ridondante;
		\item \textbf{Evitare l'uso dell'accento nelle maiuscolo}: è necessario utilizzare l'apostrofo;
		\item \textbf{Le citazioni devono essere più specifiche};
		\item \textbf{Fare attenzione all'uso delle section};
		\item \textbf{I titoli devono iniziare sempre con una lettera maiuscola};
	\end{itemize}
	\item \textbf{Italiano}:
	\begin{itemize}
		\item \textbf{Evitare il futuro:} quando possibile, il presente aiuta a rendere più chiara la frase;
	\end{itemize}
	\item \textbf{\LaTeX}:
	\begin{itemize}
		\item \textbf{Gestione cartelle in modo ricorsivo:} un file non può essere ripetuto più volte in cartelle diverse.
	\end{itemize}
\end{itemize}