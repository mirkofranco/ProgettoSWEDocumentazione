\documentclass[a4paper,12pt]{article}
\usepackage{ZeroSeven}

\titlepage{}
\author{Bianca Andreea Ciuche}
\date{2019-02-01}
\intestazioni{\includegraphics[scale=0.3]{images/logo_intestazione}}
\pagestyle{myfront}
\begin{document}
	\begin{titlepage}
		\centering
		{\huge\bfseries MegAlexa\par}
		Arricchitore di skill di Amazon Alexa
		\line(1,0){350} \\
		{\scshape\LARGE Verbale Interno del 2019-02-01 \par}
		\vspace{1cm}
		{\scshape Gruppo ZeroSeven \par}
		\logo
		%devono essere compilati questi campi ogni volta
		\begin{tabular}{c|c}
			{\hfill \textbf{Versione}} 			& 1.0.0				\\
			{\hfill\textbf{Data Redazione}} 	& 2019-02-03		\\ 
			{\hfill\textbf{Redazione}} 			&  		Bianca Andreea Ciuche		\\ 
			{\hfill\textbf{Verifica}} 				&  	  	\\ 
			{\hfill\textbf{Approvazione}} 		&  		\\ 
			{\hfill\textbf{Uso}} 					& 	Interno	\\ 
			{\hfill\textbf{Distribuzione}} 			& 			Prof. Tullio Vardanega \\ & Prof. Riccardo Cardin \\ & Gruppo ZeroSeven \\ & Zero12 s.r.l.	\\ 
			{\hfill\textbf{Email di contatto}} & zerosevenswe@gmail.com \\
		\end{tabular}
	\end{titlepage}
	
	
	
	\label{LastFrontPage}
	
	
	\newpage
	\cleardoublepage
	\begin{table}[tbph]
		\centering
		\begin{tabularx}{\textwidth}{|c|c|X|X|c|}
			\hline
			\textbf{Versione} & \textbf{Data} & \textbf{Descrizione} & \textbf{Autore} & \textbf{Ruolo} \\
			\hline
			0.0.2 & 2019-02-03 & Stesura verbale &Bianca Andreea Ciuche  & Analista \\
			\hline
			0.0.1 & 2018-12-08 & Creazione template documento & Ludovico Brocca & Amministratore\\
			\hline
		\end{tabularx}
		\caption{Diario delle modifiche}
	\end{table}
	\cleardoublepage
	\pagestyle{mymain}
	
	\tableofcontents
	\cleardoublepage
	\section{Informazioni sulla riunione}
	\subsection{Motivo della riunione} E' stata svolta una chiamata tramite Hangouts per discutere sulle varie problematiche emerse a seguito della valutazione ricevuta in Revisione dei Requisiti in data 21\textbackslash01\textbackslash2019.
	\begin{itemize}
		\item \textbf{Luogo e data}: Padova, Venerdì 1 Febbraio 2019;
		\item \textbf{Ora di inizio}: 15:15;
		\item \textbf{Ora di fine}: 16:20;
		\item \textbf{Partecipanti}:  
		\begin{itemize}
			\item Gian Marco Bratzu;
			\item Mirko Franco;
			\item Ciuche Bianca;
			\item Ludovico Brocca;
			\item Matteo Depascale;
			\item Stefano Zanatta.
		\end{itemize}
	\end{itemize}
	
	
	\section{Resoconto}
	\subsection{Argomenti discussi}
	Sono stati discussi i seguenti punti emersi dalla correzione ricevuta:
	\begin{itemize}		
		\item \textbf{Tutti i documenti:} 
			\begin{itemize}
				\item Cambiare il nome Tabella delle Modifiche in Registro delle modifiche. 
				\item Nel Registro delle modifiche il luogo di modifica deve essere riferito numericamente.
			\end{itemize}
		\item \textbf{Norme di Progetto:} 
		\begin{itemize}
			\item Introdurre un'appendice che descriva le metriche necessarie per la stesura del documento Piano di qualifica;
			\item  Introdurre un'identificazione dei codici per le decisioni che verranno prese nei verbali.
		\end{itemize}
	   	\item \textbf{Analisi dei Requisiti:} 
	   	\begin{itemize}
	   		\item Estendere il capitolo §2- Descrizione generale;
	   		\item Correggere i casi d'uso segnalati nella correzione;
	   		\item Inserire una spiegazione dei codici identificativi dei requisiti, correggere i requisiti funzionali segnalati nella correzione e inserire i giusti requisiti di vincolo;
		\end{itemize}
     	\item \textbf{Piano di Progetto:} 
     	\begin{itemize}
     		\item Inserire l'analisi dei rischi prima della pianificazione.
     		\item Rendere la pianificazione aderente a un modello di sviluppo incrementale incentrata sullo sviluppo del prodotto.
     		\item Eseguire le correzioni sulle tabelle in base alle segnalazione della correzione ricevuta;
     	\end{itemize}
     	\item \textbf{Piano di Qualifica:} 
     	\begin{itemize}
     		\item Rivedere la struttura e i contenuti del documento.
     		\item Rendere il resoconto §A a cruscotto con serie storiche e diagrammi.
        \end{itemize}
	\end{itemize}

    	\newpage
	\subsection{Tracciamento decisioni}
	\begin{table}[tbph]
		\centering
		\begin{tabularx}{\textwidth}{|C|C|}
			\hline
			\textbf{Codice } & \textbf{Decisione} \\
			\hline
			01-VER\_2019-02-01& Inserire i casi d'uso e i requisiti su PragmaDB\\
			\hline
		    02-VER\_2019-02-01& Express versione 4.16.4: framework utilizzato per NodeJS\\
		    \hline
		    03-VER\_2019-02-01& Jasmine versione 3.0.0: strumento utilizzato per effettuare i test per il codice in NodeJS\\
			\hline
			04-VER\_2019-02-01& NPM versione 6.4.1: strumento utilizzato per la gestione della build del codice prodotto \\
			\hline
			05-VER\_2019-02-01& AWS Lambda: strumento utilizzato per l'esecuzione del codice nel server\\
			\hline
			06-VER\_2019-02-01& DynamoDB: strumento utilizzato come database NoSQL\\
			\hline
		\end{tabularx}
		\caption{Tracciamento decisioni}
	\end{table}

	\label{LastPage}
\end{document}