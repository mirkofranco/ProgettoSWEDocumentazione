\documentclass[a4paper,12pt]{article}
\usepackage{ZeroSeven}

\titlepage{}
\author{Bianca Andreea Ciuche}
\date{2019-03-20}
\intestazioni{\includegraphics[scale=0.3]{images/logo_intestazione}}
\pagestyle{myfront}
\begin{document}
	\begin{titlepage}
		\centering
		{\huge\bfseries MegAlexa\par}
		Arricchitore di skill di Amazon Alexa
		\line(1,0){350} \\
		{\scshape\LARGE Verbale Interno del 2019-03-18 \par}
		\vspace{1cm}
		{\scshape Gruppo ZeroSeven \par}
		\logo
		%devono essere compilati questi campi ogni volta
		\begin{tabular}{c|c}
			{\hfill \textbf{Versione}} 			& 1.0.0				\\
			{\hfill\textbf{Data Redazione}} 	& 2019-03-20		\\ 
			{\hfill\textbf{Redazione}} 			&  		Bianca Andreea Ciuche		\\ 
			{\hfill\textbf{Verifica}} 				&  	 Matteo Depascale \\ 
			{\hfill\textbf{Approvazione}} 		&  Ludovico Brocca\\ 
			{\hfill\textbf{Uso}} 					& 	Interno	\\ 
			{\hfill\textbf{Distribuzione}} 			& 			Prof. Tullio Vardanega \\ & Prof. Riccardo Cardin \\ & Gruppo ZeroSeven 	\\ 
			{\hfill\textbf{Email di contatto}} & zerosevenswe@gmail.com \\
		\end{tabular}
	\end{titlepage}
	
	
	
	\label{LastFrontPage}
	
	
	\newpage
	\cleardoublepage
	\begin{table}[tbph]
		\centering
		\begin{tabularx}{\textwidth}{|c|c|X|X|c|}
			\hline
			\textbf{Versione} & \textbf{Data} & \textbf{Descrizione} & \textbf{Autore} & \textbf{Ruolo} \\
			\hline
			1.0.0 & 2019-03-24 & Approvazione del documento per rilascio& Ludovico Brocca &  Responsabile \\
			\hline
			0.1.0 & 2019-03-23 & Verifica verbale con esito positivo & Matteo Depascale  & Verificatore \\
			\hline
			0.0.2 & 2019-03-20 & Stesura verbale &Bianca Andreea Ciuche  & Analista \\
			\hline
			0.0.1 & 2018-12-08 & Creazione template documento & Ludovico Brocca & Amministratore\\
			\hline
		\end{tabularx}
		\caption{Registro delle modifiche}
	\end{table}
	\cleardoublepage
	\pagestyle{mymain}
	
	\tableofcontents
	\cleardoublepage
	\section{Informazioni sulla riunione}
	\subsection{Motivo della riunione} \`{E} stata svolta una chiamata tramite  \glossario{Hangouts} per discutere sull'esito ricevuto in Revisione di Progetto in data 2019-03-15 e per svolgere una suddivisione del lavoro futuro.
	\begin{itemize}
		\item \textbf{Luogo e data}: Padova, Lunedì \label{key}18 Marzo 2019;
		\item \textbf{Ora di inizio}: 15:15;
		\item \textbf{Ora di fine}: 16:50;
		\item \textbf{Partecipanti}:  
		\begin{itemize}
			\item Gian Marco Bratzu;
			\item Mirko Franco;
			\item Bianca Andreea Ciuche;
			\item Ludovico Brocca;
			\item Matteo Depascale;
			\item Andrea Deidda;
			\item Stefano Zanatta.
		\end{itemize}
	\end{itemize}
	
	
	\section{Resoconto}
	\subsection{Argomenti discussi}
	La riunione prevede i seguenti punti:
	\begin{itemize}		
		\item Sono state discusse le correzioni da effettuare sui documenti in base alla valutazione ricevuta;
		\item Si è deciso di chiedere alla  \glossario{proponente} in che lingua devono essere scritti il Manuale Utente e Manuale Sviluppatore;
		\item \`{E} stata svolta la suddivisione del lavoro per la progettazione del software;
		\item il gruppo ha trovato difficoltà nella progettazione del requisito facoltativo RFF25 (Alexa deve fornire un sistema di aiuto che dice all’utente in qualsiasi istante in quale punto dell’esecuzione si trova), in quanto nella pratica non ha molto senso come requisito, per questo è stato spostato in "facoltativo non accettato";
		\item con una analisi più approfondita, il gruppo ha ritenuto che il requisito RFF26 (I dati utente verranno automaticamente sincronizzati con il proprio account Amazon) non è qualcosa di necessario e realizzabile, per questo motivo è stato spostato in "facoltativo non accettato";
		\item il gruppo ha deciso di fare il passaggio da JavaScript a TypeScript. Questa decisione è stata presa, tenendo in considerazione che i tempi di codifica e risoluzione degli errori saranno ridotti in modo considerevole. Inoltre è stato stimato che la traduzione da JavaScript a TypeScript sarà veloce.    
	\end{itemize}

    	\newpage
	\subsection{Tracciamento decisioni}
	\begin{table}[tbph]
		\centering
		\begin{tabularx}{\textwidth}{|C|C|}
			\hline
			\textbf{Codice } & \textbf{Decisione} \\
			\hline
			01-VER\_2019-03-18& Suddivisione delle correzione per documenti \\
			\hline
		    02-VER\_2019-03-18& Chiedere alla  \glossario{proponente} un appuntamento per una riunione tramite Hangouts\\
			\hline
		    03-VER\_2019-03-18& Divisione dei compiti per la progettazione\\
			\hline
			04-VER\_2019-03-18& rimozione requitito RFF25\\
			\hline
			05-VER\_2019-03-18& rimozione requisito RFF26\\
			\hline
			05-VER\_2019-03-18& passaggio a TypeScript\\
			\hline
		\end{tabularx}
		\caption{Tracciamento decisioni}
	\end{table}

	\label{LastPage}
\end{document}