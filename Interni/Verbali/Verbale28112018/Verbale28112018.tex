\documentclass[a4paper,12pt]{article}

\usepackage{ZeroSeven}

\titlepage{}

\author{Matteo Depascale}
\date{2018-11-29}
\intestazioni{Gruppo ZeroSeven}
\pagestyle{myfront}
\begin{document}
\begin{titlepage}
	\centering
	{\huge\bfseries MegAlexa\par}
	Arricchitore di skill di Amazon Alexa
	\line(1,0){350} \\
	{\scshape\LARGE Verbale Interno del 28-11-2018 \par}
	\vspace{1cm}
	{\scshape Gruppo ZeroSeven \par}
	\logo
	%devono essere compilati questi campi ogni volta
	\begin{tabular}{c|c}
		{\hfill \textbf{Versione}} 			& 1.0.0				\\
		{\hfill\textbf{Data Redazione}} 	& 2018-11-29		\\ 
		{\hfill\textbf{Redazione}} 			&  		Matteo Depascale			\\ 
		{\hfill\textbf{Verifica}} 				&  		Bianca Ciuche		\\ 
		{\hfill\textbf{Approvazione}} 		&  		Mirko Franco	\\ 
		{\hfill\textbf{Uso}} 					& 	Interno	\\ 
		{\hfill\textbf{Distribuzione}} 			& 			Prof. Tullio Vardanega \\ & Prof. Riccardo Cardin \\ & Gruppo ZeroSeven		\\ 
		{\hfill\textbf{Email di contatto}} & zerosevenswe@gmail.com \\
	\end{tabular}
\end{titlepage}
	

	
	\label{LastFrontPage}
	

	\newpage
	\cleardoublepage
		\begin{table}[tbph]
		\centering
		\begin{tabularx}{\textwidth}{|c|c|X|X|c|}
			\hline
			\textbf{Versione} & \textbf{Data} & \textbf{Descrizione} & \textbf{Autore} & \textbf{Ruolo} \\
			\hline
			0.0.1 & 2018-11-29 & Creazione scheletro documento
			& Matteo Depascale & Verificatore\\
			\hline
			0.0.2 & 2018-11-29 & Stesura verbale & Matteo Depascale & Verificatore \\
			\hline
			0.0.3 & 2018-12-04 & Verifica con esito positivo
			& Bianca Ciuche & Verificatore \\
			\hline
			1.0.0 & 2018-12-13 & 
			Approvazione per Revisione dei Requisiti
			& Mirko Franco &  Responsabile \\
			\hline
		\end{tabularx}
		\caption{Diario delle modifiche}
	\end{table}
	
	\cleardoublepage
	\pagestyle{mymain}
	
	\tableofcontents
	\cleardoublepage
	
	\section{Informazioni sulla riunione}
	\subsection{Motivo della riunione}
	La riunione è stata convocata, attraverso il gruppo Telegram, per discutere argomenti riguardanti le milestones e i principali strumenti di supporto da adottare.
	Di seguito vengono riportate alcune informazioni dell'incontro:
	\begin{itemize}
		\item \textbf{Luogo e data}: sede del Dipartimento di Matematica "Tulio Levi-Civita", Università degli Studi Di Padova, Lunedì 28 Novembre 2018
		\item \textbf{Ora di inizio}: 14:30
		\item \textbf{Ora di fine}: 16:00
		\item \textbf{Partecipanti: } tutti i componenti del gruppo \texttt{ZeroSeven} 
	\end{itemize}
	\section{Ordine del giorno}
	Di seguito viene riportato l'ordine del giorno con il quale è stata convocata la riunione:
	\begin{itemize}
		\item Scelta delle date per l'inserimento delle milestones
		\item Strumenti di supporto da adottare
		\item Richiesta di un incontro con l'azienda proponente
		\item Stesura delle possibili domande da fare all'azienda \textit{zero12}
	\end{itemize}
	\section{Resoconto}
	\subsection{Inserimento di milestones}
	Il gruppo discute su quale sia la scelta migliore per numero e posizionamento delle milestones. Oltre a quelle obbligatorie corrispondenti alle quattro revisioni si decide di posizionarne almeno altre due. La prima in concomitanza con l'incontro dei gruppi del primo lotto previsto per metà Dicembre. La seconda si decide di posizionarla verso la fine di Febbraio, tra la Revisione dei Requisiti e la Revisione di Progettazione. Per maggiori e più approfonditi di dettagli si rimanda al \texttt{Piano di Progetto v1.0.0}.
%	Il gruppo ha scelto di inserire le milestones nelle seguenti date:
%	\begin{itemize}
%		\item 11-12 dicembre 2018, prima dell'incontro con il prof. Vardanega che avverrà il 14 dicembre
%		\item 11-12 gennaio 2019, in modo da esser preparati per la consegna dell'offerta alla proponente e alla successiva % %	%	%		\texttt{Revisione dei Requisiti (RR)}
%		\item 18-19 febbraio 2019, per la \texttt{Revisione di Progettazione (RP)}
%	\end{itemize}
%	Inoltre si è deciso che ulteriori date per le milestones verranno aggiunte non appena, grazie al proseguire del progetto, si %	%	avrà un quadro più chiaro della situazione.
	\subsection{Strumenti di supporto}
	Come prima cosa è stato deciso di non avere una sola repository su \textit{Github}, ma bensì averne due distinte: una contenente i documenti e l'altra contente il codice, in tal modo non avverrà una build ad ogni modifica dei documenti.
	Durante il corso del progetto verranno adottati i seguenti sistemi:
	\begin{itemize}
		\item\textit{Asana}: si è discusso riguardo all'utilizzo di due strumenti di integrazione di \textit{Github}, ovvero \textit{Jira} e \textit{Asana}. Si è deciso di utilizzare \textit{Asana} sia per quanto riguarda la possibilità di ottenere la loro licenza premium per 6 mesi, sia per la maggior semplicità che offre nel gestire il progetto. \\
		Verrà inoltre adottato per la sollevazione e la discussione delle issues durante il progetto
		\item \textit{TexStudio}: per la scrittura e la compilazione dei documenti \LaTeX 
		\item \textit{InstaGantt}: per la creazione dei diagrammi di \textit{Gantt}, è possibile anche integrarlo con \textit{Asana}
		\item \textit{GitFlow}: verrà integrato a \textit{GIthub} per una miglior gestibilità dei vari branch
		\item \textit{SonarQube}: per l'ispezione e la qualità del codice		
	\end{itemize}	
	Inoltre vengono avanzate alcune proposte riguardo ai diversi grafici \textit{UML} da adottare, in merito verrà presa una decisione entro la prossima riunione. Si è abbozzata anche la possibilità di utilizzare un proprio server, anche per questo argomento è stata rimandata la decisione.
	\subsection{Richiesta d'incontro}
	Viene inviata una email all'azienda \textit{Zero12} con la richiesta di un incontro al quale parteciperanno almeno 4 componenti del gruppo
	\subsection{Argomenti su cui informarsi}
	Di seguito è presente una lista degli argomenti su cui si baseranno le domande durante l'incontro con l'azienda \textit{Zero12}:
	\begin{itemize}
		\item Possibilità di realizzare il progetto solamente per \textit{Andorid}, e nel caso se è possibile fare una, seppur minima, implementazione in \textit{iOS}
		\item Spiegazione più dettagliata riguardo al \textit{Voice Dialog Flow}
		\item Quale \textit{database} verrà utilizzato e se verrà fornito dall'azienda
		\item La \textit{repository} proposta potrà essere implementata a con funzionalità aggiuntive?
		\item Lingua di preferenza per la scrittura dei documenti
		\item Qualità del codice
		\item Requisiti da soddisfare
		\item Supporto futuro 
	\end{itemize}
	
	\label{LastPage}

\end{document}
