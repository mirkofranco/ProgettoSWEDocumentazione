\documentclass[a4paper,12pt]{article}

\usepackage{ZeroSeven}

\titlepage{}

\author{Mirko Franco}
\date{26-11-2018}
\intestazioni{Gruppo ZeroSeven}
\pagestyle{myfront}
\begin{document}
\begin{titlepage}
	\centering
	{\huge\bfseries MegAlexa\par}
	Arricchitore di skill di Amazon Alexa
	\line(1,0){350} \\
	{\scshape\LARGE Verbale Interno 19-11-2018 \par}
	\vspace{1cm}
	{\scshape Gruppo ZeroSeven \par}
	\logo
	%devono essere compilati questi campi ogni volta
	\begin{tabular}{c|c}
		{\hfill \textbf{Versione}} 			& 0.0.1				\\
		{\hfill\textbf{Data Redazione}} 	& 22-11-2018		\\ 
		{\hfill\textbf{Redazione}} 			&  		Mirko Franco			\\ 
		{\hfill\textbf{Verifica}} 				&  		Matteo Depascale		\\ 
		{\hfill\textbf{Approvazione}} 		&  			\\ 
		{\hfill\textbf{Uso}} 					& 	Interno	\\ 
		{\hfill\textbf{Distribuzione}} 			& 			Prof. Tullio Vardanega \\ & Prof. Riccardo Cardin \\ & Gruppo ZeroSeven		\\ 
		{\hfill\textbf{Email di contatto}} & zerosevenswe@gmail.com \\
	\end{tabular}
\end{titlepage}
	

	
	\label{LastFrontPage}
	

	\newpage
	\cleardoublepage
	\begin{center}
		\textbf{Diario delle modifiche}
	\end{center}
	\begin{center}
		\begin{tabular}{|c|c|c|c|c|}
			\hline
			\textbf{Versione} & \textbf{Data} & \textbf{Descrizione} & \textbf{Autore} & \textbf{Ruolo} \\
			\hline
			0.0.1 & 2018-11-22 & Creazione scheletro documento & Mirko Franco & Responsabile\\
			\hline
			0.0.2 & 2018-11-22 & Stesura verbale & Mirko Franco & Responsabile \\
			\hline
		\end{tabular}
	\end{center}
	
	\cleardoublepage
	\pagestyle{mymain}
	
	\tableofcontents
	\cleardoublepage
	
	\section{Informazioni sulla riunione}
	\subsection{Motivo della riunione}
	La riunione è stata convocata, attraverso il gruppo Telegram creato subito dopo la formazione dei gruppi, per approfondire la conoscenza tra i membri del gruppo e per una prima discussione sulla scelta del capitolato e  sui possibili strumenti di supporto da utilizzare. 
	Di seguito vengono riportate alcune informazioni generali.
	\begin{itemize}
		\item \textbf{Luogo e data}: sede del Dipartimento di Matematica "Tulio Levi-Civita", Università degli Studi Di Padova, Lunedì 19 Novembre 2018
		\item \textbf{Ora di inizio}: 15:00
		\item \textbf{Ora di fine}: 16:30
		\item \textbf{Partecipanti: } tutti i componenti del gruppo \texttt{ZeroSeven} 
	\end{itemize}
	\section{Ordine del giorno}
	Di seguito viene riportato l'ordine del giorno con il quale è stata convocata la riunione:
	\begin{itemize}
		\item Scelta del nome del gruppo e del logo
		\item Strumenti di comunicazioni con l'esterno: mail di contatto
		\item Scelta di una prima disposizione di ruoli
		\item Discussione sui possibili strumenti di supporto al progetto
		\item Scelta del capitolato
	\end{itemize}
	\section{Resoconto}
	\subsection{Nome e logo}
	Dopo una breve discussione il gruppo sceglie \textit{ZeroSeven} come nome con il quale identificarsi.\\
	Si procede con delle proposte di logo da parte dei componenti del gruppo tramite degli schizzi. Viene scelto poi il logo ufficiale che verrà realizzato.
	\subsection{Mail di contatto}
	Il gruppo sceglie di utilizzare il servizio \textit{Gmail} per l'apertura di una casella di posta elettronica necessaria alla comunicazione con enti esterni, quali committente e proponente.\\
	Si decide che l'accesso e il conseguente utilizzo di questa casella di posta è riservato al responsabile, che agirà per nome e per conto dell'intero gruppo. 
	\subsection{Ruoli}
	Si discute brevemente di una prima ripartizione dei ruoli per un primo periodo. Si rimanda al \texttt{Piano di Progetto v1.0.0} per ulteriori e più approfondite informazioni.
	\subsection{Strumenti di supporto}
	Avviene una prima generale discussione su possibili strumenti da utilizzare. \\
	Viene deciso di utilizzare \textit{git} come sistema di versionamento tramite la piattaforma \textit{Github}. \\
	Non vengono prese altre decisioni in merito, ma si rimanda la discussione a future riunioni. 
	Sarà valido quanto verrà scritto nelle \texttt{Nome di Progetto v.1.0.0}.
	\subsection{Scelta del capitolato}
	Dopo aver lungamente discusso e valutato i vari capitolati i componenti del gruppo esprimono la loro preferenza per il capitolato \texttt{MegAlexa: arricchitore di skill di Amazon Alexa}. Per ulteriori e più approfonditi dettagli si rimanda allo \texttt{Studio di Fattibilità}.
	\label{LastPage}

\end{document}
