\section{Processi organizzativi}
\subsection{Gestione}
\subsubsection{Ruoli di progetto}
I componenti del gruppo di progetto sono tenuti a ripartire tra loro in modo equo e omogeneo i differenti ruoli. A tal fine i ruoli verranno assegnati a rotazione dal Responsabile di Progetto, il quale si assicurerà di non creare conflitti di interesse come as es. un programmatore chiamato poi in veste di verificatore delle proprie features.
\paragraph{Amministratore di Progetto} L'amministratore è responsabile dell'efficienza e dell'operatività dell'ambiente di sviluppo. Si occupa della redazione e dell'attuazione dei piani e delle procedure di Gestione per la qualità. Redige le norme di\glossario{progetto}per conto del responsabile e collabora alla stesura del piano di progetto. Gestisce l'archivio della documentazione di progetto, controlla versioni e configurazioni del prodotto.
\paragraph{Responsabile di Progetto}E' responsabile ultimo dei risultati del progetto. Sua responsabilità sono l'approvazione dell'emissione di documenti, l'elaborazione e l'emanazione di piani e scadenze. Coordina il gruppo e le attività. E' suo compito gestire le relazioni con il controllo di qualità interno al\glossario{progetto}. Infine approva l'offerta e i suoi allegati.
\paragraph{Analista} Si occupa dell'attività di analisi, della redazione dello studio di fattibilità e dell'analisi dei requisiti.
\paragraph{Progettista} Suo compito è l'attività di progettazione, inoltre redige la specifica tecnica, la definizione di prodotto e la parte programmatica del piano di qualifica. 
\paragraph{Verificatore} Si occupa dell'attività di\glossario{verifica}. Redige la parte retrospettiva del piano di qualifica, che illustra l'esito e la completezza delle verifiche e delle prove effettuate secondo il piano. Si occupa  inoltre dell'attività di controllo, in particolare verifica se i documenti e i prodotti dei processi rispettano le attese e sono conformi alle Norme di Progetto.
\paragraph{Programmatore} Si occupa delle attività di codifica che hanno lo scopo di realizzare il\glossario{prodotto}e le componenti di ausilio necessarie all'esecuzione delle prove di verifica e di validazione. 
\subsection{Procedure}
\subsubsection{Gestione delle comunicazioni}
La seguente sezione definisce le direttive in merito alle comunicazioni interne ed esterne del gruppo\glossario{ZeroSeven}.
\paragraph{Comunicazioni interne}
La comunicazione tra membri appartenenti al gruppo avviene tramite l'utilizzo di un gruppo\glossario{Telegram}\footnote{Guida, Documentazione, FAQ di Telegram-\url{https://telegram.org/}}.
\paragraph{Comunicazioni esterne}
Le comunicazioni esterne avvengono tramite l'utilizzo della mail \mailzeroseven, la cui gestione è assegnata al \textit{Responsabile di Progetto}.
La mail deve essere scritta in prima persona singolare, utilizzando un tono formale, è obbligatorio inoltre dichiarare nome, cognome e ruolo prima dell'invio.
\subsubsection{Gestione degli incontri}
\paragraph{Incontri interni} L'organizzazione di incontri interni avviene tramite l'utilizzo del sopracitato gruppo Telegram, ad ogni incontro interno segue la redazione di un \textit{Verbale Interno} la cui\glossario{verifica}e validazione spetterà a un \textit{Verificatore}.
\paragraph{Incontri esterni}\label{Com esterne} L'organizzazione di incontri esterni (con \textit{Proponenti},\textit{Committenti} etc.) avviene tramite l'utilizzo della mail \\ \mailzeroseven (vedi \hyperref[Com esterne]{sezione 4.2.1}), ad ogni incontro esterno segue la redazione di un \textit{Verbale Esterno}. Una volta validato tale documento, esso dovrà essere inviato alla controparte presente all'incontro.

\subsubsection{Gestione degli strumenti di coordinamento} Per la gestione di attività e compiti viene utilizzato\glossario{Asana}\footnote{Guida e Documentazione di Asana -\url{https://asana.com/guide}}, che permette di gestire facilmente progetti di grandi dimensioni.
L'assegnazione di compiti tramite questo strumento è affidata al \textit{Responsabile}, seguendo le scadenze precedentemente stabilite nel \textit{Piano di Progetto}.