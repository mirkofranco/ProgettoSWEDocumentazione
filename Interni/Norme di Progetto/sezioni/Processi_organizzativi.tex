\section{Processi organizzativi}
\subsection{Gestione}
\subsubsection{Ruoli di progetto}
I componenti del gruppo di progetto sono tenuti a ripartire tra loro in modo equo e omogeneo i differenti ruoli. A tal fine i ruoli verranno assegnati a rotazione dal Responsabile di Progetto, il quale si assicurerà di non creare conflitti di interesse come as es. un programmatore chiamato poi in veste di verificatore delle proprie features.
\paragraph{Amministratore di Progetto} L'amministratore è responsabile dell'efficienza e dell'operatività dell'ambiente di sviluppo. Si occupa della redazione e dell'attuazione dei piani e delle procedure di Gestione per la qualità. Redige le norme di progetto per conto del responsabile e collabora alla stesura del piano di progetto. Gestisce l'archivio della documentazione di progetto, controlla versioni e configurazioni del prodotto.
\paragraph{Responsabile di Progetto}E' responsabile ultimo dei risultati del progetto. Sua responsabilità sono l'approvazione dell'emissione di documenti, l'elaborazione e l'emanazione di piani e scadenze. Coordina il gruppo e le attività. E' suo compito gestire le relazioni con il controllo di qualità interno al progetto. Infine approva l'offerta e i suoi allegati.
\paragraph{Analista} Si occupa dell'attività di analisi, della redazione dello studio di fattibilità e dell'analisi dei requisiti.
\paragraph{Progettista} Suo compito è l'attività di progettazione, inoltre redige la specifica tecnica, la definizione di prodotto e la parte programmatica del piano di qualifica. 
\paragraph{Verificatore} Si occupa dell'attività di verifica. Redige la parte retrospettiva del piano di qualifica, che illustra l'esito e la completezza delle verifiche e delle prove effettuate secondo il piano. 
\paragraph{Programmatore} Si occupa delle attività di codifica che hanno lo scopo di realizzare il prodotto e le componenti di ausilio necessarie all'esecuzione delle prove di verifica e di validazione. 
\paragraph{Validatore}
\subsection{Procedure}
\subsubsection{Gestione delle comunicazioni}
\paragraph{Comunicazioni interne}
\paragraph{Comunicazioni esterne}
\subsubsection{Gestione degli incontri}
\paragraph{Incontri interni}
\paragraph{Incontri esterni}
\subsubsection{Gestione degli strumenti di coordinamento}