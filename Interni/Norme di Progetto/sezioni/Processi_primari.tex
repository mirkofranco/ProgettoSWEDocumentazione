\chapter{Processi primari}
\section{Fornitura}			
Questo\glossario{processo}ha lo scopo di trattare le norme e i termini che i membri del gruppo\glossario{ZeroSeven}sono tenuti a rispettare per diventare fornitori della proponente Zero12 s.r.l e dei committenti Prof. Tullio Vardanega e Prof. Riccardo Cardin. \\
L'aspettativa del gruppo è instaurare e mantenere un rapporto di collaborazione costante con Zero12 s.r.l. per poter soddisfare al meglio i bisogni dell'azienda cliente e rispettare i vincoli richiesti.
\subsection{Studio di fattibilità} 
Dopo la presentazione dei\glossario{capitolati}, è compito del \textit{Responsabile} convocare le riunioni necessarie per consentire al gruppo di esprimere idee e opinioni riguardo ai diversi capitolati. Le informazioni raccolte verranno utilizzate dagli \textit{Analisti} per redigere lo \textit{Studio di Fattibilità}. L'analisi si basa sui seguenti punti:
\begin{itemize}
	\item \textbf{tecnologie utilizzate:} se queste sono interessanti e soprattutto non sono state ancora applicate dalla maggior parte del gruppo;
	\item \textbf{interesse futuro:} viene valutato se si desidera perseguire una carriera, o semplicemente avere più competenze, nell'ambito di lavoro richiesto dal capitolato;
	\item \textbf{individuazione rischi:} vengono valutati gli eventuali punti critici che il gruppo potrebbe incontrare nel corso del progetto.
\end{itemize}
\subsection{Rapporti di fornitura con la Proponente}
Durante l'intero progetto si intende instaurare un costante rapporto collaborativo con la Proponente, nella persona del Sig. Stefano Dindo, con il fine di:
\begin{itemize}
	\item determinare bisogni in modo più accurato possibile;
	\item scegliere in modo collaborativo i requisiti del prodotto;
	\item accordare insieme vincoli di qualità del prodotto;
	\item stimare tempi e costi.
\end{itemize}
A seguito della consegna del prodotto, salvo ulteriori accordi, non seguirà l'attività di manutenzione del prodotto.
\subsection{Documentazione fornita}
Di seguito viene indicata la documentazione che verrà fornita alla\glossario{Proponente}e al Committente con lo scopo di rendere trasparenti le scelte di:
\begin{itemize}
	\item \textbf{pianificazione}: \texttt{Piano di Progetto v1.0.0};
	\item \textbf{processi}: \texttt{Norme di Progetto v1.0.0};
	\item \textbf{verifica e validazione}: \texttt{Piano di Qualifica v1.0.0};
	\item \textbf{obbiettivi di qualità}: \texttt{Piano di Qualifica v1.0.0};
	\item \textbf{requisiti}: \texttt{Analisi dei Requisiti v1,0,0}.
\end{itemize}
%\paragraph{Piano di Progetto}
%Il \textit{Responsabile} si occupa della scrittura del \textit{Piano di Progetto} con l'obiettivo di organizzare le attività con efficienza e %misurare l'avanzamento del\glossario{progetto}, pertanto il documento in questione dovrà contenere:
%\begin{itemize}
%	\item \textbf{Analisi dei Rischi:} vengono analizzati sia i rischi, che potrebbero sorgere durante lo svolgersi del progetto, sia le %possibili soluzioni a questi rischi, tenendo conto dei relativi costi e delle scadenz
%	\item \textbf{Pianificazione:} si organizza nel dettaglio le attività da svolgere.
%	\item \textbf{Consuntivo di periodo e Preventivo a finire:} viene redatto il consultivo di periodo alla fine di ogni attività. In questo %modo si possono confrontare i risultati ottenuti con quelli attesi. Questi ultimi sono visibili nel preventivo, il quale contiene una stima %del lavoro necessario.
%\end{itemize}
%\paragraph{Piano di Qualifica}
%Gli \textit{Amministratori} documentano le norme per la\glossario{verifica}e la validazione dei prodotti e dei processi, la verifica avviene %con costanza affinché non vengano introdotti errori. Il documento contiene:
%\begin{itemize}
%	\item \textbf{Qualità di processo}: per garantire la qualità dei processi attraverso l'utilizzo di strumenti e metriche, inoltre è %necessario adottare come standard \textit{ISO/IEC 15504}.
%	\item \textbf{Qualità di prodotto}: per garantire e misurare la qualità dei prodotti sviluppati, è necessario adottare come standard %\textit{ISO/IEC 15010}.
%\end{itemize}

%\paragraph{Verbali Esterni}
%Vengono verbalizzati tutti gli incontri avvenuti con l'azienda Zero12 s.r.l., la redazione del verbale esterno è compito degli %\textit{Analisti}. I verbali esterni verranno inclusi nella documentazione fornita, inoltre si cercherà, ove possibile, di inviare una copia %del suddetto verbale all'azienda cliente.

\section{Sviluppo}
Il\glossario{processo}di sviluppo comprende tutte le attività e i compiti svolti dal gruppo per produrre il software richiesto dal\glossario{proponente}. Dallo sviluppo del\glossario{prodotto}ci si aspetta che questo soddisfi i test di verifica e di validazioni, gli obiettivi imposti e i bisogni dell'azienda cliente. Il processo di sviluppo si svolge secondo lo standard \textit{ISO/IEC 12207}.
\subsection{Analisi dei requisiti}
Gli \textit{Analisti} si occupano di analizzare nel dettaglio i\glossario{requisiti}che il prodotto finale dovrà soddisfare, il risultato viene poi documentato nell' \textit{Analisi dei Requisiti}. Vengono utilizzati i casi d'uso per l'analisi e la ricerca dei requisiti.
Il documento redatto dovrà rispettare le specifiche di seguito riportate.
\subsubsection{Fonti dei requisiti}
Compito di analisti è quello di redigere una lista di\glossario{requisiti}. Tali requisiti possono essere ricavati da varie fonti, le quali sono:
\begin{itemize}
	\item \textbf{Capitolato}: i requisiti emersi dall'analisi del documento fornito dalla\glossario{Proponente};
	\item \textbf{Verbali Esterni}: i requisiti emersi durante incontri con la Proponente;
	\item \textbf{Interni}: i requisiti emersi tramiti analisi e discussione interna del gruppo\glossario{ZeroSeven};
	\item \textbf{Casi d'uso}: i requisiti emersi dall'analisi di uno o più casi d'uso.
\end{itemize}
\subsubsection{Classificazione dei requisiti}
Ogni\glossario{requisito}deve avere la seguente nomenclatura:
\begin{center}
	\textbf{R$\Bigl\{$A$\Bigr\}$$\Bigl\{$B$\Bigr\}$$\Bigl\{$XX$\Bigr\}$.$\Bigl\{$YY$\Bigr\}$}
\end{center}
dove:
\begin{itemize}
	\item \textbf{A:} corrisponde a uno dei seguenti requisiti:
	\begin{itemize}
		\item 1: funzionale;
		\item 2: di qualità;
		\item 3: di prestazione;
		\item 4: di vincolo.
	\end{itemize}
	\item \textbf{B:} corrisponde a uno dei seguenti requisiti:
	\begin{itemize}
		\item O: obbligatorio;
		\item F: facoltativo;
		\item D: desiderabile.
	\end{itemize}
	\item \textbf{{XX}:} numero che identifica i\glossario{requisiti};
	\item \textbf{{YY}:} numero progressivo che identifica i sottocasi, esso può, a sua volta, includere altri sottocasi.
\end{itemize}

\subsubsection{Casi d'uso}
Ogni caso d'uso deve presentare i seguenti campi:
\begin{itemize}
	\item Codice identificativo;
	\item Titolo
	\item Diagramma\glossario{UML};
	\item Attori primari;
	\item Attori secondari;
	\item Scopo e descrizione;
	\item Precondizione;
	\item Postcondizione;
	\item Flusso principale degli eventi.
\end{itemize}
\subsubsection{Codice identificativo CU}
Ogni caso d'uso deve avere la seguente nomenclatura:
\begin{center}
	\textbf{UC$\Bigl\{$XX$\Bigr\}$.$\Bigl\{$YY$\Bigr\}$}
\end{center}
dove:
\begin{itemize}
	\item \textbf{UC:} Use Case;
	\item \textbf{{XX}:} numero che identifica i casi d'uso;
	\item \textbf{{YY}:} numero progressivo che identifica i sottocasi, esso può, a sua volta, includere altri sottocasi.
\end{itemize}


%\subparagraph{Diagrammi UML}
%Per la stesura dei diagrammi UML vengono adottate le seguenti convenzioni:
%\begin{itemize}
%	\item \textbf{Versione:} lo standard utilizzato per i grafici UML è la \textit{v2.0}.
%	\item \textbf{Lingua:} la lingua utilizzata è l'italiano.
%\end{itemize}


%\subparagraph{Requisiti}
%Ogni requisito deve presentare i seguenti campi:
%\begin{itemize}
%	\item Codice identificativo
%	\item Descrizione
%	\item Fonti
%\end{itemize}


\subsection{Progettazione}
%La progettazione deve poter dimostrare che tutti i\glossario{requisiti}specificati nell'\textit{Analisi dei Requisiti} siano rispettati. Per far %ciò, c'è bisogno della stesura dei seguenti documenti:
%\begin{itemize}
%	\item Specifica tecnica
%	\item Definizione di Prodotto
%\end{itemize}
\subsubsection{Scopo}
L'attività di Progettazione precede obbligatoriamente la produzione del software e consiste nel descrivere e fornire una soluzione al problema che sia soddisfacente per tutti gli\glossario{stakeholders}. Essa serve a garantire che il prodotto sviluppato sia adeguato rispetto ai bisogni emersi nell'attività di Analisi. 
La progettazione permette di:
\begin{itemize}
	\item costruire l'architettura logica del prodotto;
	\item facilitare la codifica da parte dei \textit{Programmatori}, riducendo la complessità del problema originale, permettendo di organizzare i compiti organizzativi e possibilmente facilitando la parallelizzazione;
	\item perseguire la correttezza per costruzione.
\end{itemize}
Come da \texttt{Piano di Progetto v1.0.0} la Progettazione si svolgerà nell'arco di due periodi distinti, duranti i quali verranno prodotti rispettivamente i documenti Specifica Tecnica e Definizione di Prodotto.

\subparagraph{Specifica tecnica}
Contiene le specifiche riguardanti la progettazione ad alto livello del\glossario{prodotto}e delle sue componenti, inoltre descrive i diagrammi\glossario{UML}utilizzati per la realizzazione dell'architettura e i test di verifica. Il documento deve quindi includere:
\begin{itemize}
	\item \textbf{Diagrammi UML:}
	\begin{itemize}
		\item Diagrammi delle classi;
		\item Diagrammi dei package;
		\item Diagrammi di attività;
		\item Diagrammi di sequenza.
	\end{itemize}
	\item \textbf{Design pattern:} devono essere specificati tutti i \textit{design}\glossario{pattern} utilizzati e la scelta di questi deve essere giustificata. 
	\item \textbf{Tracciamento delle componenti:} ogni\glossario{requisito}deve riferirsi al componente che lo soddisfa.
	\item \textbf{Test di integrazione:} vengono definite delle classi di\glossario{verifica}per far si che ogni componente del sistema funzioni correttamente.
\end{itemize}

\subparagraph{Definizione di prodotto}
Descrive nel dettaglio la progettazione, integrando il contenuto della \textit{Specifica tecnica}. Inoltre specifica le definizioni delle classi e i diagrammi\glossario{UML}relativi e i test necessari alla verifica. Il documento deve quindi includere:
\begin{itemize}
	\item \textbf{Diagrammi UML:}
	\begin{itemize}
		\item Diagrammi delle classi;
		\item Diagrammi di attività;
		\item Diagrammi di sequenza;
		\item Diagrammi dei package.
	\end{itemize}
	\item \textbf{Definizione delle classi:} ogni classe deve essere descritta in maniera esaustiva;
	\item \textbf{Tracciamento delle classi:} ogni\glossario{requisito}deve essere tracciato in modo da garantire che ogni classe ne soddisfi almeno uno e che sia possibile risalire alle classi a esso associate;
	\item \textbf{Test di unità:} si definiscono dei test di unità volti a verificare che il sistema funzioni nella maniera corretta.
\end{itemize}

\subsection{Codifica}
Questa attività segue la Progettazione e consiste nell'implementazione effettiva di quanto precedentemente progettato.
Questa sezione verrà completata dopo la Revisione dei Requisiti.
\subsubsection{Stile di codifica}
Vengono elencate le norme alle quali i \textit{Programmatori} devono attenersi durante l'attività di programmazione e implementazione.
\begin{itemize}
	\item \textbf{Indentazione:} viene richiesto l'utilizzo di esattamente una tabulazione;
	\item \textbf{Parentesi:} devono essere aperte utilizzando il metodo \textit{inline}, ovvero iniziano nella stessa riga del costrutto saltando uno spazio, inoltre blocchi di codice vanno racchiusi tra parentesi graffe;
	\item \textbf{Univocità dei nomi:} il nome delle classi, dei metodi e delle variabili deve essere il più esplicativo possibile al fine di garantire una maggior comprensione;
	\item \textbf{Nomi delle costanti:} la definizione delle costanti deve essere fatta utilizzando solo lettere maiuscole.
	\item \textbf{Nomi delle classi:} il nome delle classi deve iniziare con la lettera maiuscola;
	\item \textbf{Nomi dei metodi:} il nome dei metodi deve iniziare con la lettera minuscola, se sono composti da più parole allora le parole successive iniziano con la lettera maiuscola;
	\item \textbf{Lingua:} la lingua utilizzata è l'inglese, sia per i nomi delle classi, dei metodi e delle variabili, sia per i commenti del codice scritto.
\end{itemize}

\subsubsection{Intestazione}
L'intestazione di ogni file deve essere scritta nella seguente maniera: \\
/$^{*}$\\
$^{*}$ File: nome del file \\
$^{*}$ Version: versione del file \\
$^{*}$ Date: data di creazione del file \\
$^{*}$ Author: nome di crea e successivamente di chi modifica il file \\
$^{*}$ \\
$^{*}$ License: tipo di licenza \\
$^{*}$ \\
$^{*}$ History: registro delle modifiche \\
$^{*}$ Author $\vert$$\vert$ Date $\vert$$\vert$ Description \\
$^{*}$ \\
$^{*}$\textbackslash

\subsubsection{Versionamento}
Viene inserito all'interno dell'intestazione, descritta precedentemente, utilizzando la seguente nomenclatura:
\begin{center}
	\textbf{X.Y}
\end{center}
dove:
\begin{itemize}
	\item \textbf{X:} rappresenta l'indice primario, l'avanzamento di questo corrisponde una maggior stabilità del file, inoltre comporta l'azzeramento dell'indice Y.
	\item \textbf{Y:} rappresenta l'indice di modifica, il suo incremento corrisponde a una modifica o verifica del file. 
\end{itemize}
La versione \textit{1.0} rappresenta un file completo e stabile in grado di essere testato, questo significa che le funzionalità obbligatorie sono state implementate e pertanto sono disponibili per essere testate.





