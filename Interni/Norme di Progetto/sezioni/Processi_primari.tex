\section{Processi primari}
\subsection{Fornitura}			
Questo \glossario{processo} ha lo scopo di trattare le norme e i termini che i membri del gruppo ZeroSeven sono tenuti a rispettare per diventare fornitori della proponente Zero12 s.r.l e dei committenti Prof. Tullio Vardanega e Prof. Riccardo Cardin. \\
L'aspettativa del gruppo è instaurare e mantenere un rapporto di collaborazione costante con Zero12 s.r.l. per poter soddisfare al meglio i bisogni dell'azienda cliente e rispettare i vincoli richiesti.

\subsubsection{Attività}
\paragraph{Studio di fattibilità} 
Dopo la presentazione dei capitolati, è compito del \textit{Responsabile} convocare le riunioni necessarie per consentire al gruppo di esprimere idee e opinioni riguardo ai diversi capitolati. Le informazioni raccolte verranno utilizzate dagli \textit{Analisti} per redigere lo \textit{Studio di Fattibilità}. L'analisi si basa sui seguenti punti:
\begin{itemize}
	\item \textbf{tecnologie utilizzate:} se queste sono interessanti e soprattutto non sono state ancora applicate dalla maggior parte del gruppo.
	\item \textbf{interesse futuro:} ovvero viene valutato se si desidera perseguire una carriera, o semplicemente avere più competenze, nell'ambito di lavoro richiesto dal capitolato.
	\item \textbf{individuazione rischi:} vengono valutati gli eventuali punti critici che il gruppo potrebbe incontrare nel corso del progetto.
\end{itemize}

\paragraph{Piano di Progetto}
Il \textit{Responsabile} si occupa della scrittura del \textit{Piano di Progetto} con l'obiettivo di organizzare le attività con efficienza e misurare l'avanzamento del progetto, pertanto il documento in questione dovrà contenere:
\begin{itemize}
	\item \textbf{Analisi dei Rischi:} vengono analizzati sia i rischi, che potrebbero insorgere durante lo svolgimento del progetto, sia le possibili soluzioni ai rischi tenendo conto dei costi e delle scadenze.
	\item \textbf{Pianificazione:} nel dettaglio delle attività da svolgere.
	\item \textbf{Consuntivo di periodo e Preventivo a finire:} viene redatto il consultivo di periodo al fine di ogni attività in tal modo si potranno confrontare i risultati ottenuti con quelli attesi, questi ultimi saranno visibili nel preventivo, il quale conterrà una stima del lavoro necessario.
\end{itemize}

\paragraph{Piano di Qualifica}


\paragraph{Verbali Esterni}
Vengono verbalizzati tutti gli incontri avvenuti con l'azienda Zero12 s.r.l., la redazione del verbale esterno è compito degli \textit{Analisti}. I verbali esterni verranno inclusi nella documentazione fornita e inoltre si cercherà, ove possibile, di inviare una copia del suddetto verbale all'azienda cliente.

\subsection{Sviluppo Scopo - Aspettative - Descrizione}
Il processo di sviluppo comprende tutte le attività e i compiti svolti dal gruppo per produrre il software richiesto dal proponente.

\subsubsection{Attività}
\paragraph{Analisi dei requisiti Scopo - Aspettative - Descrizione}
\subparagraph{Casi d'uso}
\subparagraph{Codice identificativo UC}
\subparagraph{Requisiti}
\subparagraph{Codice identificativo requisiti}
\paragraph{Progettazione Scopo - Aspettative - Descrizione}
\subparagraph{Specifica tecnica}
\subparagraph{Definizione di prodotto}
\paragraph{Codifica Scopo - Aspettative - Descrizione}
\subparagraph{Stile di codifica}
\subparagraph{Intestazione}
\subparagraph{Versionamento}
\subparagraph{Ricorsione}
\paragraph{Strumenti}
\subparagraph{Da definire}
\subparagraph{Da definire}
\subparagraph{Da definire}



