\section{Processi primari}
\subsection{Fornitura}			
Questo \glossario{processo} ha lo scopo di trattare le norme e i termini che i membri del gruppo ZeroSeven sono tenuti a rispettare per diventare fornitori della proponente Zero12 s.r.l e dei committenti Prof. Tullio Vardanega e Prof. Riccardo Cardin. \\
L'aspettativa del gruppo è instaurare e mantenere un rapporto di collaborazione costante con Zero12 s.r.l. per poter soddisfare al meglio i bisogni dell'azienda cliente e rispettare i vincoli richiesti.

\subsubsection{Attività}
\paragraph{Studio di fattibilità} 
Dopo la presentazione dei capitolati, è compito del \textit{Responsabile} convocare le riunioni necessarie per consentire al gruppo di esprimere idee e opinioni riguardo ai diversi capitolati. Le informazioni raccolte verranno utilizzate dagli \textit{Analisti} per redigere lo \textit{Studio di Fattibilità}. L'analisi si basa sui seguenti punti:
\begin{itemize}
	\item \textbf{tecnologie utilizzate:} se queste sono interessanti e soprattutto non sono state ancora applicate dalla maggior parte del gruppo.
	\item \textbf{interesse futuro:} viene valutato se si desidera perseguire una carriera, o semplicemente avere più competenze, nell'ambito di lavoro richiesto dal capitolato.
	\item \textbf{individuazione rischi:} vengono valutati gli eventuali punti critici che il gruppo potrebbe incontrare nel corso del progetto.
\end{itemize}

\paragraph{Piano di Progetto}
Il \textit{Responsabile} si occupa della scrittura del \textit{Piano di Progetto} con l'obiettivo di organizzare le attività con efficienza e misurare l'avanzamento del progetto, pertanto il documento in questione dovrà contenere:
\begin{itemize}
	\item \textbf{Analisi dei Rischi:} vengono analizzati sia i rischi, che potrebbero sorgere durante lo svolgersi del progetto, sia le possibili soluzioni a questi rischi, tenendo conto dei relativi costi e delle scadenze.
	\item \textbf{Pianificazione:} si organizza nel dettaglio le attività da svolgere.
	\item \textbf{Consuntivo di periodo e Preventivo a finire:} viene redatto il consultivo di periodo alla fine di ogni attività. In questo modo si possono confrontare i risultati ottenuti con quelli attesi. Questi ultimi sono visibili nel preventivo, il quale contiene una stima del lavoro necessario.
\end{itemize}

\paragraph{Piano di Qualifica}
Gli \textit{Amministratori} documentano le norme per la verifica e la validazione dei prodotti e dei processi, la verifica avviene con costanza affinché non vengano introdotti errori. Il documento contiene:
\begin{itemize}
	\item \textbf{Qualità di processo}: per garantire la qualità dei processi attraverso l'utilizzo di strumenti e metriche, inoltre è necessario adottare come standard \textit{ISO/IEC 15504}.
	\item \textbf{Qualità di prodotto}: per garantire e misurare la qualità dei prodotti sviluppati, è necessario adottare come standard\textit{ISO/IEC 15010}.
\end{itemize}

\paragraph{Verbali Esterni}
Vengono verbalizzati tutti gli incontri avvenuti con l'azienda Zero12 s.r.l., la redazione del verbale esterno è compito degli \textit{Analisti}. I verbali esterni verranno inclusi nella documentazione fornita, inoltre si cercherà, ove possibile, di inviare una copia del suddetto verbale all'azienda cliente.

\subsection{Sviluppo}
Il \glossario{processo} di sviluppo comprende tutte le attività e i compiti svolti dal gruppo per produrre il software richiesto dal proponente. Dallo sviluppo del prodotto ci si aspetta che questo soddisfi i test di verifica e di validazioni, gli obiettivi imposti e i bisogni dell'azienda cliente. Il processo di sviluppo si svolge secondo lo standard \textit{ISO/IEC 12207}.

\subsubsection{Attività}

\paragraph{Analisi dei requisiti}
Gli \textit{Analisti} si occupano di analizzare nel dettaglio i requisiti che il prodotto finale dovrà soddisfare, il risultato viene poi documentato nell' \textit{Analisi dei Requisiti}. Vengono utilizzati i casi d'uso per l'analisi e la ricerca dei requisiti.

\subparagraph{Casi d'uso}
Ogni caso d'uso deve presentare i seguenti campi:
\begin{itemize}
	\item Codice identificativo
	\item Titolo
	\item Diagramma \glossario{UML}
	\item Attori primari
	\item Attori secondari
	\item Scopo e descrizione
	\item Precondizione
	\item Postcondizione
	\item Flusso principale degli eventi
\end{itemize}

\subparagraph{Diagrammi UML}
Per la stesura dei diagrammi UML vengono adottate le seguenti convenzioni:
\begin{itemize}
	\item \textbf{Versione:} lo standard utilizzato per i grafici UML è la \textit{v2.0}.
	\item \textbf{Lingua:} la lingua utilizzata è l'italiano.
\end{itemize}

\subparagraph{Codice identificativo CU}
Ogni caso d'uso deve avere la seguente nomenclatura:

\begin{center}
	\textbf{UC$\Bigl\{$XX$\Bigr\}$.$\Bigl\{$YY$\Bigr\}$}
\end{center}
dove:
\begin{itemize}
	\item \textbf{UC:} Use Case
	\item \textbf{{XX}:} numero che identifica i casi d'uso.
	\item \textbf{{YY}:} numero progressivo che identifica i sottocasi, esso può, a sua volta, includere altri sottocasi.
\end{itemize}

\subparagraph{Requisiti}
Ogni requisito deve presentare i seguenti campi:
\begin{itemize}
	\item Codice identificativo
	\item Descrizione
	\item Fonti
\end{itemize}

\subparagraph{Codice identificativo requisiti}
Ogni requisito deve avere la seguente nomenclatura:
\begin{center}
	\textbf{R$\Bigl\{$A$\Bigr\}$$\Bigl\{$B$\Bigr\}$$\Bigl\{$XX$\Bigr\}$.$\Bigl\{$YY$\Bigr\}$}
\end{center}
dove:
\begin{itemize}
	\item \textbf{A:} corrisponde a uno dei seguenti requisiti:
	\begin{itemize}
		\item 1: funzionale
		\item 2: di qualità
		\item 3: di prestazione
		\item 4: di vincolo
	\end{itemize}
	\item \textbf{B:} corrisponde a uno dei seguenti requisiti:
	\begin{itemize}
		\item O: obbligatorio
		\item F: facoltativo
		\item D: desiderabile
	\end{itemize}
	\item \textbf{{XX}:} numero che identifica i requisiti.
	\item \textbf{{YY}:} numero progressivo che identifica i sottocasi, esso può, a sua volta, includere altri sottocasi.
\end{itemize}

\paragraph{Progettazione}
La progettazione deve poter dimostrare che tutti i requisiti specificati nell'\textit{Analisi dei Requisiti} siano rispettati. Per far ciò, c'è bisogno della stesura dei seguenti documenti:
\begin{itemize}
	\item Specifica tecnica
	\item Definizione di Prodotto
\end{itemize}

\subparagraph{Specifica tecnica}
Contiene le specifiche riguardanti la progettazione ad alto livello del prodotto e delle sue componenti, inoltre descrive i diagrammi UML utilizzati per la realizzazione dell'architettura e i test di verifica. Il documento deve quindi includere:
\begin{itemize}
	\item \textbf{Diagrammi UML:}
	\begin{itemize}
		\item Diagrammi delle classi
		\item Diagrammi dei package
		\item Diagrammi di attività
		\item Diagrammi di sequenza
	\end{itemize}
	\item \textbf{Design pattern:} devono essere specificati tutti i \glossario{design pattern} utilizzati e la scelta di questi deve essere giustificata. 
	\item \textbf{Tracciamento delle componenti:} ogni requisito deve riferirsi al componente che lo soddisfa.
	\item \textbf{Test di integrazione:} vengono definite delle classi di verifica per far si che ogni componente del sistema funzioni correttamente.
\end{itemize}

\subparagraph{Definizione di prodotto}
Descrive nel dettaglio la progettazione, integrando il contenuto della \textit{Specifica tecnica}. Inoltre specifica le definizioni delle classi e i diagrammi UML relativi e i test necessari alla verifica. Il documento deve quindi includere:
\begin{itemize}
	\item \textbf{Diagrammi UML:}
	\begin{itemize}
		\item Diagrammi delle classi
		\item Diagrammi di attività
		\item Diagrammi di sequenza
	\end{itemize}
	\item \textbf{Definizione delle classi:} ogni classe deve essere descritta in maniera esaustiva.
	\item \textbf{Tracciamento delle classi:} ogni requisito deve essere tracciato in modo da garantire che ogni classe ne soddisfi almeno uno e che sia possibile risalire alle classi a esso associate.
	\item \textbf{Test di unità:} si definiscono dei test di unità volti a verificare che il sistema funzioni nella maniera corretta.
\end{itemize}

\paragraph{Codifica}
\subparagraph{Stile di codifica}
Vengono elencate le norme alle quali i \textit{Programmatori} devono attenersi durante l'attività di programmazione e implementazione.
\begin{itemize}
	\item \textbf{Indentazione:} viene richiesto l'utilizzo di esattamente una tabulazione.
	\item \textbf{Parentesi:} devono essere aperte utilizzando il metodo \textit{inline}, ovvero iniziano nella stessa riga del costrutto saltando uno spazio, inoltre blocchi di codice vanno racchiusi tra parentesi graffe.
	\item \textbf{Univocità dei nomi:} il nome delle classi, dei metodi e delle variabili deve essere il più esplicativo possibile al fine di garantire una maggior comprensione.
	\item \textbf{Nomi delle costanti:} la definizione delle costanti deve essere fatta utilizzando solo lettere maiuscole.
	\item \textbf{Nomi delle classi:} il nome delle classi deve iniziare con la lettera maiuscola.
	\item \textbf{Nomi dei metodi:} il nome dei metodi deve iniziare con la lettera minuscola, se sono composti da più parole allora le parole successive iniziano con la lettera maiuscola.
	\item \textbf{Lingua:} la lingua utilizzata è l'inglese, sia per i nomi delle classi, dei metodi e delle variabili, sia per i commenti del codice scritto.
\end{itemize}

\subparagraph{Intestazione}
L'intestazione di ogni file deve essere scritta nella seguente maniera: \\
/$^{*}$\\
$^{*}$ File: nome del file \\
$^{*}$ Version: versione del file \\
$^{*}$ Date: data di creazione del file \\
$^{*}$ Author: nome di crea e successivamente di chi modifica il file \\
$^{*}$ \\
$^{*}$ License: tipo di licenza \\
$^{*}$ \\
$^{*}$ History: registro delle modifiche \\
$^{*}$ Author $\vert$$\vert$ Date $\vert$$\vert$ Description \\
$^{*}$ \\
$^{*}$\textbackslash

\subparagraph{Versionamento}
Viene inserito all'interno dell'intestazione, descritta precedentemente, utilizzando la seguente nomenclatura:
\begin{center}
	\textbf{X.Y}
\end{center}
dove:
\begin{itemize}
	\item \textbf{X:} rappresenta l'indice primario, l'avanzamento di questo corrisponde una maggior stabilità del file, inoltre comporta l'azzeramento dell'indice Y.
	\item \textbf{Y:} rappresenta l'indice di modifica, il suo incremento corrisponde a una modifica o verifica del file. 
\end{itemize}
La versione \textit{1.0} rappresenta un file completo e stabile in grado di essere testato, questo significa che le funzionalità obbligatorie sono state implementate e pertanto sono disponibili per essere testate.

\paragraph{Strumenti}
Di seguito vengono elencati gli strumenti utilizzati durante lo svolgimento del progetto:

\subparagraph{Instagantt}
Questo tool viene affiancato ad \textit{Asana} e permette la costruzione dei diagrammi di Gantt con le task che vengono effettivamente create dal gruppo.
\subparagraph{Trender}
Viene utilizzato per il tracciamento dei requisiti ed è disponibile alla seguente repository \url{https://github.com/campagna91/Trender}.
\subparagraph{UML Designer}
Questo software serve per la creazione dei diagrammi UML, è possibile scaricarlo al seguente indirizzo \url{http://www.umldesigner.org}.



