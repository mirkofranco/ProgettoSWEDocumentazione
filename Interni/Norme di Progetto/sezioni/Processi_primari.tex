\section{Processi primari}
\subsection{Fornitura}			
Questo \glossario{processo} ha lo scopo di trattare le norme e i termini che i membri del gruppo ZeroSeven sono tenuti a rispettare per diventare fornitori della proponente Zero12 s.r.l e dei committenti Prof. Tullio Vardanega e Prof. Riccardo Cardin. \\
L'aspettativa del gruppo è instaurare e mantenere un rapporto di collaborazione costante con Zero12 s.r.l. per poter soddisfare al meglio i bisogni dell'azienda cliente e rispettare i vincoli richiesti.

\subsubsection{Attività}
\paragraph{Studio di fattibilità} 
Dopo la presentazione dei capitolati, è compito del \textit{Responsabile} convocare le riunioni necessarie per consentire al gruppo di esprimere idee e opinioni riguardo ai diversi capitolati. Le informazioni raccolte verranno utilizzate dagli \textit{Analisti} per redigere lo \textit{Studio di Fattibilità}. L'analisi si basa sui seguenti punti:
\begin{itemize}
	\item \textbf{tecnologie utilizzate:} se queste sono interessanti e soprattutto non sono state ancora applicate dalla maggior parte del gruppo.
	\item \textbf{interesse futuro:} ovvero viene valutato se si desidera perseguire una carriera, o semplicemente avere più competenze, nell'ambito di lavoro richiesto dal capitolato.
	\item \textbf{individuazione rischi:} vengono valutati gli eventuali punti critici che il gruppo potrebbe incontrare nel corso del progetto.
\end{itemize}

\paragraph{Piano di Progetto}
Il \textit{Responsabile} si occupa della scrittura del \textit{Piano di Progetto} con l'obiettivo di organizzare le attività con efficienza e misurare l'avanzamento del progetto, pertanto il documento in questione dovrà contenere:
\begin{itemize}
	\item \textbf{Analisi dei Rischi:} vengono analizzati sia i rischi, che potrebbero insorgere durante lo svolgimento del progetto, sia le possibili soluzioni ai rischi tenendo conto dei costi e delle scadenze.
	\item \textbf{Pianificazione:} nel dettaglio delle attività da svolgere.
	\item \textbf{Consuntivo di periodo e Preventivo a finire:} viene redatto il consultivo di periodo al fine di ogni attività in tal modo si potranno confrontare i risultati ottenuti con quelli attesi, questi ultimi saranno visibili nel preventivo, il quale conterrà una stima del lavoro necessario.
\end{itemize}

\paragraph{Piano di Qualifica}
I \textit{Verificatori} documentano le norme per la verifica e la validazione dei prodotti e dei processi, la verifica avviene con costanza affinché non vengano introdotti errori. Il documento dovrà contenere:
\begin{itemize}
	\item \textbf{Visione generale della strategia di verifica:} in cui si stabiliscono le procedure di controllo qualità, tenendo conto di risorse a disposizioni e vincoli.
	\item \textbf{Controllo di qualità di processo e dei documenti:} per garantire la qualità del prodotto attraverso l'utilizzo di alcuni strumenti e metriche
	\item \textbf{Gestione della revisione:} per stabilire le modalità di comunicazione degli errori
	\item \textbf{Resoconto delle attività di verifica:} viene fatto alla fine di ogni attività e vengono riportate le metriche calcolate oltre che al resoconto sulla verifica effettuata.
\end{itemize}

\paragraph{Verbali Esterni}
Vengono verbalizzati tutti gli incontri avvenuti con l'azienda Zero12 s.r.l., la redazione del verbale esterno è compito degli \textit{Analisti}. I verbali esterni verranno inclusi nella documentazione fornita e inoltre si cercherà, ove possibile, di inviare una copia del suddetto verbale all'azienda cliente.

\subsection{Sviluppo}
Il \glossario{processo} di sviluppo comprende tutte le attività e i compiti svolti dal gruppo per produrre il software richiesto dal proponente. Dallo sviluppo del prodotto ci si aspetta che questo soddisfi i test di verifica e di validazioni, gli obiettivi imposti e i bisogni dell'azienda cliente. Il processo di sviluppo si svolge secondo lo standard \textit{ISO/IEC 12207}.

\subsubsection{Attività}

\paragraph{Analisi dei requisiti}
Gli \textit{Analisti} si occupano di analizzati nel dettaglio i requisiti che il prodotto finale dovrà soddisfare, il risultato viene poi documentato nell' \textit{Analisi dei Requisiti}. Vengono utilizzati i casi d'uso per l'analisi e la ricerca dei requisiti.

\subparagraph{Casi d'uso}
Ogni caso d'uso deve presentare i seguenti campi:
\begin{itemize}
	\item Codice identificativo
	\item Titolo
	\item Diagramma \glossario{UML}
	\item Attori primari
	\item Attori secondari
	\item Scopo e descrizione
	\item Precondizione
	\item Postcondizione
	\item Flusso principale degli eventi
\end{itemize}

\subparagraph{Codice identificativo CU}
Ogni caso d'uso deve avere la seguente nomenclatura:

\begin{center}
	\textbf{UC$\Bigl\{$XX$\Bigr\}$-$\Bigl\{$YY$\Bigr\}$}
\end{center}
dove:
\begin{itemize}
	\item \textbf{UC:} Use Case
	\item \textbf{{XX}:} iniziali dell'attore primario
	\item \textbf{{YY}:} numero progressivo che identifica i sottocasi, esso può a sua volta includere altri sottocasi.
\end{itemize}

\subparagraph{Requisiti}
\subparagraph{Codice identificativo requisiti}
\paragraph{Progettazione Scopo - Aspettative - Descrizione}
\subparagraph{Specifica tecnica}
\subparagraph{Definizione di prodotto}
\paragraph{Codifica Scopo - Aspettative - Descrizione}
\subparagraph{Stile di codifica}
\subparagraph{Intestazione}
\subparagraph{Versionamento}
\subparagraph{Ricorsione}
\paragraph{Strumenti}
\subparagraph{Da definire}
\subparagraph{Da definire}
\subparagraph{Da definire}



