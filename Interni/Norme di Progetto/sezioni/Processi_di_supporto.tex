\section{Processi di Supporto}
\subsection{Documentazione}
Il processo di Documentazione stabilisce una serie di norme e convenzioni per la stesura dei documenti così da renderli formali, validi e coerenti.
Un documento viene definito formale quando viene approvato dal \textit{Responsabile di progetto}.
Ogni documento formale verrà classificato come:
\begin{itemize}
	\item \textbf{Interno:} documenti ad uso esclusivo del gruppo ZeroSeven, quali Norme di progetto e Studio di fattibilità;
	\item \textbf{Esterno:} documenti consultabili anche da attori esterni al gruppo ZeroSeven.
	
\end{itemize}

\subsubsection{Nomenclatura}
Tutti i documenti formali, esclusi i verbali, seguiranno questo sistema di nomenclatura:
\begin{itemize}
	\item \textbf{Nome del documento:} Il nome del documento deve essere scritto in \glossario{Upper Camel Case}.
	NormeDiProgetto per il documento corrente. 
	\item \textbf{Versione:} La documentazione prodotta deve essere corredata del numero di versione secondo la seguente codifica:
	
	\textbf{v.X.Y.Z}
	
	dove:
	\begin{itemize}
		\item \textbf{X:} indica il numero crescente di uscite formali. Sarà compito del responsabile di progetto azzerare gli indici Y e Z all'uscita di ogni rilascio.
		\item \textbf{Y:} indica lo stato del documento secondo la seguente numerazione:
		\begin{enumerate}
			\setcounter{enumi}{-1}
			\item per la documentazione in fase di sviluppo;
			\item per la documentazione in fase di verifica;
			\item per la documentazione in fase di approvazione;
			\item per la documentazione approvata e formale.
		\end{enumerate}
		\item \textbf{Z:} indica il numero crescente di modifiche apportate al documento. Ogni modifica deve essere riscontrabile con il diario delle modifiche. Deve essere azzerato quando il responsabile approva il documento. 	
	\end{itemize}
\end{itemize}

I documenti verranno citati secondo il formato NomeDocumento v.X.Y.Z mentre i file saranno rinominati \texttt{NomeDocumento\_v.X.Y.Z.pdf} \\
I verbali sia interni che esterni seguiranno la nomenclatura VerbaleUsoData.

\subsubsection{Ciclo di vita documentazione}
Ogni documento formale deve passare gli stadi di ”Sviluppo”, ”Verifica” e ”Approvato”.
\begin{itemize}
	\item \textbf{Sviluppo:} inizia con la creazione del documento e termina con la conclusione	della stesura di tutte le sue parti. In questa fase i Redattori aggiungono le parti assegnate tramite ticket;
	
	\item  \textbf{Verifica:} il documento entra nella fase di verifica dopo l’assegnazione da
	parte del Responsabile. I Verificatori effettueranno le procedure di controllo
	dello stesso.
	Al termine del controllo in caso positivo il documento entra automaticamente
	in fase di ”Approvazione”, altrimenti i loro riscontri vengono consegnati
	al Responsabile di Progetto, che provvederà ad assegnare nuovamente il
	documento ad un Redattore attraverso una nuova fase di Sviluppo;
	
	\item  \textbf{Approvazione:} l’approvazione di un documento coincide con il superamento
	positivo da parte di un verificatore. 
	Sarà onere del Responsabile decidere, dopo un attenta lettura, se approvare il documento per il rilascio esterno o se è necessario modificare il documento.

	
	\item  \textbf{Approvato:} il documento è pronto per il rilascio esterno.
\end{itemize}


\subsubsection{Struttura dei documenti}
Al fine di uniformare la struttura grafica e di permettere ai membri del gruppo di concentrarsi solo sulla stesura del contenuto è stato creato un template \LaTeX.
Ogni documento sarà composto da:
\begin{itemize}
	\item Frontespizio;
	\item Diario delle modifiche;
	\item Indice;
	\item Contenuto delle pagine interne;
	\item Elenco delle immagini e tabelle.
\end{itemize}
\paragraph{Frontespizio}
La prima pagina di tutti i documenti dovrà essere così composta:
\begin{itemize}
	\item Titolo del progetto
	\item Titolo del documento
	\item Logo e nome del gruppo
	\item Descrizione in forma tabellare contenente informazioni importanti quali:
	\begin{itemize}
		\item Versione del documento;
		\item Data di Redazione;
		\item Redatore;
		\item Verifica;
		\item Approvazione;
		\item Uso;
		\item Distribuzione;
		\item Email di contatto.
	\end{itemize}
\end{itemize}

\paragraph{Diario delle modifiche}
Successivo al frontespizio deve essere sempre presente un registro riassuntivo delle modifiche del documento in forma tabellare contenente:
\begin{itemize}
\item Versione dopo la modifica;
\item Data della pubblicazione della modifica;
\item Breve descrizione della modifica effettuata;
\item Autore della modifica;
\item Ruolo.
\end{itemize}
\paragraph{Indice}
Tutti i documenti, eccezion fatta per i verbali, devono contenere un indice il quale permette una visione macroscopica del contenuto del documento,
permettendo una lettura ipertestuale e non necessariamente sequenziale.
\paragraph{Struttura delle pagine interne}
Le pagine interne dei documenti rispettano i canoni previsti dal template \LaTeX e oltre al contenuto interno sono composte da: \\

\begin{itemize}
	\item \textbf{intestazione:}
	\begin{itemize}
		\item Logo del gruppo posto a sinistra;
		\item Nome del capitolo corrente a destra. 
	\end{itemize}
	\item \textbf {Piè di pagina:}
	\begin{itemize}
	\item titolo del documento completo di versione, posto a sinistra;
	\item numero di pagina posto a destra.
	\end{itemize}

\end{itemize} 
 \subsubsection{Elenco delle immagini e tabelle} 
Al termine di ogni documento dovrà essere presente un \textit{Elenco delle figure} presenti ed un \textit{Elenco delle tabelle}.
\subsubsection{Norme tipografiche}
Questa sezione racchiude le convenzioni riguardanti tipografia, ortografia e uno stile uniforme e disciplinato per tutti i documenti.
\paragraph{Punteggiatura}
La punteggiatura (chiamata anche interpunzione) è un sistema di segni che sono usati per iniziare, terminare o separare frasi, parti di frasi o parole. \\
Ogni segno della punteggiatura va sempre unito all'ultima lettera della parola che lo precede e separato con uno spazio dalla lettera iniziale della parola che lo segue.
le lettere maiuscole vanno poste solo dopo il punto, il punto di domanda, il punto esclamativo e all’inizio di ogni elemento di un elenco puntato, oltre che dove previsto dalla lingua italiana. È inoltre utilizzata l’iniziale maiuscola nel nome del team, del progetto, dei documenti, dei ruoli di progetto, delle fasi di
lavoro e nelle parole Proponente e Committente.
\paragraph{Formati}
\begin{itemize}
\item  Date: le date presenti nei documenti devono seguire la notazione definita dallo standard ISO 8601:2004 YYYY-MM-DD

dove:
\begin{itemize}
	\item  YYYY: rappresenta l’anno scritto utilizzando quattro cifre;
	\item MM: rappresenta il mese scritto utilizzando due cifre;
	\item DD: rappresenta il giorno scritto utilizzando due cifre.
\end{itemize}
  \item  Monospace: sarà utilizzato il carattere monospace per formattare il testo contenente parti di codice, comandi, nomi di classi;
  \item  Percorsi: per gli indirizzi email e web  deve essere utilizzato il comando
  \LaTeX \url, mentre per gli indirizzi relativi va usato il formato monospace;
  \item Maiuscolo: l'utilizzo del carattere maiuscolo per l'intera parola è riservato esclusivamente agli acronimi.
\end{itemize}
\paragraph{Stile del testo}
\begin{itemize}
	\item \textbf{Corsivo:}
	il corsivo deve essere utilizzato esclusivamente nei seguenti casi:
	\begin{itemize}
		\item Citazioni: Le frasi citate andranno scritte in corsivo;
		\item Parole del \textit{Glossario}: Alla prima occorrenza le parole inserite nel glossario oltre ad avere una G a pedice saranno scritte in corsivo;
		\item Ruoli: Dovrà essere utilizzato il corsivo quando si parla di ruoli di progetto (es. \textit{Analista});
		\item Documenti: Il nome di un documento andrà scritto in corsivo (es. \textit{Norme di progetto}).
	\end{itemize}
	\item  \textbf{Grassetto:} il grassetto può essere utilizzato esclusivamente negli elenchi puntati per dare risalto al concetto sviluppato.
	\item  \textbf{Sottolineato:} Non è previsto l'uso del testo sottolineato.
	\end{itemize}
\paragraph{Norme redazionali} 

\begin{itemize}
	\item \textbf{Elenchi:} Al termine di ogni punto di un elenco verrà utilizzato il carattere (;) eccetto per l'ultimo punto per il quale verrà utilizzato il carattere (.);
	\item \textbf{Riferimenti informativi:} Ogni riferimento a prodotti, guide, software,
	libri esterno al progetto dovrà essere indicato tramite un’annotazione a piè di
	pagina.
	\item \textbf{\LaTeX:} Ogni riferimento a \LaTeX  verrà scritto utilizzando il comando \texttt{\textbackslash LaTeX};
	\item \textbf{Comandi \LaTeX:} sono stati realizzati dei comandi personalizzati al fine di evitare errori di battitura e unificare tutti i documenti facilitando il lavoro degli \textit{Analisti}.
	\begin{itemize}
		\item \textbf{ \textbackslash intestazioni}: inserisce un'intestazione personalizzata con il nome del documento e il logo ZeroSeven
		\item \textbf{ \textbackslash mailzeroseven}: inserisce l'indirizzo email del gruppo per un eventuale contatto
		\item \textbf{ \textbackslash progetto}: inserisce il nome del progetto
		\item \textbf{ \textbackslash logo}: inserisce il logo del gruppo ZeroSeven
		\item \textbf{ \textbackslash glossario}: indica un termine da inserire nel glossario(marcato da una G maiuscola a pedice).	
	\end{itemize}
	
	\item \textbf{Sigle:} Nonostante sarà preferito l'utilizzo delle parole per intero potranno essere utilizzate le seguenti sigle:
	\begin{itemize}
	\item AdR = Analisi dei Requisiti;
	\item NdP = Norme di Progetto;
	\item PdP = Piano di Progetto;
	\item PdQ = Piano di Qualifica;
	\item SdF = Studio di Fattibilità;
	\item RR = Revisione dei Requisiti;
	\item RQ = Revisione di Qualifica;
	\item RP = Revisione di Progetto;
	\item Ra = Revisione di Accettazione.
	\end{itemize}
\end{itemize}
\paragraph{Componenti grafiche}
	\begin{itemize}
	\item \textbf{Tabelle:} 
	Ogni tabella presente all'interno dei documenti deve essere accompagnata da una didascalia,	in cui deve comparire un numero identificativo incrementale per la tracciatura della stessa all'interno del documento.
	\item \textbf{Immagini:}
	Le immagini presenti all'interno dei documenti devono essere nel formato Scalable Vector Graphics (SVG). In questo modo si garantisce una maggior qualità dell'immagine in caso di ridimensionamento. Per consentire l’inclusione delle immagini nei documenti,
	le immagini dovranno essere convertite nel formato PDF. Qualora non sia possibile
	salvare le immagine in formato vettoriale è preferito il formato Portable Network
	Graphics (PNG).

	\end{itemize}
\subsection{Verifica}
La \glossario{verifica} è un processo atto ad evidenziare ed eliminare la possibile presenza di errori.
Di seguito verranno descritti gli strumenti e le pratiche utilizzate per la verifica del codice e dei documenti durante la loro realizzazione.
Durante questa prima fase di progetto la verifica si è focalizzata principalmente su documenti e diagrammi.
\subsubsection{Analisi statica}
L'analisi statica di un programma o documento permette di rilevare problemi al loro interno. Si può svolgere principalmente in due modi:
\begin{itemize}
\item \textbf{Walkthrough:} questa metodologia prevede una revisione ad ampio spettro parallelizzabile, così da stilare una prima lista degli errori più comuni. Risulta fondamentale nelle prime fasi di sviluppo del progetto dal momento che è il più semplice da imparare, di contro però è molto dispendioso in termini di efficienza;
\item \textbf{Inspection} al contrario della metodologia precedente, questa risulta essere molto più veloce perché attraverso una lista di controllo degli errori e delle misurazioni effettuate permette un'analisi più efficace delle criticità, omettendo le parti che non presentano problematiche. 
\end{itemize}
\subsubsection{Analisi dinamica}
L'analisi dinamica al contrario dell'analisi statica è applicabile unicamente al software e non ai documenti. Il processo prevede la realizzazione e l'esecuzione di test sul codice. Ogni test per essere valido deve essere ripetibile, cioè attraverso il medesimo input di partenza si deve poter risalire a un dato output. A tal fine è necessario che per ogni test vengano segnati i seguenti riferimenti: 
\begin{itemize}
\item \textbf{Ambiente:} sono i sistemi software e hardware utilizzati nel corso dei test;
\item \textbf{Stato iniziale:} lo stato di partenza del prodotto prima del test;
\item \textbf{Input};
\item \textbf{Output};
\item \textbf{Avvisi}: un eventuale insieme di istruzioni riguardanti l'esecuzione del test e i suoi risultati.
\end{itemize}
\subsubsection{Verifica diagrammi UML}
Sarà compito dei verificatori accertarsi che tutti i diagrammi UML prodotti rispettino lo standard UML e che siano corretti semanticamente.
\subsubsection{Strumenti}
\paragraph{Verifica ortografica}
Per la verifica ortografica verrà utilizzato lo strumento integrato in TexStudio, mentre l'analisi del periodo sarà effettuata dai Verificatori.
\paragraph{Indice di Gulpease}
Per il calcolo dell'/glossario{indice di Gulpease} il gruppo utilizzerà strumenti automatici disponibili online quali Farfalla Project \footnote{\url{https://farfalla-project.org/readability_static}} e Corrige!it \footnote{\url{http://www.corrige.it}}.

\subsection{Validazione}
Il processo di Validazione verrà svolto nella fase finale del progetto e ha l'obbiettivo di verificare che il prodotto sia conforme a quanto pianificato.