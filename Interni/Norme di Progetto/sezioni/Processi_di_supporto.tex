\section{Processi di Supporto}
\subsection{Documentazione}
Il processo di Documentazione stabilisce una serie di norme e convenzioni per la stesura dei documenti in modo da renderli formali, validi e coerenti.
Un documento viene definito formale quando viene approvato dal \textit{Responsabile di progetto}.
Ogni documento formale verrà classificato come interno o esterno con le seguenti differenze:
\begin{itemize}
	\item \textbf{Interno:} ad uso del gruppo ZeroSeven... 
	\item \textbf{Esterno:} ad uso anche esterno (da sistemare). 
	
\end{itemize}

\subsubsection{Nomenclatura}
Tutti i documenti formali, esclusi i verbali, seguiranno questo sistema di nomenclatura:
\begin{itemize}
	\item \textbf{Nome del documento:} Il nome del documento deve essere scritto in \textit{Upper Camel Case$_{G}$}.
	NormeDiProgetto per il documento corrente. 
	\item \textbf{Versione:} La documentazione prodotta deve essere corredata del numero di versione secondo la seguente codifica:
	
	\textbf{v.X.Y.Z}
	
	dove:
	\begin{itemize}
		\item \textbf{X:} indica il numero crescente di uscite formali. Sarà compito del responsabile di progetto azzerare gli indici Y e Z all'uscita di ogni rilascio.
		\item \textbf{Y:} indica lo stato del documento secondo la seguente numerazione:
		\begin{enumerate}
			\setcounter{enumi}{-1}
			\item per la documentazione in fase di sviluppo;
			\item per la documentazione in fase di verifica;
			\item per la documentazione in fase di approvazione;
			\item per la documentazione approvata e formale.
		\end{enumerate}
		\item \textbf{Z:} indica il numero crescente di modifiche apportate al documento. Ogni modifica deve essere riscontrabile con il diario delle modifiche. Deve essere azzerato quando il responsabile approva il documento. 	
	\end{itemize}
\end{itemize}

I documenti verranno citati secondo il formato NomeDocumento v.X.Y.Z mentre i file saranno rinominati \texttt{NomeDocumento\_v.X.Y.Z.pdf} \\
I verbali sia interni che esterni seguiranno la nomenclatura VerbaleUsoData.

\subsubsection{Ciclo di vita documentazione}
Ogni documento formale deve passare gli stadi di ”Sviluppo”, ”Verifica” e ”Approvato”.
\begin{itemize}
	\item \textbf{Sviluppo:} inizia con la creazione del documento e termina con la conclusione	della stesura di tutte le sue parti. In questa fase i Redattori aggiungono le parti assegnate tramite ticket;
	
	\item  \textbf{Verifica:} il documento entra nella fase di verifica dopo l’assegnazione da
	parte del Responsabile. I Verificatori effettueranno le procedure di controllo
	dello stesso.
	Al termine del controllo in caso positivo il documento entra automaticamente
	in fase di ”Approvazione”, altrimenti i loro riscontri vengono consegnati
	al Responsabile di Progetto, che provvederà ad assegnare nuovamente il
	documento ad un Redattore attraverso una nuova fase di Sviluppo;
	
	\item  \textbf{Approvazione:} l’approvazione di un documento coincide con il superamento
	positivo da parte di un verificatore. 
	Sarà onere del Responsabile decidere, dopo un attenta lettura, se approvare il documento per il rilascio esterno o se è necessario modificare il documento.

	
	\item  \textbf{Approvato:} il documento è pronto per il rilascio esterno.
\end{itemize}


\subsubsection{Struttura dei documenti}
Al fine di uniformare la struttura grafica e di permettere ai membri del gruppo di concentrarsi solo sulla stesura del contenuto è stato creato un template \LaTeX.
Ogni documento sarà composto da:
\begin{itemize}
	\item Frontespizio;
	\item Diario delle modifiche;
	\item Indice;
	\item Contenuto delle pagine interne;
	\item Elenco delle immagini/tabelle.
\end{itemize}
\paragraph{Frontespizio}
La prima pagina di tutti i documenti dovrà essere così composta:
\begin{itemize}
	\item Titolo del progetto
	\item Titolo del documento
	\item Logo e nome del gruppo
	\item Descrizione in forma tabellare contenente informazioni importanti quali:
	\begin{itemize}
		\item Versione del documento;
		\item Data di Redazione;
		\item Redatore;
		\item Verifica;
		\item Approvazione;
		\item Uso;
		\item Distribuzione;
		\item Email di contatto.
	\end{itemize}
\end{itemize}

\paragraph{Diario delle modifiche}
Successivo al frontespizio deve essere sempre presente un registro riassuntivo delle modifiche del documento in forma tabellare contenente:
\begin{itemize}
\item Versione dopo la modifica;
\item Data della pubblicazione della modifica;
\item Breve descrizione della modifica effettuata;
\item Autore della modifica;
\item Ruolo.
\end{itemize}
\paragraph{Indice}
Tutti i documenti, eccezion fatta per i verbali, devono contenere un indice il quale permette una visione macroscopica del contenuto del documento,
permettendo una lettura ipertestuale e non necessariamente sequenziale.
\paragraph{Struttura delle pagine interne}
Le pagine interne dei documenti rispettano i canoni previsti dal template \LaTeX e oltre al contenuto interno sono composte da: \\

**********************************************

(Valutare se modificare il template)
\begin{itemize}
	\item \textbf{intestazione:}
	\begin{itemize}
		\item Logo del gruppo posto a sinistra;
		\item Nome del capitolo corrente a destra. 
	\end{itemize}
	\item \textbf {Piè di pagina:}
	\begin{itemize}
	\item titolo del documento completo di versione, posto a sinistra;
	\item numero di pagina posto a destra.
	\end{itemize}

\end{itemize}
 ********************************************
 
 \subsubsection{Elenco delle immagini/tabelle}
 ...
 
 ...
\subsubsection{Norme tipografiche}
Questa sezione racchiude le convenzioni riguardanti tipografia, ortografia e uno stile
uniforme e disciplinato per tutti i documenti.
\paragraph{Punteggiatura}
La punteggiatura (chiamata anche interpunzione) è un sistema di segni che sono usati per iniziare, terminare o separare frasi, parti di frasi o parole. \\
Ogni segno della punteggiatura va sempre unito all'ultima lettera della parola che lo precede e separato con uno spazio dalla lettera iniziale della parola che lo segue.
le lettere maiuscole vanno poste solo dopo il punto, il punto di domanda, il punto esclamativo e all’inizio di ogni elemento di un elenco puntato, oltre che dove previsto dalla lingua italiana. È inoltre utilizzata l’iniziale maiuscola nel nome del team, del progetto, dei documenti, dei ruoli di progetto, delle fasi di
lavoro e nelle parole Proponente e Committente.
\paragraph{Formati}
\begin{itemize}
\item  Date: le date presenti nei documenti devono seguire la notazione definita dallo
standard ISO(?) YYYY-MM-DD

dove:
\begin{itemize}
	\item  YYYY: rappresenta l’anno scritto utilizzando quattro cifre;
	\item MM: rappresenta il mese scritto utilizzando due cifre;
	\item DD: rappresenta il giorno scritto utilizzando due cifre.
\end{itemize}
  \item  Monospace: sarà utilizzato il carattere monospace per formattare il testo contenente parti di codice, comandi, nomi di classi;
  \item  Percorsi: per gli indirizzi email e web  deve essere utilizzato il comando
  \LaTeX \url, mentre per gli indirizzi relativi va usato il formato monospace;
  \item Maiuscolo: l'utilizzo del carattere maiuscolo per l'intera parola è riservato esclusivamente agli acronimi.
  \item etc...
\end{itemize}
\paragraph{Stile del testo}
\begin{itemize}
	\item \textbf{Corsivo:}
	il corsivo deve essere utilizzato esclusivamente nei seguenti casi:
	\begin{itemize}
		\item Citazioni: Le frasi citate andranno scritte in corsivo;
		\item Parole del \textit{Glossario}: Alla prima occorrenza le parole inserite nel glossario oltre ad avere una G a pedice saranno scritte in corsivo;
		\item Ruoli: Dovrà essere utilizzato il corsivo quando si parla di ruoli di progetto (es. \textit{Analista});
		\item Documenti: Il nome di un documento andrà scritto in corsivo (es. \textit{Norme di progetto}).
	\end{itemize}
	\item  \textbf{Grassetto:} il grassetto può essere utilizzato esclusivamente negli elenchi puntati per dare risalto al concetto sviluppato.
	\item \textbf{\LaTeX:} Ogni riferimento a \LaTeX verrà scritto utilizzando il comando \texttt{\textbackslash LaTeX};
	\end{itemize}
\paragraph{Altre norme} 
Le seguenti norme serviranno a rendere la stesura dei documenti un processo coerente (frase da rivedere!!!).
\begin{itemize}
	\item \textbf{Elenchi:} Al termine di ogni punto di un elenco verrà utilizzato il carattere (;) eccetto per l'ultimo punto per il quale verrà utilizzato il carattere (.);
	\item \textbf{Riferimenti informativi:} Ogni riferimento a prodotti, guide, software,
	libri esterno al progetto dovrà essere indicato tramite un’annotazione a piè di
	pagina.
	\item \textbf{Comandi \LaTeX:} sono stati realizzati dei comandi personalizzati al fine di evitare errori di battitura e unificare tutti i documenti facilitando il lavoro dei Redattori. (COMANDI ANCORA DA CREARE!!!)
	*******************************
	comandi che andrebbero creati... nome del gruppo, parola del glossario, nome del proponente e del committente, nomi dei documenti e altri che mi verranno in mente.
	*******************************
	\item 
\end{itemize}

\subsection{Attività}
\paragraph{Analisi dei requisiti}
Questo documento sarà redatto dagli \textit{Analisti$_{G}$} e dovrà includere
\begin{itemize}
	\item{\textbf{Casi d'uso:}}  devono essere descritti i casi d’uso con una descrizione testuale e
	un diagramma rappresentativo dove utile per una più immediata comprensione;
	\item{\textbf{Requisiti:}}  devono essere descritti i casi d’uso con una descrizione testuale e
	un diagramma rappresentativo dove utile per una più immediata comprensione;
	\item{\textbf{Tracciamento dei requisiti:}}  ogni requisito deve essere tracciato alle fonti da cui è stato generato.
\end{itemize}
\paragraph {Casi d'uso}
Ogni caso d'uso è descritto dalla seguente struttura:
\begin{itemize}
\item Codice identificativo;
\item Titolo;
\item Diagramma UML;
\item Attori primari;
\item Attori secondari;
\item Scopo e descrizione;
\item Precondizione;
\item Postcondizione;
\item Flusso base degli eventi;
\item Inclusioni;
\item Estensioni;
\end{itemize}
Ogni caso d’uso è identificato da un codice, che segue il seguente formalismo: \\

\textbf{ UC [codice padre].[codice livello]} \\ 

Dove:
\begin{itemize}
\item{\textbf{ codice padre:}}  numero che identifica univocamente i casi d’uso;
\item{\textbf{ codice livello:}}  numero progressivo che identifica i sottocasi. Può a sua volta includere altri livelli.
\end{itemize}
\paragraph{Requisiti}
Ogni requisito è strutturato come segue:

\begin{itemize}
	\item Codice identificativo;
	\item Descrizione;
	\item Fonti;
\end{itemize}
Ogni requisito ha il seguente formalismo:\\

\textbf{ R[Importanza][Tipologia][codice padre].[codice livello]} \\

Dove:
\begin{itemize}
	\item \textbf{Importanza:} identifica uno dei seguenti gradi di necessità:
	\begin{itemize}
		\item \textbf{O:} requisito obbligatorio;
		\item \textbf{D:} requisito desiderabile;
		\item \textbf{F:} requisito facoltativo.
		
	\end{itemize}
	\item \textbf{Tipologia:}  identifica uno dei seguenti tipi di requisito:
	\begin{itemize}
	\item 1: requisito funzionale;
	\item 2: requisito prestazionale;
	\item 3: requisito qualitativo;
	\item 4: requisito progettuale.
	
	\end{itemize}
	\item \textbf{codice padre:} numero che identifica in modo univoco i requisiti;
	\item \textbf{codice livello:} numero progressivo che identifica i sottorequisiti e può a sua volta includere altri livelli.
\end{itemize}
