\newpage
\section{Introduzione}
\subsection{Scopo del documento}
Il seguente documento espone le norme che i membri del gruppo\glossario{ZeroSeven}dovranno obbligatoriamente adottare nello svolgimento del progetto "\glossario{MegAlexa}".
Ogni membro ha il dovere di leggere il documento e di seguire le regole in esso esposte per garantire  un modo di lavorare comune, massimizzare l'efficenza e l'efficacia riducendo il numero di errori.
Nel documento verranno definite le  norme riguardanti:
\begin{itemize}
		\item L'identificazione dei ruoli e delle relative mansioni da svolgere;
		\item L'interazione tra i membri del gruppo e con le entità esterne;
		\item Le modalità di lavoro durante le fasi di\glossario{progetto};
		\item La stesura dei documenti;
		\item L'ambiente di sviluppo
\end{itemize}
Le norme verranno prodotte incrementalmente, al progressivo maturare delle esigenze di progetto, trattando prima quelle più impellenti e ricorrenti e successivamente quelle che interverranno più avanti, sempre garantendo che ogni attività da svolgere sia stata precedentemente normata.

\subsection{Scopo del prodotto}
Lo scopo del prodotto è di realizzare una \textit{skill$_{G}$} per \textit{Alexa$_{G}$} di \textit{Amazon$_{G}$} in grado di avviare dei \textit{workflow$_{G}$} creati dagli utenti tramite interfaccia \textit{web$_{G}$} o mobile app per \textit{iOS$_{G}$} e \textit{Android$_{G}$}.

\subsection{Glossario}
I termini tecnici, gli acronimi e le abbreviazioni che necessitano di un chiarimento
o di una spiegazione verranno riportati in modo chiaro e conciso nel
documento Glossario vx.x.x, al fine di evitare ogni ambiguità di linguaggio
e massimizzare la comprensione dei documenti. Ogni occorrenza di vocaboli
presenti nel Glossario è marcata da una “G” maiuscola in pedice.

\subsection{Riferimenti}
\subsubsection{Normativi}
\begin{itemize}
	\item \glossario{Capitolato} C4
	\footnote{\url{https://www.math.unipd.it/~tullio/IS-1/2018/Progetto/C4.pdf}};
	
	\item ISO/IEC 12207
	\footnote{\url{https://www.iso.org/standard/43447.html}};
	
	\item ISO 8601
	\footnote{\url{https://www.iso.org/iso-8601-date-and-time-format.html}};
\end{itemize}
\subsubsection{Informativi}
\begin{itemize}
	\item Software Engineering - Ian Sommerville - 9th Edition (2010);
	
	\item Slide del corso "Ingegneria del Software" - Amministrazione\\ di progetto\footnote{\url{https://www.math.unipd.it/~tullio/IS-1/2018/Dispense/FC1.pdf}};
	
	\item Sito di\glossario{GitHub}\footnote{\url{https://github.com/}};
	
	\item Piano di Progetto;
	\item Piano di Qualifica;
	
	\item \glossario{SonarQube}\footnote{\url{https://www.sonarqube.org/}};
\end{itemize}
\newpage
