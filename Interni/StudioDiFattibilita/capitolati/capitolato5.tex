\chapter{Capitolato C5: GaiaGo}
\section{Descrizione Generale}

Il quinto\glossario{capitolato}propone la creazione di un servizio di Car Sharing Peer to Peer, per condividere la propria automobile con altre persone.


\section{Finalit\`a del progetto}
L'obiettivo è lo sviluppo di un sistema software che mette in contatto coloro che vogliono condividere la propria macchina e chi ha bisogno di utilizzarla.\\
Il possessore dell'automobile comunica all'applicazione i giorni di dispobilit\`a del suo veicolo.
Chi ha bisogno di affittare un'automobile può vedere quali sono i veicoli disponibili ed effettuare una prenotazione.

\section{Tecnologie interessate}
\begin{itemize}
\item \textbf{Framework Octalysis}: tecnologia per lo sviluppo Peer to Peer;
\item \textbf{Henshin movens}: piattaforma software per applicazioni che iteragiscono con città intelligenti e la mobilità;
\item \textbf{Node.Js$_{G}$}: piattaforma per l'esecuzione di codice JavaScript Server-side.
\end{itemize}
\section{conclusione}
Le tecnologie utilizzate sono ritenute poco interessanti dai membri, oltre al fatto che servizi di car sharing sono già disponibili nel mercato. Per questi motivi, il\glossario{capitolato}non è stato preso in considerazione. 
