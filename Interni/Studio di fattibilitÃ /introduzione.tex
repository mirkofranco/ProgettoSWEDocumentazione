\chapter{Introduzione}
\section{Scopo del prodotto}
Lo scopo del progetto è quello di sviluppare un applicativo Web e Mobile in grado di creare delle routine personalizzate per gli utenti gestibili tramite \glossario{Alexa} di \glossario{Amazon}. L'obbiettivo è quello di creare \glossario{skill} in grado di avviare \glossario{workflow} creati dagli utenti fornendogli dei \glossario{connettori}.
\section{Glossario}
Al fine di evitare ogni ambiguità di linguaggio e massimizzare la comprensione dei documenti, i termini tecnici, di dominio, gli acronimi e le parole che necessitano di essere chiarite, sono riportate nel \texttt{Glossario v1.0.0}.\\
Ogni occorrenza di vocaboli presenti nel \texttt{Glossario} è marcata da una "G" maiuscola in pedice.
\section{Riferimenti}
\subsection{Normativi}
\begin{itemize}
	\item Norme di Progetto
\end{itemize}
\subsection{Informativi}
\begin{itemize}
\item \textbf{Capitolato 1: Butterfly} - monitor per processi CI/CD;
\item \textbf{Capitolato 2: Colletta} - piattaforma raccolta dati di analisi di testo
\item \textbf{Capitolato 3: G\&B} - monitoraggio intelligente di processi DevOps
\item \textbf{Capitolato 4: MegAlexa} - arricchitore di skill di Amazon Alexa
\item \textbf{Capitolato 5: P2PCS} - piattaforma di peer-to-peer car sharing
\item \textbf{Capitolato 6: Soldino} - piattaforma Ethereum per pagamenti IVA

\end{itemize}
