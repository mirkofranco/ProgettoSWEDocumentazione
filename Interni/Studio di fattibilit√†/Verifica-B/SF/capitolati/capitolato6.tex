\chapter{Capitolato C6: Soldino}
\section{Descrizione Generale}

Il \textit{capitolato$_{G}$} C6 prevede lo sviluppo di Soldino, una piattaforma con lo scopo di agevolare la compravendita di beni e servizi, automatizzando il pagamento dell'iva.\\
Questo servizio verrà distrubito dal Governo e sarà basato su Ethereum. 


\section{Finalit\`a del progetto}
Soldino è rivolto principalmente a tre fasce d'utenza:
\begin{itemize}
	\item governo
	\item imprenditori
	\item cittadini
\end{itemize}
La compravendita di beni avviene tramite \textit{smart contracts}. La moneta utilizzata è il \textit{Cubit}: una valuta elettronica basata su Ethereum.
Il governo può creare e distribuire Cubit. Gli imprenditori possono registrarsi al sistema, creare i beni o servizi e pagare le imposte al governo. I cittadini possono acquistare beni o servizi dagli imprenditori (in particolare dalle loro aziende) e registrarsi per diventare imprenditori.\\
In particolare, i macro moduli da sviluppare sono gli \textit{smart contracts} e l'interfaccia utente. Gli \textit{smart contracts} si interfacciano con la rete Ethereum per la gestione dei \textit{Cubit}. L'interfaccia utente deve permettere a tutti di utilizzare i servizi, in modo semplice e trasparente.


\section{Tecnologie interessate}
\begin{enumerate}
	\item Truffle: framework per lo sviluppo di Ethereum
	\item Metamask: permette un accesso sicuro alla rete Ethereum
	\item Surge: web server per il front end
	\item Ethereum network Ropsten: servizio online di staging, per testare la rete Ethereum prima di introdurla nella rete principale
	
\end{enumerate}
\section{Potenziali Criticità}
Gli aspetti negativi sono:
\section{conclusione}
Ethereum è una tecnologia interessante, ma il team la considera troppo specifica e complessa da imparare. Per questo motivo, Soldino non è stato considerato. 