
\chapter{Capitolato scelto C4: MegAlexa}
\section{Informazioni sul Capitolato}
\begin{itemize}
 \item \textbf{Nome:} MegAlexa;
\item \textbf{Proponente:}Zero12;
\item \textbf{Committenti:}Prof.Tullio Vardegna, Prof.Riccardo Cardin;
\end{itemize}

\section{Descrizione generale}
L'obiettivo di MegAlexa è la  realizzazione di un applicativo web o mobile in grado di creare dei Workflow per delle skill che poi verranno integrate ad \emph{Alexa di Amazon}.
L'applicativo mobile può essere per smartphone con sistema operativo Android oppure iOS.
\section{Finalit\`a del progetto}
All'utente verranno forniti dei connettori che possono essere inseriti all'interno di un Workflow che poi verrà eseguito tramite controllo vocale.  
Per il controllo vocale verrà utilizzato \emph{Amazon Alexa} la quale sarà collegata ad \emph{Amazon Web Services} per gestire ed elaborare i dati raccolti utilizzando vari servizi come \emph{API Gateway}, \emph{Aurora Serverless} e \emph{DynamoDB}.
I risultati poi verranno forniti tramite GUI web, o mobile, oppure "a voce" da \emph{Amazon Alexa} stessa.
\section{Tecnologie interessate}
\begin{itemize}
	\item \textbf{Alexa}: assistente vocale di Amazon basato su cloud;
	\item \textbf{Amazon Web Services}: insieme di servizi di cloud computing utili per il progetto tra cui:
	\begin{itemize}
	\item \textbf{API Gateway}: servizio per la creazione, manutenzione e protezione di API;
	\item \textbf{Aurora Serverless}: consente di eseguire il codice senza considerare il server;
	\item \textbf{DynamoDB}: database non relazionale per applicazioni che necessitano di elevate prestazioni.
	\end{itemize}
	\item \textbf{Node.js}: piattaforma per il motore Javascript V8
	\item \textbf{HTML5, CSS3, Javascript, Bootstrap} - tecnologie di base per la gestione di interfaccia web lato client;
	\item \textbf{Swift, Kotlin}: Linguaggi di programmazione rispettivamente per Android e iOS.
\end{itemize}
\section{Potenziali Criticità}
Gli aspetti negativi sono:
\section{Conclusione}
Il fatto che il progetto proponga di lavorare con una tecnologia nata di recente come \emph{Amazon Alexa} ha portato il team a scegliere questo capitolato. Inoltre il gruppo ha trovato particolarmente interessante l'idea di sviluppare un applicativo mobile. Infine per lo studio delle tecnologie coinvolte, le quali possono tornare utili in futuro.