\chapter{Capitolato C2: Colletta}
\section{Descrizione generale}
Il capitolato in esame richiede lo sviluppo di una piattaforma per la raccolta dati, principalmente per aiutare l'azienda nei processi di \texttt{Machine Learning} dei propri sistemi informatici.

\section{Finalit\`a del progetto}
Lo scopo principale del progetto è raccogliere dati utili per l'apprendimento automatico assistito dei sistemi informatici. Sebbene gli esempi riportati nella presentazione ipotizzassero una soluzione attraverso l'implementazione di un applicativo in grado di gestire esercizi di analisi grammaticale, tale requisito non risulta vincolante, ed è possibile avanzare proposte alternative.

\section{Tecnologie interessate}
L'azienda proponente ha lasciato ampie libertà sulle scelte riguardanti le tecnologie da utilizzare nello sviluppo di questo progetto. Tuttavia sono fortemente consigliate le seguenti.
\begin{itemize}
	\item \textbf{FreeLing/Hunpos} nel caso in cui si proceda nella realizzazione di una piattaforma perl'analisi grammaticale, tali tecnologie permettono di etichettare le singole parole di una frase (\texttt{pos-tagging}) con codici specifici che determinano la loro funzione.
	\item \textbf{Firebase} per la memorizzazione di grandi quantità di dati
\end{itemize}
\section{Potenziali Criticità}
Gli aspetti negativi sono:
\section{Conclusione}
Il capitolato in esame è stato scartato, le motivazioni di tale scelta vanno ricercate nella volontà del gruppo di sviluppare un progetto comprendente tecnologie nuove e innovative, per un posizionamento più competitivo nel mercato del lavoro. Si ritiene che lo sviluppo di una piattaforma di raccolta dati non soddisfi appieno questi prerequisiti. 