\chapter{Introduzione}
\section{Scopo del documento}
Il seguente documento  descrive le motivazioni che hanno portato il gruppo alla scelta del capitolato C4 “MegAlexa”. Vengono successivamente descritti i restanti capitolati, fornendo chiarimenti sulla loro esclusione.
\section{Scopo del prodotto}
Lo scopo del prodotto è la realizzazione  di  una skill$_{G}$ per Alexa$_{G}$ di Amazon$_{G}$ in grado di avviare dei workflow$_{G}$ creati dagli utenti tramite interfaccia web$_{G}$ o mobile app per iOS$_{G}$ e Android$_{G}$. 
\section{Glossario} 

 I termini tecnici, gli acronimi e le abbreviazioni che necessitano  di un chiarimento o di una spiegazione verranno riportati in modo chiaro e conciso nel documento Glossario vx.x.x, al fine di evitare ogni ambiguità di linguaggio e massimizzare la comprensione dei documenti.
 Ogni occorrenza di vocaboli presenti nel Glossario è marcata da una “G”
 maiuscola in pedice.
 \begin{center}
 	
 \end{center}

\section{Riferimenti}
\subsection{Normativi}
\begin{itemize}
	\item Norme di Progetto: %todo
\end{itemize}
\subsection{Informativi}
\begin{itemize}
\item \textbf{Capitolato 1: Butterfly} - monitor per processi CI/CD;\\
\url{https://www.math.unipd.it/~tullio/IS-1/2018/Progetto/C1.pdf}

\item \textbf{Capitolato 2: Colletta} - piattaforma raccolta dati di analisi di testo\\
\url{https://www.math.unipd.it/~tullio/IS-1/2018/Progetto/C2.pdf}
\item \textbf{Capitolato 3: G\&B} - monitoraggio intelligente di processi DevOps\\
\url{https://www.math.unipd.it/~tullio/IS-1/2018/Progetto/C3.pdf}
\item \textbf{Capitolato 4: MegAlexa} - arricchitore di skill di Amazon Alexa\\
\url{https://www.math.unipd.it/~tullio/IS-1/2018/Progetto/C4.pdf}
\item \textbf{Capitolato 5: P2PCS} - piattaforma di peer-to-peer car sharing\\
\url{https://www.math.unipd.it/~tullio/IS-1/2018/Progetto/C5.pdf}
\item \textbf{Capitolato 6: Soldino} - piattaforma Ethereum per pagamenti IVA\\
\url{https://www.math.unipd.it/~tullio/IS-1/2018/Progetto/C6.pdf}

\end{itemize}
