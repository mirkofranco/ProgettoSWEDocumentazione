\chapter{Capitolato C1: Butterfly}
\section{Descrizione generale}
Il primo capitolato prevede la realizzazione di una piattaforma in grado di gestire le segnalazioni provenienti dagli strumenti utilizzati nei processi di integrazione e rilascio continui in aziende di grandi dimensioni. 

\section{Finalit\`a del progetto}
Viene proposto lo sviluppo di un applicativo in grado di raccogliere e smistare le segnalazioni provenienti da tecnologie coinvolte nei processi di Continuos Delivery e Continuos Integration.
Viene delineata una distinzione tra componenti \texttt{Producer} e \texttt{Consumer}, dove i primi generano i messaggi, e i secondi sono i destinatari di quest'ultimi.\\
\\
\textbf{Producer}
	\begin{itemize}
	\item \textbf{GitLab}: Strumento utilizzato per il versionamento del codice
	\item \textbf{SonarQube}: Utilizzato per l'analisi statica della qualità del codice
	\item \textbf{Redmine}: Piattaforma web per la gestione generale di progetti di dimensioni considerevoli.
	\end{itemize}
\textbf{Consumer}
	\begin{itemize}	
		\item \textbf{Telegram}
		\item \textbf{Slack}
		\item \textbf{Email}
	\end{itemize}

Lo scopo è fornire, attraverso un pattern \texttt{Publisher/Subscriber}, una piattaforma in grado di interagire con il maggior numero possibile di tali tecnologie. 	
	
	
\section{Tecnologie interessate}
	\begin{itemize}
		\item Utilizzo dell'architettura \textbf{REST}(Representational State Transfer) per la strutturazione di un sistema distribuito.
		\item A scelta \textbf{Java/NodeJs/Python} per lo sviluppo dei componenti applicativi.
		\item \textbf{Kafka}: Sistema di messaggistica instantanea distribuito, ideale per la gestione di pattern \texttt{Publisher/Subscriber} sopra citati. 
		\item \textbf{Docker} per la containerizzazione e gestione dell'infrastruttura
	\end{itemize}


\section{Conclusione}
Il capitolato in esame non è stato preso in considerazione; tale scelta va attribuita alla grande mole di tecnologie per cui è necessaria una conoscenza approfondita.