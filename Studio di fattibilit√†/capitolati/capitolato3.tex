\chapter{Capitolato C3}
\section{Descrizione generale}
Nel capitolato C3 si richiede che venga realizzato un plug-in che applichi dei metodi di intelligenza artificiale a flussi di dati e che sia integrabile a \emph{Grafana}.
\section{Finalit\`a del progetto}
L'obbiettivo del progetto è quello di realizzare, usando \emph{Javascript}, un plug-in che dovrà essere in grado di soddisfare i seguenti punti:
\begin{enumerate}
	\item Leggere da un file json la definizione della \emph{rete Bayesiana};
	\item Associare ai nodi, di tale rete, i dati prelevati dal flusso di monitoraggio;
	\item Effettuare un nuovo calcolo delle probabilit\`a della rete secondo regole temporali prestabilite;
	\item Fornire a \emph{Grafana} i dati dei nodi della rete esclusi dal flusso di monitoraggio;
	\item Il plug-in dovrà fornire i risultati in modo da poter essere poi utilizzati per la creazione di una dashboard con grafici, per la loro visualizzazione. 
\end{enumerate}
\section{Tecnologie interessate}
\begin{itemize}
	\item \textbf{DevOps}: servizio cloud che permette di gestire lo scambio di fatture tra persone e il ministero;
	\item \textbf{Grafana}: prodotto Open Source utilizzato per il monitoraggio dei dati;
	\item \textbf{Javascript}: tecnologie di base per la gestione di interfaccia web lato client;
	\item \textbf{Reti Bayesiane}: modello grafico probabilistico;
	\item \textbf{Jsbayes}: libreria Open Source utilizzata per la gestione dei calcoli della \emph{rete Bayesiana}.
\end{itemize}
\section{Conclusione}
Alla prima lettura, il capitolato ha attirato l'attenzione del team ma, dopo un'approfondita riflessione, si è deciso di escluderlo dato che si trattava di lavorare con le \emph{reti Bayesiane}.