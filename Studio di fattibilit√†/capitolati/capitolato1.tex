\chapter{Capitolato C1: Progetto Butterfly}
\section{Descrizione generale}
Il primo capitolato propone lo sviluppo di un applicativo in grado di 
raccogliere le segnalazioni provenienti da tutte le tecnologie coinvolte nei processi di Continuos Delivery e Continuos Integration, accentrandole e standardizzandole.

\section{Finalit\`a del progetto}

 


\section{Tecnologie interessate}
\subsection{Tecnologie su cui sviluppare}
\begin{itemize}
	\item Java/NodeJs/Python%todo
	\item Kafka%todo
	\item Docker%todo
	\item Api Rest%todo
\end{itemize}

\subsection{}
\begin{itemize}
	\item \textbf{Git}: Strumento usato per il versionamento del codice
	\item \textbf{Jenkins}: Applicativo in grado di facilitare la Continuos Integration
	\item \textbf{SonarQube}: Utilizzato per l'analisi statica della qualità del codice
	\item \textbf{Redmine}: Piattaforma web per la gestione generale di progetti di dimensioni considerevoli.
\end{itemize}


\section{Conclusione}
Il capitolato in esame non è stato preso in considerazione; tale scelta va attribuita alla grande mole di tecnologie per cui è necessaria una conoscenza approfondita.